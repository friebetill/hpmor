\chapter{Das Stanford-Gefängnis-Experiment, Teil 2}

\emph{Anmerkung des Übersetzers: dieses und die nächsten Kapitel sind der
Grund, warum ich diese Geschichte ab 16 eingestuft habe....Askaban ist ein
furchtbarer Ort.}

Das Adrenalin floss bereits in Harrys Adern, sein Herz hämmerte bereits in
seiner Brust, dort in dem abgedunkelten und bankrotten Laden. Professor Quirrell
hatte seine Erklärung beendet und in einer Hand hielt Harry einen winzigen
Holzzweig, der der Portschlüssel sein würde.

\emph{Das war es, das war der Tag und der Moment, an dem Harry begann, die
Rolle zu spielen. Sein erstes richtiges Abenteuer, einen Kerker, den es zu
durchqueren galt, eine böse Regierung, der es zu trotzen galt, ein Mädchen in
Not, das gerettet werden musste.}

Harry hätte ängstlicher sein sollen, zurückhaltender, aber stattdessen fühlte er
nur, dass es an der Zeit war und überfällig, der Held zu werden, von denen er in
seinen Büchern gelesen hatte; seine Reise zu dem zu beginnen, von dem er immer
gewusst hatte, dass er dazu bestimmt war.

Den ersten Schritt auf dem Weg zu tun, der zu Kimball Kinnison und Captain
Picard und Batman und definitiv nicht zu Raistlin Majere führte.

Soweit Harrys Gehirn aus den frühmorgendlichen Zeichentrickfilmen wusste, sollte
man, wenn man erwachsen wurde, erstaunliche Kräfte erlangen und das Universum
retten, das war es, was Harrys Gehirn bei Erwachsenen gesehen und als Vorbild
für den Reifungsprozess angenommen hatte, und Harry wollte unbedingt anfangen,
erwachsen zu werden. Und wenn das Muster der Geschichte vorsah, dass der Held
als Ergebnis seines ersten Abenteuers einen Teil seiner Unschuld verliert, dann
schien es zumindest jetzt, in diesem noch unschuldigen Moment, für ihn an der
Zeit und überfällig zu sein, diesen Schmerz zu erfahren. So wie das Ablegen von
Kleidern, die ihm zu klein waren; oder wie endlich in die nächste Stufe des
Spiels aufzusteigen, nachdem er elf Jahre lang in Welt 3, Level 2 von Super
Mario Brothers feststeckte. Harry hatte genug Romane gelesen, um zu ahnen, dass
er sich danach nicht mehr so begeistert fühlen würde, also genoss er es, solange
es anhielt.

Es gab ein knallendes Geräusch, als etwas in Harrys Nähe verschwand, und dann
war keine Zeit mehr für heldenhaftes Grübeln. Harrys Hand schnappte nach dem
kleinen Holzzweig. Ein Haken zerrte bewegungslos an Harrys Bauch, als sich der
Portschlüssel aktivierte, der sich diesmal viel stärker anfühlte als die
kleineren Transporte zwischen dem Hogwarts-Gelände und der Winkelgasse -

und ihn mitten in ein gewaltiges Donnergrollen fallen ließ, das verging, und ein
Peitschenhieb kalten Regens, der ihm ins Gesicht peitschte, wobei das Wasser
Harrys Brille beschmierte und ihn in einem Augenblick blendete und die Welt in
ein verschwommenes Bild verwandelte, selbst als er begann, auf die tosenden
Ozeanwellen weit unten zu fallen. Er war hoch, hoch, hoch über der leeren
Nordsee angekommen. Der Schock des tosenden Sturms brachte Harry fast dazu, den
Besen loszulassen, den Professor Quirrell ihm gegeben hatte, was keine gute Idee
gewesen wäre. Es dauerte fast eine ganze Sekunde, bis Harry seinen Verstand
zusammennahm und seinen Besen mit einem leichten Schwung wieder nach oben
brachte.

\glqq{}Ich bin hier\grqq{}, sagte eine unbekannte Stimme aus einem Stück leerer
Luft über ihm; tief und kiesig, die Stimme des bleichen, schlaksigen, bärtigen
Mannes, in den sich Professor Quirrell mit Vielsaft Trank verwandelt hatte,
bevor er sich und seinen Besen desillusionierte.

\glqq{}Ich bin hier\grqq{}, sagte Harry unter dem Unsichtbarkeitsumhang hindurch.
Er selbst hatte keinen Vielsaft Trank benutzt. Einen anderen Körper zu tragen,
behinderte die eigene Magie, und Harry konnte all seine wenige Magie gebrauchen;
daher sah der Plan vor, dass Harry fast immer unsichtbar bleiben sollte, anstatt
den Trank zu benutzen.

(Keiner von ihnen hatte den Namen des anderen ausgesprochen. Man benutzte echte
Namen einfach zu keinem Zeitpunkt während einer illegalen Mission, selbst wenn
man unsichtbar über einem anonymen Fleckchen Wasser in der Nordsee schwebte. Das
tat man einfach nicht. \emph{Es wäre dumm.})

Sorgfältig hielt er sich mit einer Hand am Besen fest, während der Regen und der
Wind um ihn herum heulten, und mit einem ebenso vorsichtigen Griff hob Harry
seinen Zauberstab und rückte seine Brille zurecht. Dann, als die Linsen klar
waren, schaute Harry sich um. Er war von Wind und Regen umgeben, es mochte
vielleicht fünf Grad Celsius sein, wenn er Glück hatte; er hatte sich bereits
einen wärmenden Zauber auferlegt, nur weil er im Februar draußen war, aber der
hielt den treibenden kalten Tropfen nicht stand. Schlimmer als Schnee, der Regen
durchnässte jede Oberfläche. Der Unsichtbarkeitsumhang machte alles unsichtbar,
aber er bedeckte nicht alles, und das bedeutete, dass er nicht alle vor Regen
schützte. Harrys Gesicht war der vollen Wucht des eindringenden Wassers
ausgesetzt, und es floss direkt in seinen Nacken und durchnässte sein Hemd, auch
die Ärmel seiner Roben und die Manschetten seiner Hose und seine Schuhe, das
Wasser nahm jedes Stückchen Stoff als Möglichkeit, sich einzuschleichen.

\glqq{}Hier entlang\grqq{}, sagte die vielsagende Stimme, und ein Funke grünen
Lichts leuchtete vor Harrys Besen auf und huschte dann in eine Richtung davon,
die Harry wie jede andere Richtung erschien. Durch den blendenden Regen folgte
Harry. Manchmal verlor er ihn, diesen kleinen grünen Funken, und jedes Mal, wenn
er ihn verlor, rief Harry, und der Funke tauchte ein paar Sekunden später wieder
vor ihm auf. Als Harry den Trick verstanden hatte, dem Funken zu folgen,
beschleunigte er, und Harry trat den Besen in den höchsten Gang und folgte ihm.
Der Regen peitschte ihn stärker und fühlte sich so an, wie Harry sich
vorstellte, dass es sich anfühlen musste, das Gesicht voller Schrotkugeln zu
bekommen, aber seine Brille blieb klar und schützte seine Augen. Erst ein paar
Minuten später, bei voller Geschwindigkeit des Besens, erhaschte Harry einen
Blick auf einen riesigen Schatten durch den Regen, der weit über das Wasser
ragte.

Und er spürte ein fernes, hohles Echo der Leere, das von dort ausstrahlte, wo
der Tod wartete, und das über Harrys Geist hinwegspülte und sich um ihn herum
teilte, wie eine Welle, die sich an Stein bricht. Diesmal kannte Harry seinen
Feind, und sein Wille war stählern und ganz aus Licht.

\glqq{}Ich kann die Dementoren bereits spüren\grqq{}, sagte die kiesige Stimme
des verwandelten Quirrell. \glqq{}Das habe ich nicht erwartet, nicht so bald.\grqq{}

\glqq{}Denk an die Sterne\grqq{}, sagte Harry über ein fernes Donnergrollen
hinweg. \glqq{}Lass keine Wut in dir aufkommen, nichts Negatives, denk einfach an
die Sterne, wie es sich anfühlt, sich selbst zu vergessen und körperlos durchs
All zu fallen. Halte dich an diesem Gedanken fest, wie an einer
Okklumentikbarriere über deinem ganzen Geist. Die Dementoren werden
Schwierigkeiten haben, daran vorbeizukommen.\grqq{}

Einen Moment lang herrschte Schweigen, dann: \glqq{}Interessant.\grqq{}

Der grüne Funke hob sich, und Harry neigte seinen Besen leicht nach oben, um ihm
zu folgen, auch als er sie in eine Nebelbank steuerte, eine Wolke, die tief über
dem Wasser schwebte. Bald schwebten sie über dem riesigen dreiseitigen
Metallgebäude, das sich weit unten abzeichnete, und zwar leicht schräg. Das
Dreieck aus Stahl war hohl, nicht massiv, es war ein Gebäude aus drei dicken,
massiven Wänden und ohne Zentrum. Die Auroren, die Wache hielten, befanden sich
in der obersten Etage und an der Südseite des Gebäudes, wie Professor Quirrell
gesagt hatte, geschützt durch ihre Patronuszauber. Der legale Eingang nach
Askaban befand sich auf dem Dach in der südwestlichen Ecke des Gebäudes. Den die
beiden natürlich nicht benutzen würden. Stattdessen würden sie einen Korridor
benutzen, der direkt unter der nördlichen Ecke des Gebäudes verlief. Professor
Quirrell würde zuerst hinuntergehen und direkt an der Nordspitze ein Loch in das
Dach und seine Schutzwände stechen und eine Illusion zurücklassen, um die Lücke
zu verdecken. Die Gefangenen wurden in den Seiten des Gebäudes gehalten, in
Stockwerken, die ihren Verbrechen entsprachen. Und ganz unten, in der äußersten
Mitte und Tiefe von Askaban, lag ein Nest mit mehr als hundert Dementoren.
Gelegentlich wurde eine Ladung Erde hineingeschüttet, um das Niveau zu halten,
da die den Dementoren direkt ausgesetzte Materie zu Schlamm und Nichts
zerfiel...

\glqq{}Warte eine Minute\grqq{}, sagte die raue Stimme, \glqq{}folge mir dann
schnell, und geh vorsichtig durch.\grqq{}

\glqq{}Verstanden\grqq{}, sagte Harry leise.

Der Funke erlosch, und Harry begann zu zählen, eintausend, zweitausend,
dreitausend ... ... einundsechzigtausend, und Harry tauchte ab, der Wind
kreischte um ihn herum, als er abtauchte, hinunter zu der riesigen
Metallstruktur, hinunter zu dem Ort, an dem er die Schatten des Todes spüren
konnte, die auf ihn warteten, das Licht aussaugend und Leere ausstrahlend,
während die Metallstruktur größer und größer wurde.

Schlicht und gesichtslos ragte die riesige graue Form auf, bis auf eine einzelne
erhabene kastenförmige Struktur in der südwestlichen Ecke. Die nördliche Ecke
war einfach leer, Professor Quirrells Loch war nicht zu erkennen. Harry zog
scharf an, als er sich der Nordecke näherte, und verschaffte sich mehr
Sicherheitsabstand, als er im Flugunterricht eingehalten hätte, aber nicht zu
viel. Sobald er zum Stillstand gekommen war, begann er, seinen Besen langsam
wieder abzusenken, auf das, was wie das feste Dach der Spitze der Nordecke
aussah. Der Abstieg durch das illusorische Dach, während er unsichtbar war, war
eine seltsame Erfahrung, und dann fand sich Harry in einem Metallkorridor
wieder, der von einem schwachen orangefarbenen Licht erhellt wurde - das, wie
Harry nach einem erschrockenen Blick feststellte, von einer altmodischen,
ummantelten Gaslampe kam...

\emph{... denn in der Gegenwart von Dementoren würde die Magie nach einiger
Zeit versagen, ausgelaugt werden.}

Harry stieg von seinem Besen ab. Der Sog der Leere war jetzt stärker. Sie teilte
sich und umfloss Harry, ohne ihn zu berühren. Sie waren weit entfernt, aber sie
waren viele, die Wunden in der Welt; Harry hätte mit geschlossenen Augen auf sie
zeigen können.

\glqq{}\textbf{\emph{Wirke deinen Patronussss}}\grqq{}, zischte eine Schlange vom
Boden, die in dem schwachen orangefarbenen Licht mehr verfärbt als grün aussah.
Sogar in Parsel kam der Ton des Stresses durch.

Harry war überrascht; Professor Quirrell hatte gesagt, dass Animagi in ihrer
Animagus-Form viel weniger anfällig für Dementoren waren.\emph{ (Aus demselben
Grund, aus dem die Patronusse Tiere waren, nahm Harry an.)} Wenn Professor
Quirrell in seiner Schlangengestalt so viel Ärger hatte, was wäre dann mit ihm
passiert, wenn er in der menschlichen Gestalt war, die ihm erlaubte, seine Magie
zu benutzen...? Harrys Zauberstab hob sich bereits in seiner Hand.

\emph{Dies würde der Anfang sein. Selbst wenn es nur eine Person war, nur eine
Person, die er vor der Dunkelheit retten konnte, selbst wenn er noch nicht
mächtig genug war, um alle Gefangenen von Askaban in Sicherheit zu teleportieren
und die dreieckige Hölle bis auf den Grund niederzubrennen... Trotzdem war es
ein Anfang,} \emph{es war ein Anfang, es war eine Anzahlung auf alles, was Harry
mit seinem Leben zu erreichen gedachte. Kein Warten mehr, kein Hoffen, kein
bloßes Versprechen, es würde alles hier beginnen. Hier und jetzt.}

Harrys Zauberstab schnellte nach unten und zeigte auf die Dementoren, die weit
unten warteten. \glqq{}Expecto
Patronum!\grqq{}./p>

Die glühende humanoide Gestalt flammte auf. Es war nicht mehr das sonnenhelle
Ding, das es vorher gewesen war... wahrscheinlich, weil Harry sich nicht ganz
davon abhalten konnte, an all die anderen Gefangenen in ihren Zellen zu denken,
an die, die er nicht hier war, um sie zu retten. Vielleicht war es aber auch
besser so. Harry würde diesen Patronus eine Weile aufrechterhalten müssen, und
es wäre vielleicht besser, wenn er nicht ganz so hell wäre. Der Patronus wurde
bei diesem Gedanken noch ein wenig dunkler; und dann noch weiter, als Harry
versuchte, etwas weniger Kraft in ihn zu stecken, bis schließlich die leuchtende
humanoide Gestalt nur noch etwas heller leuchtete als der hellste Tierpatronus,
und Harry spürte, dass er ihn nicht weiter dimmen konnte, ohne zu riskieren, ihn
ganz zu verlieren.

Und dann: \glqq{}\emph{Es isst sstabi}l\grqq{}, zischte Harry und begann, seinen
Besen in seinen Beutel zu stecken. Sein Zauberstab blieb in seiner Hand, und ein
leichtes, nachhaltiges Strömen von ihm ersetzte die leichten Verluste durch
seinen Patronus.

Die Schlange verschwamm zu der Form eines schlaksigen, blassen Mannes, der
Professor Quirrells Zauberstab in einer Hand und einen Besen in der anderen
hielt. Der schlaksige Mann taumelte, als er wieder zu sich kam, und lehnte sich
für einen Moment an die Wand. \glqq{}Gut gemacht, wenn auch vielleicht ein
bisschen langsam\grqq{}, murmelte die kiesige Stimme. Professor Quirrells
Trockenheit lag darin, auch wenn sie nicht zur Stimme passte, ebenso wenig wie
der ernste Blick auf dem dickbärtigen Gesicht. \glqq{}Ich kann sie überhaupt
nicht mehr spüren.\grqq{} Einen Moment später fuhr der Besen in die Robe des Mannes
und verschwand. Dann erhob sich der Zauberstab des Mannes und tippte auf seinen
Kopf, und mit einem Geräusch wie eine knackende Eierschale verschwand er erneut.
In der Luft erblühte ein schwacher grüner Funke, und Harry, immer noch in den
Unsichtbarkeitsumhang gehüllt, folgte ihm.

Hätte man von außen zugeschaut, hätte man nichts gesehen als einen kleinen
grünen Funken, der durch die Luft schwebte, und einen silbern glänzenden
Humanoiden, der ihm folgte.

Sie gingen hinunter und hinunter und hinunter, vorbei an einer Gaslampe nach der
anderen und gelegentlich an einer riesigen Metalltür, bis sie schließlich in
völliger Stille in Askaban ankamen. Professor Quirrell hatte eine Art Barriere
errichtet, durch die er hören konnte, was in der Nähe vor sich ging, aber keine
Geräusche konnten nach außen dringen, und keine Geräusche konnten Harry
erreichen.

Harry war nicht ganz in der Lage gewesen, seinen Verstand davon abzuhalten, sich
zu fragen, warum diese Stille herrschte, oder seinen Verstand davon abzuhalten,
die Antwort zu geben. Die Antwort, die er bereits auf einer wortlosen Ebene der
Vorahnung kannte, die ihn dazu veranlasst hatte, vergeblich zu versuchen, nicht
daran zu denken.

\emph{Irgendwo hinter diesen riesigen Metalltüren schrien Menschen.}

Die silberne humanoide Gestalt schwankte, hellte sich auf und verdunkelte sich,
jedes Mal, wenn Harry daran dachte. Harry war gesagt worden, er solle sich mit
einem Blasenkopf-Zauber belegen.

\emph{Um zu verhindern, dass er etwas riecht.}

Der ganze Enthusiasmus und Heldentum hatte sich bereits abgenutzt, wie Harry
gewusst hatte das es sein würde, es hatte nicht einmal für seine Verhältnisse
lange gedauert, der Prozess hatte sich schon vollendet, als sie das erste Mal
eine dieser Metalltüren passierten. Jede Metalltür war mit einem riesigen
Schloss verriegelt, einem Schloss aus einfachem, unmagischem Metall, das einen
Hogwartsschüler im ersten Jahr nicht aufgehalten hätte -

\emph{wenn man noch einen Zauberstab hatte, wenn man noch seine Magie hatte,
was die Gefangenen nicht hatten.}

Diese Metalltüren waren nicht die Türen einzelner Zellen, hatte Professor
Quirrell gesagt, jede öffnete sich in einen Korridor, in dem sich eine Gruppe
von Zellen befand. Irgendwie half das ein wenig, nicht zu denken, dass jede Tür
direkt einem Gefangenen entsprach, der direkt dahinter wartete. Stattdessen
könnte es mehr als einen Gefangenen geben, was die emotionale Wirkung
abschwächte; genau wie die Studie, die zeigte, dass die Leute mehr spendeten,
wenn man ihnen sagte, dass ein bestimmter Geldbetrag nötig war, um das Leben
eines Kindes zu retten, als wenn man ihnen sagte, dass derselbe Gesamtbetrag
nötig war, um acht Kinder zu retten...

Harry fiel es immer schwerer, nicht daran zu denken, und jedes Mal, wenn er
es tat, schwankte das Licht seines Patronus.

Sie kamen zu der Stelle, wo der Gang nach links abbog, an der Ecke des
dreieckigen Gebäudes. Wieder gab es absteigende Metallstufen, eine weitere
Treppe; wieder gingen sie hinunter.

Einfache Mörder wurden nicht in die unterste Zelle gesteckt. Es gab immer einen
tieferen Ort, an den man gehen konnte, eine noch schlimmere Strafe zu
befürchten. Egal wie tief man schon gesunken war, die Regierung des magischen
Britanniens hatte noch eine Drohung gegen einen, wenn man noch Schlimmeres tat.
Aber Bellatrix Black war die Todesserin, die mehr Angst einflößte als jeder
andere außer Lord Voldemort selbst, eine schöne und tödliche Zauberin, die ihrem
Meister absolut treu ergeben war; sie war, wenn so etwas möglich war, sogar noch
sadistischer und bösartiger als Du-weißt-schon-wer, als ob sie versuchte, ihren
Meister auszustechen...

...das war es, was die Welt von ihr wusste, was die Welt von ihr glaubte. Aber
davor, so hatte Professor Quirrell Harry erzählt, vor dem Erscheinen des
schrecklichsten Dieners des Dunklen Lords, hatte es ein Mädchen in Slytherin
gegeben, das ruhig gewesen war, das sich meistens zurückgehalten und niemandem
etwas zuleide getan hatte. Danach hatte man sich erfundene Geschichten über sie
erzählt, Erinnerungen, die sich im Nachhinein veränderten \emph{(Harry kannte
die Forschungen dazu gut).} Aber zu der Zeit, als sie noch die Schule besuchte,
war die begabteste Hexe in Hogwarts als sanftes Mädchen bekannt gewesen
\emph{(hatte Professor Quirrell gesagt).} Ihre wenigen Freunde waren überrascht
gewesen, als sie sich den Todessern angeschlossen hatte, und noch überraschter,
dass sie so viel Dunkelheit hinter diesem traurigen, wehmütigen Lächeln
versteckt hatte. Das war es, was Bellatrix einst gewesen war, die
vielversprechendste Hexe ihrer eigenen Generation, bevor der Dunkle Lord sie
stahl und brach, sie zerbrach und neu formte, sie auf einer tieferen Ebene und
mit dunkleren Künsten als jeder Imperius an sich band. Zehn Jahre hatte
Bellatrix dem Dunklen Lord gedient, getötet, wen er sie töten ließ, gefoltert,
wen er sie foltern ließ. Und dann war der Dunkle Lord endlich besiegt worden.
Und Bellatrix' Albtraum hatte sich fortgesetzt.

\emph{Irgendwo in Bellatrix mochte etwas sein, das immer noch schrie, das die
ganze Zeit geschrien hatte, etwas, das ein psychiatrischer Heiler zurückbringen
konnte; oder es mochte nicht sein, Professor Quirrell konnte es nicht wissen.
Aber so oder so, sie könnten... ...sie könnten sie zumindest aus Askaban
herausholen...}

Bellatrix Black war in die unterste Ebene von Askaban gesteckt worden. Harry
hatte Mühe, sich nicht vorzustellen, was er sehen würde, wenn sie in ihre Zelle
kämen. Bellatrix musste am Anfang fast keine Angst vor dem Tod gehabt haben,
wenn sie überhaupt noch am Leben war. Sie stiegen eine weitere Treppe hinunter
und kamen dem Tod und Bellatrix so viel näher, das Klacken ihrer unsichtbaren
Schuhe war das einzige Geräusch, das Harry hören konnte. Das schwache
orangefarbene Licht der Gaslaternen, der schwache grüne Funke, der durch die
Luft schwebte, die leuchtende Gestalt, die ihnen folgte und deren silbernes
Licht von Zeit zu Zeit schwankte. Nachdem sie viele Male hinabgestiegen waren,
kamen sie rechtzeitig zu einem Korridor, der nicht in einer Treppe endete, und
einer letzten Metalltür, vor der der grüne Funke stehen blieb.

Harrys Herz hatte sich ein wenig beruhigt, da sie weit in die Tiefen von Askaban
hinabgestiegen waren, ohne dass etwas passiert war. Aber jetzt hämmerte es
wieder in seiner Brust. Sie waren ganz unten, und die Schatten des Todes waren
ganz nah dran.

Ein leises metallisches Klicken kam aus dem Schloss, als Professor Quirrell den
Weg öffnete. Harry holte tief Luft und erinnerte sich an alles, was Professor
Quirrell ihm gesagt hatte.

\emph{Der schwierige Teil würde nicht nur darin bestehen, die vorgetäuschte
Persönlichkeit richtig hinzubekommen, um Bellatrix Black selbst zu täuschen,
der schwierige Teil würde darin bestehen, seinen Patronus gleichzeitig am
Laufen zu halten...}

Der grüne Funke erlosch, und einen Moment später schimmerte eine meterhohe
Schlange ins Dasein, die nicht mehr unsichtbar war. Die Metalltür bewegte sich
mit einem langsamen, knarrenden Geräusch, als Harry mit seiner unsichtbaren Hand
darauf drückte, sie einen Spalt öffnete und hindurchspähte. Er sah einen geraden
Korridor, der in massivem Stein endete. Dort gab es kein Licht außer dem, das
von Harrys Patronus hereinkam. Es war hell genug, dass er die äußeren
Gitterstäbe der acht Zellen sehen konnte, die in den Korridor eingelassen waren,
aber er konnte das Innere nicht sehen; aber was noch wichtiger war, er sah
niemanden im Korridor selbst.

\glqq{}\emph{Ich sehe nichts}\grqq{}, zischte Harry.

Die Schlange huschte weiter, schlängelte sich schnell über den Boden. Einen
Moment später - \glqq{}\textbf{\emph{Sie ist allein}}\grqq{}, zischte die
Schlange.

\emph{Bleib hier}, dachte Harry an seinen Patronus, der direkt neben der Tür
Stellung bezog, als würde er sie bewachen; dann stieß Harry die Tür weiter auf
und folgte hinein.

Die erste Zelle, die Harry erblickte, enthielt eine ausgetrocknete Leiche, die
Haut war grau und fleckig, das Fleisch an einigen Stellen durchgescheuert, so
dass die Knochen darunter zum Vorschein kamen, keine Augen -

Harry schloss die Augen. Das konnte er immer noch, er war immer noch unsichtbar,
er verriet nichts, indem er die Augen schloss.

Er hatte es schon gewusst, er hatte es auf Seite sechs seines Verwandlungsbuches
gelesen, dass man in Askaban blieb, bis seine Haftzeit abgelaufen war. Wenn man
vorher starb, behielten sie einen dort, bis sie den Leichnam freigaben. War die
Strafe lebenslänglich, ließ man die Leiche einfach in der Zelle, bis die Zelle
gebraucht wurde, und warf sie dann in die Grube der Dementoren. Aber es war
trotzdem ein Schock zu sehen, dass diese Leiche ein Mensch gewesen war, den man
einfach dort gelassen hatte -

Das Licht im Raum schwankte.

\emph{Ganz ruhig,} dachte Harry in seinem Innersten. Es wäre nicht gut für
Professor Quirrell, wenn dieser Patronus bei seinen traurigen Gedanken erlöschen
würde. So nahe bei den Dementoren könnte der Verteidigungsprofessor einfach tot
umfallen, wo er stand.

\emph{Ganz ruhig, Harry James Potter-Evans-Verres, ganz ruhig!}

Mit diesem Gedanken öffnete Harry wieder die Augen, er hatte keine Zeit zu
verlieren.

In der zweiten Zelle, die er betrachtete, lag nur noch ein Skelett.

\emph{Und hinter den Gittern der dritten Zelle sah er Bellatrix Black.}

Etwas Kostbares und Unersetzliches in Harry verdorrte wie trockenes Gras. Man
konnte erkennen, dass die Frau kein Skelett war, dass ihr Kopf kein Schädel war,
denn die Textur der Haut unterschied sich immer noch von der von Knochens, egal
wie weiß und blass sie geworden war, während sie allein in der Dunkelheit
wartete. Entweder gab man ihr nicht viel zu essen, oder was sie aß, entzogen ihr
die Schatten des Todes; denn ihre Augen schienen unter den Lidern
zusammengeschrumpft, ihre Lippen sahen zu verschrumpelt aus, um sie über den
Zähnen zu schließen. Die Farbe schien aus der schwarzen Kleidung, die sie im
Gefängnis getragen hatte, ausgelaugt, als hätten die Dementoren auch diese
ausgelaugt. Sie sollten gewagt sein, diese Kleider, und jetzt lagen sie lose
über einem Skelett und enthüllten verschrumpelte Haut.

\emph{Ich bin hier, um sie zu retten, ich bin hier, um sie zu retten, ich bin
hier, um sie zu retten,} dachte Harry verzweifelt zu sich selbst, immer und
immer wieder mit einer Anstrengung, die an Okklumentik erinnerte, und er wollte,
dass sein Patronus nicht erlosch, dass er blieb und Bellatrix vor den Dementoren
beschützte -

in seinem Herzen, in seinem Kern, hielt Harry an all seinem Mitleid und seinem
Mitgefühl fest, an seinem Willen, sie vor der Dunkelheit zu retten; der silberne
Glanz, der durch die offene Tür hereinkam, hellte sich auf, noch während er es
dachte.

Und in einem anderen Teil von ihm, als würde er einfach einen anderen Teil
seines Verstandes eine Gewohnheit ausführen lassen, ohne ihr viel Aufmerksamkeit
zu schenken...  > <p
style=\grqq{}.ext-align:center\grqq{}.> <p
style=\grqq{}.ext-align:center\grqq{}.Ein kalter Ausdruck kam über Harrys
Gesicht, unsichtbar unter der Kapuze.  <p
style=\grqq{}.ext-align:center\grqq{}. \glqq{}\emph{Hallo, meine liebe
Bella}\grqq{}, sagte ein kühles Flüstern. \glqq{}\emph{Hast du mich vermisst?}\grqq{}


