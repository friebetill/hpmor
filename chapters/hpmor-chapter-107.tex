\chapter{Die Wahrheit, Teil 5}

Der Verteidigungsprofessor hatte einen Kessel aufgestellt, den er mit einem
Schwung seines Zauberstabs in die richtige Position brachte und mit einem
weiteren Schwung ein Feuer darunter entfachte. Ein kurzes Kreisen des Fingers
des Verteidigungsprofessors hatte einen langstieligen Löffel in Bewegung
gesetzt, der den Kessel weiter rührte, ohne dass er gehalten wurde. Die
indigoblauen Blütenblätter schienen im weißen Licht der Wände zu leuchten und
wölbten sich in einer Weise nach innen, die den Eindruck eines Wunsches nach
Privatsphäre vermittelte. Die ersten dieser Blüten waren sofort in den Trank
gegeben worden, aber dann hatte der Kessel noch eine Weile vor sich hin gerührt.
Der Verteidigungsprofessor hatte eine Position eingenommen, von der aus er Harry
sehen konnte, indem er nur den Kopf leicht drehte, und Harry wusste, dass er
sich im Blickfeld des Verteidigungsprofessors befand. In der Ecke wartete ein
Phönix aus verfluchtem Feuer, und etwas von dem nahen Stein begann zu glänzen,
als es zu größerer Glätte schmolz. Die brennenden Flügel warfen ein
karmesinrotes Licht ab, das allem im Raum einen blutigen Schimmer verlieh und
sich in scharlachroten Funken von den Gläsern spiegelte.

\glqq Die Zeit drängt\grqq{}, sagte Professor Quirrell. \glqq Stell deine
Fragen, wenn du welche hast.\grqq{}

\emph{Warum, Professor Quirrell, warum, warum bist du so, warum machst du dich zum Monster, warum Lord Voldemort, ich weiß, du willst vielleicht nicht dasselbe wie ich, aber ich kann mir nicht vorstellen bei dem was du willst, dass dies der beste Weg ist, es zu bekommen...}

Das war es, was Harrys Gehirn wissen wollte. Was Harry wissen \emph{musste},
war... ein Ausweg aus dem, was als Nächstes passieren würde. Aber der
Verteidigungsprofessor hatte gesagt, dass er nicht über seine Zukunftspläne
sprechen würde. Es war schon seltsam genug, dass der Verteidigungsprofessor
bereit war, über den Rest zu reden, das musste einer seiner Regeln
widersprechen...

\glqq Ich denke nach\grqq{}, sagte Harry laut.

Professor Quirrell lächelte leicht. Er benutzte einen Stößel, um die erste
magische Zutat des Zaubertranks zu mahlen, ein leuchtend rotes Sechseck. \glqq
Das verstehe ich gut\grqq{}, sagte der Verteidigungsprofessor. \glqq Aber denk
nicht zu lange nach, Kind.\grqq{}

\emph{Ziele: Lord Voldemort daran hindern, Menschen zu schaden, einen Weg
finden, ihn zu töten oder zu neutralisieren, aber zuerst den Stein holen und
Hermine wiederbeleben... }
\emph{...Professor Quirrell überzeugen, damit aufzuhören... }

Harry schluckte, drückte die Emotionen hinunter und versuchte, das Wasser nicht
in seine Augen kommen zu lassen. Tränen würden wahrscheinlich keinen guten
Eindruck auf Lord Voldemort machen. Professor Quirrell runzelte bereits die
Stirn, obwohl er aus der Richtung seines Blickes ein Blatt untersuchte, das in
lebhaften Schattierungen von Weiß, Grün und Violett gefärbt war.

Es gab keinen offensichtlichen Weg, eines der Ziele zu erreichen, noch nicht.
Alles, was Harry tun konnte, war, die Fragen zu stellen, die am ehesten
nützliche Informationen zu liefern schienen, auch wenn er noch keinen Plan
hatte.

\emph{Also fragen wir einfach nach dem, was am interessantesten erscheint?
}sagte Harrys Ravenclaw-Seite. \emph{Da bin ich dabei.}

\emph{Halt die Klappe}, sagte Harry zu der Stimme; und beschloss dann, bei
näherem Nachdenken, dass er nicht länger so tun würde, als wäre sie da.

Vier Themen kamen Harry in den Sinn, die unter dem Gesichtspunkt der Neugier auf
wichtige Dinge Prioritäten waren. Vier Fragen also, vier wichtige Themen, die er
versuchen sollte, unterzubringen, während dieser Trank noch gebraut wurde. Vier
Fragen...

\glqq Ich stelle die erste Frage\grqq{}, sagte Harry. \glqq Was geschah wirklich
in der Nacht des 31. Oktober 1981?\grqq{}

\emph{Warum war diese Nacht anders als alle anderen Nächte... }

\glqq Ich hätte gerne die ganze Geschichte. Bitte.\grqq{}

Die Frage, wie und warum Lord Voldemort seinen Scheintod überlebt hatte, schien
für die weitere Planung von Bedeutung zu sein.

\glqq Ich habe erwartet, dass du das fragen würdest\grqq{}, sagte Professor
Quirrell und ließ eine Glockenblume und einen weißen Glitzerstein in den Trank
fallen. \glqq Zunächst einmal ist alles, was ich dir über den Horkrux-Zauber
erzählt habe, wahr; das solltest du wissen, da ich in Parsel gesprochen
habe.\grqq{}

Harry nickte.

\glqq Innerhalb von Sekunden, nachdem du die Details des Zaubers erfahren hast,
hast du den zentralen Fehler erkannt und angefangen, darüber nachzudenken, wie
der Zauber verbessert werden könnte. Glaubst du, der junge Tom Riddle war
anders?\grqq{}

Harry schüttelte den Kopf.

\glqq Nun, er war es\grqq{}, sagte Professor Quirrell. \glqq Wann immer ich
versucht war, an dir zu verzweifeln, habe ich mich daran erinnert, dass ich in
deinem Alter ein Idiot war. Als ich fünfzehn war, machte ich mir einen Horcrux,
wie es mir ein bestimmtes Buch gezeigt hatte, indem ich den Tod von Abigail
Myrtle unter den Augen des Basilisken von Slytherin ausnutzte. Ich plante, jedes
Jahr, nachdem ich Hogwarts verlassen hatte, einen neuen Horkrux zu machen und
das als meinen Notfallplan zu bezeichnen, falls meine anderen Hoffnungen auf
Unsterblichkeit nicht in Erfüllung gehen würden. Im Nachhinein betrachtet, griff
der junge Tom Riddle nach Strohhalmen. Der Gedanke, einen besseren Horkrux zu
machen, sich nicht mit dem Zauberspruch zu begnügen, den ich bereits gelernt
hatte ... dieser Gedanke kam mir erst, als ich die Dummheit der normalen
Menschen begriffen hatte und mir klar wurde, welche ihrer Torheiten ich
nachgeahmt hatte. Aber mit der Zeit lernte ich die Angewohnheit die du von mir
geerbt hast: in jedem Fall zu fragen, wie es besser gemacht werden könnte. Sich
mit dem Zauberspruch aus einem Buch zu begnügen, wenn er nur eine schwache
Ähnlichkeit mit dem hatte, was ich wirklich wollte? Absurd! Und so machte ich
mich daran, einen besseren Zauberspruch zu erschaffen.\grqq{}

\glqq Du hast jetzt wahre Unsterblichkeit?\grqq{} Harry war sich bewusst, dass
dies eine Frage war, die wichtiger war als Krieg und Strategie, selbst wenn es
um alles andere ging.

\glqq In der Tat\grqq{}, sagte Professor Quirrell. Er hielt in seiner Arbeit an
den Zaubertränken inne und drehte sich zu Harry um; in den Augen des Mannes lag
ein Ausdruck der Freude, den Harry dort noch nie gesehen hatte. \glqq In allen
Dunklen Künsten, die ich finden konnte, in allen verbotenen Geheimnissen, zu
denen mir das Ungeheuer von Slytherin Schlüssel gab, in allen Überlieferungen,
an die man sich unter Zauberern erinnert, fand ich nur Andeutungen und
Bruchstücke von dem, was ich brauchte. Also formte ich es neu und entwarf ein
neues Ritual, das auf neuen Prinzipien basierte. Dieses Ritual brannte jahrelang
in meinem Kopf, ich perfektionierte es in meiner Fantasie, dachte über seine
Bedeutung nach und nahm feine Anpassungen vor, während ich darauf wartete, dass
sich die Absicht stabilisierte. Endlich wagte ich es, mein Ritual zu beschwören,
ein erfundenes Opferritual, das auf einem von aller bekannten Magie unerprobten
Prinzip beruht. Und ich lebte und lebe immer noch.\grqq{} Der
Verteidigungsprofessor sprach mit leisem Triumph, als sei die Tat selbst so
groß, dass keine Worte ihr je gerecht werden könnten. \glqq Ich benutze das Wort
'Horkrux' immer noch, aber nur aus Sentimentalität. Es ist ein völlig neues
Ding, die größte aller meiner Schöpfungen.\grqq{}

\glqq Als eine meiner Fragen, die du zu beantworten versprochen hast, frage ich,
wie man diesen Zauberspruch wirkt\grqq{}, sagte Harry.

\glqq Abgelehnt.\grqq{} Der Verteidigungsprofessor wandte sich wieder seinem
Zaubertrank zu und warf eine graumelierte weiße Feder und eine Glockenblume
hinein. \glqq Ich hatte gedacht dich zu unterrichten wenn du älter bist, denn
kein Tom Riddle wäre sonst zufrieden; aber ich habe es mir anders
überlegt.\grqq{}

Erinnerungen sind manchmal schwer abzurufen, und Harry hatte versucht, sich zu
erinnern, ob Professor Quirrell schon einmal Andeutungen zu diesem Thema gemacht
hatte. Irgendetwas an Professor Quirrells Formulierung löste eine Erinnerung
aus:

\emph{Vielleicht wirst du es erfahren, wenn du älter bist... }

\glqq Es gibt noch physische Verankerungen für deine Unsterblichkeit\grqq{},
sagte Harry laut. \glqq Es ähnelt dem alten Horkrux-Zauber sehr, was ein
weiterer Grund ist, warum du sie immer noch Horkruxe nennst.\grqq{}

Es war gefährlich, das laut zu sagen, aber Harry musste es wissen.

\glqq Wenn ich mich irre, kannst du es jederzeit in Parsel leugnen."

Professor Quirrell lächelte böse. \glqq Deine Vermutung ist richtig, Junge, so
viel es dir auch nützen mag.\grqq{}

\emph{Leider war das keine schwer zu deckende Schwachstelle, wenn der Feind schlau war.}

Normalerweise hätte Harry den Vorschlag nicht gemacht, nur für den Fall, dass
der Feind nicht selbst darauf gekommen wäre, aber in diesem Fall hatte er ihn
bereits gemacht.

\glqq Ein Horcux in einen aktiven Vulkan geworfen, so schwer, dass er in der
Erdkruste versinkt\grqq{}, sagte Harry schwerfällig. \glqq Derselbe Ort, an dem
ich den Dementor fallen lassen wollte, wenn ich ihn nicht zerstören konnte. Und
dann hast du mich gefragt, wo ich sonst etwas verstecken würde, wenn ich nicht
wollte, dass es jemals wieder jemand findet. Ein Horkrux, kilometerweit unten
vergraben, in einem anonymen Kubikmeter der Erde. Ein Horkrux, den du in den
Marianengraben geworfen hast. Ein Horkrux, der hoch oben in der Stratosphäre
schwebt, unsichtbar. Selbst du weißt nicht, wo sie sind, weil du die genauen
Details aus deinem Gedächtnis gelöscht hast. Und der letzte Horkrux ist die
Pioneer-11-Schallplatte, die du in die NASA geschmuggelt und modifiziert hast.
Von ihr hast du dein Bild von den Sternen, wenn du den Sternenlicht-Zauber
sprichst. Feuer, Erde, Wasser, Luft, Leere.\grqq{}

\emph{So etwas wie ein Rätsel („Something like a Riddle“, anm. des
Übersetzers)}, hatte der Verteidigungsprofessor es genannt, und deshalb hatte
Harry es sich gemerkt.
\emph{Etwas wie ein Rätsel.}

\glqq In der Tat\grqq{}, sagte der Verteidigungsprofessor. \glqq Ich war etwas
schockiert, als du dich so schnell daran erinnert hast, aber ich nehme an, es
macht keinen Unterschied; alle fünf sind außerhalb meiner oder deiner
Reichweite.\grqq{}

\emph{Das mochte nicht stimmen, vor allem, wenn es einen Weg gab, die magische Verbindung irgendwie zurückzuverfolgen und den Standort zu bestimmen ... obwohl Voldemort vermutlich sein Bestes getan hätte, um sie zu verschleiern ... aber was die Magie angerichtet hatte, konnte die Magie vielleicht besiegen. Pioneer 11 mochte nach Zauberer-Maßstäben weit weg sein, aber die NASA wusste genau, wo sie sich befand, und sie war wahrscheinlich viel besser zu erreichen, wenn man die Ziolkowsky-Raketengleichung mit Hilfe von Magie dazu bringen konnte, sich zu verpissen... }

Ein plötzlicher Anflug von Sorge machte sich in Harrys Kopf breit. Es gab keine
Regel, die besagte, dass der Verteidigungsprofessor die Wahrheit darüber gesagt
haben musste, welche interstellare Sonde er mit einem Horcrux belegt hatte, und
wenn Harry sich richtig erinnerte, waren Kommunikation und Ortung der
Pioneer-10-Sonde kurz nach dem Vorbeiflug am Jupiter verloren gegangen.

\emph{Warum hätte Professor Quirrell nicht einfach beide horcruxen können? }

Der naheliegende nächste Gedanke kam Harry. Es war etwas, das nicht angedeutet
werden sollte, wenn der Feind nicht daran gedacht hatte. Aber es schien äußerst
wahrscheinlich, dass der Feind daran gedacht hatte.

\glqq \emph{Sag, Herr Lehrer\grqq{} }, zischte Harry, \glqq \emph{würde es dich
umbringen, diese fünf Anker zu zerstören?}\grqq{}

\glqq \textbf{\emph{Warum fragsssst du?}}\grqq{}, zischte der
Verteidigungsprofessor, mit einem Tonfall, den die Parsel Sprache als
schlangenhaftes Amüsement übersetzte. \glqq \textbf{\emph{Vermutest du, dass die
Antwort 'nein' lautet?}}\grqq{}

Harry wusste nicht, wie er antworten sollte, obwohl er stark vermutete, dass es
ohnehin keine Rolle spielte.

\glqq \textbf{\emph{Dein Verdacht ist richtig, Junge! Diese fünf zu vernichten,
würde mich nicht sterblich machen.}}\grqq{}

Harrys Kehle fühlte sich wieder ein wenig trocken an.
\emph{Wenn der Zauber keine katastrophalen Kosten mit sich brachte ...}

\glqq Wie viele Anker hast du gemacht?"

\glqq \textbf{\emph{Würde ich normalerweise nicht sagen, aber esss issst klar,
dass du es schon erraten hast.}}\grqq{}

Das Lächeln des Verteidigungsprofessors wurde breiter. \glqq
\textbf{\emph{Antwort issst, dass ich es nicht weiß. Ss-hörte auf zu zählen,
ssirgendwo bei hundertundsieben. Ich habe es mir einfach zur Gewohnheit gemacht,
jedes Mal, wenn ich jemanden unter vier Augen ermordet habe.}}\grqq{}

\emph{Über 100 Morde unter vier Augen, bevor Lord Voldemort aufgehört hatte zu zählen. Und noch schlimmer: }

\glqq Dein Unsterblichkeitszauber erfordert immer noch den Tod eines Menschen?
Warum?\grqq{}

\glqq \textbf{\emph{Große Schöpfung erhält Leben und Magie in Geräten, die durch
das Opfern von Leben und Magie anderer geschaffen wurden.}}\grqq{} Wieder dieses
zischende Schlangenlachen. \glqq \textbf{\emph{Ich mochte die falsche
Beschreibung des vorigen Horkrux-Spruchs so sehr, war so enttäuscht, als ich die
Wahrheit darüber erkannte, dass die Gedanken an eine verbesserte Version in
dieser Form auftauchte.}}\grqq{}

Harry war sich nicht sicher, warum der Verteidigungsprofessor ihm all diese
lebenswichtigen Informationen gab, aber es musste einen Grund geben, und das
machte ihn nervös.

\glqq Du bist also wirklich ein körperloser Geist, der von Quirinus Quirrell
Besitz ergreift.\grqq{}

\glqq \textbf{\emph{Jawohl. Ich werde schnell zurückkehren, wenn dieser Körper
getötet wird. Ich werde sehr verärgert sein, und rachsüchtig. Ich sage dir das,
Junge, damit du keine Dummheiten machst.}}\grqq{}

\glqq Ich verstehe\grqq{}, sagte Harry.

Er tat sein Bestes, um seine Gedanken zu ordnen, sich zu erinnern, was er als
Nächstes fragen wollte, während der Verteidigungsprofessor seinen Blick wieder
auf den Trank richtete. Die linke Hand des Mannes träufelte eine zerstoßene
Muschel in den Kessel, während seine rechte Hand eine weitere Glockenblume
hineinfallen ließ.

\glqq Was ist also am 31. Oktober passiert? Du... hast versucht, den kleinen
Harry Potter in einen Horcrux zu verwandeln, entweder die neue oder die alte
Art. Du hast es absichtlich getan, weil du es Lily Potter gesagt hast.\grqq{}
Harry holte tief Luft. Jetzt, da er wusste warum die Schauer da waren, konnte er
sie ertragen.

\glqq \emph{Nun gut, ich akzeptiere die Abmachung. Du sollst sterben und das
Kind soll leben. Jetzt lass deinen Zauberstab fallen, damit ich dich ermorden
kann.}\grqq{}

Im Nachhinein war es klar, dass Harry sich an dieses Ereignis hauptsächlich aus
der Perspektive von Lord Voldemort erinnert hatte, und erst ganz zum Schluss
hatte er es durch die Augen des kleinen Harry Potter gesehen.

\glqq Was hast du getan? Warum hast du es getan?\grqq{}

\glqq Trelawneys Prophezeiung\grqq{}, sagte Professor Quirrell. Seine Hand
klopfte mit einem Kupferstreifen auf eine Glockenblume, bevor er sie einwarf.
\glqq Ich habe lange darüber nachgedacht, nachdem Snape die Prophezeiung zu mir
gebracht hat. Prophezeiungen sind nie etwas Triviales. Und wie soll ich das so
ausdrücken, dass du nicht auf dumme Gedanken kommst ... nun, ich werde es sagen,
und wenn du dumm bist, werde ich verärgert sein. Ich war fasziniert von der
Behauptung der Prophezeiung, dass jemand mir ebenbürtig sein würde, weil es
bedeuten könnte, dass diese Person am anderen Ende einer intelligenten
Unterhaltung stehen könnte. Nach fünfzig Jahren, in denen ich von schwafelnder
Dummheit umgeben war, kümmerte es mich nicht mehr, ob meine Reaktion als
literarisches Klischee angesehen werden könnte. Ich wollte mir diese Gelegenheit
nicht entgehen lassen, ohne vorher darüber nachzudenken. Und dann, siehst du,
hatte ich eine clevere Idee.\grqq{} Professor Quirrell seufzte. \glqq Mir ist
eingefallen, wie ich die Prophezeiung auf meine eigene Weise erfüllen könnte, zu
meinem eigenen Vorteil. Ich würde das Baby als mir ebenbürtig kennzeichnen,
indem ich den alten Horkrux-Zauber so wirke, dass mein eigener Geist auf das
leere Gehirn des Babys geprägt wird; es wäre eine reinere Kopie von mir selbst,
da es kein altes Ich gäbe, das sich mit dem neuen vermischen würde. In einigen
Jahren, wenn mir die Herrschaft über Britannien langweilig geworden wäre und ich
mich anderen Dingen zuwandte, würde ich mit dem anderen Tom Riddle vereinbaren,
dass er erscheinen sollte, um mich zu besiegen, und er würde über das Britannien
herrschen, das er gerettet hatte. Wir würden das Spiel für immer gegeneinander
spielen und unser Leben inmitten einer Welt voller Narren interessant halten.
Ich wusste, dass ein Dramatiker voraussagen würde, dass wir beide damit enden
würden, dass wir uns gegenseitig zerstören; aber ich dachte lange darüber nach
und entschied, dass wir beide es einfach ablehnen würden, das Drama
auszuspielen. Das war meine Entscheidung, und ich war zuversichtlich, dass es
dabei bleiben würde; beide Tom Riddles, dachte ich, wären zu intelligent, um
diesen Weg wirklich zu gehen. Die Prophezeiung schien anzudeuten, dass, wenn ich
alles bis auf einen Rest von Harry Potter zerstöre, unsere Geister nicht so
verschieden sein würden und wir in derselben Welt existieren könnten."

Etwas ging schief\grqq{}, sagte Harry. \glqq Etwas, das das Dach des Hauses der
Potters in Godric's Hollow weggesprengt, mir die Narbe auf der Stirn beschert
und deinen verbrannten Körper zurückgelassen hat."

Professor Quirrell nickte. Seine Hände waren bei der Arbeit an den Zaubertränken
langsamer geworden. \glqq Die Resonanz in unserer Magie\grqq{}, sagte Professor
Quirrell leise. \glqq Als ich den Geist des Babys so geformt hatte, dass er
meinem eigenen glich …"

Harry erinnerte sich an den Moment in Askaban, als Professor Quirrells
Tötungsfluch mit seinem Patronus zusammengestoßen war. Die brennende, reißende
Qual in seiner Stirn, als ob sein Kopf kurz davor gewesen wäre, sich in zwei
Hälften zu teilen.

\glqq Ich kann gar nicht zählen, wie oft ich an diese Nacht gedacht habe, meinen
Fehler im Geiste wiederholt und mir überlegt habe, was ich besser hätte tun
sollen\grqq{}, sagte Professor Quirrell. \glqq Später habe ich entschieden, dass
ich meinen Zauberstab aus der Hand hätte werfen und in meine Animagus-Form
wechseln sollte. Aber in dieser Nacht... in dieser Nacht habe ich instinktiv
versucht, die chaotischen Fluktuationen in der Magie zu kontrollieren, selbst
als ich spürte, wie ich von innen heraus verbrannte. Das war die falsche
Entscheidung, und ich habe versagt. So wurde mein Körper zerstört, während ich
den Geist des kleinen Harry Potter überschrieb; jeder von uns zerstörte den
anderen bis auf einen Rest. Und dann..." Professor Quirrells Gesichtsausdruck
war beherrscht. \glqq Und dann, als ich im Inneren meiner Horkruxe wieder zu
Bewusstsein kam, stellte sich heraus, dass meine großartige Schöpfung nicht so
funktionierte, wie ich gehofft hatte. Ich hätte in der Lage sein sollen, mich
aus meinen Horkruxen zu befreien und von jedem Opfer Besitz zu ergreifen, das
mir zustimmte oder das zu schwach war, um sich mir zu widersetzen. Das war der
Teil meiner großartigen Schöpfung, der meine Absicht verfehlte. Wie beim
ursprünglichen Horkrux-Zauber konnte ich nur in ein Opfer eindringen, das den
physischen Horkrux kontaktiert... und ich hatte meine unzähligen Horkruxe an
Orten versteckt, wo sie niemand jemals finden würde. \textbf{Dein Instinkt ist
richtig, Junge, das wäre kein guter Zeitpunkt zum Lachen!}"

Harry blieb ganz still.

Die Herstellung des Tranks war zu einer vorübergehenden Pause gekommen, einer
Pause, in der keine Zutaten hinzugefügt wurden, während der Kessel eine Zeit
lang köchelte.

\glqq Ich habe die meiste Zeit damit verbracht, mir die Sterne
anzuschauen\grqq{}, sagte Professor Quirrell, seine Stimme war jetzt leiser. Der
Verteidigungsprofessor hatte sich von dem Zaubertrank abgewandt und starrte auf
die weiß erleuchteten Wände des Raumes. \glqq Meine letzte Hoffnung waren die
Horcruxe, die ich in der hoffnungslosen Idiotie meiner Jugend versteckt hatte.
Ich hatte sie in uralte Artefakte eingearbeitet, statt in anonyme Kieselsteine;
ich hatte sie unter Giftbrunnen inmitten eines Sees von Inferi versteckt, statt
sie mit einem Portschlüssel ins Meer zu teleportieren. Wenn jemand so etwas
finden und ihren lächerlich großen Schutz durchdringen würde ... aber das schien
eine ferne Hoffnung zu sein. Ich war nicht sicher, ob ich jemals wieder
verkörpert sein würde. Doch wenigstens war ich unsterblich. Das schlimmste aller
Schicksale war abgewendet worden, so viel hatte meine großartige Schöpfung
bewirkt. Ich hatte nur noch wenig zu hoffen und wenig zu fürchten. Ich
beschloss, nicht wahnsinnig zu werden, da es keinen Vorteil zu bringen schien.
Stattdessen starrte ich zu den Sternen hinaus und dachte nach, während die Sonne
langsam hinter mir verschwand. Ich dachte über die Fehler meines vergangenen
Lebens nach; es waren viele, in diesem Rückblick. In meiner Vorstellungskraft
konstruierte ich mächtige neue Rituale, die ich versuchen könnte, wenn ich frei
wäre meine Magie noch einmal zu benutzen, und von meiner Unsterblichkeit
überzeugt bin. Ich grübelte länger als zuvor über alte Rätsel nach, ich hatte
mich einst für geduldig gehalten und lernte jetzt ganz neue Dimensionen der
Geduld. Ich wusste, dass ich im Falle meiner Freilassung weitaus mächtiger sein
würde als in meinem vorherigen Leben; aber damit rechnete ich nicht." Professor
Quirrell wandte sich wieder dem Trank zu. \glqq Neun Jahre und vier Monate nach
jener Nacht gelang es einem wandernden Abenteurer namens Quirinus Quirrell, den
Schutzwall zu überwinden, der einen meiner frühesten Horcruxe bewachte. Den Rest
kennst du. \textbf{\emph{Und jetzt, Junge, darfst du sagen, was wir beide
wissen, dass du denkst.}}"

\glqq Ähm\grqq{}, sagte Harry. \glqq Es scheint nicht sehr klug zu sein, das zu
sagen -"

\glqq In der Tat, Mr. Potter. Es ist nicht sehr klug, mir das zu sagen. Nicht
einmal ein bisschen. \textbf{Nicht im Geringsten!} Aber ich weiß, dass du es
denkst, und du wirst es weiter denken, und ich werde es weiter wissen, bis du es
sagst. \textbf{Also sprich.}"

\glqq Also. Ähm. Mir ist klar, dass dies etwas ist, das im Nachhinein
offensichtlicher ist als im Voraus, und ich schlage gewiss nicht vor, dass du
jetzt versuchst, den Fehler zu korrigieren, aber wenn ich ein Dunkler Lord wäre
und zufällig von einem Kind höre, dem prophezeit wurde, mich zu besiegen, nun,
es gibt es einen bestimmten Zauber, der unblockbar und unaufhaltsam ist und
jedes einzelne Mal bei allem funktioniert, was ein Gehirn hat -"

\glqq \textbf{Ja, danke, Mr. Potter, dieser Gedanke ist mir in den nächsten neun
Jahren mehrmals gekommen.}\grqq{} Professor Quirrell hob eine weitere
Glockenblume auf und begann sie in seiner bloßen Faust zu zerbröseln. \glqq Ich
habe dieses Prinzip zum Kernstück meines Lehrplans für Kampfmagie gemacht,
nachdem ich seine zentrale Bedeutung auf die harte Tour gelernt hatte. Es war
nicht die erste Regel auf der Liste des jüngeren Tom Riddle. Nur durch harte
Erfahrung lernen wir, welche Prinzipien Vorrang vor anderen Prinzipien haben;
als bloße Worte klingen sie alle gleich überzeugend. Im Nachhinein wäre es
besser gewesen, wenn ich Bellatrix an meiner Stelle zum Haus der Potters
geschickt hätte; aber ich hatte eine Regel, die mir sagte, dass ich in solchen
Angelegenheiten selbst gehen muss und nicht versuchen darf, einen vertrauten
Leutnant zu schicken. Ja, ich habe den Tötungsfluch in Erwägung gezogen, aber
ich habe mich gefragt, ob der Tötungsfluch, wenn er auf ein Kleinkind gewirkt
wird, irgendwie abprallen und mich treffen würde, wodurch die Prophezeiung
erfüllt würde. Woher sollte ich das wissen?"

\glqq Dann nimm eine Axt, es ist schwer, einen prophezeiungserfüllenden Zauber
von einer Axt abprallen zu lassen\grqq{}, sagte Harry und hielt dann den Mund.

\glqq Ich habe beschlossen, dass es der sicherste Weg ist, zu versuchen, die
Prophezeiung auf meine eigene Weise zu erfüllen\grqq{}, sagte Professor
Quirrell. \glqq Unnötig zu sagen, dass ich das nächste Mal, wenn ich eine
Prophezeiung höre, die mir nicht gefällt, sie an jeder möglichen Stelle
zerreißen werde, anstatt zu versuchen, mitzuspielen."

Professor Quirrell zerdrückte eine Rose, als wolle er den Saft aus ihr
herauspressen, wobei er immer noch seine bloße Faust benutzte.

\glqq Und jetzt denken alle, dass der Junge-der-lebte irgendwie immun gegen den
Tötungsfluch ist, obwohl Tötungsflüche keine Häuser ruinieren oder verbrannte
Leichen hinterlassen, weil es ihnen nicht in den Sinn gekommen ist, dass Lord
Voldemort jemals einen anderen Zauber benutzen würde."

Harry blieb wieder still. Es war Harry in den Sinn gekommen, dass es eine andere
offensichtliche Möglichkeit gab, wie Lord Voldemort seinen Fehler hätte
vermeiden können. Etwas, das mit einer Muggel-Erziehung vielleicht leichter zu
erkennen wäre als mit der Sichtweise eines Zauberers. Harry hatte sich noch
nicht entschieden, ob er Professor Quirrell von seinem Gedanken erzählen sollte;
es gab sowohl Vor- als auch Nachteile, auf diesen speziellen Fehler hinzuweisen.

Nach einiger Zeit nahm Professor Quirrell die nächste Zutat für Zaubertränke in
die Hand, eine Strähne von etwas, das wie Einhornhaar aussah.

\glqq Ich sage dir das als Warnung\grqq{}, sagte Professor Quirrell. \glqq
Erwarte nicht, dass ich noch einmal neun Jahre aufgehalten werde, wenn du meinen
Körper irgendwie zerstörst. Ich habe die Horkruxe gleich an bessere Orte
gesetzt, und jetzt ist selbst das unnötig. Dank dir habe ich erfahren, wo ich
den Stein der Auferstehung finden kann. Der Stein der Auferstehung bringt die
Toten natürlich nicht zurück, aber er birgt eine uralte Magie, um den Anschein
eines Geistes zu projizieren, die älter ist als meine eigene. Und da ich einer
bin, der den Tod besiegt hat, erkannte der Ring mich als seinen Meister an und
erhörte alle meine Wünsche. Ich habe es nun in meine große Schöpfung
aufgenommen." Professor Quirrell lächelte leicht. \glqq Ich hatte viele Jahre
zuvor erwogen, dieses Gerät zu einem Horkrux zu machen, mich damals aber dagegen
entschieden, da ich erkannte, dass der Ring Magie unbekannter Natur besitzt...
ach, solche Ironien spielt das Leben mit uns. Aber ich schweife ab. Du, Junge,
hast das verursacht. Du hast meinen Geist befreit, damit er fliegen kann, wohin
er will, und das günstigste Opfer verführen kann, indem du zu lax mit deinen
Geheimnissen umgehst. Es ist eine Katastrophe für jeden, der sich mir
widersetzt, und du hast sie mit einem Finger, der auf eine Teeuntertasse
gezeichnet wurde, herbeigeführt. Diese Welt wird ein sicherer Ort für alle sein,
wenn du die Umsicht lernst, die Zaubergeborene in der Wiege beigebracht
bekommen. \textbf{\emph{Und all }}\textbf{\emph{das, was ich gerade gesagt habe,
ist die Wahrheit.}}"

Harry schloss die Augen, und seine eigene Hand massierte seine Stirn; wenn er
sie von außen gesehen hätte, hätte es wie der Spiegel von Professor Quirrell in
tiefen Gedanken ausgesehen.

\emph{Das Problem, Professor Quirrell zu besiegen, sah immer schwieriger aus,
selbst nach den Maßstäben der Art von unmöglichen Problemen, die Harry bereits
gelöst hatte. Wenn es das Ziel von Professor Quirrell war, diese Schwierigkeit
zu kommunizieren, dann hatte er Erfolg. }

Harry fing an, ernsthaft die Möglichkeit in Betracht zu ziehen, dass es besser
wäre, sich anzubieten Britannien als Voldemorts nichtmörderischer Stellvertreter
zu regieren, wenn Professor Quirrell selbst nur zustimmen würde, damit
aufzuhören, ständig Menschen zu töten. Sogar meistens. Aber das würde
wahrscheinlich nicht passieren.

Harry starrte auf seine Hände, von wo aus er sich auf den Boden gesetzt hatte,
und fühlte Traurigkeit, die in Verzweiflung überging.

\emph{Der Lord Voldemort, der Harry seine dunkle Seite gegeben hatte, hatte so lange über Dinge nachgedacht und seine eigenen Gedankengänge reflektiert ... und war als der ruhige, klar denkende und immer noch Serienmörder Professor Quirrell herausgekommen.}

Professor Quirrell fügte dem Trank des Glanzes eine Prise goldenes Haar hinzu,
und das erinnerte Harry daran, dass die Zeit weiterlief; die Locken aus hellem
Haar waren seltener als die Glockenblumen.

\glqq Ich stelle meine zweite Frage\grqq{}, sagte Harry. \glqq Erzähl mir etwas
über den Stein der Weisen. Kann er noch etwas anderes, außer Verwandlung
dauerhaft zu machen? Ist es möglich, mehr Steine herzustellen, und warum ist das
so schwierig?"

Professor Quirrell war über den Trank gebeugt, und Harry konnte sein Gesicht
nicht sehen. \glqq Nun gut, ich werde dir die Geschichte des Steins so erzählen,
wie ich sie mir zurechtgelegt habe. Die einzige Macht des Steins besteht darin,
eine vorübergehende Form in eine wahre und dauerhafte Substanz zu verwandeln -
eine Macht, die absolut jenseits gewöhnlicher Magie liegt. Beschwörungen wie das
Schloss Hogwarts werden durch eine konstante Quelle der Magie aufrechterhalten.
Selbst Metamorphmagi können keine goldenen Fingernägel manifestieren und sie
dann zum Verkauf abschneiden. Es wird vermutet, dass der Metamorphmagus-Fluch
lediglich die Substanz des Fleisches umgestaltet, so wie ein Muggelschmied Eisen
mit Hammer und Zange bearbeitet; und der Körper enthält kein Gold. Wenn Merlin
selbst Gold aus dünner Luft erschaffen konnte, ist das in der Geschichte nicht
überliefert. Der Stein, so können wir schon vor der Forschung vermuten, muss
also ein sehr altes Ding sein. Im Gegensatz dazu ist Nicholas Flamel der Welt
erst seit sechs Jahrhunderten bekannt. Dann ist die naheliegende nächste Frage,
die man stellen muss, wenn man die Geschichte des Steins zurückverfolgen will
welche?"

\glqq Ähm\grqq{}, sagte Harry. Er rieb sich die Stirn und konzentrierte sich.
\emph{Wenn der Stein alt war, aber die Welt Nicholas Flamel erst seit sechs Jahrhunderten kannte...}
\glqq Gab es einen anderen sehr langlebigen Zauberer, der etwa zur gleichen Zeit
verschwand, als Nicholas Flamel auftauchte?"

\glqq Fast\grqq{}, sagte Professor Quirrell. \glqq Erinnerst du dich, dass es
vor sechs Jahrhunderten eine Dunkle Dame gab, die als unsterblich galt, die
Zauberin Baba Yaga? Es hieß, sie könne jede Wunde an sich selbst heilen, sich in
jede beliebige Form verwandeln... sie besaß offensichtlich den Stein der
Unsterblichkeit. Und dann, eines Jahres, erklärte sich Baba Yaga bereit, in
Hogwarts Kampfmagie zu unterrichten, im Rahmen eines alten und respektierten
Waffenstillstands."

Professor Quirrell sah... wütend aus, ein Blick, wie Harry ihn selten gesehen
hatte.

\glqq Aber man traute ihr nicht, und so wurde ein Fluch heraufbeschworen. Manche
Flüche sind leichter zu sprechen, wenn sie einen selbst und andere gleichermaßen
binden; Slytherins Parselmund-Fluch ist ein Beispiel dafür. In diesem Fall
wurden Baba Yagas Unterschrift und die Unterschriften aller Schüler und Lehrer
von Hogwarts in ein uraltes Gerät gelegt, das als Feuerkelch bekannt ist. Baba
Yaga schwor, keinen Tropfen des Blutes der Schüler zu vergießen und ihnen nichts
wegzunehmen, was ihnen gehörte. Im Gegenzug schworen die Schüler, keinen Tropfen
von Baba Yagas Blut zu vergießen und ihr nichts wegzunehmen, was ihr gehörte. So
unterschrieben sie alle, wobei der Feuerkelch Zeuge sein und den Übertreter
bestrafen sollte."

Professor Quirrell hob eine neue Zutat auf, einen losen Goldfaden, der um eine
Prise einer übel aussehenden Substanz gewickelt war.

\glqq In ihr sechstes Jahr in Hogwarts ging also eine Hexe namens Perenelle. Und
obwohl Perenelle gerade erst in die Schönheit ihrer Jugend hineingewachsen war,
war ihr Herz bereits schwärzer als das von Baba Yaga -"

\glqq \emph{Du} nennst sie böse?\grqq{} sagte Harry und merkte dann, dass er
gerade den Trugschluss des ad hominem tu quoque begangen hatte.

\glqq Sei still, Junge, ich erzähle die Geschichte. Wo war ich? Ah, ja,
Perenelle, die Schöne und Begehrliche. Perenelle verführte die Dunkle Lady über
Monate hinweg, mit sanften Berührungen und Flirten und dem schüchternen
Vortäuschen von Unschuld. Das Herz der Dunklen Dame war erobert, und sie wurden
Liebende. Und dann, eines Nachts, flüsterte Perenelle, wie sie von Baba Yagas
Macht, ihre Gestalt zu verändern, gehört hatte und wie dieser Gedanke ihre
Begierde entflammt hatte; so überredete Perenelle Baba Yaga, mit dem Stein in
der Hand zu ihr zu kommen und in einer einzigen Nacht viele Gestalten
anzunehmen, zu ihrem Vergnügen. Unter anderem befahl Perenelle Baba Yaga, die
Gestalt eines Mannes anzunehmen; und sie lagen zusammen in der Gestalt eines
Mannes und einer Frau. Aber Perenelle war bis zu dieser Nacht eine Jungfrau
gewesen. Und da sie in jenen Tagen alle ziemlich altmodisch waren, betrachtete
der Feuerkelch dies als das Vergießen von Perenelles Blut und die Einnahme
dessen, was ihr gehörte; so wurde Baba Yaga dazu verleitet, ihren Schwur zu
brechen, und der Kelch machte sie wehrlos. Dann tötete Perenelle die ahnungslose
Baba Yaga, als sie in Perenelles Bett schlief, tötete die Dunkle Dame, die sie
geliebt hatte und friedlich nach Hogwarts gekommen war, im Rahmen eines
Waffenstillstands; und das war das Ende des Paktes, durch den Dunkle Zauberer
und Hexen in Hogwarts Kampfmagie lehrten. In den nächsten Jahrhunderten wurde
der Feuerkelch dazu benutzt, sinnlose Turniere zwischen den Schulen zu
beaufsichtigen, und dann lag er in einer stillgelegten Kammer in Beauxbatons,
bis ich ihn schließlich stahl."

Professor Quirrell ließ einen blass-beige-rosa Zweig in den Kessel fallen, und
seine Farbe änderte sich zu Weiß, sobald er die Oberfläche berührte.

\glqq Aber ich schweife ab. Perenelle nahm den Stein von Baba Yaga und nahm die
Gestalt und den Namen von Nicholas Flamel an. Sie behielt auch ihre Identität
als Perenelle und nannte sich Flamels Frau. Die beiden sind zusammen in der
Öffentlichkeit aufgetreten, aber das könnte auf verschiedene Arten geschehen
sein."

\glqq Und die Herstellung des Steins?\grqq{}, fragte Harry, während sein Gehirn
daran arbeitete, all das zu verarbeiten. \glqq Ich habe ein alchemistisches
Rezept dafür gesehen, in einem Buch -"

\glqq Eine weitere Lüge. Perenelle ließ es so aussehen, als ob 'Nicholas Flamel'
sich das Recht verdient hätte, ewig zu leben, indem er einen großen Zauber
vollbrachte, den jeder versuchen konnte. Und sie gab anderen einen falschen Weg
vor, anstatt den einen wahren Stein zu suchen, wie Perenelle den von Baba Yaga
gesucht hatte."

Professor Quirrell sah ziemlich säuerlich aus.

\glqq Es sollte dich nicht überraschen, dass ich Jahre damit verschwendet habe,
dieses falsche Rezept zu meistern. Als nächstes wirst du fragen, warum ich
Perenelle nicht entführt, gefoltert und getötet habe, nachdem ich die Wahrheit
erfahren hatte."

Das war in der Tat keine Frage, die Harry in den Sinn gekommen war.

Professor Quirrell sprach weiter. \glqq Die Antwort ist, dass Perenelle die
Ambitionen von dunklen Zauberern wie mir vorausgesehen und ihnen zuvorgekommen
ist. Nicholas Flamel' legte öffentlich den unbrechbaren Schwur ab, sich auf
keinen Fall dazu zwingen zu lassen, seinen Stein aufzugeben - \emph{um die
Unsterblichkeit vor den Begehrlichen zu schützen,} behauptete er, als wäre das
ein öffentlicher Dienst. Ich befürchtete, dass der Stein für immer verloren
wäre, wenn Perenelle sterben würde, ohne zu sagen, wo er versteckt war, und ihr
Gelübde verhinderte Versuche der Folter. Außerdem hatte ich die Hoffnung,
Perenelles Wissen zu erlangen, wenn ich die richtige Strategie finden könnte, um
es ihr zu entlocken. Obwohl Perenelle anfangs nur wenig eigenes Wissen besaß,
hat sie das Leben von Zauberern, die größer waren als sie selbst, als Geisel
genommen, indem sie im Tausch gegen Geheimnisse ein paar Tropfen Heilung und im
Tausch gegen Macht eine kleine Umkehrung des Alters anbot. Perenelle lässt sich
nicht herab, anderen wirkliche Jugend zu schenken - aber wenn du von einem
Zauberer hörst, der mit grauem Bart zweihundertfünfzig Jahre alt wurde, kannst
du sicher sein, dass ihre Hand im Spiel war. In meiner Generation hatten die
Jahrhunderte Perenelle genug Vorteil verschafft, dass sie Albus Dumbledore als
Gegengewicht zum Dunklen Lord Grindelwald aufziehen konnte. Als ich als Lord
Voldemort auftauchte, ließ Perenelle Dumbledore noch weiter aufsteigen und
verteilte einen weiteren Tropfen ihres gehorteten Wissens, wann immer Lord
Voldemort einen Vorteil zu erlangen schien. Ich hatte das Gefühl, dass ich in
der Lage sein müsste, mir in dieser Situation etwas Gescheites einfallen zu
lassen, aber ich tat es nicht. Ich griff sie nicht direkt an, denn ich war mir
meiner großen Schöpfung nicht sicher; es war nicht ausgeschlossen, dass ich
eines Tages bei ihr um eine Portion umgekehrtes Alter betteln gehen musste."

Professor Quirrell ließ zwei Glockenblumen auf einmal in den Trank fallen, und
sie schienen zu verschmelzen, als sie die blubbernde Flüssigkeit berührten.

\glqq Aber jetzt bin ich mir meiner Schöpfung sicher, und so habe ich
beschlossen, dass die Zeit gekommen ist, den Stein mit Gewalt zu nehmen."

Harry zögerte. \glqq Ich würde gerne deine Antwort in Parsel hören, war das
alles wahr?"

\glqq \textbf{\emph{Nichts davon ist mir als falsch bekannt}}\grqq{}, sagte
Professor Quirrell. \glqq \textbf{\emph{Eine Geschichte zu erzählen bedeutet,
gewisse Lücken zu füllen; ich war nicht anwesend, um zu beobachten, als
Perenelle Baba Yaga verführte. Ich denke, die Grundlagen sollten weitgehend
korrekt sein.\grqq{} }}

Harry bemerkte eine Spur von Verwirrung.

\glqq Dann verstehe ich nicht, warum der Stein hier in Hogwarts ist. Wäre es
nicht die beste Verteidigung, ihn einfach unter einem anonymen Felsen in
Grönland zu verstecken?"

\glqq Vielleicht hat sie meine Fähigkeiten als besonders guter Finder
respektiert\grqq{}, sagte der Verteidigungsprofessor.

Er schien auf seinen Kessel konzentriert zu sein, während er eine Glockenblume
in ein Gefäß mit einer Flüssigkeit tauchte, die mit dem Zaubertränkesymbol für
Regenwasser beschriftet war.

\emph{Wir sind uns sehr ähnlich, der Verteidigungsprofessor und ich, in mancher Hinsicht, wenn nicht in anderen. Wenn ich mir vorstelle, was ich tun würde, angesichts seines Problems...}

\glqq Hast du allen vorgegaukelt, du könntest den Stein finden?\grqq{} sagte
Harry laut. \glqq Damit Perenelle ihn in Hogwarts unterbringen würde, wo
Dumbledore ihn bewachen könnte?"

Der Verteidigungsprofessor seufzte und blickte nicht vom Kessel auf. \glqq Ich
nehme an, dass es nicht möglich ist, diese Stratege müßig vor dir zu verstecken.
Ja, nachdem ich von Quirrell Besitz ergriffen hatte und zurückgekehrt war,
setzte ich eine Strategie um, die ich mir beim Betrachten der Sterne ausgedacht
hatte. Zuerst sorgte ich dafür, dass ich als Verteidigungsprofessor in Hogwarts
angenommen wurde, denn es wäre nicht gut, wenn Verdacht geschöpft würde, während
ich noch auf der Suche nach einer Anstellung war. Als das erledigt war, sorgte
ich dafür, dass bei einer von Perenelles Fluchbrecher-Expeditionen eine
gefälschte, aber glaubwürdige Inschrift entdeckt wurde, die beschrieb, wie
\emph{die Krone der Schlange} benutzt werden konnte, um den Stein aufzuspüren,
wo auch immer er versteckt war. Unmittelbar danach, bevor Perenelle die Krone
aufkaufen konnte, wurde sie gestohlen; außerdem hinterließ ich deutliche
Hinweise darauf, dass der Dieb die Macht besaß, mit Schlangen zu sprechen. So
dachte Perenelle, dass ich unfehlbar den Standort des Steins finden könnte und
dass er einen Wächter brauchte, der mächtig genug war, mich zu besiegen. So kam
es, dass der Stein in Hogwarts aufbewahrt wurde, in Dumbledores Reich. Genau wie
ich es beabsichtigt hatte, da ich ja schon für ein Jahr Zugang zu Hogwarts
hatte. Ich denke, das ist alles, was dich betrifft, und ich rede nicht von
Zukunftsplänen."

Harry runzelte die Stirn.

\emph{Professor Quirrell hätte ihm das nicht sagen dürfen. Es sei denn, die Strategie war irgendwie irrelevant für eine zukünftige Täuschung von Perenelle geworden...? Oder es sei denn, der Verteidigungsprofessor hatte mit seiner schnellen Antwort gehofft, die Leute zu dem Schluss kommen zu lassen, dass es sich um einen Doppelbluff handelte und dass die Schlangenkrone den Stein wirklich finden konnte... }

Harry beschloss, diese Antwort in Parsel nicht zu hinterfragen.

Eine weitere Locke hellen Haares, das weiß, aber nicht vom Alter gezeichnet zu
sein schien, wurde sanft in den Kessel geträufelt und erinnerte Harry erneut
daran, dass sie ein Zeitlimit hatten.

Harry überlegte, aber er sah keinen weiteren Weg, um diese Frage weiter zu
verfolgen; es gab keinen bekannten Weg, weitere Steine der Weisen herzustellen
und keinen offensichtlichen Weg, solche zu erfinden, was wahrscheinlich die
objektiv schlechteste Nachricht war, die Harry den ganzen Tag über gehört hatte.

Harry nahm einen tiefen Atemzug. \glqq Ich stelle meine dritte Frage\grqq{},
sagte Harry. \glqq Was ist die Wahrheit hinter diesem ganzen Schuljahr? All die
Verschwörungen, die du durchgeführt hast, all die Verschwörungen, von denen du
weißt."

\glqq Hm\grqq{}, sagte Professor Quirrell und ließ eine weitere Glockenblume in
den Trank fallen, begleitet von einer Pflanze, die die Form eines kleinen
Kreuzes hatte. \glqq Mal sehen ... die schockierendste Wendung ist, dass sich
herausstellt, dass der Verteidigungsprofessor insgeheim Lord Voldemort ist."

\glqq Tja, offensichtlich\grqq{}, sagte Harry mit einer gehörigen Portion
selbstgesteuerter Bitterkeit.

\glqq Wo soll ich dann anfangen?"

\glqq Warum hast du Hermine getötet?" Die Frage rutschte ihm einfach heraus.

Professor Quirrells bleiche Augen blickten von dem Trank auf, beobachteten ihn
aufmerksam. \glqq Man sollte meinen, das sollte offensichtlich sein - aber ich
kann dir wohl nicht verübeln, dass du dem misstraust, was offensichtlich
erscheint. Um das Ziel eines obskuren Komplotts zu verstehen, muss man seine
Folgen beobachten und sich fragen, wer sie beabsichtigt haben könnte. Ich habe
Miss Granger getötet, um deine Position gegenüber Lucius Malfoy zu verbessern,
denn meine Pläne sahen nicht vor, dass er so viel Einfluss auf dich hat. Ich
gebe zu, ich bin beeindruckt, wie weit du es geschafft hast, diese Öffnung zu
nutzen."

Harry biss die Zähne zusammen, was ihn einige Mühe kostete. \glqq Das ist nach
deinem gescheiterten Versuch, Hermine den Mordversuch an Draco anzuhängen und
sie nach Askaban zu schicken, warum? Weil dir der Einfluss, den sie auf mich
hatte, nicht gefiel?"

\glqq Mach dich nicht lächerlich\grqq{}, sagte Professor Quirrell. \glqq Wenn
ich nur Miss Granger hätte beseitigen wollen, hätte ich die Malfoys nicht ins
Spiel gebracht. Ich habe dein Spiel mit Draco Malfoy beobachtet und fand es
amüsant, aber ich wusste, dass es nicht lange weitergehen konnte, bevor Lucius
davon erfuhr und sich einmischte; und dann hätte dir deine Torheit großen Ärger
eingebracht, denn Lucius würde es nicht auf die leichte Schulter nehmen. Hättest
du nur während des Zaubergamot-Prozesses verlieren können, verlieren, wie ich es
dir beigebracht hatte, dann hätte es in nur zwei weiteren Wochen stichhaltige
Beweise gegeben, dass Lucius Malfoy, nachdem er die scheinbare Niedertracht
seines Sohnes entdeckt hatte, Professor Sprout dazu gebracht hatte, den
Blutkühlungszauber auf Mr. Malfoy anzuwenden und den Zauber der falschen
Erinnerung auf Miss Granger zu wirken. Lucius wäre vom politischen Spielbrett
gefegt worden, ins Exil, wenn nicht gar nach Askaban geschickt worden; Draco
Malfoy hätte den Reichtum des Hauses Malfoy geerbt, und dein Einfluss auf ihn
wäre unangefochten gewesen. Stattdessen musste ich diesen Plan mitten im Lauf
abbrechen. Du hast es geschafft, den eigentlichen Plan komplett zu stören, indem
du das Doppelte deines gesamten Vermögens geopfert hast, indem du Lucius Malfoy
die perfekte Gelegenheit gegeben hast, seine wahre Sorge um seinen Sohn zu
beweisen. Du hast ein unglaubliches Anti-Talent für Einmischungen, muss ich
sagen."

\glqq Und du dachtest auch\grqq{}, sagte Harry, der sich trotz der Muster seiner
dunklen Seite anstrengen musste, um seine Stimme ruhig und kühl zu halten, \glqq
dass zwei Wochen in Askaban Miss Grangers Gemüt verbessern und sie dazu bringen
würden, keinen schlechten Einfluss mehr auf mich zu haben. Also hast du
irgendwie dafür gesorgt, dass es Zeitungsberichte gibt, die fordern, dass sie
nach Askaban geschickt wird, anstatt einer anderen Strafe."

Professor Quirrells Lippen zogen sich zu einem dünnen Lächeln zusammen. \glqq
Gut überlegt, Junge. Ja, ich dachte, sie könnte als deine Bellatrix dienen. Ein
solches Ergebnis hätte dich auch ständig daran erinnert, wie viel Respekt dem
Gesetz gebührt, und dir geholfen, eine angemessene Einstellung gegenüber dem
Ministerium zu entwickeln."

\glqq Dein Plan war dumm und kompliziert und hatte keine Chance, zu
funktionieren." Harry wusste, dass er taktvoller sein sollte, dass er sich auf
mehr von dem einließ, was Professor Quirrell Torheit nannte, aber in diesem
Moment konnte er sich nicht dazu bringen, sich darum zu kümmern.

\glqq Es war weniger kompliziert als Dumbledores Komplott, die drei Armeen in
der Weihnachtsschlacht zu vereinen, und nicht viel komplizierter als mein
eigenes Komplott, dich glauben zu lassen, Dumbledore hätte Mr. Zabini erpresst.
Die Einsicht, die dir fehlt, Mr. Potter, ist, dass dies keine Intrigen waren,
die gelingen \emph{mussten}."

Professor Quirrell rührte weiter beiläufig den Trank und lächelte.

\glqq Es gibt Intrigen, die gelingen müssen, bei denen man die Kernidee so
einfach wie möglich hält und alle Vorsichtsmaßnahmen trifft. Es gibt auch
Intrigen, bei denen es akzeptabel ist, zu scheitern, und bei denen man sich
etwas Spielraum gönnen oder die Grenzen seiner Fähigkeit, mit Komplikationen
umzugehen, austesten kann. Es war ja nicht so, dass mich ein Fehlschlag bei
einem dieser Intrigen umgebracht hätte."

Professor Quirrell lächelte nicht mehr.

\glqq Unsere Reise nach Askaban war von der ersten Art, und ich war weniger
amüsiert über deine Possen dort."

\glqq Was genau hast du mit Hermine gemacht?" Ein Teil von Harry wunderte sich
über die Gleichmäßigkeit in seiner Stimme.

\glqq Vergiftungen und falsche Erinnerungszauber. Ich konnte mich nicht darauf
verlassen, dass irgendetwas anderes von den Hogwarts-Wachzaubern und der
Untersuchung, von der ich wusste, dass dein Geist sie durchschauen würde,
unentdeckt bleiben würde." Ein Aufflackern von Frustration ging über Professor
Quirrells Gesicht. \glqq Ein Teil dessen, was du zu Recht als Komplikation
bezeichnest, liegt darin, dass die erste Version meines Plans nicht so verlief
wie geplant, und ich musste ihn abändern. Ich kam zu Miss Granger in den Gängen
und gab mich als Professor Sprout aus, um ihr eine Verschwörung anzubieten. Mein
erster Überredungsversuch schlug fehl. Ich versuchte es erneut mit einer neuen
Präsentation. Der zweite Köder schlug fehl. Der dritte Köder schlug fehl. Der
zehnte Köder schlug fehl. Ich war so frustriert, dass ich anfing, meine gesamte
Bibliothek an Verkleidungen durchzugehen, einschließlich derer, die eher zu Mr.
Zabini passen. Trotzdem funktionierte nichts. Das Kind wollte ihren kindlichen
Kodex nicht verletzen."

\glqq Du hast nicht das Recht, sie kindisch zu nennen, Professor." Harrys Stimme
klang seltsam in seinen eigenen Ohren. \glqq Ihr Kodex hat funktioniert. Er hat
dich daran gehindert, sie auszutricksen. Der ganze Sinn von deontologischen
ethischen Anordnungen ist, dass Argumente für ihre Verletzung oft viel weniger
vertrauenswürdig sind, als sie aussehen. Du hast kein Recht, ihre Regeln zu
kritisieren, wenn sie genau wie beabsichtigt funktioniert haben."

\emph{Nachdem sie Hermine wiederbelebt hatten, würde Harry ihr erzählen, dass Lord Voldemort selbst nicht in der Lage gewesen war, sie in Versuchung zu führen, etwas Falsches zu tun, und dass er sie deshalb getötet hatte.}

\glqq Das ist wohl richtig\grqq{}, sagte Professor Quirrell. \glqq Es gibt ein
Sprichwort, das besagt, dass selbst eine stehen gebliebene Uhr zweimal am Tag
recht hat, und ich glaube nicht, dass Miss Granger wirklich vernünftig war.
Dennoch, Regel Zehn: Man darf nicht über die Unwürdigkeit der Gegner schimpfen,
nachdem sie einen Plan vereitelt haben. Wie dem auch sei. Nach zwei vollen
Stunden gescheiterter Versuche wurde mir klar, dass ich zu stur war und dass ich
Miss Granger nicht brauchte, um genau den Teil auszuführen, den ich für sie
geplant hatte. Ich gab meine ursprüngliche Absicht auf und vermittelte Miss
Granger stattdessen falsche Erinnerungen daran, wie sie Mr. Malfoy dabei
beobachtete, wie er ein Komplott gegen sie schmiedete, und zwar unter Umständen,
die implizierten, dass sie weder dir noch den Behörden davon erzählen sollte. Am
Ende war es Mr. Malfoy, der mir den Einstieg verschafft hat, den ich brauchte,
ganz durch Zufall."

Professor Quirrell ließ eine Glockenblume und ein Stück Pergament in den Kessel
fallen.

\glqq Warum zeigen die Wachzauber an, dass der Verteidigungsprofessor Hermine
getötet hat?"

\glqq Ich trug den Bergtroll als falschen Zahn, als Dumbledore mich gegenüber
den Hogwarts- Wachzaubern als den Verteidigungsprofessor identifizierte." Ein
leichtes Lächeln. \glqq Andere lebende Waffen können nicht verwandelt werden;
sie würden die Entzauberung nicht die erforderlichen sechs Stunden überleben, um
nicht von Zeitumkehrern aufgespürt zu werden. Die Tatsache, dass ein Bergtroll
als Mordwaffe benutzt wurde, war ein klares Zeichen dafür, dass der Attentäter
eine Ersatzwaffe brauchte, die sicher verwandelt werden konnte. Kombiniert mit
den Beweisen der Wachzauber und Dumbledores eigenem Wissen, wie er mich in
Hogwarts identifiziert hatte, hätte man daraus schließen können, wer dafür
verantwortlich war - theoretisch. Allerdings hat mich die Erfahrung gelehrt,
dass solche Rätsel viel schwieriger zu lösen sind, wenn man die Lösung nicht
schon kennt, und ich hielt es für ein kleines Risiko. Ah, das erinnert mich
daran, dass ich selbst eine Frage habe."

Der Verteidigungsprofessor warf Harry nun einen prüfenden Blick zu. \glqq Was
hat mich zuletzt verraten, im Korridor vor diesen Räumen?"

Harry schob andere Emotionen beiseite, um Kosten und Nutzen einer ehrlichen
Antwort abzuwägen, und kam zu dem Schluss, dass der Verteidigungsprofessor viel
mehr Informationen preisgab, als er bekam (\emph{warum}?), und dass es das Beste
war, nicht den Anschein von Zurückhaltung zu erwecken.

\glqq Das Wichtigste\grqq{}, sagte Harry, \glqq war, dass es zu unwahrscheinlich
war, dass alle zur gleichen Zeit in Dumbledores Korridor angekommen waren. Ich
habe versucht, mit der Hypothese zu arbeiten, dass jeder, der ankam, koordiniert
sein musste, auch du."

\glqq Aber ich hatte gesagt, dass ich Snape folge\grqq{}, sagte der
Verteidigungsprofessor. \glqq War das nicht plausibel?"

\glqq Das war es, aber ...\grqq{} sagte Harry. \glqq Ähm. Die Gesetze, die
bestimmen, was eine gute Erklärung ausmacht, sprechen nicht von plausiblen
Ausreden, die man hinterher hört. Sie sprechen von den Wahrscheinlichkeiten, die
wir im Voraus festlegen. Das ist der Grund, warum die Wissenschaft die Leute
dazu bringt, Vorhersagen zu machen, anstatt den Erklärungen zu vertrauen, die
sich die Leute hinterher ausdenken. Und ich hätte nicht vorhergesagt, dass du
Snape folgst und so auftauchen würdest. Selbst wenn ich im Voraus gewusst hätte,
dass du eine verfolgbare Spur an Snapes Zauberstab anbringen kannst, hätte ich
nicht erwartet, dass du es tust und ihm genau dann folgst. Da deine Erklärung
mir nicht das Gefühl gab, dass ich das Ergebnis im Voraus vorausgesehen hätte,
blieb es eine Unwahrscheinlichkeit. Ich begann mich zu fragen, ob Sprouts
Drahtzieher vielleicht auch dein Auftauchen arrangiert hat. Und dann wurde mir
klar, dass die Notiz an mich selbst nicht wirklich von meinem Zukunfts-Ich kam,
und das verriet es völlig."

\glqq Ah\grqq{}, sagte der Verteidigungsprofessor und seufzte. \glqq Nun, ich
denke, es hat sich alles zum Besten gewendet. Du hast es erst zu spät begriffen;
und es hätte sowohl Unannehmlichkeiten als auch Vorteile gehabt, wenn du
unwissend geblieben wärst."

\glqq Was um alles in der Welt wolltest du denn machen? Der Grund, warum ich so
sehr versucht habe, es herauszufinden, war, dass die ganze Sache einfach so
seltsam war."

\glqq Das hätte auf Dumbledore zeigen sollen, nicht auf mich\grqq{}, sagte
Professor Quirrell und runzelte die Stirn. \glqq Tatsache ist, dass Miss
Greengrass erst in einigen Stunden in diesem Korridor eintreffen sollte ...
obwohl ich annehme, dass es nicht allzu überraschend ist, dass sie sich
zusammengetan haben, da Mr. Malfoy ihr den Hinweis gegeben hat, den ich ihm
gegeben habe. Wäre Mr. Nott allein gekommen, hätte sich das Ganze weniger
farcenhaft abgespielt. Aber ich halte mich für einen Spezialisten in der
Steuerung von Kämpfen und konnte dafür sorgen, dass der Kampf so verlief, wie
ich es wollte. Ich nehme an, dass es am Ende etwas lächerlich aussah."

Der Verteidigungsprofessor ließ eine Pfirsichscheibe und eine Glockenblume in
den Kessel fallen.

\glqq Aber lass uns die Diskussion über den Spiegel verschieben, bis wir ihn
erreicht haben. Hatten Sie noch Fragen zu Miss Grangers bedauerlichem und
hoffentlich vorübergehendem Ableben?"

\glqq Ja\grqq{}, sagte Harry mit gleichmäßiger Stimme. \glqq Was hast du mit den
Weasley-Zwillingen gemacht? Dumbledore dachte - ich meine, die Schule sah, wie
der Schulleiter zu den Weasley-Zwillingen ging, nachdem Hermine verhaftet worden
war. Dumbledore dachte, dass du als Voldemort die Weasley-Zwillinge aufgesucht,
ihre Karte gefunden und an sich genommen hast und ihr Erinnerung gelöscht hast?"

\glqq Dumbledore hatte völlig recht\grqq{}, sagte Professor Quirrell und
schüttelte wie verwundert den Kopf. \glqq Er war auch ein völliger Narr, die
Hogwarts-Karte im Besitz dieser beiden Idioten zu lassen. Ich hatte einen
unangenehmen Schock, nachdem ich die Karte wiedergefunden hatte; sie zeigte
meinen und deinen Namen korrekt an! Die Weasley-Idioten hatten es für eine bloße
Fehlfunktion gehalten, besonders nachdem du deinen Umhang und deinen
Zeitumkehrer erhalten hattest. Wenn Dumbledore die Karte selbst behalten hätte -
wenn die Weasleys jemals mit Dumbledore darüber gesprochen hätten - aber das
haben sie nicht, zum Glück."

\glqq Das würde ich gerne sehen\grqq{}, sagte Harry.

Ohne den Blick vom Kessel zu nehmen, zog Professor Quirrell ein gefaltetes
Pergament aus seiner Robe, zischte: \glqq \textbf{\emph{Zeig deine
Umgebung}}\grqq{}, und warf Harry das gefaltete Pergament zu.

Es flog zielsicher durch die Luft, ein Hauch von Unheil lag über Harrys Sinnen,
als es sich auf ihn zubewegte, und dann flatterte es sanft zu Harrys Füßen.
Harry hob das Pergament auf und entfaltete es.

Zuerst schien das Pergament leer zu sein. Dann, als würde eine unsichtbare Feder
darüber fahren, erschienen die Umrisse von Wänden und Türen, alle in
handschriftlichen Linien gezeichnet. Die Schrift umriss eine Reihe von Kammern,
von denen die meisten leer waren; die letzte Kammer in der Reihe hatte ein
wirres Gekritzel in der Mitte, als ob die Karte ihre eigene Verwirrung
ausdrücken wollte; und die vorletzte Kammer zeigte zwei Namen, die an den
Positionen in der Kammer geschrieben waren, die Harry saß und Professor Quirrell
stand. \emph{ Tom M. Riddle. }
\emph{Tom M. Riddle. }

Harry starrte auf das Pergament, ein unangenehmer Schauer überkam ihn. Es war
eine Sache zu hören, dass Lord Voldemort behauptete, dein Name sei Tom Riddle;
es war eine andere Sache, festzustellen, dass die Magie von Hogwarts dem
zustimmte.

\glqq Hast du diese Karte manipuliert, um dieses Ergebnis zu erzielen, oder ist
sie dir durch Zufall erschienen?"

\glqq \textbf{\emph{War Zufall}}\grqq{}, antwortete Professor Quirrell mit einem
Unterton von zischendem Lachen. \glqq \textbf{\emph{Keine Tricks.}}"

Harry faltete die Karte und warf sie zurück in Professor Quirrells Richtung;
eine Kraft fing sie in der Luft auf, bevor sie den Boden erreichte, und zog die
Karte zurück in Professor Quirrells Robe.

Der Verteidigungsprofessor sprach. \glqq Ich möchte noch anmerken, dass Snape
Miss Granger und ihre Untergebenen zu den Schulschlägern geführt und manchmal
schützend eingegriffen hat."

\glqq Das wusste ich."

\glqq Interessant\grqq{}, sagte Professor Quirrell.

\glqq Hat Dumbledore auch davon erfahren? Antwort in Parsel."

\glqq \emph{Nicht so weit ich weiß}\grqq{}, zischte Harry.

\glqq Faszinierend\grqq{}, sagte Professor Quirrell. \glqq
\textbf{\emph{Vielleicht interessiert dich das auch: Der Zaubertränkehersteller
musste sgeheim arbeiten, weil sein Plan dem Plan des Schulleiters
entgegenstand.\grqq{} }}

Harry dachte darüber nach, während Professor Quirrell auf den Trank blies, als
wolle er ihn abkühlen, obwohl das Feuer unter dem Kessel immer noch brannte;
dann fügte er eine Prise Dreck und einen Tropfen Wasser und eine Glockenblume
hinzu.

\glqq Bitte erklär mir das\grqq{}, sagte Harry.

\glqq Ist es dir nie in den Sinn gekommen, dich zu fragen, warum Dumbledore
Severus Snape als Leiter des Hauses Slytherin ausgewählt hat? Zu sagen, dass es
eine Tarnung für seine Arbeit als Dumbledores Spion war, erklärt nichts. Snape
hätte auch nur Zaubertrankmeister sein können und gar nicht der Leiter von
Slytherin. Snape hätte zum Hüter des Geländes und der Schlüssel gemacht werden
können, wenn er innerhalb von Hogwarts bleiben musste! Warum der Leiter des
Hauses Slytherin? Ist dir nicht in den Sinn gekommen, dass das nach Dumbledores
Moralvorstellungen keine guten Auswirkungen auf die Slytherins haben kann?"

Der Gedanke war Harry nicht in genau dieser Form gekommen, nein... \glqq Ich
habe mich etwas Ähnliches gefragt. Ich habe das Dilemma nicht in dieser präzisen
Form formuliert."

\glqq Und jetzt, wo Sie es haben, ist die Lösung offensichtlich?"

\glqq Nein\grqq{}, sagte Harry.

\glqq Das ist enttäuschend. Du hast nicht genug Zynismus gelernt, du hast die
Flexibilität dessen, was Moralisten Moral nennen, nicht begriffen. Um ein
Komplott zu ergründen, muss man sich die Konsequenzen ansehen und fragen, ob sie
vielleicht beabsichtigt sind. Dumbledore hat das Haus Slytherin absichtlich
sabotiert - \textbf{\emph{sssieh mich nicht so an, Junge, ich spreche die
Wahrheit.}} Während des letzten Zaubererkrieges füllten Slytherins meine Reihen
an Untergebenen auf, und andere Slytherins im Zaubergamot unterstützten mich.
Betrachte es aus Dumbledores Perspektive und bedenke, dass er kein angeborenes
Verständnis für die Art der Slytherins hat. Stelle dir vor, Dumbledore wird
zunehmend traurig über dieses Hogwarts-Haus, das die Quelle von so viel Unheil
zu sein scheint. Und dann, siehe da, setzt Dumbledore als Leiter von Slytherin
die Person Snape ein. Snape! Severus Snape! Ein Mann, der seinem Haus weder List
noch Ehrgeiz beibringt, ein Mann, der lasche Disziplin durchsetzt und seine
Kinder schwach macht! Ein Mann, der die Schüler anderer Häuser beleidigen würde,
der den Namen Slytherins bei allen anderen ruiniert! Ein Mann, dessen Nachname
im magischen Britannien unbekannt und sicher nicht adlig war, der halb in Lumpen
herumlief! Glaubst du, Dumbledore war sich der Konsequenzen nicht bewusst? Wenn
Dumbledore derjenige war, der sie herbeigeführt hat, und ein Motiv hatte, sie
herbeizuführen? Ich vermute, Dumbledore hat sich eingeredet, dass im nächsten
Zaubererkrieg mehr Menschenleben gerettet würden, wenn Voldemorts zukünftige
Todesser geschwächt würden."

Professor Quirrell ließ einen Eissplitter in den Kessel fallen, der langsam
schmolz, als er den Oberflächenschaum berührte.

\glqq Wenn man diesen Prozess lange genug fortsetzt, würde kein Kind mehr nach
Slytherin gehen wollen. Das Haus würde ausgemustert werden, und wenn der Hut den
Namen weiterhin rief, würde er zu einem Zeichen der Schmach unter den Kindern
werden, die danach auf die anderen drei Häuser verteilt würden. Von diesem Tag
an würde Hogwarts drei aufrechte Häuser des Mutes und der Gelehrsamkeit und des
Fleißes haben, ohne dass ein Haus der schlechten Kinder hinzukäme; gerade so,
als ob die drei Gründer von Hogwarts am Anfang weise genug gewesen wären,
Salazar Slytherin ihre Gesellschaft zu verweigern. Ich nehme an, das war
Dumbledores beabsichtigtes Endspiel; ein kurzfristiges Opfer für das größere
Wohl." Professor Quirrell lächelte sardonisch. \glqq Und Lucius hat das alles
geschehen lassen, ohne zu protestieren oder, wie ich annehme, ohne zu bemerken,
dass etwas schief gelaufen ist. Ich fürchte, dass meine ehemaligen Diener in
meiner Abwesenheit in diesem Kampf des Verstandes ziemlich unterlegen waren."

Harry hatte ein paar Schwierigkeiten, das zu verstehen, beschloss aber nach
einigem Nachdenken, dass jetzt nicht der richtige Zeitpunkt war, um zu
versuchen, es herauszufinden. Ob Lord Voldemort es glaubte, war nicht
entscheidend; Harry würde diese Anschuldigung selbst bewerten müssen. Professor
Quirrells Erwähnung seiner Diener hatte Harry an etwas anderes erinnert, das zu
fragen er... verpflichtet war, wie Harry annahm. Die schlechte Nachricht war
vorhersehbar. An jedem anderen Tag wäre es furchtbar gewesen. Heute würden sie
einfach in der Flut untergehen.

\glqq Bellatrix Black\grqq{}, sagte Harry. \glqq Was war die Wahrheit über sie?"

\glqq Sie war innerlich gebrochen, bevor ich sie jemals getroffen habe\grqq{},
sagte Professor Quirrell. Er hob etwas auf, das wie ein weiß-graues Gummiband
aussah, und hielt es über den Kessel; als das Gummi in den Dampf gehalten wurde,
färbte es sich schwarz. \glqq Legilimenz bei ihr anzuwenden war ein Fehler. Aber
dieser Blick zeigte mir, wie einfach es sein würde, sie dazu zu bringen, sich in
mich zu verlieben, also tat ich es. Von da an war sie die treueste aller meiner
Dienerinnen, die einzige, der ich \emph{fast} vertrauen konnte. Ich hatte nicht
die Absicht, ihr zu geben, was sie von mir wollte; also schenkte ich sie den
Brüdern Lestrange zur Benutzung, und die drei waren auf ihre eigene Art
glücklich."

\glqq Das bezweifle ich\grqq{}, sagte Harrys Mund, meist auf Autopilot. \glqq
Wenn das wahr wäre, hätte sich Bellatrix nicht daran erinnern können, wer die
Lestrange-Brüder waren, als wir sie in Askaban fanden."

Professor Quirrell zuckte mit den Schultern. \glqq Da magst du recht haben."

\glqq Was haben wir dort eigentlich gemacht?"

\glqq Herausfinden, wo Bellatrix meinen Zauberstab hingelegt hatte. Ich hatte
den Todessern von meiner Unsterblichkeit erzählt, in der - inzwischen als
vergeblich erwiesenen - Hoffnung, dass sie wenigstens für ein paar Tage
zusammenbleiben würden, wenn ich zu sterben scheine. Bellatrix' Anweisungen
lauteten, meinen Zauberstab von dort zu holen, wo mein Körper erschlagen worden
war, und diesen Zauberstab zu einem bestimmten Friedhof zu bringen, wo mein
Geist vor ihr erscheinen sollte."

Harry schluckte. Ihm kam das Bild von Bellatrix Black in den Sinn, wie sie auf
dem Friedhof wartete, wartete, wartete, in zunehmender Verzweiflung ... es war
kein Wunder, dass sie nicht strategisch gedacht hatte, als sie den
Longbottom-Haushalt angriff.

\glqq Was hast du mit Bellatrix gemacht, als sie raus war?"

\glqq Wir haben sie an einen ruhigen Ort geschickt, damit sie wieder zu Kräften
kommt\grqq{}, sagte Professor Quirrell. Ein kaltes Lächeln. \glqq Ich hatte noch
eine Verwendung für sie, oder besser gesagt, für einen bestimmten Teil von ihr,
und über meine Zukunftspläne werde ich keine Fragen beantworten."

Harry atmete tief durch und versuchte, die Kontrolle zu behalten. \glqq Gab es
in diesem Schuljahr noch andere geheime Verschwörungen?"

\glqq Oh, eine ganze Reihe, aber nicht mehr viele, die dich betreffen, nicht
dass mir das aus dem Stegreif einfällt. Der wahre Grund, warum ich verlangte,
den Erstklässlern den Patronus-Zauber beizubringen, war, einen Dementor vor
deine eigene Person zu bringen, und dann habe ich dafür gesorgt, dass dein
Zauberstab dorthin fällt, wo der Dementor dich weiterhin durch ihn aussaugen
kann. Es war keine Bosheit dabei, nur die Hoffnung, dass du etwas von deinem
wahren Gedächtnis zurückerhältst. Das war auch der Grund, warum ich dafür
gesorgt habe, dass einige Hexen dich während deines Vorfalls auf dem Dach aus
der Luft heruntergezogen haben, damit ich erscheinen konnte, um dein Leben zu
retten; nur für den Fall, dass während des Dementor-Vorfalls, den ich für kurz
danach geplant hatte, irgendein Verdacht auf mich fallen würde.
\textbf{\emph{Auch keine Böswilligkeit dort.}} Ich habe einige der Angriffe auf
Miss Grangers Gruppe arrangiert, damit die Angriffe abgewehrt werden konnten;
ich mag Schulschläger nicht besonders. Ich denke, das sind alle geheimen
Verschwörungen, die dich in diesem Schuljahr betreffen, \textbf{\emph{es sei
denn, ich habe ssetwas vergessen}}."

\emph{Lebenslektion gelernt, }sagte sein Hufflepart. \emph{Versuche, der
Versuchung zu widerstehen, dich wahllos in das Leben anderer Leute einzumischen.
Wie zum Beispiel in das Leben von Padma Patil. Wenn du nicht so enden willst,
meine ich}.

Eine Prise rotbrauner Staub wurde sanft in den Kessel mit den Zaubertränken
gesiebt, und Harry stellte seine vierte und letzte Frage, diejenige, die die
geringste Priorität zu haben schien, aber dennoch wichtig war.

\glqq Was war dein Ziel während des Zaubererkrieges? Ich meine, was -" Seine
Stimme schwankte. \glqq Was war der Sinn der ganzen Sache?"

Sein Gehirn wiederholte endlos: \emph{Warum, warum, warum Lord Voldemort .}..

Professor Quirrell hob eine Augenbraue. \glqq Sie haben dir von David Monroe
erzählt, nicht wahr?\grqq{}

\glqq Ja, du warst sowohl David Monroe als auch Lord Voldemort während des
Zaubererkrieges, diesen Teil habe ich verstanden. Du hast David Monroe
umgebracht sich als er verkleidet und hast David Monroes Familie ausgelöscht
aus, damit sie keinen Unterschied bemerken würden -\grqq{}

\glqq In der Tat.\grqq{}

\glqq Du hattest vor, die Seite zu kontrollieren, die den Zaubererkrieg gewinnt,
egal, welche Seite gewinnt. Aber warum musste eine Seite Voldemort sein? Ich
meine, wäre es nicht einfacher gewesen, die öffentliche Unterstützung mit
jemandem zu gewinnen, der weniger... mit jemandem, der weniger Voldemort
ist?\grqq{}

Professor Quirrells Hammer machte einen ungewöhnlich lauten Knall, als er weiße
Schmetterlingsflügel zerquetschte und sie mit einer weiteren Glockenblume
vermischte.

\glqq Ich hatte geplant\grqq{}, sagte Professor Quirrell barsch, \glqq dass Lord
Voldemort gegen David Monroe verliert. Der Fehler in dieser Strategie war die
absolute Erbärmlichkeit von -" Professor Quirrell hielt inne. \glqq Nein, ich
erzähle die Geschichte aus dem Zusammenhang gerissen. Hör zu, Junge, als ich
meine große Schöpfung erdacht hatte und in die Fülle meiner Magie gekommen war,
dachte ich, die Zeit sei gekommen, die politische Macht in die Hand zu nehmen.
Es würde unbequem sein, gewiss, und meine Zeit auf eine Weise in Anspruch
nehmen, die nicht angenehm war. Aber ich wusste, dass die Muggel irgendwann die
Welt zerstören oder Krieg gegen die Zauberer führen würden oder beides, und es
musste etwas getan werden, wenn ich nicht bis in alle Ewigkeit in einer toten
oder langweiligen Welt umherwandern wollte. Nachdem ich die Unsterblichkeit
erlangt hatte, brauchte ich ein neues Ziel, um meine Jahrzehnte zu füllen, und
die Muggel daran zu hindern, alles zu ruinieren, schien ein Ziel von akzeptablem
Umfang und Schwierigkeitsgrad zu sein. Es ist eine Quelle ständiger Belustigung
für mich, dass ausgerechnet ich der Einzige bin, der wirklich etwas für dieses
Ziel tut. Obwohl ich annehme, dass es für die sterblichen Insekten Sinn machen
würde, sich nicht um das Ende ihrer Welt zu kümmern; warum sollten sie auch,
wenn sie sowieso sterben werden und sich die Unannehmlichkeiten ersparen können,
auf dem Weg dorthin irgendetwas Schwieriges zu tun? Aber ich schweife ab. Ich
sah, wie Dumbledore durch seinen Sieg über Grindelwald zur Macht aufgestiegen
war, also dachte ich, ich könnte das Gleiche tun. Ich hatte mich schon vor
langer Zeit an David Monroe gerächt - er war ein Ärgernis aus meinem Jahrgang in
Slytherin - also dachte ich daran, auch seine Identität zu stehlen und seine
Familie auszulöschen, um mich zum Erben seines Hauses zu machen. Und ich ersann
auch einen großartigen Gegner für David Monroe, den furchterregendsten Dunklen
Lord, den man sich vorstellen kann, unvorstellbar klug; bei weitem gefährlicher
als Grindelwald, denn seine Intelligenz würde auf all die Weise perfektioniert
sein, auf die Grindelwald fehlerhaft und selbstzerstörerisch gewesen war. Ein
Dunkler Lord, der sein gerissenes Äußerstes tun würde, um die Allianzen zu
zerstören, die ihn bekämpfen würden, ein Dunkler Lord, der durch seine
oratorischen Fähigkeiten die tiefste Loyalität von seinen Anhängern fordern
würde. Der furchtbarste Dunkle Lord, der jemals Großbritannien oder die Welt
bedroht hatte, das war der, den David Monroe besiegen würde."

Professor Quirrells Hammer schlug mit zwei weiteren Schlägen auf eine
Glockenblume und dann auf eine andere blasse Blume.

\glqq Aber obwohl ich auf meinen Wanderungen manchmal die Rolle des Dunklen
Zauberers gespielt hatte, hatte ich nie die Identität eines vollwertigen Dunklen
Lords mit Untergebenen und einer politischen Agenda angenommen. Ich hatte keine
Übung in dieser Aufgabe, und ich erinnerte mich an die Geschichte von Dark
Evangel und das Desaster ihres ersten öffentlichen Auftritts. Nach dem, was sie
hinterher sagte, hatte sie sich eigentlich als wandelnde Katastrophe und Apostel
der Finsternis bezeichnen wollen, aber in der Aufregung des Augenblicks stellte
sie sich stattdessen als Apostroph der Finsternis vor. Danach musste sie zwei
ganze Dörfer vernichten, bevor sie jemand ernst nahm."

\glqq Du hast dich also entschlossen, zuerst ein Experiment im kleinen Rahmen zu
versuchen\grqq{}, sagte Harry. Eine Übelkeit stieg in ihm auf, denn in diesem
Moment verstand Harry, er sah sich selbst gespiegelt; der nächste Schritt war
genau das, was Harry selbst getan hätte, wenn er keine Spur von Ethik gehabt
hätte, wenn er innerlich so leer gewesen wäre. \glqq Du hast dir eine
Wegwerf-Identität geschaffen, um zu lernen, wie die Dinge funktionieren, und um
deine Fehler aus dem Weg zu räumen."

\glqq In der Tat. Bevor ich ein wirklich schrecklicher Dunkler Lord wurde, gegen
den David Monroe kämpfen musste, schuf ich mir zum Üben die Persona eines
Dunklen Lords mit glühend roten Augen, der sinnlos grausam zu seinen
Untergebenen war und eine politische Agenda aus nacktem persönlichem Ehrgeiz
kombiniert mit Blutpurismus verfolgte, wie sie von Betrunkenen in der
Nockturngasse vertreten wurde. Meine ersten Untergebenen wurden in einer Taverne
angeworben, bekamen Umhänge und Totenkopfmasken und wurden aufgefordert, sich
als Todesser vorzustellen."

Das kranke Gefühl des Verstehens vertiefte sich in der Magengrube von Harry.
\glqq Und du hast dich Voldemort genannt."

\glqq Genau, \emph{General Chaos.}" Professor Quirrell grinste von dort, wo er
am Kessel stand. \glqq Ich wollte, dass es ein Anagramm meines Namens ist, aber
das hätte nur funktioniert, wenn man mir praktischerweise den zweiten Vornamen
'Marvolo' gegeben hätte, und dann wäre es zu weit hergeholt gewesen. Unser
eigentlicher zweiter Vorname ist Morfin, falls es dich interessiert. Aber ich
schweife ab. Ich dachte, Voldemorts Karriere würde nur ein paar Monate dauern,
höchstens ein Jahr, bevor die Auroren seine Untergebenen zu Fall brachten und
der entbehrliche Dunkle Lord verschwand. Wie du sehen kannst, hatte ich meine
Konkurrenz maßlos überschätzt. Und ich konnte mich nicht recht dazu durchringen,
meine Untergebenen zu quälen, wenn sie mir schlechte Nachrichten brachten, ganz
gleich, was die Dunklen Lords im Theater taten. Ich konnte es nicht ganz
schaffen, die Lehren des Blutpurismus so zusammenhanglos zu argumentieren, als
wäre ich ein Betrunkener. Ich habe nicht versucht, clever zu sein, als ich meine
Untergebenen auf ihre Missionen schickte, aber ich habe ihnen auch keine völlig
sinnlosen Befehle gegeben -"

Professor Quirrell gab ein reumütiges Grinsen von sich, das man in einem anderen
Zusammenhang vielleicht als charmant hätte bezeichnen können.

\glqq Einen Monat später warf sich Bellatrix Black vor mir nieder, und nach drei
Monaten verhandelte Lucius Malfoy mit mir bei einem Glas teuren Feuerwhiskeys.
Ich seufzte, gab alle Hoffnung auf die Zaubererwelt auf und begann als David
Monroe, mich diesem furchterregenden Lord Voldemort entgegenzustellen."

\glqq Und was geschah dann -"

Ein Knurren verzerrte Professor Quirrells Gesicht. \glqq Die absolute
Unzulänglichkeit jeder einzelnen Institution in der Zivilisation des magischen
Britanniens ist das, was passiert ist! Du kannst es nicht begreifen, Junge! Ich
kann es nicht begreifen! Man muss es sehen, und selbst dann kann man es nicht
glauben! Du wirst vielleicht bemerkt haben, dass von deinen Mitschülern, die
über die Berufe ihrer Familie sprechen, drei von vier eine Tätigkeit in
irgendeinem Teil des Ministeriums zu erwähnen scheinen. Du wirst dich fragen,
wie ein Land es schaffen kann, drei seiner vier Bürger in der Bürokratie zu
beschäftigen. Die Antwort ist, dass, wenn sie sich nicht alle gegenseitig daran
hindern würden, ihre Arbeit zu tun, keiner von ihnen mehr etwas zu tun hätte!
Die Auroren waren als Einzelkämpfer kompetent, sie kämpften gegen Dunkle
Zauberer und nur die Besten überlebten, um neue Rekruten auszubilden, aber ihre
Führung war in absoluter Unordnung. Das Ministerium war so sehr mit der
Weiterleitung von Papieren beschäftigt, dass das Land keine effektive Opposition
gegen Voldemorts Angriffe hatte, außer mir, Dumbledore und einer Handvoll
untrainierter Irregulärer. Ein unfähiger, inkompetenter, feiger Faulpelz,
Mundungus Fletcher, galt als wichtiger Aktivposten im Orden des Phönix - denn da
er ansonsten arbeitslos war, brauchte er keinen weiteren Job zu machen der seine
Zeit beanspruchte! Ich versuchte, Voldemorts Angriffe zu schwächen, um zu sehen,
ob es möglich war, dass er verlor; sofort setzte das Ministerium weniger Auroren
ein, um sich mir entgegenzustellen! Ich hatte Maos Kleines Rotes Buch gelesen,
ich hatte meine Todesser in Guerillataktik ausgebildet - für nichts! Umsonst!
Ich griff das gesamte magische Britannien an und in jedem Gefecht waren meine
Truppen dem Gegner zahlenmäßig überlegen! In meiner Verzweiflung befahl ich
meinen Todessern, systematisch jeden einzelnen inkompetenten Leiter der
Abteilung für magische Strafverfolgung zu ermorden. Ein Bürohengst nach dem
anderen meldete sich trotz des Schicksals seiner Vorgänger freiwillig für höhere
Posten und rieb sich vergnügt die Hände über die Aussicht auf eine Beförderung.
Jeder von ihnen dachte, sie würden nebenbei einen Deal mit Lord Voldemort
abschließen. Wir brauchten sieben Monate, um uns durch sie alle zu morden, und
kein einziger Todesser hat gefragt, warum wir uns die Mühe gemacht haben. Und
dann, selbst mit dem Aufstieg von Bartemius Crouch zum Direktor und Amelia Bones
als Chef-Auror, war es immer noch zu wenig. Ich hätte allein besser kämpfen
können. Dumbledores Hilfe war seine moralische Zurückhaltung nicht wert, und
Crouchs Hilfe war seinen Respekt vor dem Gesetz nicht wert.\grqq{}

Professor Quirrell drehte das Feuer unter dem Zaubertrank auf.

\glqq Und irgendwann\grqq{}, sagte Harry durch die Herzensangst hindurch, \glqq
hast du gemerkt, dass du als Voldemort einfach mehr Spaß hast.\grqq{}

\glqq Es ist die am wenigsten nervige Rolle, die ich je gespielt habe. Wenn Lord
Voldemort sagt, dass etwas getan werden soll, gehorchen die Leute ihm und
widersprechen nicht. Ich musste meinen Impuls nicht unterdrücken, Leute, die
Idioten sind, zu foltern; ausnahmsweise war das alles Teil der Rolle. Wenn mir
jemand das Spiel unangenehm machte, sagte ich einfach Avadakedavra, unabhängig
davon, ob das strategisch klug war, und sie haben mich nie wieder belästigt."

Professor Quirrell hackte beiläufig einen kleinen Wurm in Stücke.

\glqq Aber meine wahre Epiphanie kam an einem bestimmten Tag, als David Monroe
versuchte, eine Einreisegenehmigung für einen asiatischen Ausbilder in
Kampftaktik zu bekommen, und ein Ministeriumsbeamter dies mit einem süffisanten
Lächeln ablehnte. Ich fragte den Ministeriumsangestellten, ob er verstehe, dass
diese Maßnahme sein Leben retten sollte, und der Ministeriumsangestellte
lächelte nur noch mehr. Dann warf ich in meiner Wut Masken und Vorsicht
beiseite, ich benutzte meine Legilimation, ich tauchte meine Finger in die
Kloake seiner Dummheit und riss die Wahrheit aus seinem Geist. Ich verstand
nicht, und ich wollte verstehen. Mit meiner Beherrschung der Legilimenz zwang
ich sein winziges Schreiberhirn, Alternativen auszuleben und zu sehen, was sein
Schreiberhirn davon halten würde, wenn Lucius Malfoy oder Lord Voldemort oder
Dumbledore an meiner Stelle stünden.\grqq{}

Professor Quirrells Hände waren langsamer geworden, während er zart kleine
Streifen aus einem Stück Kerzenwachs herausschälte.

\glqq Was ich an diesem Tag endlich begriffen habe, ist kompliziert, Junge,
deshalb habe ich es auch nicht früher verstanden. Für dich werde ich es trotzdem
versuchen zu beschreiben. Heute weiß ich, dass Dumbledore nicht an der Spitze
der Welt steht, auch wenn er der Oberste Mugwump des Internationalen Bundes ist.
Die Leute reden offen schlecht über Dumbledore, sie kritisieren ihn stolz und
ins Gesicht, in einer Weise, wie sie es nicht wagen würden, Lucius Malfoy die
Stirn zu bieten. Du hast dich Dumbledore gegenüber respektlos verhalten, Junge,
weißt du, warum du das getan hast?"

\glqq Ich ... bin mir nicht sicher\grqq{}, sagte Harry. Dass er Tom Riddles
übrig gebliebene neurale Muster hatte, war sicherlich eine naheliegende
Hypothese.

\glqq Wölfe, Hunde und sogar Hühner kämpfen untereinander um die Vorherrschaft.
Was ich schließlich aus dem Verstand dieses Schreibers verstand, war, dass für
ihn Lucius Malfoy die Dominanz hatte, Lord Voldemort hatte die Dominanz, und
David Monroe und Albus Dumbledore hatten sie nicht. Indem wir uns auf die Seite
des Guten stellten, indem wir erklärten, im Licht zu leben, hatten wir uns
unbedrohlich gemacht. In Großbritannien hat Lucius Malfoy die Vorherrschaft,
denn er kann deine Kredite einfordern oder Bürokraten des Ministeriums gegen
deinen Laden schicken oder dich im Tagespropheten verunglimpfen, wenn du dich
offen gegen seinen Willen stellst. Und der mächtigste Zauberer der Welt hat
keine Dominanz, denn jeder weiß, dass er\grqq{}, Professor Quirrells Lippen
kräuselten sich, \glqq ein Held aus Geschichten ist, unerbittlich zurückhaltend
und zu bescheiden für Rache. Sag, Kind, hast du schon einmal ein Drama gesehen,
in dem der Held, bevor er sich bereit erklärt, sein Land zu retten, so viel Gold
verlangt, wie ein Anwalt für einen Gerichtsprozess erhalten könnte?"

\glqq Eigentlich gab es viele solcher Helden in der Muggel-Literatur, ich nenne
nur mal Han Solo"

\glqq Nun, im magischen Drama ist das nicht so. Da gibt es nur bescheidene
Helden wie Dumbledore. Es ist die Fantasie des mächtigen Sklaven, der sich nie
wirklich über dich erheben wird, nie deinen Respekt einfordert, dich nicht
einmal um Lohn bittet. Verstehst du jetzt?"

\glqq Ich... glaube schon\grqq{}, sagte Harry. Frodo und Sam aus Herr der Ringe
schienen tatsächlich dem Archetyp eines völlig unbedrohlichen Helden zu
entsprechen. \glqq Willst du damit sagen, dass die Leute \emph{so} über
Dumbledore denken? Ich glaube nicht, dass die Hogwartsschüler ihn als einen
Hobbit sehen."

\glqq In Hogwarts bestraft Dumbledore gewisse Übertretungen gegen seinen Willen,
also ist er bis zu einem gewissen Grad gefürchtet - obwohl die Schüler sich
immer noch erlauben, ihn mehr als nur im Flüsterton zu verspotten. Außerhalb des
Schlosses wird Dumbledore belächelt; man begann, ihn als verrückt zu bezeichnen,
und er spielte die Rolle eines Narren. Tritt in die Rolle eines Retters aus dem
Theater, und die Leute sehen in dir einen Sklaven, auf dessen Dienste sie
Anspruch haben und den zu kritisieren ihr Vergnügen ist; denn es ist das
Privileg der Herren, sich zurückzulehnen und hilfreiche Korrekturen zu rufen,
während die Sklaven arbeiten. Nur in den Erzählungen der alten Griechen, aus
einer Zeit, als die Menschen in ihren Wahnvorstellungen noch nicht so hoch
entwickelt waren, sieht man den Helden, der auch respektiert ist. Hektor,
Aeneas, Achilles das waren Helden, die ihr Recht auf Rache an denen behielten,
die sie beleidigten, die Gold und Juwelen als Bezahlung für ihre Dienste
verlangen konnten, ohne Empörung auszulösen. Und wenn Lord Voldemort Britannien
eroberte, hätte er sich herablassen können sich im Sieg edel zu zeigen; und
niemand würde sein Wohlwollen für selbstverständlich halten, noch ihm
Korrekturen zuschreien, wenn sein Werk nicht nach ihrem Geschmack war. Wenn er
siegte, würde er wahren Respekt haben. An jenem Tag im Ministerium begriff ich,
dass ich mich durch meinen Neid auf Dumbledore genauso getäuscht hatte wie
Dumbledore selbst. Ich begriff, dass ich mich die ganze Zeit um den falschen
Platz bemüht hatte. Das solltest du wissen, Junge, denn du hast es dir erlaubt
schlecht über Dumbledore zu reden, aber hast es nie gewagt hast, schlecht über
mich zu reden. Sogar in deinen eigenen Gedanken, wette ich, denn der Instinkt
sitzt tief. Du wusstest, dass es dich teuer zu stehen kommt, den starken und
rachsüchtigen Professor Quirrell zu verspotten, aber dass es dich nichts kostet,
den schwachen und harmlosen Dumbledore zu beleidigen."

\glqq Danke\grqq{}, sagte Harry durch den Schmerz hindurch, \glqq für diese
wertvolle Lektion, Professor Quirrell, ich sehe, du hast Recht mit dem, was mein
Verstand getan hat."

\emph{Obwohl Tom Riddles Erinnerungen wahrscheinlich auch etwas damit zu tun hatten, dass er Dumbledore manchmal grundlos angegriffen hatte, war Harry in der Nähe von Professor McGonagall nicht so gewesen... die zugegebenermaßen die Macht hatte, Hauspunkte abzuziehen und nicht Dumbledores Ausstrahlung von Toleranz hatte... nein, es stimmte trotzdem, Harry wäre sogar in seinen eigenen Gedanken respektvoller gewesen, wenn es nicht sicher gewesen wäre Dumbledore gegenüber respektlos zu sein. }

Das war also David Monroe gewesen, und das war Lord Voldemort gewesen... Damit
war die rätselhafteste Frage immer noch nicht beantwortet, und Harry war sich
nicht sicher, ob es klug wäre, sie zu stellen. Wenn Lord Voldemort es irgendwie
geschafft hatte, nicht daran zu denken, und Professor Quirrell es dann immer
noch geschafft hatte, während neun Jahren der Betrachtung nicht daran zu denken,
dann war es nicht klug, es zu sagen... oder vielleicht war es das doch; die
Qualen des Zaubererkrieges waren nicht gut für Großbritannien gewesen.

Harry entschied und sprach.

\glqq Eine Sache, die mich verwirrt hat, war, warum der Zaubererkrieg so lange
gedauert hat\grqq{}, wagte Harry zu sagen. \glqq Ich meine, vielleicht
unterschätze ich die Schwierigkeiten, mit denen Lord Voldemort konfrontiert war
-"

\glqq Du willst wissen, warum ich nicht einige der stärkeren Zauberer, die sich
meinem Imperius hätten widersetzen können umgebracht und das Ministerium in, oh,
vielleicht drei Tagen übernommen habe."

Harry nickte stumm.

Professor Quirrell sah nachdenklich aus; seine Hand siebte Grasschnitt in den
Kessel, Stück für Stück. Diese Zutat stand, wenn Harry sich richtig erinnerte,
zu etwa vier Fünfteln am Ende des Rezepts.

\glqq Das habe ich mich auch gefragt\grqq{}, sagte der Verteidigungsprofessor
schließlich, \glqq als ich Trelawneys Prophezeiung von Snape hörte, und ich habe
sowohl über die Vergangenheit als auch über die Zukunft nachgedacht. Wenn du
mein vergangenes Ich gefragt hättest, warum er den Imperius nicht benutzt hat,
hätte er von dem Bedürfnis gesprochen, als Herrscher gesehen zu werden, die
Bürokratie des Ministeriums offen zu kommandieren, bevor es an der Zeit war,
seine Augen nach außen zu wenden. Er hätte bemerkt, wie ein schneller und
stiller Sieg später Herausforderungen bringen könnte. Er hätte auf das Hindernis
hingewiesen, das Dumbledore und seine unglaublichen Verteidigungsfähigkeiten
darstellen. Und er hätte ähnliche Ausreden für jeden anderen schnellen Weg
gehabt, den er in Betracht zog. Irgendwie war es nie der richtige Zeitpunkt, um
meine Pläne in die Endphase zu bringen, es gab immer noch eine Sache, die zuerst
erledigt werden musste. Dann hörte ich die Prophezeiung und ich wusste, dass es
an der Zeit war, denn die Zeit selbst nahm Notiz von mir. Dass die Zeit des
Zögerns vorbei war. Und ich blickte zurück und merkte, dass das irgendwie schon
seit Jahren so lief. Ich glaube..." Das gelegentliche Stückchen Gras tropfte
immer noch von seiner Hand, aber Professor Quirrell schien es nicht zu beachten.
\glqq Ich dachte, als ich unter dem Sternenlicht über meine Vergangenheit
nachdachte, dass ich mich zu sehr daran gewöhnt hatte, gegen Dumbledore zu
spielen. Dumbledore war intelligent, er gab sich alle Mühe, listig zu sein, er
wartete nicht darauf, dass ich zuschlug, sondern präsentierte mir
Überraschungen. Er machte bizarre Züge, die sich auf faszinierende und
unvorhersehbare Weise auswirkten. Im Nachhinein betrachtet gab es viele
offensichtliche Pläne, um Dumbledore zu vernichten; aber ich glaube, ein Teil
von mir wollte nicht wieder Solitär statt Schach spielen. Erst als ich die
Aussicht hatte, einen weiteren Tom Riddle zu erschaffen, gegen den ich ein
Komplott schmieden konnte, jemanden, der sogar noch würdiger war als Dumbledore,
war ich bereit, das Ende meines Krieges in Betracht zu ziehen. Ja, im Nachhinein
klingt das dumm, aber manchmal sind unsere Gefühle dümmer, als wir unseren
Verstand dazu bringen können, es zuzugeben. Ich hätte eine solche Politik
niemals absichtlich unterstützt. Es hätte gegen die Regeln neun, sechzehn,
zwanzig und zweiundzwanzig verstoßen, und das ist zu viel, selbst wenn man sich
amüsiert. Aber immer wieder zu entscheiden, dass es noch eine Sache gibt, die
noch zu tun ist, einen weiteren Vorteil, der noch zu erlangen ist, ein weiteres
Stück, das ich einfach an seinen Platz bringen muss, bevor ich eine angenehme
Zeit in meinem Leben verlasse und mich der langweiligeren Herrschaft über
Britannien zuwende ... nun, selbst ich bin nicht immun gegen einen solchen
Fehler, wenn ich nicht merke, dass ich ihn mache."

Und da wusste Harry, was am Ende dieser Geschichte passieren würde, nachdem der
Stein der Weisen wiedergefunden worden war.

\emph{Am Ende der Sache würde Professor Quirrell ihn umbringen. Professor Quirrell wollte ihn nicht töten. Es war möglich, dass Harry die einzige Person auf der Welt war, gegen die Professor Quirrell keinen Tötungsfluch anwenden konnte. Aber Professor Quirrell dachte, er müsse es tun, aus welchem Grund auch immer. Deshalb hatte Professor Quirrell beschlossen, dass es notwendig war, den Trank des Glanzes auf die lange Art zu brauen. Deshalb hatte sich Professor Quirrell so leicht dazu überreden lassen, diese Fragen zu beantworten, endlich mit jemandem über sein Leben zu sprechen, der es verstehen könnte. Genau wie Lord Voldemort das Ende des Zaubererkrieges hinausgezögert hatte, um länger gegen Dumbledore spielen zu können.}

Harry konnte sich nicht mehr genau daran erinnern, was Professor Quirrell vorhin
gesagt hatte, dass er Harry nicht umbringen würde. Es war nichts Direktes im
Sinne von \glqq Ich habe absolut nicht vor, dich auf irgendeine Art und Weise zu
töten, es sei denn, du bestehst unbedingt darauf, etwas Dummes zu tun\grqq{}
gewesen. Harry hatte selbst gezögert, das Versprechen zu weit zu treiben und auf
eindeutigen Bedingungen zu bestehen, denn Harry hatte bereits gewusst, dass er
Lord Voldemort neutralisieren musste, und er hatte erwartet, dass eine präzisere
Sprache diese Tatsache offenbaren würde, wenn sie versuchten, wirklich
verbindliche Versprechen auszutauschen. Es hätte also sicherlich Schlupflöcher
gegeben, was auch immer gesagt worden war. Die Erkenntnis war kein besonderer
Schock, nur ein gesteigertes Gefühl der Dringlichkeit; ein Teil von Harry hatte
das bereits gewusst und nur auf einen Vorwand gewartet, um es den Überlegungen
mitzuteilen. Es waren hier zu viele Dinge gesagt worden, die Professor Quirrell
niemandem mit einer Lebenserwartung, die in mehr als Stunden gemessen wird,
offenbaren würde.

Die überwältigende Isolation und Einsamkeit des Lebens, das Professor Quirrell
beschrieben hatte, mochte erklären, warum er bereit war, gegen seine Regeln zu
verstoßen und mit Harry zu reden, wenn man bedachte, dass Harry bald sterben
würde und dass die Welt nicht wirklich wie ein Theaterstück funktionierte, in
dem der Bösewicht, der seine Pläne offenbarte, immer daran scheiterte, den
Helden danach zu töten. Aber Harrys Tod musste sicherlich irgendwo in diesen
Zukunftsplänen vorkommen.

Harry schluckte und kontrollierte seine Atmung. Professor Quirrell hatte gerade
ein Büschel Rosshaar in den Zaubertrank gegeben, und das war sehr spät im
Zaubertrank, wenn Harry sich richtig erinnerte. Es waren auch nicht mehr viele
Glockenblumen auf dem Haufen, die hinzugefügt werden konnten. Es war
wahrscheinlich an der Zeit, sich nicht mehr so viele Gedanken über das Risiko zu
machen und dieses Gespräch alles in allem weniger konservativ zu führen.

\glqq Wenn ich auf einen von Lord Voldemorts Fehlern hinweise\grqq{}, sagte
Harry, \glqq bestraft er mich dann dafür?"

Professor Quirrell hob die Augenbrauen. \glqq Nicht, wenn der Fehler ein echter
ist. Ich möchte nicht, dass du mich moralisch verurteilst. Aber ich würde weder
den Überbringer schlechter Nachrichten verfluchen, noch den Untergebenen, der
einen ehrlichen Versuch unternimmt, auf ein Problem hinzuweisen. Selbst als Lord
Voldemort könnte ich mich nie zu dieser Dummheit hinreißen lassen. Natürlich gab
es einige Dummköpfe, die meine Politik mit Schwäche verwechselten, die
versuchten, sich vorzudrängen, indem sie mich in ihrem öffentlichen Rat
niederdrückten und meinten, ich müsse das als Kritik tolerieren." Professor
Quirrell lächelte erinnerungsvoll. \glqq Die Todesser waren ohne sie besser
dran, und ich rate dir, nicht denselben Fehler zu machen."

Harry nickte, ein leichter Schauer durchlief ihn. \glqq Ähm, als du mir erzählt
hast, was in Godric's Hollow passiert ist, in der Halloween-Nacht, 1981, ich
meine, ähm... dachte ich, ich hätte einen weiteren Fehler in deiner
Argumentation gesehen. Eine Möglichkeit, wie du die Katastrophe hättest
vermeiden können. Aber, ähm, ich glaube, du hast einen blinden Fleck, eine Art
von Strategie, die du nicht in Betracht ziehst, so dass du es nicht einmal im
Nachhinein gesehen hast..."

\glqq Ich hoffe, du hast nicht vor, irgendetwas Dummes zu sagen, nach dem Motto:
\emph{'Versuch nicht, Leute zu töten'}\grqq{}, sagte Professor Quirrell. \glqq
Ich werde wütend sein, wenn das der Fall sein sollte."

\glqq \emph{Ein echter Irrtum, angesichts deiner Ziele. Wirst du mich verletzen,
wenn ich dir gegenüber die Rolle des Lehrers einnehme und dich unterrichte? Oder
wenn der Irrtum ssssimpeö und offensichtlich ist, und du dich ssdumm fühlssst?}"

\glqq \textbf{\emph{Nein}}\grqq{}, zischte Professor Quirrell. \glqq
\textbf{\emph{Nicht, wenn Lektion wahr isssst.}}"

Harry schluckte. \glqq Ähm. Warum hast du das Horkruxsystem nicht getestet,
bevor du es tatsächlich benutzen musstest?"

\glqq Testen?\grqq{}, fragte Professor Quirrell. Er blickte von dem brauenden
Trank auf und Empörung kam in seine Stimme. \glqq Was soll das heißen, testen?!"

\glqq Warum hast du nicht getestet, ob das Horkruxsystem richtig funktioniert,
bevor du es an Halloween gebraucht hast?"

Professor Quirrell sah angewidert aus. \glqq Du bist lächerlich - ich wollte
nicht sterben, Mr. Potter, und das war die einzige Möglichkeit, meine großartige
Schöpfung zu testen! Was hätte es gebracht, wenn ich mein Leben lieber früher
als später riskiert hätte? Wie wäre ich besser dran gewesen?"

Harry schluckte einen Kloß im Hals hinunter. \glqq Es gab eine Möglichkeit für
dich, dein Horkruxsystem zu testen, ohne zu sterben. Die allgemeine Lektion ist
wichtig. Verstehst du es jetzt?"

\glqq Nein\grqq{}, sagte Professor Quirrell nach einer Weile. Der
Verteidigungsprofessor zerbröselte vorsichtig eine der letzten Glockenblumen
zusammen mit einer Strähne langen blonden Haares und ließ sie dann in den Trank
fallen, der nun heller blubberte. Nur noch zwei weitere Glockenblumen lagen auf
dem Tisch für den Zaubertrank. \glqq Ich hoffe für dich, dass die Lektion
vernünftig ist."

\glqq Angenommen, Professor, ich hätte gelernt, wie man den verbesserten
Horkrux-Zauber wirkt, und ich wäre bereit, ihn zu benutzen. Was würde ich dann
damit machen?"

Professor Quirrell antwortete sofort. \glqq Du würdest eine Person finden, die
du moralisch verabscheust und von deren Tod du dich überzeugen kannst, dass er
andere Leben retten würde, und sie ermorden, um einen Horkrux zu erschaffen."

\glqq Und was dann?"

\glqq Mehr Horkruxe machen\grqq{}, sagte der Verteidigungsprofessor. Er hob ein
Glas mit etwas auf, das wie Drachenschuppen aussah.

\glqq Davor\grqq{}, sagte Harry.

Nach einer Weile schüttelte der Verteidigungsprofessor den Kopf. \glqq Ich sehe
es immer noch nicht, und du wirst mit diesem Spiel aufhören und es mir sagen."

\glqq Ich würde Horcruxe für meine Freunde machen. Wenn dir jemals auch nur ein
einziger anderer Mensch auf der ganzen Welt wirklich etwas bedeutet hätte, wenn
es nur eine Person gegeben hätte, die deiner Unsterblichkeit einen Sinn gegeben
hätte, jemanden, von dem du wolltest, dass er für immer mit dir lebt -" Harrys
Kehle schnürte sich zu. \glqq Dann, dann wäre die Idee, einen Horcrux für jemand
anderen zu machen, nicht so ein kontraintuitiver Gedanke gewesen." Harry
blinzelte heftig. \glqq Du hast einen blinden Fleck, wenn es um Strategien geht,
die beinhalten, nette Dinge für andere Menschen zu tun, und zwar so sehr, dass
es dich davon abhält, deine egoistischen Werte zu erreichen. Du denkst... es ist
nicht dein Stil, nehme ich an. Dieser... bestimmte Teil deines
Selbstverständnisses... ist es, der dich diese neun Jahre gekostet hat."

Die Pipette mit dem Minzöl, den der Verteidigungsprofessor in der Hand hielt,
füllte Tropfen für Tropfen Flüssigkeit in den Kessel.

\glqq Ich verstehe ...\grqq{}, sagte der Professor für Verteidigung langsam.
\glqq Ich verstehe. Ich hätte Rabastan das fortgeschrittene Horcrux-Ritual
beibringen und ihn zwingen sollen, die Erfindung zu testen. Ja, das ist im
Nachhinein höchst einleuchtend. Im Übrigen hätte ich Rabastan anweisen können,
sich selbst an einem Wegwerfkind zu versuchen, um zu sehen, was passiert, bevor
ich mich nach Godric's Hollow begab, um dich zu erschaffen." Professor Quirrell
schüttelte amüsiert den Kopf. \glqq Nun. Ich bin froh, dass ich das jetzt
erkenne und nicht zehn Jahre früher; ich hatte damals genug, um mir Vorwürfe zu
machen."

\glqq Du siehst keine schönen Wege, um die Dinge zu tun, die du tun
willst\grqq{}, sagte Harry. Er hörte einen Ton der Verzweiflung in seiner
eigenen Stimme. \glqq Selbst wenn eine nette Strategie effektiver wäre, siehst
du sie nicht, weil dein Selbstbild von dir sagt, dass du nicht nett bist sondern
böse."

\glqq Das ist eine faire Beobachtung\grqq{}, sagte Professor Quirrell. \glqq In
der Tat, jetzt, wo du darauf hingewiesen hast, sind mir gerade ein paar nette
Dinge eingefallen, die ich noch heute tun kann, um meine Ziele zu erreichen."

Harry schaute ihn nur an.

Professor Quirrell lächelte. \glqq Das ist eine gute Lektion, Mr. Potter. Von
nun an werde ich, bis ich den Trick gelernt habe, sorgfältig nach listigen
Strategien Ausschau halten, die darin bestehen, anderen Menschen Gutes zu tun.
Vielleicht gehe ich hin und übe Taten des guten Willens, bis mein Verstand
leicht in diese Richtung geht."

Ein kalter Schauer lief Harry über den Rücken. Professor Quirrell hatte dies
ohne das geringste sichtbare Zögern gesagt. Lord Voldemort war sich
\emph{absolut} sicher, dass er niemals erlöst werden konnte. Er hatte nicht die
geringste Angst, dass ihm das passieren könnte. Die vorletzte Glockenblume wurde
sanft in den Trank fallen gelassen.

\glqq Irgendwelche anderen wertvollen Lektionen, die du Lord Voldemort
beibringen möchtest, Junge?\grqq{}, sagte Professor Quirrell. Er blickte von dem
Trank auf und grinste, als wüsste er genau, was Harry dachte.

\glqq Ja\grqq{}, sagte Harry, wobei seine Stimme fast brach. \glqq Wenn es dein
Ziel ist, Glück zu erlangen, dann fühlt es sich besser an, nette Dinge für
andere Menschen zu tun als für dich selbst."

\glqq Glaubst du wirklich, dass ich nie daran gedacht habe, Junge?" Das Lächeln
war verschwunden. \glqq Hältst du mich für dumm? Nach meinem Abschluss in
Hogwarts bin ich jahrelang durch die Welt gezogen, bevor ich als Lord Voldemort
nach Großbritannien zurückgekehrt bin. Ich habe mehr Gesichter aufgesetzt, als
ich mir die Mühe machte, zu zählen. Glaubst du, ich habe nie versucht, den
Helden zu spielen, nur um zu sehen, wie sich das anfühlt? Ist dir der Name
Alexander Chernyshov schon mal begegnet? Unter diesem Deckmantel suchte ich ein
verlassenes Höllenloch auf, das von einem dunklen Zauberer beherrscht wurde, und
befreite die elenden Bewohner aus ihrer Knechtschaft. Sie weinten Tränen der
Dankbarkeit für mich. Es fühlte sich nicht nach etwas Besonderem an. Ich blieb
sogar in der Nähe und tötete die nächsten fünf dunklen Zauberer, die versuchten,
das Kommando über den Ort zu übernehmen. Ich gab meine eigenen Galleonen aus -
nun, nicht meine eigenen Galleonen, aber das Prinzip ist dasselbe - um ihr
kleines Land zu verschönern und einen Anschein von Ordnung einzuführen. Sie
krochen umso mehr und nannten eines von drei ihrer Kinder Alexander. Ich fühlte
immer noch nichts, also nickte ich vor mich hin, schrieb es als einen fairen
Versuch ab und machte mich auf den Weg."

Und warst du dann glücklich als Lord Voldemort?" Harrys Stimme hatte sich
erhoben, war wild geworden.

Professor Quirrell zögerte, dann zuckte er mit den Schultern. \glqq Es scheint,
dass Sie die Antwort darauf bereits kennen."

\glqq Warum dann? Warum sollte man Voldemort sein, wenn es einen nicht einmal
glücklich macht?" Harrys Stimme brach. \glqq Ich bin du, ich bin dir
nachempfunden, also weiß ich, dass Professor Quirrell nicht nur eine Maske ist!
Ich weiß, dass er jemand ist, der du wirklich hättest sein können! Warum nicht
einfach so bleiben? Nimm deinen Fluch von der Verteidigungsposition und bleib
einfach hier, benutze den Stein der Weisen, um David Monroes Gestalt anzunehmen
und lass den echten Quirinus Quirrell frei, wenn du sagst, dass du aufhörst,
Menschen zu töten, schwöre ich, niemandem zu sagen, wer du wirklich bist, sei
einfach Professor Quirrell, für immer! Deine Schüler würden dich schätzen, die
Schüler meines Vaters schätzen ihn -"

Professor Quirrell kicherte über den Kessel, während er ihn umrührte. \glqq Es
leben vielleicht fünfzehntausend Zauberer im magischen Britannien, Kind. Früher
waren es mehr. Es gibt einen Grund, warum sie Angst haben, meinen Namen
auszusprechen. Du würdest mir das verzeihen, weil dir mein Unterricht in
Kampfmagie gefallen hat?"

\emph{Sekundiert}, sagte Harrys innerer Hufflepuff. \emph{Im Ernst, was soll
das?}

Harry behielt seinen Kopf erhoben, obwohl er zitterte. \glqq Es steht mir nicht
zu, dir irgendetwas zu verzeihen, was du getan hast. Aber es ist besser als ein
weiterer Krieg."

\glqq Ha\grqq{}, sagte der Verteidigungsprofessor. \glqq Wenn du jemals einen
Zeitdreher findest, der vierzig Jahre zurückreicht und die Geschichte verändern
kann, solltest du das Dumbledore unbedingt mitteilen, bevor er Tom Riddles
Bewerbung für den Verteidigungsposten ablehnt. Aber leider fürchte ich, dass
Professor Riddle in Hogwarts kein dauerhaftes Glück gefunden hätte."

\glqq Warum nicht?"

\glqq Weil ich dann immer noch von Idioten umgeben gewesen wäre und ich sie
nicht hätte töten können\grqq{}, sagte Professor Quirrell milde. \glqq Idioten
zu töten ist meine große Freude im Leben, und ich wäre dir dankbar, wenn du
nicht schlecht darüber redest, bis du es selbst ausprobiert hast."

\glqq Es gibt etwas, das dich glücklicher machen würde als das\grqq{}, sagte
Harry, wobei seine Stimme wieder brach. \glqq Das muss es doch geben."

\glqq Warum?\grqq{}, fragte Professor Quirrell. \glqq Ist das irgendein
wissenschaftliches Gesetz, das mir noch nicht begegnet ist? Erzähl mir davon."

Harry öffnete den Mund, fand aber keine Worte, es musste doch etwas geben, wenn
er nur das Richtige sagen könnte -

\glqq Und du\grqq{}, sagte Professor Quirrell, \glqq hast auch kein Recht, von
Glück zu sprechen. Glück ist nicht das, was du über alles wertschätzt. Das hast
du von Anfang an entschieden, ganz am Anfang dieses Jahres, als der Sprechende
Hut dir Hufflepuff angeboten hat. Das weiß ich, weil ich vor vielen Jahren ein
ähnliches Angebot und eine ähnliche Warnung erhalten habe, die ich genauso
abgelehnt habe wie du. Darüber hinaus gibt es nicht viel mehr zu sagen Tom
Riddle." Der Verteidigungsprofessor wandte sich wieder dem Kessel zu.

Bevor Harry etwas erwidern konnte, warf Professor Quirrell die letzte
Glockenblume hinein, und ein Schwall glühender Blasen stieg aus dem Kessel auf.

\glqq Ich glaube, wir sind hier fertig\grqq{}, sagte Professor Quirrell. \glqq
Wenn du noch Fragen hast müssen sie warten."

Harry erhob sich zittrig auf seine Füße; auch als Professor Quirrell den Kessel
nahm und eine lächerlich große Menge der sprudelnden Flüssigkeit, mehr als in
ein Dutzend Kessel zu passen schien, auf das purpurne Feuer schüttete, das den
Eingang bewachte. Das violette Feuer erlosch schlagartig.

\glqq Und jetzt der Spiegel\grqq{}, sagte Professor Quirrell, zog den
Unsichtbarkeitsumhang aus seinem Umhang und ließ ihn vor Harrys Schuhe fallen.


Anm. des Übersetzers: Wer sich immer darüber gewundert hat warum Harry ist wie
er ist und das kein normales Kind so wäre.... nun wisst ihr den Grund. Eine
Kopie eines unwissenden Lord Voldemort war dieses Jahr in Hogwarts.

