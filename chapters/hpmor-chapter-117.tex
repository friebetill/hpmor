\chapter{Was sich zu beschützen lohnt: Professor Quirrell}

Die Sonne schien auf das den grünen Rasen herab, schlug Funken aus reflektiertem
Weiß von jedem vorbeiziehenden Tautropfen oder reflektierenden Blatt, das sich
zufällig richtig positionierte - ein klarer blauer Himmel für eine Beerdigung.

Harry hatte es abgelehnt, die Trauerrede zu halten. Er hatte sich noch einmal
geweigert. Professor Flitwick hatte ihn schon vor Wochen im Mai darum gebeten,
um Harry Zeit zu geben, seinen Text zu schreiben, bevor es notwendig werden
würde, zu sprechen; und Harry hatte auch damals Nein gesagt. So fiel die Wahl
auf einen Gryffindor aus dem sechsten Jahr, Oliver Habryka, der die vierthöchste
Punktzahl aller Schüler bei Quirrell hatte und General einer Armee gewesen war.
Der siebzehnjährige Junge war groß und nicht besonders ansehnlich in soliden
schwarzen Roben; statt einer roten Krawatte trug er eine violette Krawatte, wie
sie Professor Quirrell manchmal bevorzugt hatte. Er sprach, unter den gegebenen
Umständen, ex tempore. Die vorhergehenden, im Voraus geschriebenen Laudationes
waren abgelegt worden; Oliver Habryka hatte ein Pergament in der linken Hand,
aber er schaute es nicht einmal an.

\glqq{}Professor Quirrell war sehr krank\grqq{}, sagte der große Junge, seine
schwankende Stimme ging in ein Schülergeflüster über, das gelegentlich von einem
dumpfen Schluchzen unterbrochen wurde. \glqq{}Ich denke, wenn Professor Quirrell
in der Lage gewesen wäre, in der Fülle seiner Kräfte zu kämpfen, hätte
Du-weißt-schon-wer ihn nicht so leicht besiegen können, und vielleicht auch gar
nicht. Man sagt, dass David Monroe der Einzige war, vor dem Du-weißt-schon-wem
jemals Angst hatte. Aber\grqq{}, Olivers Stimme stockte, \glqq{}Professor
Quirrell war nicht im Vollbesitz seiner Kräfte. Er war sehr krank. Er hatte
Schwierigkeiten, selbst zu gehen. Und er ging, um sich dem Dunklen Lord zu
stellen, allein."

Dann gab es eine Pause, in der die Schüler eine Weile weinten.

Oliver wischte sich die Tränen mit seinem Ärmel weg und sprach wieder. \glqq{}Wir
wissen nicht genau, was passiert ist\grqq{}, sagte Oliver. \glqq{}Ich vermute,
der Dunkle Lord hat ihn ausgelacht. Vielleicht hat er sich über den Professor
lustig gemacht, weil er ihn herausgefordert hat, als er nicht einmal mehr
aufstehen konnte. Nun, jetzt lacht er nicht mehr, oder?"

Es gab heftiges Nicken von den Schülern; alle, die Harry sehen konnte, von
Gryffindor bis Slytherin.

\glqq{}Vielleicht wusste der Dunkle Lord einen Weg, Professor Quirrell zu heilen,
Du-weißt-schon-wer ist ja von den Toten auferstanden. Vielleicht bot er
Professor Quirrell sein Leben an, wenn Professor Quirrell ihm dienen würde.
Professor Quirrell lächelte und sagte dem Dunklen Lord, dass es an der Zeit sei,
dass sie ein Spiel spielen, das \emph{\glqq{}Wer ist der gefährlichste Zauberer
der Welt\grqq{}} heißt.\emph{\glqq}

\emph{Wenn du es nicht weißt, erfinde nicht einfach irgendwas.}
Aber Harry hat nichts gesagt. Es war das, was Lord Voldemort vielleicht versucht
hätte, es war das, was Professor Quirrell vielleicht zurück gesagt hätte.

\glqq{}Und sie sagen uns nicht alles\grqq{}, sagte Oliver, \glqq{}aber wir können
erraten, was als Nächstes geschah. Wir alle wissen, dass Hermine Granger, die
eine der besten Schülerinnen des Professors war, Anfang des Jahres von einem
Troll getötet wurde, es muss der Dunkle Lord gewesen sein, der das getan hat,
genauso wie er ihr den Blutkühlungszauber angehängt hat. Professor Quirrell
wusste, dass der Dunkle Lord dahinter steckte, also stahl er Miss Grangers
Leiche und konservierte sie, bewahrte sie sicher auf -"

\emph{Diese Idee konnte man ihm nicht verübeln.}

\glqq{}Dann ging Professor Quirrell hinaus, um sich dem Dunklen Lord zu stellen.
Der Dunkle Lord tötete Professor Quirrell. Und Hermine Granger erwachte wieder
zum Leben. Es heißt, sie sei jetzt lebendig und ganz. Und vielleicht noch etwas
mehr. Als der Dunkle Lord versuchte, sich ihrer zu bemächtigen, war alles, was
von ihm übrig war, seine verbrannten Roben und seine Hände um Miss Grangers
Kehle. So wie Harry Potter durch die Liebe und das Opfer seiner Mutter vor dem
Tötungsfluch bewahrt wurde, muss Professor Quirrell, der sich bereitwillig dem
Dunklen Lord allein stellte, Hermine Grangers Geist zurückgerufen haben, von wo
auch immer sie war -" Olivers Stimme pausierte erneut.

\glqq{}Nicht einfach so\grqq{}, sagte Harry aus der ersten Sitzreihe, seine
eigene Stimme heiser. Er musste an diesem Punkt etwas sagen, bevor es außer
Kontrolle geriet. Wenn es nicht schon außer Kontrolle geraten war. \glqq{}David
Monroe war ein mächtiger Zauberer, mächtiger als jeder außer ihm und mir wusste.
Ich glaube nicht, dass man jemanden von den Toten zurückholen kann, nur indem
man sich selbst opfert. Niemand sollte es auf diese Weise versuchen.\grqq{}

\emph{So eine schöne Geschichte. Sie hätte wahr sein sollen. Es hätte wahr sein
sollen.}

\glqq{}Ich weiß nicht viel über die Person hinter dem Professor\grqq{}, sagte
Oliver Habryka, nachdem er sich wieder unter Kontrolle hatte. \glqq{}Ich weiß,
dass David Monroe kein glücklicher Mann war. Er konnte nie einen Patronus-Zauber
wirken.\grqq{}

Wieder sammelten sich Tränen in Harrys Augen. \emph{Es war nicht richtig, es war
nicht fair, Voldemort hatte so viele Menschen getötet, er hätte zusammen mit
seinen Anhängern sterben müssen, er hatte keine Sonderbehandlung verdient.}
\emph{Aber es war nicht nur Harrys Schwäche gewesen, es waren die Horcruxe
gewesen, Voldemort hätte nicht einfach getötet werden können. Also konnte Harry
es zugeben, er war froh, er war froh, dass Professor Quirrell nicht ganz weg
war...}

\glqq{}Aber ich, weiß\grqq{}, sagte Oliver, Tränen glitzerten auf seinen eigenen
Wangen, \glqq{}Professor Quirrell, ist glücklich, wo auch immer, er ist
jetzt.\grqq{}

An Harrys linker Hand leuchtete ein winziger Smaragd hell in der Morgensonne.

\emph{Nicht der Himmel, nicht irgendein ferner Stern, nicht ein anderer Ort,
sondern ein besserer Mensch, ich werde es dir zeigen, eines Tages werde ich dir
zeigen, wie man glücklich ist -}

Der große Junge blickte auf ein Pergament hinunter, das er in der anderen Hand
hielt und das er zum ersten Mal zu Rate zog.

\glqq{}Professor Quirrell\grqq{}, sagte Oliver, seine Stimme wurde nun feuriger
und schneller, \glqq{}war mit Abstand der beste Professor für Kampfmagie, den
Hogwarts je hatte. Salazar Slytherin hätte nicht halb so ein guter Lehrer sein
können, egal welche Zaubersprüche er kannte. Professor Quirrell hat uns zu
Beginn dieses Jahres gesagt, dass das, was er uns gelehrt hat, immer unser
festes Fundament in den Künsten der Verteidigung sein wird. Und das wird es auch
sein. Für immer. Wir werden es den neuen Schülern nächstes Jahr beibringen,
egal, wen wir als Professor haben. Die älteren Schüler werden die jüngeren
unterrichten. Das ist die Lösung für den Fluch der Verteidigungsposition. Wir
werden nicht herumsitzen und darauf warten, dass die Autorität uns unterrichtet.
Und wir werden dafür sorgen, dass Professor Quirrells Lehren in Hogwarts nie
aussterben.\grqq{}

Harry schaute zu der Stelle, an der Professor - nein, Schulleiterin McGonagall -
saß, und sah, wie die Schulleiterin stumm nickte, mit einem Blick, der traurig
und streng und stolz war.

\glqq{}Sie haben uns noch nicht zu Miss Granger gelassen\grqq{}, sagte Oliver.
Seine Stimme zitterte. \glqq{}Das Mädchen, das wiederauferstanden ist. Aber ich
werde immer an den Verteidigungsprofessor denken, wenn ich sie sehe. Sein Opfer
lebt in ihr weiter, genau wie seine Lehren in uns weiterleben.\grqq{} Oliver
warf einen Blick auf die Stelle, an der Harry saß, dann sah er wieder auf das
Pergament hinunter. \glqq{}Auf Professor Quirrell also, den besten Slytherin, den
es je gab, das, was jeder Slytherin sein sollte! Ein dreifaches Hoch auf
ihn!\grqq{}  > <p
style=\grqq{}.ext-align:center\grqq{}.\textbf{\glqq{}Huzzah!}   <p
style=\grqq{}.ext-align:center\grqq{}.  <p
style=\grqq{}.ext-align:center\grqq{}.\textbf{Hussa!}   <p
style=\grqq{}.ext-align:center\grqq{}.  <p
style=\grqq{}.ext-align:center\grqq{}.\textbf{Huzzah!\grqq{}}

Diesmal blieb niemand still, kein einziger Schüler, den Harry sehen konnte.

