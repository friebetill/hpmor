\partchapter{Sich seiner selbst bewusst sein}{1}

\lettrine{\loq A}{bbott}, Hannah!\grqq{} Pause. \glqq{}HUFFLEPUFF!\grqq{}

\glqq{}Bones, Susan!\grqq{} Pause. \glqq{}HUFFLEPUFF!\grqq{}

\glqq{}Boot, Terry!\grqq{} Pause. \glqq{}RAVENCLAW!\grqq{}

Harry blickte kurz zu seinem neuen Hauskameraden hinüber, mehr um einen kurzen
Blick auf sein Gesicht zu werfen als alles andere. Er versuchte immer noch, sich
von seiner Begegnung mit den Geistern unter Kontrolle zu bringen. Das Traurige,
das wirklich Traurige, das wirklich wirklich Traurige war, dass er sich
tatsächlich wieder unter Kontrolle zu haben schien. Es schien unpassend zu sein.
Als hätte er mindestens einen Tag brauchen sollen. Vielleicht ein ganzes Leben.

\emph{Vielleicht einfach nie.}

\glqq{}Ecke, Michael!\grqq{} Lange Pause.

\glqq{}RAVENCLAW!\grqq{}

Am Rednerpult vor dem riesigen Kopftisch stand Professor McGonagall und schaute
sich scharf um, während sie einen Namen nach dem anderen rief, obwohl sie nur
für Hermine und ein paar andere gelächelt hatte.

Hinter ihr, auf dem höchsten Stuhl am Tisch - eigentlich eher ein goldener Thron
- saß ein verhutzelter und bebrillter Alter mit einem silberweißen Bart, der
fast bis zum Boden reichte, wenn er sichtbar wäre, und wachte mit wohlwollender
Miene über die Sortierung; so stereotyp im Aussehen, wie ein weiser alter
Zauberer nur sein konnte, ohne tatsächlich orientalisch zu sein.

(Obwohl Harry gelernt hatte, sich vor stereotypen Erscheinungen zu hüten, seit
er zum ersten Mal Professor McGonagall begegnet war und dachte, dass sie gackern
müsste.)

Der uralte Zauberer hatte jedem sortierten Schüler applaudiert, mit einem
unerschütterlichen Lächeln, das frisch erfreut für jeden schien.

Links neben dem goldenen Thron saß ein Mann mit scharfen Augen und mürrischem
Gesicht, der niemandem applaudiert hatte und der es irgendwie schaffte, jedes
Mal, wenn Harry ihn ansah, direkt zu ihm zurückzuschauen.

Weiter links stand der bleiche Mann, den Harry im Tropfenden Kessel gesehen
hatte, dessen Augen wie in Panik auf die umstehende Menge blickten und der
gelegentlich in seinem Sitz zu zucken schien; aus irgendeinem Grund ertappte
sich Harry immer wieder dabei, ihn anzustarren.

Links von diesem Mann eine Reihe von drei älteren Hexen, die sich nicht
sonderlich für die Schüler zu interessieren schienen.

Dann zur rechten Seite des hohen goldenen Stuhls eine rundliche Hexe mittleren
Alters mit einem gelben Hut, die jedem Schüler außer den Slytherins applaudiert
hatte.

Ein kleiner Mann, der auf dem Stuhl stand, mit einem überlangem weißen Bart, der
jedem Schüler applaudiert hatte, aber nur die Ravenclaws anlächelte.

Und ganz rechts, den gleichen Platz wie drei kleinere Wesen einnehmend, das
bergige Wesen, das sie alle begrüßt hatte, nachdem sie aus dem Zug ausgestiegen
waren, und das sich Hagrid, Hüter der Schlüssel und des Geländes nannte.

\glqq{}Ist der Mann, der auf seinem Stuhl steht, der Leiter von Ravenclaw?\grqq{}
flüsterte Harry Hermine zu.

Ausnahmsweise antwortete Hermine nicht sofort; sie bewegte sich ständig von
einer Seite zur anderen, starrte den Sprechenden Hut an und zappelte so
energisch, dass Harry dachte, ihre Füße könnten den Boden verlassen.

\glqq{}Ja, das ist er\grqq{}, sagte eine der Vertrauensschüler, die sie
begleitet hatten, eine junge Frau, die das Blau von Ravenclaw trug. Miss
Clearwater, wenn Harry sich richtig erinnerte. Ihre Stimme war ruhig,
vermittelte aber einen Hauch von Stolz.

\glqq{}Das ist der Zaubereiprofessor von Hogwarts, Filius Flitwick, der
kenntnisreichste Zaubereimeister der Welt und ein ehemaliger Champion im
Duellieren -\grqq{}

\glqq{}Warum ist er so klein?\grqq{}, zischte ein Schüler, an dessen Namen sich
Harry nicht erinnerte. \glqq{}Ist er ein Halbblut?!\grqq{}

Ein kühler Blick von der jungen Vertrauensschülerin. \glqq{}Der Professor hat
tatsächlich Kobold Vorfahren -\grqq{}

\glqq{}Was?!\grqq{} sagte Harry unwillkürlich, was Hermine und vier andere
Schüler dazu veranlasste, ihn zum Schweigen zu bringen. Jetzt bekam Harry einen
überraschend einschüchternden Blick von der Ravenclaw Vertrauensschülerin
zugeworfen.

\glqq{}Ich meine -\grqq{}, flüsterte Harry. \glqq{}Nicht, dass ich ein Problem
damit hätte - es ist nur - ich meine - wie ist das möglich? Man kann nicht
einfach zwei verschiedene Spezies miteinander vermischen und lebensfähige
Nachkommen bekommen! Es müsste die genetischen Anweisungen für jedes Organ, das
sich zwischen den beiden Spezies unterscheidet, durcheinanderbringen - das wäre
so, als würde man versuchen,\grqq{}

sie hatten keine Autos, also konnte er nicht die Analogie mit den
durcheinandergebrachten Motoren verwenden, \glqq{}eine halbe Kutsche halb Boot
oder so zu bauen...\grqq{}

Der Ravenclaw Vertrauensschüler sah Harry immer noch ernst an. \glqq{}Warum
kannst du nicht ein Halbkutschen-Halbboot haben?\grqq{}

\glqq{}Hssh!?\grqq{}, schrie ein anderer Vertrauensschüler, obwohl die
Ravenclaw-Hexe immer noch leise gesprochen hatte.

\glqq{}Ich meine -\grqq{}, sagte Harry noch leiser und versuchte herauszufinden,
wie er die Frage beantworten sollte, ob sich die Kobolde aus den Menschen
entwickelt hatten, oder ob sie sich aus einem gemeinsamen Vorfahren der Menschen
wie dem Homo erectus entwickelt hatten, oder ob die Kobolde irgendwie aus den
Menschen gemacht worden waren - ob sie, sagen wir, unter einem vererbbaren
Zauber, dessen magische Wirkung abgeschwächt wurde, wenn nur ein Elternteil ein
'Kobold' war, genetisch immer noch menschlich waren, was erklären würde, wie
Kreuzung möglich war, und in welchem Fall Kobolde nicht ein unglaublich
wertvoller zweiter Datenpunkt dafür wären, wie sich Intelligenz in anderen
Spezies neben dem Homo sapiens entwickelt hatte

- jetzt, wo Harry darüber nachdachte, hatten die Kobolde in Gringotts nicht sehr
wie wirklich außerirdische, nicht-menschliche Intelligenzen gewirkt, nicht wie
Puppenspieler oder so -

\glqq{}Ich meine, wo kommen Kobolde überhaupt her?\grqq{}

\glqq{}Litauen\grqq{}, flüsterte Hermine abwesend, die Augen immer noch fest auf
den Sprechenden Hut gerichtet. Jetzt erntete Hermine ein Lächeln von der
Vertrauensschülerin.

\glqq{}Schon gut\grqq{}, flüsterte Harry. Am Rednerpult rief Professor
McGonagall: \glqq{}Goldstein, Anthony!\grqq{}

\glqq{}RAVENCLAW!\grqq{}

Hermine, die neben Harry saß, hüpfte so heftig auf ihren Zehenspitzen, dass ihre
Füße bei jedem Aufprall den Boden verließen.

\glqq{}Goyle, Gregory!\grqq{} Es herrschte ein langer, angespannter Moment der
Stille unter dem Hut. Fast eine Minute.

\glqq{}SLYTHERIN!\grqq{}

\glqq{}Granger, Hermine!\grqq{}

Hermine riss sich los und rannte mit voller Wucht auf den Sprechenden Hut zu,
hob ihn auf und stülpte sich den fleckigen alten Hut hart über den Kopf, was
Harry zusammenzucken ließ. Hermine war diejenige gewesen, die ihm die Sache mit
dem Sprechenden Hut erklärt hatte, aber sie behandelte ihn ganz sicher nicht wie
ein unersetzliches, lebenswichtiges, 800 Jahre altes Artefakt vergessener Magie,
das im Begriff war, komplizierte Telepathie in ihrem Kopf zu vollführen, und das
in keiner besonders guten körperlichen Verfassung zu sein schien.

\glqq{}RAVENCLAW!\grqq{}

Und da wir gerade von ihren vorhersehbaren Schlussfolgerungen sprechen. Harry
verstand nicht, warum Hermine so angespannt gewesen war. In welchem verrückten
alternativen Universum würde dieses Mädchen nicht nach Ravenclaw sortiert
werden? Wenn Hermine Granger nicht nach Ravenclaw ging, dann gab es keinen guten
Grund für die Existenz des Hauses Ravenclaw.

Hermine kam am Ravenclaw-Tisch an und wurde pflichtbewusst bejubelt; Harry
fragte sich, ob der Jubel lauter oder leiser gewesen wäre, wenn sie geahnt
hätten, was für eine Konkurrenz sie an ihrem Tisch begrüßen würden. Harry kannte
Pi bis 3,141592, weil die Genauigkeit auf einen Teil in einer Million für die
meisten praktischen Zwecke ausreichend war. Hermine kannte hundert Ziffern von
Pi, weil so viele Ziffern hinten in ihrem Mathelehrbuch abgedruckt waren.

Neville Longbottom ging nach Hufflepuff, worüber Harry froh war. Wenn dieses
Haus wirklich die Loyalität und Kameradschaft beinhaltete, für die es angeblich
stand, dann würde ein Haus voller zuverlässiger Freunde Neville sehr gut tun.

\emph{Schlaue Kinder in Ravenclaw, böse Kinder in Slytherin, Möchtegern-Helden
in Gryffindor und alle, die die eigentliche Arbeit machen, in Hufflepuff.}

(Obwohl Harry recht gehabt hatte, zuerst einen Ravenclaw- Vertrauensschüler zu
befragen. Die junge Frau hatte nicht einmal von ihrer Lektüre aufgeschaut oder
Harry erkannt, sondern nur einen Zauberstab in Nevilles Richtung gestupst und
etwas gemurmelt. Daraufhin hatte Neville einen benommenen Gesichtsausdruck
angenommen und zum fünften Waggon von vorne und dem vierten Abteil auf der
linken Seite gewandert war, in dem sich tatsächlich seine Kröte befunden hatte).

\glqq{}Malfoy, Draco!\grqq{}, kam nach Slytherin, und Harry atmete ein wenig
erleichtert auf. Es hatte wie eine sichere Sache ausgesehen, aber man konnte nie
wissen, welches winzige Ereignis den Verlauf seines Masterplans durchkreuzen
könnte.

Professor McGonagall rief

\glqq{}Perks, Sally-Anne!\grqq{}, und aus den versammelten Kindern löste sich
ein blasses, schmächtiges Mädchen, das seltsam ätherisch aussah - als würde sie
auf mysteriöse Weise verschwinden, sobald man aufhörte, sie anzusehen, und nie
wieder gesehen werden oder sich auch nur erinnern. Und dann (mit einem Ton der
Beklemmung, der so fest aus ihrer Stimme und ihrem Gesicht gehalten wurde, dass
man sie schon sehr gut kennen musste, um ihn zu bemerken) atmete Minerva
McGonagall tief ein und rief:

\glqq{}Potter, Harry!\grqq{}

Es herrschte eine plötzliche Stille in der Halle. Alle Gespräche verstummten.
Alle Augen wandten sich zum Starren. Zum ersten Mal in seinem ganzen Leben hatte
Harry das Gefühl, dass er vielleicht Lampenfieber haben könnte. Harry verdrängte
dieses Gefühl sofort. Ganze Räume voller Leute, die ihn anstarrten, war etwas,
an das er sich gewöhnen musste, wenn er im magischen Britannien leben oder
überhaupt irgendetwas anderes Interessantes mit seinem Leben anfangen wollte. Er
setzte ein selbstbewusstes und falsches Lächeln auf und hob einen Fuß, um einen
Schritt nach vorne zu machen -

\glqq{}Harry Potter!\grqq{}, rief die Stimme von Fred oder George Weasley, dann
die des anderen Weasley-Zwillings, und einen Moment später hatte der gesamte
Gryffindor-Tisch und bald darauf ein guter Teil von Ravenclaw und Hufflepuff den
Ruf aufgegriffen.

\glqq{}Harry Potter! Harry Potter! Harry Potter!!!\grqq{}

Und Harry Potter schritt nach vorne. Viel zu langsam, stellte er fest, als er
begonnen hatte, aber da war es schon zu spät, um sein Tempo zu ändern, ohne dass
es unangenehm aussah.

\glqq{}Harry Potter! Harry Potter! HARRY POTTER!\grqq{}

Mit einer nur allzu guten Vorstellung von dem, was sie sehen würde, drehte sich
Minerva McGonagall um und blickte hinter sich auf den Rest des Haupttisches.
Trelawney fächelte sich hektisch Luft zu, Filius schaute neugierig zu, Hagrid
klatschte mit, Sprout schaute ernst, Vector und Sinistra waren verwirrt und
Quirrell starrte ins Leere. Albus lächelt wohlwollend.

Und Severus Snape, hatte seinen leeren Weinkelch so fest umklammert, dass sich
das Silber langsam verformte.

Mit einem breiten Grinsen, den Kopf drehend, um sich erst zur einen und dann zur
anderen Seite zu verbeugen, während er zwischen den vier Tischen der Häuser
hindurchging, schritt Harry Potter in gemessenem Tempo vorwärts, wie ein Prinz,
der sein Schloss erbt.

\glqq{}Rettet uns vor noch mehr Dunklen Lords!\grqq{}, rief einer der
Weasley-Zwillinge, und dann rief der andere Weasley-Zwilling: \glqq{}Vor allem,
wenn sie Professoren sind!\grqq{} unter allgemeinem Gelächter von allen Tischen
außer Slytherin.

Minervas Lippen bildeten eine weiße Linie. Sie würde sich mit den
Weasley-Rabauken über den letzten Teil unterhalten, wenn sie dachten, sie sei
machtlos, weil es der erste Schultag war und Gryffindor keine Punkte abziehen
konnte. Wenn sie sich nicht um Nachsitzen kümmerten, würde sie etwas anderes
finden. Dann, mit einem plötzlichen Keuchen des Entsetzens, schaute sie in
Severus' Richtung, der sicher erkannte, dass der Potter-Junge keine Ahnung haben
musste, von wem sie sprach - Severus' Gesicht hatte sich von Wut in eine Art
angenehme Gleichgültigkeit verwandelt. Ein schwaches Lächeln spielte um seine
Lippen. Er blickte in die Richtung von Harry Potter, nicht auf den
Gryffindor-Tisch, und seine Hände hielten die zerstörten Überreste eines
ehemaligen Weinkelches.

Harry Potter ging mit einem starren Lächeln nach vorne, fühlte sich innerlich
warm und gleichzeitig irgendwie schrecklich. Sie jubelten ihm für eine Arbeit
zu, die er mit einem Jahr gemacht hatte. Eine Aufgabe, die er nicht wirklich
beendet hatte. Irgendwo, irgendwie, war der Dunkle Lord noch am Leben. Hätten
sie auch so laut gejubelt, wenn sie das gewusst hätten? Aber die Macht des
Dunklen Lords war schon einmal gebrochen worden. Und Harry würde sie wieder
beschützen. Wenn es tatsächlich eine Prophezeiung gab und sie so lautete.

Na ja, eigentlich ist es egal, was irgendeine verdammte Prophezeiung besagt. All
die Menschen, die an ihn glaubten und ihm zujubelten - Harry konnte es nicht
ertragen, dass das falsch war. Aufzublitzen und zu verblassen wie so viele
andere Wunderkinder. Eine Enttäuschung zu sein. Seinem Ruf als Symbol des Lichts
nicht gerecht zu werden, ganz egal, wie er ihn erlangt hatte. Er würde auf jeden
Fall, egal wie lange es dauert und selbst wenn es ihn umbringt, ihre Erwartungen
erfüllen.

Und diese Erwartungen dann noch übertreffen, so dass man sich rückblickend
wunderte, dass man einst so wenig von ihm verlangt hatte.

\glqq{}HARRY POTTER! HARRY POTTER! HARRY POTTER!\grqq{}

Harry machte seine letzten Schritte in Richtung des Sprechenden Hutes. Er
verbeugte sich vor dem Orden des Chaos am Gryffindor-Tisch, dann drehte er sich
um und verbeugte sich noch einmal zur anderen Seite der Halle und wartete, bis
der Applaus und das Kichern verklungen waren.

(Im Hinterkopf fragte er sich, ob der Sprechende Hut ein Bewusstsein hatte in
dem Sinne, dass er sich seines eigenen Bewusstseins bewusst war, und wenn ja, ob
er damit zufrieden war, nur einmal im Jahr mit Elfjährigen sprechen zu dürfen.
Sein Lied hatte das angedeutet:

(\emph{Oh, ich bin der Sortierhut und mir geht's gut, ich schlafe das ganze Jahr
und arbeite einen Tag.}.., anm. des Übersetzers: \emph{\glqq{}I am the sorting
hat and I'm okay, I sleep one year and work one day.\grqq{}})

Als wieder einmal Stille im Raum herrschte, setzte sich Harry auf den Hocker und
setzte sich vorsichtig das 800 Jahre alte telepathische Artefakt vergessener
Magie auf den Kopf. Er dachte so angestrengt nach, wie er nur konnte:

\emph{Sortiere mich noch nicht! Ich habe Fragen, die ich dir stellen muss!}
\emph{Wurde jemals mein Gedächtnis gelöscht? Hast du den Dunklen Lord als Kind
sortiert und kannst du mir etwas über seine Schwächen sagen?} \emph{Kannst du
mir sagen, warum ich den Bruderstab des Dunklen Lords bekommen habe?} \emph{Ist
der Geist des Dunklen Lords an meine Narbe gebunden und werde ich deshalb
manchmal so wütend?}

\emph{Das sind die wichtigsten Fragen, aber wenn du noch einen Moment Zeit hast,
kannst du mir etwas darüber sagen, wie ich die verlorene Magie, die dich
erschaffen hat, wiederentdecken kann?}

In die Stille von Harrys Geist, in der vorher nur eine Stimme zu hören gewesen
war, kam eine zweite, unbekannte Stimme, die sehr besorgt klang:

\textbf{\glqq{}Oh je. Das ist noch nie passiert...\grqq{}}