\chapter{Was sich zu beschützen lohnt, Teil 0}

Zuerst war Anna froh gewesen, dass das letzte Quidditch-Match so lange dauerte-
als Gryffindor war sie bei der Sache mit dem Hauspokal nur Zuschauerin, es war
ja nicht so, dass Gryffindor jemals gewonnen hätte. Im Gegensatz dazu war die
letztjährige Weltmeisterschaft im Quidditch, für die ihre Familie sehr teure
Karten gekauft hatte, in zehn Minuten vorbei gewesen, was schrecklich war. Die
modernen Quidditchspiele waren zu kurz geworden, der Schnatz viel zu schnell
gefangen. Es war ein weit verbreitetes Problem unter Kennern: Die
Besenverzauberungen hatten sich weiterentwickelt, während der Schnatz die
gleiche Regelgeschwindigkeit beibehalten hatte, mit dem Ergebnis, dass die
Quidditchspiele immer kürzer geworden waren. Auf professioneller Ebene hatte
sich der Quidditch-Sport auf einen Wettbewerb reduziert, bei dem es darum ging,
wer die tiefsten Taschen für den experimentellen Rennbesen seines Suchers hatte,
und der Rest der Spieler hätte genauso gut von der Tribüne aus zusehen können.
Jeder wusste, dass etwas getan werden musste, die Situation hatte sich seit
Jahrhunderten verschlimmert, und jetzt war sie unerträglich. Aber das
Quidditch-Komitee der \emph{International Confederation of Wizards} war in die
üblichen Streitereien der I.C.W. verwickelt, schreiende Disputationen zwischen
Deutschen und Bulgaren, und irgendwie konnte sich niemand darauf einigen, wie
genau man die Regeln in Ordnung bringen sollte.

Für Anna schien der richtige Weg offensichtlich, einfach den Schnatz schnell
genug zu machen, um die vier- oder fünfstündigen Spiele des frühen neunzehnten
Jahrhunderts und des Goldenen Zeitalters des Quidditch wiederherzustellen. Nur
waren die Belgier der Meinung, dass ein professionelles Spiel zwei Stunden
dauern sollte, wie in der Zeit, als Belgien den Quidditch dominiert hatte, und
die verrückten Italiener wollten zu den wochenlangen Quidditchspielen des
vierzehnten Jahrhunderts zurückkehren, und die noch verrückteren britischen
Puristen redeten immer wieder von gelegentlichen eintägigen Quidditchspielen als
Beweis dafür, dass sich Besen nicht wirklich verbessert haben konnten, da alles
in den alten Tagen besser war, was aber nicht ganz so war, wie das Interdikt von
Merlin funktionierte.

Sie war hundertprozentig auf der Seite von Harry Potter, dass es für Hogwarts an
der Zeit war, diese schwafelnden Langweiler aufzugeben und einfach die Regeln zu
ändern, und zwar hier und jetzt. Aber nicht durch die Abschaffung des Schnatzes,
das ging bis zum \emph{Kwidditch} im elften Jahrhundert zurück. Es spielte keine
Rolle, ob die damalige Schulleiterin von Hufflepuff die Neuerung zuerst
eingeführt hatte, weil einer ihrer Schüler das Spiel spielen wollte, aber nicht
in die üblichen Rollen passte. Der Schnatz hatte sich international
durchgesetzt, weil es spannender war, wenn das Spiel immer in der nächsten
Minute enden konnte. Anna hatte diesen Standpunkt in den letzten dreißig Minuten
lautstark vertreten und dabei ganz vergessen, auf das Spiel zu achten. Dank
eines glücklichen Zufalls bei der Platzwahl war sie in der Nähe des Jungen, der
lebte, und seines Schildes gewesen, und so hatte sie es geschafft, ihre Position
von Anfang an abzustecken. Im Hinterkopf war ihr bewusst, dass, wenn sich die
Quidditch-Regeln wirklich ab hier und jetzt ändern würden, dies das Wichtigste
war, was sie jemals tun würde. Sie konnte fast spüren, wie sich der Druck der
Zeit um sie herum drehte, als würde sich das Schicksal von Quidditch selbst
heute entscheiden, und sie stand mitten drin... obwohl sie in Wahrsagerei
natürlich nicht genug Punkte bekommen hatte, um so etwas wirklich zu spüren.

Sie bemerkte kaum, als der Junge-der-lebte irgendwann aufstand, um auf die
Toilette zu gehen. Der Junge-der-lebte fiel ihr auf, als er zurückstapfte; Harry
Potter sah ein bisschen müde und wackelig aus, obwohl seine Uniform so gepflegt
aussah, als hätte er gerade eine neue angezogen. Sie bemerkte eine halbe Stunde
später, als Harry Potter ein wenig zu schwanken schien und sich dann
zusammenkauerte, wobei seine Hände seine Stirn verdeckten; es sah aus, als würde
er an seiner Stirnnarbe herumstochern. Der Gedanke machte sie leicht beunruhigt;
jeder wusste, dass mit Harry Potter etwas nicht stimmte, und wenn Potters Narbe
ihn schmerzte, dann war es möglich, dass ein versiegeltes Grauen kurz davor war,
aus seiner Stirn herauszubrechen und alle zu verschlingen. Sie verwarf diesen
Gedanken jedoch und fuhr fort, den historisch Unwissenden lauthals
Quidditch-Fakten zu erklären. Sie bemerkte es auf jeden Fall, als Harry Potter
aufstand, die Hände immer noch auf der Stirn, und seine Hände fallen ließ, um zu
zeigen, dass seine berühmte Blitznarbe jetzt feuerrot und entzündet war. Sie
blutete, und das Blut tropfte an Potters Nase herunter. Sie hörte mitten im Satz
auf zu sprechen.

Die anderen Leute drehten sich um, um zu sehen, worauf sie starrte.

\glqq{}Professor McGonagall?\grqq{} sagte Harry Potter mit schwankender Stimme.

In seinen Augenwinkeln standen Tränen, was sie schockierte; der Junge-der-lebte
schien nicht die Art von Person zu sein, die in Tränen ausbricht. Harry Potter
hob seine Stimme weiter an, als fiele es ihm schwer zu sprechen. \glqq{}Ähm,
Professor McGonagall?"

Professor McGonagall wandte sich von der Stelle ab, an der sie sich mit dem
Hufflepuff-Quidditch-Team stritt. Die Augen der Leiterin von Gryffindor weiteten
sich schockiert, und dann schob sie Leute aus dem Weg, rannte fast.

\glqq{}Harry!\grqq{}, sagte sie. \glqq{}Deine Narbe!"

Stille breitete sich aus, in einem sich ausweitenden Kreis.

\glqq{}Ich glaube\grqq{}, sagte Harry, seine Stimme immer noch schwankend, aber
lauter, \glqq{}ich glaube, er ist zurück. Ich glaube, ich sehe - durch Voldemorts
Geist -"

Anna machte bei Du-weißt-schon-wem einen Schritt zurück und fiel fast über eine
Tribüne. Ein älterer Junge, der neben ihr stand, stieß einen Schrei des
Entsetzens aus, und dann kreischte der Junge-der-lebte noch lauter. \glqq{}ER
TÖTET SIE!\grqq{}, schrie Harry Potter. Das halbe Quidditch-Stadion drehte sich
um und sah ihn an. \glqq{}Das Ritual!\grqq{}, schrie Harry Potter. \glqq{}Das Blut
seiner Diener! Das Blut, das Leben! Er beschwor sie, er nahm ihre Köpfe, ihr
Blut, das Leben, um sein eigenes zu erneuern - DER FINSTERE HERR ERHEBT SICH,
VOLDEMORT IST ZURÜCKGEKEHRT!"

Madam Hooch stieß einen schrillen Pfiff aus, und die Quidditchbesen, die nicht
schon in der Luft stehen geblieben waren, wurden langsamer. Sie war sich selbst
nicht sicher, ob das ein Scherz war; wenn ja, steckte Junge-der-lebte oder
nicht, in größeren Schwierigkeiten, als sie sich vorstellen konnte.

Professor McGonagall hob ihren Zauberstab in Position für einen Schweigezauber
und Harry Potter fing ihre Hand auf.

\glqq{}Warten Sie -" Harry Potter keuchte, seine Stimme war leiser, aber immer
noch laut genug, dass sie und die Leute in ihrer Nähe sie deutlich hören
konnten. \glqq{}Er kann aufgehalten werden - ich sehe seinen Geist, seinen Fehler
- er kann jetzt aufgehalten werden - DER WEG IST NOCH OFFEN! SIE IST IHM AUF DEN
FERSEN! SIE, DIE VOLDEMORT GETÖTET HAT!" Harrys Stimme erhob sich weiter, als
Annas eigener Mund in plötzlicher Verwirrung aufklappte. \glqq{}KEHRE ZURÜCK!
KEHRE ZURÜCK, KEHRE ZURÜCK, LEBE UND HALTE IHN AUF! HALTE IHN AUF, HERMINE!" Und
dann verstummte Harry Potter.

Er sah sich nach den Leuten um, die ihn anstarrten. Sie hatte gerade
beschlossen, dass das alles ein unglaublich geschmackloser Streich sein musste,
als ein entferntes, aber scharfes \textbf{\emph{BUMM}} die Luft erfüllte.

Harry Potter schwankte und fiel auf die Knie, während ihr das Herz bis zum Hals
schlug. Eine Explosion von aufgeregtem Geplapper erhob sich um sie herum. Sie
konnte noch die Worte aus Harry Potters Mund hören, als Professor McGonagall
neben ihm kniete.

\glqq{}Es hat funktioniert\grqq{}, keuchte Harry Potter laut, \glqq{}sie hat ihn
erwischt, er ist weg."

\glqq{}Was?!\grqq{}, rief Professor McGonagall, dann blickte sie sich um. \glqq{}
Ruhig! Ruhe, ihr alle! Harry, was ist passiert?"

Harry Potter sprach schnell, aber laut. \glqq{}Voldemort - hat versucht, Hermine
wiederzubeleben - er hat Todesser herbeigerufen und dann hat er sie noch einmal
getötet, hat ihr Blut und ihr Leben gestohlen - Hermines Körper war da, ich weiß
nicht warum, vielleicht wollte Voldemort ihn für irgendetwas benutzen -
Voldemort kam zurück, er hat sich selbst wiederbelebt, aber Hermine ist ihm
gefolgt und sie hat ihn vernichtet, er ist weg, es ist vorbei. Es geschah auf
einem Friedhof in der Nähe von Hogwarts, es ist\grqq{}, Harry Potter erhob sich,
immer noch schwankend, \glqq{}ich glaube, es ist in dieser Richtung." Harry
Potter zeigte in die ungefähre Richtung, aus der der Knall gekommen war, \glqq{}
Ich bin mir nicht sicher, wie weit. Das Geräusch von dort hat zwanzig Sekunden
gebraucht, um hierher zu kommen, also vielleicht zwei Minuten auf einem Besen -"

Mit einer Bewegung, die so geschmeidig war, dass sie unbewusst wirkte, nahm
Professor McGonagall Haltung an und sagte: \glqq{}Expecto Patronum.\grqq{} Sie
wandte sich an die leuchtende Katze, die daraufhin erschien. \glqq{}Geh zu Albus,
sag ihm, er soll sofort kommen -"

\glqq{}Dumbledore ist weg!\grqq{}, rief Harry Potter. \glqq{}Der Schulleiter ist
weg, Professor McGonagall! Der Dunkle Lord hat ihn in eine Falle gelockt, er hat
eine Art Falle umgedreht, die der Schulleiter geplant hat, und Dumbledore wurde
außerhalb der Zeit gefangen, er ist weg!"

Das entsetzte Geplapper um sie herum wurde immer lauter.

\glqq{}Geh zu Albus!\grqq{} Professor McGonagall wandte sich an ihren Patronus.
Die Mondkatze schaute McGonagall nur traurig an, und Anna sog vor plötzlichem
Entsetzen den Atem ein und hatte das Gefühl, als hätte ihr jemand in den Magen
geschlagen. Es war echt, es war alles echt, das war kein Scherz.

\glqq{}Professor McGonagall, Hermine lebt!" Harry Potter erhob wieder seine
Stimme. \glqq{}Sie lebt wirklich und ist kein Inferius oder so, und sie ist immer
noch da auf dem Friedhof!"

\glqq{}Ein Besen!\grqq{} rief Professor McGonagall. Sie drehte sich zu den
Spielern um, die regungslos über dem Quidditchfeld schwebten. \glqq{}Ich brauche
einen Besen. JETZT!"

Trotz allem hob Anna eine Hand in stummem Protest, fing sich dann aber wieder,
als die Ravenclaw- und Slytherin-Sucher heranrauschten (mit ausgezeichnetem
strategischem Gespür, da sie eigentlich nichts taten). Harry Potter war bereits
dabei, einen weiteren Besen aus seinem Beutel zu holen, einen für mehrere
Personen.

Professor McGonagall sah dies und nickte fest. \glqq{}Sie bleiben hier, Mr.
Potter, es sei denn, es gibt einen hervorragenden Grund, warum Sie dort sein
müssen. Ich werde sofort gehen."

\glqq{}Das dürfen Sie nicht!\grqq{}, quietschte Professor Flitwick, der sich
winzig durch die Menge geschoben hatte und ab und zu jemandem unter die Beine
lief. Seine Augen waren weit aufgerissen, er sah aus, als wolle er in Ohnmacht
fallen. \glqq{}Du musst in Hogwarts bleiben, Minerva! Du - du bist die -"

Professor Flitwick schien Schwierigkeiten beim Sprechen zu haben. Professor
McGonagall drehte sich herum, um Professor Flitwick anzusehen, und blieb dann
stehen, während ihr das Blut aus dem Gesicht wich. Dann nahm sie Harry Potter
den Besen aus der Hand und reichte ihn dem kleinen Halbkobold Professor.

\glqq{}Filius\grqq{}, sagte sie knackig. Alle aufkeimende Panik war aus ihrer
Stimme verschwunden, sie sprach jetzt in ihrem schottischen Akzent, als würde
sie am Montag zum Unterricht sprechen. \glqq{}Such den Friedhof, von dem Mr.
Potter gesprochen hat, und finde Miss Granger. Apparieren Sie mit ihr nach St.
Mungo's und bleiben Sie dann bei ihr."

\glqq{}Ich glaube -\grqq{} sagte Harry Potter heiser. \glqq{}Ich glaube,
Verwandlung könnte dort im Kampf eingesetzt worden sein - Professor Quirrell hat
versucht, Voldemort zu bekämpfen - treffen Sie Vorsichtsmaßnahmen -"

Filius Flitwick nickte, und schwang sich auf den Besen.

\glqq{}Professor Quirrell ist tot!\grqq{}, jammerte Harry Potter. Die Angst in
seiner Stimme war deutlich zu hören. \glqq{}Er ist tot! Der Dunkle Lord hat ihn
umgebracht! Sein Körper -\grqq{} Harry Potter verschluckte sich. \glqq{}Er ist
dort, auf dem Friedhof."

Sie stolperte wieder zurück, fühlte es wie einen weiteren Schlag in den Magen.
Professor Quirrell war - einer ihrer Lieblingsprofessoren gewesen, er hatte sie
dazu gebracht, alles zu überdenken, was sie über Slytherin geglaubt hatte, sie
hatte auf eine entfernte Weise gewusst, dass er wahrscheinlich sehr bald sterben
würde, aber zu hören, dass er wirklich, wirklich tot war...

Der Junge-der-lebte setzte sich auf die Bank, als ob seine Beine ihn nicht mehr
tragen könnten. Professor McGonagall wandte sich an die Menge und berührte ihren
Zauberstab an ihrer Kehle. \glqq{}QUIDDITCH IST VORBEI\grqq{}, dröhnte ihre
verstärkte Stimme. \glqq{}GEHT ZURÜCK IN EURE SCHLAFSÄLE -"

\glqq{}Nicht!\grqq{}, schrie Harry Potter.

Professor McGonagall drehte sich um und sah ihn an.

Tränen liefen dem Jungen-der-lebte über die Wangen, er sah aus, als hätte die
Unterbrechung ihn selbst genauso überrascht wie alle anderen. \glqq{}Es war
Professor Quirrells letzter Plan\grqq{}, sagte Harry Potter und seine Stimme
brach. Der Junge-der-lebte schaute zu den Quidditchspielern, die nun in die Nähe
geflogen waren, als würde er sie direkt ansprechen. \glqq{}Sein letzter Plan..."

Harry Potter wurde von Professor McGonagall in den Krankenflügel gebracht. Die
anderen Professoren rannten los, um wer-weiß-was zu beaufsichtigen, und ließen
nur die Professoren Sinistra und Hooch zurück. Im Stadion kursierten wilde
Gerüchte; Anna wiederholte alles, was sie gehört hatte, so gut sie konnte.

Irgendetwas war mit Dumbledore passiert, einige Todesser waren herbeigerufen und
getötet worden (nein, Harry Potter hatte nicht gesagt, welche), Professor
Quirrell war ausgezogen, um sich dem Dunklen Lord zu stellen und dafür
gestorben, Du-weißt-schon-wer war zurückgekehrt und wieder gestorben, Professor
Quirrell war tot, \emph{ja} er war tot.

Mit der Zeit wanderten die meisten Schüler zurück in ihre Schlafsäle, um zu
schlafen, wenn sie konnten. Anna blieb im Stadion und sah sich den Rest des
Spiels an, ignorierte das Bedürfnis ihres Körpers nach Schlaf und ihre Augen,
die oft von Tränen trüb wurden. Die Ravenclaw-Mannschaft schlug sich tapfer.
Aber es gab kein Quidditch-Team, das die Slytherins an diesem Tag hätte besiegen
können.

Die Morgendämmerung färbte den Himmel, als die Slytherins das letzte Spiel, den
Quidditch-Pokal und den Hauspokal gewannen.

