\chapter{Vorsichtsmaßnahmen, Teil 3}

Juni 1992. Professor Quirrell war sehr krank. Nachdem er im Mai das Blut des
Einhorns getrunken hatte, schien es ihm für eine Weile besser zu gehen, aber der
Hauch von intensiver Macht, der ihn danach umgeben hatte, hatte nicht einmal
einen Tag angehalten. Mitte Mai hatten Professor Quirrells Hände wieder
gezittert, wenn auch auf subtile Weise. Die medizinische Kur des
Verteidigungsprofessors war zu früh unterbrochen worden, wie es schien und vor
sechs Tagen war Professor Quirrell während des Abendessens zusammengebrochen.

Madam Pomfrey hatte versucht, Professor Quirrell den Unterricht zu verbieten,
und Professor Quirrell hatte sie vor allen Leuten angeschrien. Der
Verteidigungsprofessor hatte geschrien, dass er trotzdem sterben würde und seine
verbleibende Zeit so nutzen würde, wie er wollte. Also hatte Madam Pomfrey dem
Verteidigungsprofessor blinzelnd verboten, irgendetwas anderes zu tun, als seine
Klassen zu unterrichten. Sie hatte um Freiwillige gebeten, die ihr helfen
würden, Professor Quirrell in einen Raum im Hogwarts-Krankenhaus zu bringen.
Mehr als hundert Schüler waren aufgestanden, nur die Hälfte trug grün.

Der Verteidigungsprofessor saß während der Mahlzeiten nicht mehr am Haupttisch.
Er sprach keine Zaubersprüche mehr während des Unterrichts. Die ältesten
Schüler, die die meisten Quirrell-Punkte hatten, halfen ihm beim Unterrichten,
die Siebtklässler, die bereits im Mai ihre Verteidigungs-U.T.Z.s abgelegt
hatten. Sie wechselten sich ab, um ihn von seinem Zimmer im Krankenflügel zum
Unterricht zu begleiten, und brachten ihm das Essen. Professor Quirrell
beaufsichtigte seine Kampfmagie-Stunden von einem Stuhl aus, sitzend.

Hermine sterben zu sehen, hatte mehr geschmerzt als das hier, aber das war viel
schneller vorbei gewesen.

\emph{Das ist der wahre Feind.}

Das hatte Harry schon gedacht, nachdem Hermine gestorben war. Gezwungen zu sein,
Professor Quirrell sterben zu sehen, Tag für Tag, Woche für Woche, hatte nicht
viel dazu beigetragen, seine Meinung zu ändern.

\emph{Das ist der wahre Feind, dem ich mich stellen muss, }dachte Harry in der
Verteidigungsstunde am Mittwoch, als er Professor Quirrell dabei beobachtete,
wie er sich zu weit zu einer Seite seines Stuhls lehnte, bevor ihn der Assistent
des siebten Jahrgangs an diesem Tag auffangen konnte.

\emph{Alles andere sind nur Schatten und Ablenkungen. }

Harry hatte Trelawneys Prophezeiung in Gedanken durchgespielt und sich gefragt,
ob der wahre Dunkle Lord vielleicht gar nichts mit Lord Voldemort zu tun hatte.

\emph{Geboren von denen, die sich ihm dreimal widersetzt haben,} schien stark an
die Peverell-Brüder und die drei Heiligtümer des Todes zu erinnern - obwohl
Harry nicht ganz klar war, wie der Tod ihn als ebenbürtig hätte markieren
können, was eine Art vorsätzliche Handlung von Seiten des Todes zu implizieren
schien.

\emph{Das allein ist der wahre Feind, }dachte Harry. \emph{Nach ihm werden
Professor McGonagall, Mum und Dad sterben, sogar Neville zu seiner Zeit,}
\emph{es sei denn, die Wunde in der Welt kann vorher geheilt werden.}

Es gab nichts, was Harry tun konnte. Madam Pomfrey tat bereits für Professor
Quirrell, was Magie tun konnte, und Magie schien den Muggeltechniken strikt
überlegen zu sein, wenn es um Heilung ging. Es gab nichts, was Harry tun konnte.
Nichts, was er tun konnte. Gar nichts. Überhaupt nichts. Harry hob die Hand und
klopfte an die Tür, für den Fall, dass die Person dort ihn nicht mehr entdecken
konnte.

\glqq Was ist los?\grqq{}, kam eine angestrengte Stimme aus dem Krankenzimmer.

\glqq Ich bin's."

Es gab eine lange Pause.

\glqq Komm rein\grqq{}, sagte die Stimme.

Harry schlüpfte hinein, schloss die Tür hinter sich und wirkte den
Schweigezauber. Er stellte sich so weit wie möglich von Professor Quirrell weg,
nur für den Fall, dass seine eigene Magie dem Professor Unbehagen bereitete.
Obwohl das Gefühl des Unheils mit jedem Tag schwächer wurde, schwand es.
Professor Quirrell lag zurück in seinem Krankenbett, nur sein Kopf wurde von
einem Kissen gestützt. Eine Decke aus baumwollartigem Stoff, rot mit schwarzen
Nähten, bedeckte ihn bis zur Brust. Ein Buch schwebte vor seinen Augen, umrandet
von einem fahlen Schein, der auch einen schwarzen Würfel umgab, der neben dem
Bett lag. Es war also nicht die eigene Magie des Verteidigungsprofessors,
sondern ein Gerät irgendeiner Art.

Das Buch war '\emph{Denksport Physik}' von Epstein, dasselbe Buch, das Harry vor
ein paar Monaten an Draco ausgeliehen hatte. Harry hatte schon vor einigen
Wochen aufgehört, sich über seinen möglichen Missbrauch Gedanken zu machen.

\glqq Das -" sagte Professor Quirrell und hustete, es klang nicht ganz richtig.
\glqq Das ist ein faszinierendes Buch ... wenn ich nur geahnt hätte ..." Ein
Lachen, vermischt mit einem weiteren Husten. \glqq Warum habe ich gedacht, dass
die Muggelkünste... mich nicht interessierten würden? Dass sie... mir nichts
nützen? Warum habe ich mir nie die Mühe gemacht, es... experimentell zu
testen... wie du sagen würdest? Für den Fall... dass meine Annahme... falsch
war? Das erscheint mir im Nachhinein einfach nur töricht..."

Harry hatte mehr Schwierigkeiten zu sprechen als Professor Quirrell. Wortlos
griff Harry in seine Tasche und legte ein Tuch auf den Boden, das er entfaltete
und einen kleinen weißen, glatten und runden Kieselstein zum Vorschein brachte.

\glqq Was ist das?\grqq{}, fragte der Verteidigungsprofessor.

\glqq Es ist ein verwandeltes Einhorn." Harry hatte in den Büchern nachgeschaut,
hatte erfahren, dass er sich einem Einhorn ohne Angst nähern konnte, da er zu
jung war, um sexuelle Gedanken zu haben. In denselben Büchern hatte nichts
darüber gestanden, dass Einhörner intelligent waren. Harry hatte bereits
bemerkt, dass jede intelligente magische Spezies zumindest teilweise humanoid
war, vom Meervolk über Zentauren bis hin zu Riesen, von Elfen über Kobolde bis
hin zu Veelas. Alle hatten im Wesentlichen menschenähnliche Emotionen, viele
waren dafür bekannt, sich mit Menschen zu kreuzen. Harry hatte bereits
herausgefunden, dass die Magie keine neue Intelligenz erschafft, sondern nur die
Form der genetisch menschlichen Wesen verändert.

Einhörner waren equinoid, waren nicht einmal teilweise humanoid, sprachen nicht,
benutzten keine Werkzeuge, sie waren mit ziemlicher Sicherheit nur magische
Pferde. Wenn es richtig war, eine Kuh zu essen, um sich einen Tag lang zu
ernähren, dann musste es auch richtig sein, das Blut eines Einhorns zu trinken,
um den Tod wochenlang hinauszuzögern. Man konnte nicht beides haben.

Harry war also in den Verbotenen Wald gegangen, mit seinem Umhang. Er hatte den
Hain der Einhörner abgesucht, bis er sie sah, ein stolzes Geschöpf mit
reinweißem Fell und violettem Haar, mit drei blauen Flecken auf der Flanke.
Harry war hinübergegangen, und die saphirblauen Augen hatten ihn neugierig
angestarrt. Harry hatte mehrmals mit seinen Schuhen die Sequenz 1-2-3 auf den
Boden geklopft. Das Einhorn hatte keine Anzeichen gezeigt, darauf zu reagieren.
Harry hatte hinübergegriffen, ihren Huf in die Hand genommen und mit dem Huf des
Einhorns dieselbe Sequenz getippt. Das Einhorn hatte ihn nur neugierig
angeschaut.

Aber irgendetwas daran, dem Einhorn die mit Schlaftrunk versetzten Zuckerwürfel
zu verfüttern, hatte sich immer noch wie Mord angefühlt.

\emph{Diese Magie gibt ihrer Existenz ein Gewicht an Bedeutung, das kein bloßes Tier besitzen kann ... etwas Unschuldiges zu töten, um sich selbst zu retten, das ist eine sehr schwere Sünde.}

Diese beiden Sätze, von Professor McGonagall und von dem Zentauren, waren Harry
immer wieder durch den Kopf gegangen, als das weiße Einhorn gegähnt, sich auf
den Boden gelegt und seine Augen zum letzten Mal geschlossen hatte. Die
Verwandlung hatte eine Stunde gedauert, und Harrys Augen hatten während der
Arbeit immer wieder getränt. Der Tod des Einhorns war vielleicht noch nicht
eingetreten, aber er würde noch früh genug kommen, und es lag nicht in Harrys
Natur, sich jeglicher Verantwortung entziehen zu wollen. Harry würde einfach
hoffen müssen, dass, wenn man das Einhorn nicht tötete, um sich selbst zu
retten, wenn man es tat, um einem Freund zu helfen, es am Ende akzeptabel sein
würde.

Professor Quirrells Augenbrauen waren in Richtung seines Haaransatzes
geklettert. Seine Stimme war weniger weich, hatte etwas von seiner normalen
Schärfe, als er sagte: \glqq Ich verbiete dir, das noch einmal zu tun."

\glqq Ich habe mich gefragt, ob du das sagen würdest\grqq{}, sagte Harry. Er
schluckte wieder. \glqq Aber dieses Einhorn ist schon so oder so dem Untergang
geweiht, also kannst du es genauso gut nehmen, Professor ..."

\glqq Warum hast du das getan?"

\emph{Wenn der Verteidigungsprofessor das wirklich nicht verstand, war er was Freundschaft anging langsamer im Kopf als jeder andere, den Harry je getroffen hatte.}

\glqq Ich dachte immer, ich könnte nichts tun\grqq{}, sagte Harry. \glqq Ich
wurde müde, es zu denken."

Professor Quirrell schloss seine Augen. Sein Kopf lehnte sich zurück in das
Kissen. \glqq Du hast Glück gehabt\grqq{}, sagte der Verteidigungsprofessor mit
sanfter Stimme, \glqq dass ein Einhorn in verwandelter Gestalt... nicht die
Hogwarts-Zauberwächter ausgelöst hat, denn ein fremdes Wesen... Ich werde es...
außerhalb des Geländes bringen müssen, um es zu nutzen... aber das lässt sich
bewerkstelligen. Ich werde der Krankenschwester sagen, dass ich mir den See
ansehen möchte... Ich muss dich bitten, die Verwandlung aufrechtzuerhalten,
bevor du gehst, und danach sollte es lange genug dauern... und mit meiner
letzten Kraft kann ich den Todesalarm auflösen, der zur Bewachung der Herde
aufgestellt wurde... der, da das Einhorn noch nicht tot, sondern nur verwandelt
ist, noch nicht ausgelöst worden sein wird... Du hattest großes Glück, Mr.
Potter."

Harry nickte. Er begann zu sprechen, hielt dann aber wieder inne. Die Worte
schienen ihm erneut im Hals zu stecken. Du hast bereits den erwarteten Nutzen
berechnet, wenn es funktioniert, und wenn es schief geht. Du hast
Wahrscheinlichkeiten zugeordnet, du hast multipliziert, und dann hast du die
Antwort verworfen und dich auf dein neues Bauchgefühl verlassen, das dasselbe
war.
\emph{Also sag es.}

\glqq Kennst du\grqq{}, sagte Harry unsicher, \glqq überhaupt irgendeine
Möglichkeit, durch die dein Leben gerettet werden könnte?"

Die Augen des Verteidigungsprofessors weiteten sich. \glqq Warum ... fragst du
mich das, Junge?"

\glqq Es gibt ... einen Zauberspruch, von dem ich gehört habe, ein Ritual -"

\glqq Sei still\grqq{}, sagte der Verteidigungsprofessor. Einen Augenblick
später lag eine Schlange auf dem Bett. Selbst die Augen der Schlange waren
stumpf. Sie erhob sich nicht. \glqq \textbf{\emph{Sprich weiter}}\grqq{},
zischte die Schlange, ihre flackernde Zunge war die einzige Bewegung.

\glqq \emph{Es gibt... es gibt ein Ritual, von dem ich von dem Schulmeister
gehört habe, durch das er glaubt, dass der Dunkle Lord weiterleben könnte. Es
wird genannt} -\grqq{} und Harry hielt inne, als ihm klar wurde, dass er das
Wort in Parsel aussprechen konnte. \glqq \emph{Horkrux. Es erfordert einen Tod,
habe ich gehört. Aber wenn du in irgendwie kannst könntest du versuchen, das
Ritual anzupassen, auch unter großem Risiko für den neuen Zauber, so dass es mit
einem anderen Opfer durchgeführt werden kann. Es würde die ganze Welt verändern,
wenn du ssicher bist - obwohl ich nichts über den sZauber weiß - der
sSchulmeister meinte, es würde ein Stück sSeele abreißen, obwohl ich nicht
ssehe, wie das wahr sein könnte }-"

Die Schlange stieß ein zischendes Lachen aus, ein seltsames, scharfes Lachen,
fast hysterisch. \glqq \textbf{\emph{Du erzählst mir von diesem Zauber? Mir?!
Ausgerechnet mir?! Du musst in Zukunft mehr Vorsicht walten lassen, Junge. Aber
das ist nicht wichtig. Ich habe von dem Horkrux-Zauber schon vor langer Zeit
erfahren. Er ist bedeutungslos.\grqq{} }}

\glqq Bedeutungslos?\grqq{} sagte Harry laut und überrascht.

\glqq \textbf{\emph{Wäre von Anfang an ein sinnloser Zauber, wenn sSeele
existieren würde. Ein Stück sSeele reißen? Das ist eine Lüge. Ein Irrweg, um das
wahre Geheimnis zu verbergen. Nur wer nicht an die Wahrheit glaubt, wird weiter
nachdenken und erkennen, wie der Zauber wirklich funktioniert. Erforderlicher
Mord isst überhaupt kein ssOpferritual. Ssplötzlicher Tod erzeugt manchmal
Geissst, wenn Magie ausbricht und sich auf nahe Dinge einbrennt. Der
Horkrux-Zauber kanalisiert den Todesfluch durch den Verursacher, erzeugt den
eigenen Geist anstelle des Opfers und prägt den Geist in ein spezielles Gerät
ein. Ein anderes Opfer hebt das Horkrux-Gerät auf, das Gerät prägt deine
Erinnerungen in sie ein. Aber nur eine Erinnerung an die Zeit, in der das
Horcrux-Gerät gebaut wurde. Siehst du den Makel?}}"

Das brennende Gefühl war wieder in Harrys Kehle. \glqq \emph{Keine
Gedankenkontinuität des} \glqq  es gab kein Schlangenwort für Bewusstsein \glqq
-\emph{ ichs, man würde weiterleben aber die neuen Erinnerungen wären nicht im
Horkrux gesssspeichert. Und nicht wiederhergestellt.}"

\glqq \textbf{\emph{Ja, du ssssiehst es. Und Merlins Interdikt verhindert, dass
mächtige ssssZauber durch ein ssssolches Gerät weitergegeben werden, da es nicht
wirklich lebendig ist. Dunkle Zauberer, die versssuchen, so zurückzukehren, sind
schwächer und werden leicht entlarvt. Keiner hat mit solchen Mitteln lange
überlebt. Die Eigenschaften ändern sich, vermischen sich mit denen des Opfers.
Der Tod ist nicht wirklich besiegt. Das wahre Selbst ist verloren, wie du
sagssst. Nicht gut genug für mich. Ich gebe zu, ich habe es schon lange in
Erwägung gezogen.}}"

Ein Mann lag wieder im Krankenbett. Der Verteidigungsprofessor atmete, dann gab
er einen kläglichen Hustenlaut von sich. \glqq Kannst du mir die Ritualanleitung
für den Zauberspruch geben?\grqq{} sagte Harry nach kurzem Überlegen. \glqq
Vielleicht gibt es eine Möglichkeit, die Fehler zu verbessern, mit genügend
Forschung. Einen Weg, es ethisch zu tun und es funktionieren zu lassen."

\emph{ Zum Beispiel, indem man den Transfer in einen Klonkörper mit leerem Gehirn durchführt, anstatt in ein unschuldiges Opfer, was auch die Genauigkeit des Persönlichkeitstransfers verbessern könnte... obwohl das immer noch die anderen Probleme übrig ließ. }

Professor Quirrell gab einen kurzen Laut von sich, der ein Lachen gewesen sein
könnte. \glqq Weißt du, Junge\grqq{}, flüsterte Professor Quirrell, \glqq ich
hatte gedacht... dir alles beizubringen... die Samen all der Geheimnisse, die
ich kannte... von einem lebenden Geist zum anderen... damit du später, wenn du
die richtigen Bücher gefunden hast, in der Lage wärst, zu verstehen... Ich hätte
mein Wissen an dich weitergegeben, \emph{mein Erbe...} wir hätten angefangen,
sobald du mich gefragt hättest... aber du hast nie gefragt."

Selbst der Kummer, der Harry wie dickes Wasser umgab, wich dem, dem schieren
Ausmaß der verpassten Gelegenheit. \glqq Ich sollte - ? Ich wusste nicht, dass
ich das sollte - !"

Ein weiteres hustendes Glucksen. \glqq Ah ja... der unwissende Muggelgeborene...
im Erbe, wenn nicht im Blut... das bist du. Aber ich denke... es ist besser
so... dass du meinen Weg nicht gehst... es war kein guter Weg, am Ende."

\glqq Es ist noch nicht zu spät, Professor!\grqq{} sagte Harry. Ein Teil von
Harry schrie, dass er selbstsüchtig sei, und dann schrie ein anderer Teil das
nieder; es würde andere Leute geben, die helfen würden.

\glqq Doch, es ist zu spät... und du wirst mich nicht... vom Gegenteil
überzeugen... Ich habe... es mir anders überlegt... wie ich schon sagte... Ich
bin zu voll... von Geheimnissen, die besser unbekannt geblieben wären...
\emph{sieh mich an}."

Harry schaute hin. Er sah ein noch immer faltenloses Gesicht, das alt und
gequält aussah, unter einem Kopf, dem rasch die Haare ausfielen, sogar die
Seiten sahen jetzt schütter aus; Harry sah ein Gesicht, das er immer für scharf
gehalten hatte, das sich jetzt als dünn entpuppte, Muskeln und Fett schwanden
aus dem Gesicht, wie aus den Armen darunter, wie die skelettierte Form von
Bellatrix Black, die er in Askaban gesehen hatte - Harry sah wieder weg.

\glqq Siehst du\grqq{}, flüsterte der Professor. \glqq Ich möchte nicht
klischeehaft klingen... Mr. Potter... aber die Wahrheit ist... die Künste, die
man Dunkel nennt... sind wirklich nicht gut für einen Menschen... am
Ende.\grqq{} Professor Quirrell atmete ein, atmete aus.

Eine Zeit lang herrschte Stille im Krankenzimmer, die beiden wurden nur von dem
kunstvoll verzierten Stein der Wände beobachtet.

\glqq Gibt es noch etwas... Ungesagtes zwischen uns?\grqq{}, fragte Professor
Quirrell. \glqq Ich werde heute nicht sterben... wohlgemerkt... nicht jetzt...
aber ich weiß nicht, wie lange... ich in der Lage sein werde, mich zu
unterhalten."

\glqq Es gibt ...\grqq{}, sagte Harry und schluckte erneut. \glqq Es gibt eine
Menge Dinge, viel zu viele Dinge, aber.... es ist vielleicht das Falsche zu
fragen, aber ich will - diese eine Frage nicht unbeantwortet lassen – Schlange?"

Eine Schlange lag auf dem Bett.

\glqq \emph{Ich habe gelernt, wie der Tötungsfluch funktioniert. Erfordert
wahren Hass, um ihn zu beschwören, nicht viel Hass, aber man muss das Ziel tot
sehen wollen, sagen sie. Im Gefängnis mit den Lebensfressern hast du den
Tötungsfluch auf den Wächter gewirkt - du sagtest, du wolltest ihn nicht tot
sehen - war das gelogen? Hier und jetzt, in dieser Situation, kannst du die
Wahrheit sagen, auch wenn du befürchtest, dass es ein schlechtes Licht auf dich
wirft - es sollte jetzt keine Rolle spielen, Lehrer. Ich will es wissen. Ich
muss es wissen. Ich werde dich nicht im Stich lassen, so oder so.}"

Ein Mann lag auf dem Bett. \glqq Hör gut zu\grqq{}, flüsterte Professor
Quirrell. \glqq Ich werde dir ein Rätsel erzählen ... ein Rätsel eines
gefährlichen Zaubers ... wenn du die Antwort auf dieses Rätsel kennst ... wirst
du auch ... die Antwort auf deine Frage kennen ... hörst du zu?"

Harry nickte.

\glqq Es gibt eine Einschränkung... für den Tötungsfluch. Um ihn einmal zu
sprechen... in einem Kampf... muss man genug hassen... um den anderen tot sehen
zu wollen. Um ihn zweimal zu sprechen... muss man genug hassen, um zweimal zu
töten, ihm mit eigenen Händen die Kehle durchzuschneiden, ihn sterben zu sehen
und es dann noch einmal tun. Nur sehr wenige ... können genug hassen ... um
jemanden ... fünfmal zu töten ... sie würden ... gelangweilt werden."

Der Verteidigungsprofessor atmete einige Male durch, bevor er fortfuhr.

\glqq Aber wenn man sich die Geschichte ansieht... findet man einige dunkle
Zauberer... die den Tötungsfluch... immer und immer wieder aussprechen konnten.
Eine Hexe aus dem neunzehnten Jahrhundert... die sich Dark Evangel nannte... die
Auroren nannten sie A. K. McDowell. Sie konnte den tödlichen Fluch ein Dutzend
Mal in einem Kampf aussprechen. Frag dich, so wie ich mich gefragt habe, was ist
das Geheimnis, das sie kannte? Was ist tödlicher als Hass... und fließt ohne
Grenzen?"

\emph{Eine zweite Stufe des Avada-Kedavra-Zaubers, genau wie beim Patronus-Zauber...}

\glqq Das ist mir egal\grqq{}, antwortete Harry.

Der Verteidigungsprofessor gluckste feucht. \glqq Gut. Lernst schnell. Du siehst
es also..."

Eine Pause der Verwandlung. \glqq \textbf{\emph{Ich habe den Wächter nicht tot
gewünscht. Ich sprach Tötungsfluch, aber nicht mit Hass.}}" Und dann ein Mann.

Harry schluckte schwer. Es war sowohl besser als auch schlechter, als Harry
vermutet hatte; und charakteristisch genug für Professor Quirrell.
\emph{Eine zerrissene Seele, mit Sicherheit; aber Professor Quirrell hatte nie behauptet, ganz zu sein.}

\glqq Sonst noch etwas ... zu sagen?\grqq{}, fragte der Mann im Bett.

\glqq Bist du absolut sicher\grqq{}, sagte Harry, \glqq dass es nichts gibt, von
dem du jemals gehört hast, das dich retten könnte, Professor? In all deinen
Überlieferungen? Das Finden und Vereinigen aller drei Heiligtümer des Todes, ein
uraltes Artefakt, das Merlin hinter einem Rätsel versiegelt hat, das niemand je
entschlüsselt hat? Du hast gesehen, was ich alles kann. Dass ich gut im Lösen
von Rätseln bin. Du weißt, dass ich manchmal Dinge herausfinden kann, die andere
Zauberer nicht können. Ich -" Harrys Stimme brach. \glqq Mir ist dein Leben
lieber als dein Tod, Professor Quirrell."

Es gab eine lange Pause.

\glqq Eine Sache\grqq{}, flüsterte Professor Quirrell. \glqq Eine Sache... die
es tun könnte... oder auch nicht... aber sie zu erlangen... liegt jenseits
deiner Macht, oder meiner..."

\emph{Oh, das war nur die Vorbereitung für eine Nebenquest,} sagte Harrys
innerer Kritiker. Alle anderen Teile schrien, dass dieser Teil die Klappe halten
sollte. \emph{Das Leben funktionierte nicht so.} \emph{Antike Artefakte konnten
gefunden werden, aber nicht in einem Monat, nicht wenn man Hogwarts nicht
verlassen konnte und noch im ersten Jahr war.}

Professor Quirrell atmete tief ein und aus. \glqq Es tut mir leid... das kam...
zu dramatisch rüber. Mach dir keine... großen Hoffnungen... Mr. Potter. Du hast
um etwas gebeten, egal, wie unwahrscheinlich es ist. Es gibt... einen bestimmten
Gegenstand... namens..." Eine Schlange lag auf dem Bett. <p
style=\grqq{}.ext-align:center\grqq{}. \glqq \textbf{\emph{Der Stein der
Weisen}}\grqq{}, zischte die Schlange.</p>

\emph{Wenn es die ganze Zeit über ein massenhaft herstellbares Mittel zur sicheren Unsterblichkeit gegeben hätte und sich niemand darum gekümmert hätte, wäre Harry ausgerastet und hätte alle umgebracht.}

\glqq \emph{Ich habe davon in einem Buch gelesen\grqq{}, zischte Harry. \glqq
Habe festgestellt, dass es ein offensichtlicher Mythos ist. Es gäbe keinen
Grund, warum das sselbe Gerät Unsterblichkeit und endloses Gold bieten sollte.
Es sei denn, jemand erfand nur glückliche Geschichten. Ganz zu schweigen davon,
dass jeder vernünftige Mensch nach Wegen suchen würde, um mehr Steine zu
produzieren. Ich dachte besonders an dich, Herr Lehrer.}"

Ein zischendes, kaltes Lachen. \glqq \textbf{\emph{Gedankengang issst weise,
aber nicht klug genug. Wie bei Horcrux Zauber verbirgt die Absurdität das wahre
Geheimnis. Wahrer Stein isst nicht, was die Legende sagt. Hat andere Macht als
das was die Legende behauptet. Der vermeintliche Schöpfer des Steins war nicht
der, der ihn gemacht hat. Derjenige, der ihn jetzt besitzt, wurde nicht geboren,
um ihn jetzt zu benutzen. Dennoch ist der Stein in Wahrheit ein mächtiges
Heilmittel. Hast du gehört, was man darüber spricht?}}"

\glqq \emph{Nur in dem Buch.}"

\glqq \textbf{\emph{Derjenige, der den Stein besitzt, hat verschiedene Kräfte
von vielen Überlieferungen. Er lehrte den Schulmeister viele Geheimnisse.
Schulmeissster hat nichts über den Besitzer von Stein gesagt, gar nichts über
Macht von Stein? Keine Andeutungen?}}"

\glqq \emph{Nicht, dass ich mich erinnern könnte}\grqq{}, antwortete Harry
ehrlich.

\glqq \textbf{\emph{Ah}}\grqq{}, zischte die Schlange. \glqq \textbf{\emph{nun
gut.}}"

\glqq \emph{Ich könnte ihn fragen -}"

\glqq \textbf{\emph{Nein! Frag ihn nicht, Junge. Er würde Fragen nicht gut
aufnehmen.}}"

\glqq \emph{Aber wenn der Stein nur heilt -}"

\glqq \textbf{\emph{Der Schulmeister glaubt das nicht, würde das nicht glauben.
Zuviele haben den Sssstein gesucht, oder die Überlieferungen des Besitzers.
Kannst nicht fragen. Darfst nicht fragen. Versuche nicht, ssStein selbst zu
erlangen. Ich verbiete es.}}"

Ein Mann lag wieder auf dem Bett. \glqq Ich bin an... meiner Grenze...\grqq{},
sagte Professor Quirrell. \glqq Ich muss... meine Kräfte zurückgewinnen... bevor
ich... in den Wald gehe... mit deinem Geschenk. Geh jetzt... aber halte die
Verwandlung aufrecht... bevor du gehst."

Harry streckte die Hand aus, berührte den weißen Kieselstein, der in dem Tuch
lag, und erneuerte die Verwandlung darauf. \glqq Es sollte es eine Stunde und
dreiundfünfzig Minuten anhalten\grqq{}, sagte Harry.

\glqq Deine Studien ... machen sich gut."

Das war viel länger, als Harrys Verwandlungen zu Beginn des Schuljahres gedauert
hatten. Die Zaubersprüche des zweiten Jahres fielen ihm jetzt leicht, ohne
Anstrengung; was nicht verwunderlich war, da er in weniger als zwei Monaten
zwölf Jahre alt werden würde. Harry hätte sogar einen Gedächtniszauber wirken
können, wenn es für jemanden gut gewesen wäre, jede Erinnerung zu vergessen, die
mit seinem linken Arm zu tun hatte.

Er kletterte die Machtleiter hinauf, langsam, von ganz weit unten.

Der Gedanke war mit einem Potential für Traurigkeit verbunden, ein Gedanke, dass
sich eine Tür öffnete, während sich eine andere schloss; was Harry ebenfalls
verwarf.

Die Tür zum Krankenflügel schloss sich hinter Harry, während der Junge-der-lebte
schnell und zielstrebig ging und dabei seinen Unsichtbarkeitsumhang herausholte.
Vermutlich würde Professor Quirrell bald um Hilfe rufen; und ein älteres
Schülertrio würde den Verteidigungsprofessor an einen ruhigen Ort führen,
vielleicht in den Wald, mit der Ausrede, er könne den See sehen oder so.
Irgendwohin, wo der Verteidigungsprofessor unbemerkt ein Einhorn essen konnte,
nachdem Harrys Verwandlung nachgelassen hatte. Und dann würde Professor Quirrell
gesünder sein, eine Zeit lang. Seine Kraft würde zu ihm zurückkehren, so stark,
wie er immer war, für eine viel kürzere Zeit. Es würde nicht von Dauer sein.

Harry ballte die Fäuste, als er voranging, die Spannung strahlte seine
Armmuskeln hinauf.

\emph{Wenn die Behandlungskur des Verteidigungsprofessors nicht unterbrochen worden wäre, von Harry und den Auroren, die er nach Hogwarts gebracht hatte....}

.. Es war dumm, sich selbst die Schuld zu geben, Harry wusste, dass es dumm war,
und irgendwie tat es sein Gehirn trotzdem. Als würde sein Gehirn suchen,
sorgfältig einen Weg finden und auswählen, dass es seine Schuld war, egal wie
weit es gehen musste. Als wäre es die einzige Möglichkeit, die sein Gehirn
kannte, um zu trauern, wenn alles seine Schuld war.

Ein Trio von Slytherins aus dem siebten Schuljahr ging an Harrys unsichtbarer
Gestalt auf dem Flur vorbei und steuerte auf das Büro der Heiler zu, wo der
Professor wartete und sehr ernst und besorgt aussah.

\emph{Trauerten andere Menschen auch so? Oder war es ihnen auf einer gewissen Ebene egal, wie Professor Quirrell dachte? }

\emph{Es gibt noch eine zweite Ebene des Tötungsfluchs.}

Harrys Gehirn hatte das Rätsel sofort gelöst, in dem Moment, als er es zum
ersten Mal hörte; als wäre das Wissen schon immer in ihm gewesen und hätte nur
darauf gewartet, sich zu offenbaren. Harry hatte einmal irgendwo gelesen, dass
das Gegenteil von Glück nicht Traurigkeit sei, sondern Langeweile; und der Autor
hatte weiter gesagt, dass man, um das Glück im Leben zu finden, sich nicht
fragen sollte, was einen glücklich machen würde, sondern was einen mit
Begeisterung füllen würde. Und mit der gleichen Argumentation war Hass nicht das
wahre Gegenteil von Liebe. Selbst Hass war eine Art von Respekt, den man der
Existenz von jemandem entgegenbringen konnte. Wenn man sich genug um jemanden
sorgte, um sein Sterben dem Leben vorzuziehen, bedeutete das, dass man an ihn
dachte. Es war schon viel früher zur Sprache gekommen, vor dem Prozess, in einem
Gespräch mit Hermine; als sie etwas über die Vorurteile des magischen
Britanniens gesagt hatte, mit beträchtlicher und aktueller Berechtigung. Und
Harry hatte gedacht - aber nicht gesagt -, \emph{dass sie wenigstens nach
Hogwarts gelassen worden war, um bespuckt zu werden.} Nicht wie bestimmte
Menschen in bestimmten Ländern, die, wie es hieß, genauso menschlich waren wie
alle anderen; die angeblich intelligente Wesen waren, die mehr wert waren als
jedes bloße Einhorn. \emph{ Aber die trotzdem nicht in Muggelbritannien leben
durften.}

Zumindest in diesem Punkt hatte kein Muggel das Recht, einem Zauberer in die
Augen zu sehen. Das magische Britannien mag Muggelgeborene diskriminieren, aber
zumindest ließ es sie hinein, damit sie persönlich bespuckt werden konnten.

\emph{Was ist tödlicher als Hass, und fließt ohne Grenzen?}

\glqq Gleichgültigkeit\grqq{}, flüsterte Harry laut, das Geheimnis eines
Zaubers, den er nie würde sprechen können; und schritt weiter in Richtung
Bibliothek, um alles zu lesen, was er finden konnte, überhaupt irgendetwas, über
den \emph{Stein der Weisen}.

