\chapter{Tabubrüche, Nachwirkungen 3, Abstand}

Langsam und schwer ging die lange Treppe, die zum Gipfel von Ravenclaw führte.
Von innen wirkte die Treppe wie eine gerade Steigung, obwohl man von außen sehen
konnte, dass sie logischerweise eine Spirale sein musste. Man konnte die Spitze
des Ravenclaw-Turms nur erreichen, wenn man diesen langen harten Aufstieg ohne
Abkürzungen machte, Steinstufe für Steinstufe; unter Harrys Schuhen hindurch,
gehoben von seinen müden Beinen.

Harry hatte Hermine sicher ins Bett gebracht. Er hatte sich lange genug im
Ravenclaw-Gemeinschaftsraum aufgehalten, um ein paar Unterschriften zu sammeln,
die Hermine später nützlich sein könnten. Nicht viele Schüler hatten
unterschrieben; Zauberer waren nicht darauf trainiert worden, nach den Regeln
der Muggelwissenschaft zu denken, die da lauten: \emph{Halt den Mund oder steck
den Hals raus und mach eine Vorhersage oder hör auf, so zu tun, als ob du an
deine Theorie glaubst. }Die meisten von ihnen hatten nichts Inkongruentes darin
gesehen, dass sie zu nervös waren, eine Vereinbarung zu unterschreiben, die
besagte, dass Hermine sie für den Rest ihres Lebens in der Hand haben würde,
wenn sie sich irrten, während sie nach außen hin überzeugt wirkten, dass sie
schuldig war. Aber allein, dass er die Unterschriften verlangte, würde den Punkt
machen, nachdem die Wahrheit herauskam, falls jemand Hermine jemals wieder einer
dunklen Sache verdächtigte. Zumindest würde sie das nicht zweimal durchmachen
müssen. Danach hatte Harry den Gemeinschaftsraum schnell verlassen, denn es fiel
ihm immer schwerer, sich an all die freundlich-versöhnlichen Gefühle zu
erinnern, die er sich ausgedacht hatte.

Manchmal dachte Harry, dass die tiefste Spaltung in seiner Persönlichkeit nichts
mit seiner dunklen Seite zu tun hatte; vielmehr war es die Kluft zwischen dem
altruistischen und vergebenden, abstrakten, vernünftigen Harry und dem
frustrierten und wütenden Harry im Moment.

Die runde Plattform auf der Spitze des Ravenclaw-Turms war nicht der höchste Ort
in Hogwarts, aber der Ravenclaw-Turm ragte aus dem Hauptteil des Schlosses
heraus, sodass man vom Astronomieturm aus nicht auf die oberste Plattform
hinuntersehen konnte. Ein ruhiger Ort zum Nachdenken, wenn man furchtbar viel
nachzudenken hatte. Ein Ort, an den nur wenige andere Schüler kamen - es gab
einfachere Nischen der Privatsphäre, wenn man Privatsphäre wollte.

Das nächtliche Fackellicht von Hogwarts war weit unten. Die Plattform selbst bot
wenig Hindernisse; die Treppe kam aus einem unbedeckten Spalt im Boden, statt
aus einer aufrechten Tür. Von diesem Ort aus waren die Sterne so gut sichtbar,
wie sie es auf der Erde immer waren. Der Junge legte sich in der Mitte der
Plattform nieder, achtete nicht auf seine Gewänder, die schmutzig werden
könnten, und ließ seinen Kopf auf dem gefliesten Boden ruhen, so dass die
Wirklichkeit, abgesehen von ein paar halb sichtbaren Zinnen aus Stein am Rande
der Vision und einem Splitter der Mondsichel, zum Sternenlicht wurde. Die
Lichtpunkte in dunklem Samt funkelten, schwankten und kehrten zurück, eine
andere Art von Schönheit als ihr gleichmäßiger Glanz in der Stillen Nacht. Harry
starrte abstrakt hinaus, seine Gedanken waren bei anderen Dingen.

\emph{An diesem Tag hat dein Krieg gegen Voldemort begonnen...} Das hatte
Dumbledore gesagt, nach dem Vorfall mit der Rettung von Bellatrix aus Askaban.
Das war ein Fehlalarm gewesen, aber der Satz drückte das Gefühl gut aus. Vor 2
Nächten hatte sein Krieg begonnen, und Harry wusste nicht, mit wem. Dumbledore
dachte, es sei Lord Voldemort, der von den Toten zurückgekehrt war und seinen
ersten Zug gegen den Jungen machte, der ihn beim letzten Mal besiegt hatte.
Professor Quirrell hatte Draco mit Erkennungszaubern versehen, weil er
befürchtete, dass der verrückte Schulleiter von Hogwarts versuchen würde, Harry
den Tod von Lucius' Sohn anzuhängen. Oder Professor Quirrell hatte die ganze
Sache eingefädelt, und so hatte er gewusst, wo er Draco finden würde. Für
Severus Snape war der Hogwarts-Verteidigungsprofessor ein offensichtlicher
Verdächtiger, sogar der offensichtlichste Verdächtige. Und Severus Snape selbst
war vielleicht oder vielleicht auch nicht mal im Entferntesten vertrauenswürdig.
Jemand hatte Harry den Krieg erklärt, der erste Schlag sollte Draco und Hermine
gleichermaßen ausschalten, und nur ganz knapp hatte Harry Hermine gerettet. Man
konnte es nicht als Sieg bezeichnen. Draco war aus Hogwarts entfernt worden, und
wenn das nicht der Tod war, war nicht klar, wie man das wieder rückgängig machen
konnte, oder in welcher Verfassung Draco sein würde, wenn er zurückkam. Das Land
des magischen Britanniens hielt Hermine nun für eine versuchte Mörderin, was sie
vielleicht dazu bringen würde, das Vernünftige zu tun und zu gehen. Harry hatte
sein gesamtes Vermögen geopfert, um seinen Verlust ungeschehen zu machen, und
diese Karte konnte nur einmal gespielt werden. Eine unbekannte Macht hatte ihn
getroffen, und wenn dieser Schlag auch teilweise abgelenkt worden war, so hatte
er doch sehr hart getroffen. Wenigstens hatte seine dunkle Seite nichts von ihm
verlangt als Gegenleistung für die Rettung von Hermine. Vielleicht, weil seine
dunkle Seite keine imaginäre Stimme war wie die von Hufflepuff; Harry konnte
sich vorstellen, dass sein Hufflepuff-Teil andere Dinge von ihm wollte, aber
seine dunkle Seite war nicht so. Seine \glqq dunkle Seite\grqq{} war, soweit
Harry das beurteilen konnte, eine andere Art, die Harry manchmal war. Im Moment
war Harry nicht wütend; und der Versuch zu fragen, was der \glqq dunkle
Harry\grqq{} wollte, war ein Telefon, das unbeantwortet klingelte. Der Gedanke
kam ihm sogar ein wenig seltsam vor; \emph{konnte man einer anderen Art, wie man
manchmal war, etwas zu verdanken haben? }

Harry starrte hinauf zu den zufälligen Sternen, den verstreuten, funkelnden
Lichtern, die das menschliche Gehirn nicht anders konnte, als sie zu imaginären
Konstellationen zusammenzufügen.

Und dann war da noch dieses Versprechen, das Harry geschworen hatte. Draco
sollte Harry helfen, das Haus Slytherin zu retten. Und Harry sollte sich
denjenigen zum Feind machen, von dem er nach bestem Wissen und Gewissen glaubte,
dass er Narcissa Malfoy getötet hatte. Wenn Narcissa sich nicht selbst die Hände
schmutzig gemacht hatte, wenn sie tatsächlich lebendig verbrannt worden war,
wenn der Mörder nicht ausgetrickst worden war - das waren die einzigen
Bedingungen, an die sich Harry erinnern konnte. Wahrscheinlich hätte er sie
aufschreiben sollen, oder besser noch, er hätte gar nicht erst ein Versprechen
mit so vielen Vorbehalten abgeben sollen. Es gab plausible Auswege, für die Art
von Person, die sich einen Ausweg rationalisieren ließ. Dumbledore hatte nicht
wirklich gestanden. Er hat nicht offen gesagt, dass er es getan hat. Es gab
plausible Gründe für einen schuldigen Dumbledore, sich so zu verhalten. Aber es
war auch das, was man erwarten würde, wenn jemand anderes Narcissa verbrannt
hätte und Dumbledore die Schuld auf sich genommen hätte.

Harry schüttelte den Kopf und strich sich erst eine Seite seines Haares, dann
die andere gegen den Steinboden. Es gab immer noch einen letzten Ausweg, Draco
konnte ihn jederzeit aus dem Schwur entlassen. Er konnte Draco zumindest die
Situation schildern und mit ihm über Optionen sprechen, wenn sie sich wieder
trafen. Es schien keine sehr wahrscheinliche Aussicht auf Entlassung zu sein -
aber die Vorstellung, etwas ehrlich zu besprechen, war genug, um den Teil in ihm
zu befriedigen, der die Einhaltung des Eides forderte. Selbst wenn es nur einen
Aufschub bedeutete, war es besser, als sich einen guten Mann zum Feind zu
machen.

\emph{Aber ist Dumbledore ein guter Mann?} fragte die Stimme von Hufflepuff.
\emph{Wenn Dumbledore jemanden bei lebendigem Leibe verbrannte - war es nicht
so, dass gute Menschen zwar töten dürfen, aber niemals Leid dabei zufügen
dürfen? }

\emph{Vielleicht hat er sie auf der Stelle getötet,} sagte Slytherin, \emph{und
dann Lucius über den Teil mit dem Verbrennen bei lebendigem Leib angelogen. }

\emph{Aber... wenn es eine Möglichkeit gäbe, dass die Todesser magisch nachweisen könnten, wie Narcissa gestorben ist....und wenn man bei einer Lüge erwischt wird, die Familien auf der Lichtseite gefährdet... }

\emph{Sei vorsichtig, was wir geschickt rationalisieren, warnte} Gryffindor.

\emph{Man muss mit Auswirkungen auf den Ruf rechnen, wie andere Leute einen
behandeln,} sagte Hufflepuff. \emph{Wenn man entscheidet, dass es einen
ausreichenden Grund gibt, eine Frau bei lebendigem Leib zu verbrennen, ist eine
der vorhersehbaren Nebenwirkungen, dass gute Leute entscheiden, dass man die
Grenze überschritten hat und gestoppt werden muss. Dumbledore hätte damit
rechnen müssen. Er hat kein Recht, sich zu beschweren. }

\emph{Oder er erwartet, dass wir schlauer sind, }sagte Slytherin. \emph{Können
wir jetzt, wo wir so viel von der Wahrheit wissen - egal, wie genau die ganze
Geschichte ist -, wirklich glauben, dass Dumbledore ein schrecklicher,
furchtbarer Mensch ist, der unser Feind sein sollte? Mitten in einem
schrecklichen, blutigen Krieg hat Dumbledore einen }\emph{feindlichen Zivilisten
in Brand gesteckt? Das ist nur nach den Maßstäben von Comics schlecht, nicht
nach irgendeinem realistischen historischen Standard. }

Harry starrte in den Nachthimmel und erinnerte sich an die Geschichte. Im echten
Leben, in echten Kriegen... Während des Zweiten Weltkriegs hatte es ein Projekt
zur Sabotage des Nazi-Atomwaffenprogramms gegeben. Jahre zuvor hatte Leo
Szilard, der als erster die Möglichkeit einer Spaltungskettenreaktion erkannt
hatte, Fermi davon überzeugt, die Entdeckung, dass gereinigter Graphit ein
billiger und effektiver Neutronenmoderator war, nicht zu veröffentlichen. Fermi
hatte veröffentlichen wollen, um des großen internationalen Projekts der
Wissenschaft willen, das über dem Nationalismus stand. Aber Szilard hatte Rabi
überredet, und Fermi hatte sich an das Mehrheitsvotum ihrer winzigen
Drei-Personen-Verschwörung gehalten. Und so war Jahre später der einzige
Neutronenmoderator, den die Nazis kannten, Deuterium. Die einzige
Deuteriumquelle unter der Kontrolle der Nazis war eine erbeutete Anlage im
besetzten Norwegen gewesen, die durch Bomben und Sabotage ausgeschaltet worden
war, was insgesamt vierundzwanzig zivile Todesopfer forderte. Die Nazis hatten
versucht, das bereits aufbereitete Deuterium an Bord einer zivilen norwegischen
Fähre, der SS Hydro, nach Deutschland zu verschiffen. Knut Haukelid und seine
Helfer waren vom Nachtwächter der zivilen Fähre entdeckt worden, als sie sich an
Bord schlichen, um sie zu sabotieren. Haukelid hatte dem Nachtwächter gesagt,
dass sie vor der Gestapo fliehen würden, und der Nachtwächter hatte sie gehen
lassen. Haukelid hatte erwogen, den Nachtwächter zu warnen, aber das hätte die
Mission gefährdet, also hatte Haukelid ihm nur die Hand geschüttelt. Und das
zivile Schiff war im tiefsten Teil des Sees gesunken, mit acht toten Deutschen,
sieben toten Besatzungsmitgliedern und drei toten zivilen Schaulustigen. Einige
der norwegischen Retter des Schiffes waren der Meinung gewesen, dass die
anwesenden deutschen Soldaten dem Ertrinken überlassen werden sollten, aber
diese Ansicht hatte sich nicht durchgesetzt, und die deutschen Überlebenden
waren gerettet worden. Und das war das Ende des Nazi-Atomwaffenprogramms
gewesen.

\emph{Was bedeutete, dass Knut Haukelid unschuldige Menschen getötet hatte. Einer davon, der Nachtwächter des Schiffes, war ein guter Mensch gewesen. Jemand, der Haukelid unter Gefahr für sich selbst geholfen hatte, aus der Güte seines Herzens, aus den höchsten moralischen Gründen, und der im Gegenzug zum Ertrinken geschickt wurde.}

Im Nachhinein, im kalten Licht der Geschichte, hatte es so ausgesehen, als wären
die Nazis doch nie nahe daran gewesen, Atomwaffen zu bekommen. Und Harry hatte
nie etwas gelesen, was darauf hindeutete, dass Haukelid falsch gehandelt hatte.
Das war der Krieg im wirklichen Leben. Im Hinblick auf den Gesamtschaden und
darauf, wen es getroffen hatte, war das, was Haukelid getan hatte, wesentlich
schlimmer als das, was Dumbledore vielleicht Narcissa Malfoy angetan hatte, oder
was Dumbledore möglicherweise getan hatte, um die Prophezeiung an Lord Voldemort
durchsickern zu lassen, damit er Harrys Eltern angriff. Wenn Haukelid ein
Comic-Superheld gewesen wäre, hätte er irgendwie alle Zivilisten von der Fähre
geholt, er hätte die deutschen Soldaten eher direkt angegriffen, als einen
einzigen unschuldigen Menschen sterben zu lassen, aber Knut Haukelid war kein
Superheld gewesen. Und Albus Dumbledore war es auch nicht gewesen.

Harry schloss die Augen und schluckte ein paar Mal schwer gegen das plötzliche
Erstickungsgefühl an. Es war schlagartig sehr klar, dass, während Harry
herumlief und versuchte, die Ideale der Aufklärung zu leben, Dumbledore
derjenige war, der tatsächlich in einem Krieg gekämpft hatte. Gewaltfreie Ideale
waren billig zu halten, wenn man als Wissenschaftler in der Protego-Blase lebte,
die von den Polizisten und Soldaten aufgespannt wurde, deren Handlungen man in
Frage stellen durfte. Albus Dumbledore schien mit Idealen begonnen zu haben, die
mindestens so stark waren wie Harrys eigene, wenn nicht sogar stärker; und
Dumbledore hatte seinen Krieg nicht überstanden, ohne Feinde zu töten und
Freunde zu opfern.

\emph{Bist du so viel besser als Haukelid und Dumbledore, Harry Potter, dass du in der Lage sein wirst, ohne ein einziges Opfer zu kämpfen? Selbst in der Welt der Comics ist ein Superheld wie Batman nur deshalb erfolgreich, weil die Comic-Leser es nur bemerken, wenn wichtige, namentlich genannte Charaktere sterben, und nicht, wenn der Joker irgendeinen zufälligen, namenlosen Passanten erschießt, um seine Schurkerei zu zeigen. Batman ist ein Mörder, nicht weniger als der Joker, bei all den Leben, die der Joker genommen hat, die Batman hätte retten können, indem er ihn tötet. Das ist es, was der Mann namens Alastor versuchte, Dumbledore zu sagen, und im Nachhinein bedauerte Dumbledore, dass er so lange gebraucht hatte, seine Meinung zu ändern. Willst du wirklich versuchen, dem Weg des Superhelden zu folgen und nie ein einziges Stück zu opfern oder einen einzigen Feind zu töten?}

Erschöpft wandte Harry seine Aufmerksamkeit für einen Moment von dem Dilemma ab,
öffnete die Augen wieder und betrachtete die Halbkugel der Nacht, die ihm keine
Entscheidungen abverlangte. Am Rande seines Blickfeldes die blassweiße Sichel
des Mondes, dessen Licht vor eineinviertel Sekunden, in etwa 375.000 Kilometern
Entfernung im Raum der Gleichzeitigkeit, die Erde verlassen hatte. Darüber und
zur Seite hin, Polaris, der Nordstern; der erste Stern, den Harry am Himmel zu
identifizieren gelernt hatte, indem er dem Rand des Großen Wagens folgte. Das
war eigentlich ein Fünf-Sterne-System mit einem brillanten zentralen Überriesen,
434 Lichtjahre von der Erde entfernt. Es war der erste \glqq Stern\grqq{},
dessen Namen Harry von seinem Vater gelernt hatte, und zwar vor so langer Zeit,
dass er nicht einmal erraten konnte, wie alt er gewesen war. Der trübe Nebel,
der die Milchstraße war, so viele Milliarden entfernte Sterne, dass sie zu einem
undeutlichen Fluss wurden, die Ebene einer Galaxie, die sich über 100.000
Lichtjahre erstreckte. Wenn Harry ein Gefühl des Staunens empfunden hatte, als
man ihm das zum ersten Mal gesagt hatte, dann war er zu jung gewesen, um sich
jetzt, über ein paar Jahre hinweg, an dieses erste Mal zu erinnern. Im Zentrum
des Sternbildes Andromeda, dem Stern Andromeda, der eigentlich die
Andromeda-Galaxie war. Die der Milchstraße nächstgelegene Galaxie, 2,4
Milliarden Lichtjahre entfernt, mit schätzungsweise einer Billion Sternen.
Zahlen wie diese ließen die \glqq Unendlichkeit\grqq{} im Vergleich dazu
verblassen, denn die \glqq Unendlichkeit\grqq{} war einfach funktionslos und
leer. Der Gedanke, dass die Sterne \glqq unendlich\grqq{} weit entfernt waren,
war viel weniger beängstigend als der Versuch, herauszufinden, was 2,4
Milliarden Lichtjahre in Metern bedeuteten. 2,4 Milliarden Lichtjahre mal 31
Millionen Sekunden in einem Jahr mal ein Photon, das sich mit 300.000.000 Metern
pro Sekunde bewegt...

Es war ein seltsamer Gedanke, dass solche Entfernungen nicht unerreichbar weit
weg sein könnten. Magie war im Universum existent, Dinge wie Zeitumkehrer und
fliegende Besen. Hatte jemals ein Zauberer versucht, die Geschwindigkeit eines
Portschlüssels oder eines Phönix zu messen? Und das menschliche Verständnis von
Magie konnte unmöglich auch nur in die Nähe der zugrundeliegenden Gesetze
kommen. Was könnte man mit Magie machen, wenn man sie wirklich verstehen würde?
Vor einem Jahr war Papa zu einer Konferenz an die Australian National University
in Canberra gefahren, wo er ein eingeladener Redner war, und er hatte Mama und
Harry mitgenommen. Und sie hatten alle zusammen das National Museum of Australia
besucht, denn es hatte sich herausgestellt, dass es in Canberra im Grunde nichts
anderes zu tun gab. In den Glasvitrinen waren Steinschleudern zu sehen, die von
den australischen Ureinwohnern hergestellt worden waren - sie sahen aus wie
riesige hölzerne Schuhlöffel, aber sie waren geglättet und geschnitzt und in
mühevoller Kleinarbeit verziert worden. In den 40.000 Jahren, seit der
anatomisch moderne Mensch aus Asien nach Australien eingewandert war, hatte
niemand Pfeil und Bogen erfunden. Es machte einem wirklich bewusst, wie wenig
naheliegend die Idee des Fortschritts war. Warum sollte man überhaupt an
Erfindungen als etwas Wichtiges denken, wenn alle Heldengeschichten deiner
Geschichte von großen Kriegern und Verteidigern handelten und nicht von Thomas
Edison? Wie hätte irgendjemand ahnen können, während er in mühevoller
Kleinarbeit eine Steinschleuderer schnitzte, dass die Menschen eines Tages
Raketenschiffe und Atomenergie erfinden würden? Hätten Sie in den Himmel blicken
können, auf das strahlende Licht der Sonne, und daraus schließen können, dass
das Universum größere Kraftquellen enthält als bloßes Feuer? Hätten Sie erkannt,
dass, wenn die fundamentalen physikalischen Gesetze es zulassen, die Menschen
eines Tages die gleichen Energien anzapfen würden wie die Sonne? Auch wenn
nichts, was man sich mit Steinschleudern oder geflochtenen Beuteln vorstellen
konnte - kein Muster des Laufens durch die Savanne und nichts, was man durch das
Jagen von Tieren erlangen konnte - das auch nur in der Vorstellung erreichen
würde? Es war ja nicht so, dass die modernen Muggel in die Nähe der Grenzen
dessen gekommen wären, was laut Muggelphysik möglich war. Und doch lebten die
meisten Muggel wie Jäger und Sammler, die konzeptionell an ihre Steinschleudern
gebunden waren, in einer Welt, die durch die Grenzen dessen definiert war, was
man mit Autos und Telefonen machen konnte. Auch wenn die Muggelphysik explizit
Möglichkeiten wie molekulare Nanotechnologie oder den Penrose-Prozess zur
Energiegewinnung aus schwarzen Löchern zuließ, legten die meisten Menschen das
in der gleichen Abteilung ihres Gehirns ab, in der auch Märchen und
Geschichtsbücher gespeichert sind, weit weg von ihrer persönlichen Realität: Vor
langer Zeit und weit weg, vor sehr langer Zeit.

Kein Wunder also, dass die Welt der Zauberer in einem konzeptionellen Universum
lebte, das nicht durch fundamentale Gesetze der Magie begrenzt war, die niemand
kannte, sondern nur durch die Oberflächenregeln der bekannten Flüche und
Zaubersprüche. Man konnte die Art und Weise, wie Magie heutzutage praktiziert
wurde, nicht beobachten, ohne an das National Museum of Australia erinnert zu
werden, sobald man erkannte, was man sah. Selbst wenn Harrys erste Vermutung
falsch gewesen war, war es so oder so immer noch unvorstellbar, dass die
fundamentalen Gesetze des Universums einen Sonderfall für menschliche Lippen
enthielten, die den Ausdruck \emph{\glqq Wingardium Leviosa\grqq{} } formten.
Und doch reichte selbst dieses geringe Verständnis von Magie aus, um Dinge zu
tun, von denen die Muggelphysik sagte, dass sie für immer unmöglich sein
sollten: der Zeitumkehrer, das von Aguamenti aus dem Nichts gezauberte Wasser.
Was waren die ultimativen Möglichkeiten von Erfindungen, wenn die
zugrundeliegenden Gesetze des Universums einem Elfjährigen mit einem Stock
erlaubten, fast jede Beschränkung der Muggelversion der Physik zu verletzen? Wie
ein Jäger und Sammler, der versucht, zur Sonne hinaufzuschauen und zu erraten,
dass das Universum so geformt sein muss, dass Kernenergie möglich ist... Da
fragt man sich, ob zwanzigtausend Millionen Millionen Millionen Kilometer nicht
doch zu weit weg sind.

Es gab einen Schritt jenseits des abstrakt denkenden Harrys, den er gehen
konnte, wenn er genug Zeit hatte, sich zu sammeln und die richtige Umgebung
vorfand. Wenn man zu den Sternen hinaufschauten, konnte man versuchen, sich
vorzustellen, was die fernen Nachfahren der Menschheit über dein Dilemma denken
würden - in hundert Millionen Jahren, wenn sich die Sterne durch große
galaktische Bewegungen in völlig neue Positionen gedreht haben würden, jede
Konstellation verstreut. Es war ein elementares Theorem der Wahrscheinlichkeit,
dass man, wenn man wusste, wie die Antwort nach der Aktualisierung auf
zukünftige Beweise lauten würde, diese Antwort jetzt annehmen sollte. Wenn man
sein Ziel kannte, war man bereits dort. Und in Analogie, wenn auch nicht ganz
nach dem Theorem, sollte man, wenn man erraten kann, was die Nachkommen der
Menschheit über etwas denken würden, dies als seine eigene beste Vermutung
annehmen. Von diesem Standpunkt aus erschien die Idee, zwei Drittel des
Zaubergamot auszulöschen, viel weniger verlockend als noch ein paar Stunden
zuvor. Selbst wenn man es tun musste, selbst wenn man sicher wusste, dass es das
Beste für das magische Britannien wäre und dass die gesamte Geschichte der Zeit
schlechter aussehen würde, wenn man es nicht täte... selbst als Notwendigkeit
wäre der Tod von empfindungsfähigen Wesen immer noch eine Tragödie. Ein weiteres
Element der Leiden der Erde; der ältesten Erde, von der alles ausging, vor
langer Zeit und weit weg, vor sehr langer Zeit.

\emph{Er ist nicht wie Grindelwald. Es ist nichts Menschliches mehr in ihm. Ihn
musst du vernichten. Spare dir deine Wut dafür auf, und nur dafür} - Harry
schüttelte leicht den Kopf, wobei er die Sterne in seiner Vision ein wenig
kippte, während er auf dem Steinboden lag und nach oben und nach außen und in
der Zeit nach vorne blickte. Selbst wenn Dumbledore recht hatte und der wahre
Feind völlig verrückt und böse war... in hundert Millionen Jahren würde die
organische Lebensform, die als Lord Voldemort bekannt war, wahrscheinlich nicht
viel anders aussehen als all die anderen verwirrten Kinder der alten Erde. Was
auch immer Lord Voldemort mit sich selbst angestellt hatte, welche dunklen
Rituale auch immer auf rein menschlicher Ebene so schrecklich unwiderruflich
erschienen, es würde mit der Technologie von hundert Millionen Jahren nicht
unheilbar sein. Ihn zu töten, selbst wenn man es tun müsste, um das Leben
anderer zu retten, wäre nur ein weiterer Tod für zukünftige empfindungsfähige
Wesen, über den man traurig sein könnte. Wie konnte man zu den Sternen aufsehen
und etwas anderes glauben?

Harry starrte hinauf zu den funkelnden Lichtern der Ewigkeit und fragte sich,
was die Kinder der Kinder der Kinder von dem denken würden, was Dumbledore
vielleicht mit Narcissa getan hatte. Aber selbst wenn man versuchte, die Frage
so zu formulieren, zu fragen, was die Nachkommen der Menschheit denken würden,
stützte sie sich immer noch nur auf das eigene Wissen, nicht auf das der
anderen. Die Antwort kam immer noch aus deinem Inneren, und sie könnte immer
noch falsch sein. Wenn man selbst die hundertste Dezimalstelle von Pi nicht
kannte, dann wusste man auch nicht, wie die Kinder der Kinder der Kinder sie
berechnen würden, so trivial die Tatsache auch war.

Langsam - er lag schon länger da und betrachtete die Sterne, als er geplant
hatte - setzte sich Harry vom Boden auf. Er drückte sich auf die Füße, die
Muskeln protestierten, und ging zum Rand der Steinplattform auf der Höhe des
Ravenclaw-Turms hinüber. Die steinernen Zinnen, die den Rand des Turms umgaben,
waren nicht hoch, nicht hoch genug, um sicher zu sein. Sie waren eindeutig eher
Markierungen als Geländer. Harry ging nicht zu nahe an den Rand heran; es hatte
keinen Sinn, ein Risiko einzugehen. Als er auf das Hogwarts-Gelände
hinunterblickte, verspürte er vorhersehbar ein Schwindelgefühl, das wackelige
Leiden, das man Vertigo nennt. Sein Gehirn war anscheinend alarmiert, weil der
Boden unter ihm so weit entfernt war. Er mochte ganze 50 Meter entfernt sein.
Die Lektion, so schien es, war, dass Dinge unglaublich nah sein mussten, bevor
das Gehirn sie gut genug erfassen konnte, um Angst zu empfinden. Es war ein
seltenes Gehirn, das für irgendetwas ein starkes Gefühl haben konnte, wenn es
nicht räumlich, zeitlich, in der Nähe, in Reichweite war...

Früher hatte Harry sich vorgestellt, nach Askaban zu gehen, würde die Planung
und Kooperation eines erwachsenen Verbündeten erfordern. Portschlüssel, Besen,
Unsichtbarkeitszauber. Eine Möglichkeit, unbemerkt in die unteren Etagen zu
gelangen, um in die zentrale Grube zu gelangen, wo die Schatten des Todes
warteten. Und das hatte gereicht, um die Aussicht wegzuschieben, in die Zukunft,
sicher getrennt vom Jetzt. Er hatte bis heute nicht erkannt, dass es so einfach
sein könnte, Fawkes zu finden und dem Phönix zu sagen, dass es an der Zeit war.
Erinnerungen stiegen wieder auf, Erinnerungen, die Harry nie lange vergessen
konnte. Obwohl die Steine unter seinen Füßen nicht glatt wie Metall waren,
obwohl sich der mondbeschienene Himmel um ihn herum erstreckte, war es irgendwie
sehr einfach, sich vorzustellen, dass er in einem langen Metallkorridor gefangen
war, der von schummrigem orangefarbenem Licht erhellt wurde.

Die Nacht war still, still genug, dass die Erinnerungen deutlich hörbar waren.

<p style=\grqq{}.ext-align:center\grqq{}.\emph{Nein, ich habe es nicht so
gemeint, bitte stirb nicht! }</p> <p
style=\grqq{}.ext-align:center\grqq{}.\emph{Nein, ich wollte das nicht, bitte
stirb nicht! }</p> <p style=\grqq{}.ext-align:center\grqq{}.\emph{Nimm es mir
nicht weg, bitte nicht - }</p>

Die Welt verschwamm, und Harry wischte sich mit dem Ärmel über die Augen. Wenn
Hermine diejenige hinter dieser Tür gewesen wäre - wenn man Hermine nach Askaban
gesteckt hätte, hätte Harry den Phönix gerufen und wäre dorthin gegangen und
hätte jeden einzelnen Dementor weggebrannt, und es hätte keinen Unterschied
gemacht, wie verrückt es war oder was er sonst mit seinem Leben hätte anfangen
wollen. Das war einfach - das war - das war einfach so. Und die Frau, die hinter
dieser Tür war - gab es da nicht irgendwo jemanden, für den auch sie wertvoll
war? War es nicht nur Harrys Distanz zu ihrem Leben, die sein Gehirn davon
abhielt, sich nach Askaban treiben zu lassen, um sie um jeden Preis zu retten?
Was hätte es gebraucht, um ihn zu zwingen? Hätte er ihr Gesicht kennen müssen?
Ihren Namen? Ihre Lieblingsfarbe? Wäre er nach Askaban getrieben worden, um
Tracey Davis zu retten? Wäre er dorthin gezwungen worden, um Professor
McGonagall zu retten? Mum und Dad - das stand gar nicht zur Debatte. Und diese
Frau hatte gesagt, sie sei die Mutter von jemandem. Wie viele Menschen hatten
sich die Macht gewünscht, Askaban zu durchbrechen? Wie viele Gefangene von
Askaban träumten nachts von einer solch wundersamen Rettung?\emph{ Keiner. Das
ist ein glücklicher Gedanke.}

Vielleicht sollte er Askaban stürmen. Er musste nur Fawkes finden und ihm sagen,
dass es an der Zeit war. Das Zentrum der Dementorengrube visualisieren, wie er
es vom Besenstiel aus gesehen hatte, und sich vom Phönix dorthin bringen lassen.
Den Wahren Patronus-Zauber aus nächster Nähe wirken und zum Teufel mit dem, was
danach kam. Alles, was er tun musste, war, Fawkes zu finden. Es konnte so
einfach sein, wie an die Flamme zu denken, den Feuervogel in seinem Herzen zu
rufen -

\emph{Ein Stern blitzte in der Nacht auf.}
Während Harrys Augen reflexartig aufsprangen, trainiert auf Meteoritenschauer,
war ein anderer Teil von ihm überrascht, dass das astronomische Phänomen immer
noch da war; ein schwacher Stern, dessen Helligkeit langsam sichtbar zunahm. Es
gab einen erschrockenen Moment, in dem Harry sich fragte, ob er nicht einen
Meteor sah, sondern eine Nova oder Supernova - konnte man sie so heller werden
sehen? Sollte das erste Stadium einer Nova diese gelb-orange Farbe haben? Dann
bewegte sich der neue Stern wieder, und er schien zu wachsen und auch heller zu
werden. Er sah plötzlich näher aus, nicht mehr so weit weg, dass die Entfernung
keine Rolle mehr spielte. Wie das, was man für einen Stern hielt, sich als
Flugzeug entpuppte, eine leuchtende Form, deren Umrisse man tatsächlich sehen
konnte... ...nein, nicht ein Flugzeug...

Die Erkenntnis schien sich in einer Welle von Kribbeln in Harrys Brust
auszubreiten, Schweiß bereitete sich darauf vor auszubrechen. ...ein Vogel. Ein
durchdringender Schrei spaltete die Nacht und hallte von den Dächern von
Hogwarts wider. Die sich nähernde Kreatur verströmte im Flug Feuer und warf
goldene Flammen wie Funken aus seinen Federn, während die mächtigen Flügel
schlugen und wieder schlugen. Selbst als es sich in einem großen Bogen erhob, um
ein paar Schritte von Harry entfernt zu schweben, selbst als die Flammen, die
seinen Weg umgaben, kleiner wurden, schien die Kreatur nicht schwächer, nicht
weniger hell; als ob eine unsichtbare Sonne auf sie schien und sie erleuchtete.
Große glänzende Flügel, rot wie ein Sonnenuntergang, und Augen wie glühende
Perlen, die mit goldenem Feuer und Entschlossenheit loderten. Der Schnabel des
Phönix öffnete sich und stieß ein lautes Krächzen aus, das Harry so verstand,
als wäre es ein gesprochenes Wort:

<p style=\grqq{}.ext-align:center\grqq{}.\textbf{KOMM!}</p>

Ohne es zu merken, stolperte der Junge vom Rand des Daches zurück, die Augen
immer noch auf den Phönix gerichtet, sein ganzer Körper zitterte und spannte
sich an, die Fäuste ballten sich an seiner Seite und lösten sich wieder; er wich
zurück, wich weg. Der Phönix krächzte wieder, ein verzweifelter, flehender Laut.
Diesmal waren es keine Worte, aber es waren Gefühle, ein Echo von allem, was
Harry jemals über Askaban gefühlt hatte, und jede Versuchung zu handeln, einfach
etwas dagegen zu tun, das verzweifelte Bedürfnis, jetzt etwas zu tun und nicht
länger zu zögern, alles ausgedrückt durch den Schrei eines Vogels.

<p style=\grqq{}.ext-align:center\grqq{}.\textbf{Los geht's. Es ist an der Zeit.
}</p>

Die Stimme, die sprach, kam aus Harrys Innerem, nicht aus dem Phönix; aus so
tiefem Inneren, dass man ihr keinen eigenen Namen wie '\emph{Gryffindor}' geben
konnte. Alles, was er zu tun hatte, war, nach vorne zu treten und die Krallen
des Phönix zu berühren, und sie würde ihn dorthin bringen, wo er sein musste, wo
er immer dachte, dass er sein sollte, hinunter in die zentrale Grube von
Askaban. Harry konnte das Bild in seinem Geist sehen, das mit unerträglicher
Klarheit leuchtete, das Bild von sich selbst, wie er plötzlich vor freudiger
Erleichterung lächelte, als er alle seine Ängste abwarf und sich entschied -

\glqq Aber ich -\grqq{} Harry flüsterte, ohne zu wissen, was er sagte. Harry hob
seine zitternden Hände, um sich über die Augen zu wischen, aus denen Tränen
getreten waren, während der Phönix mit großen Flügelschlägen vor ihm schwebte.
\glqq Aber ich - es gibt auch noch andere Menschen, die ich retten muss, andere
Dinge, die ich tun muss -\grqq{}

Der Feuervogel stieß einen durchdringenden Schrei aus, und der Junge wich zurück
wie vor einem Schlag. Es war kein Befehl, es war kein Einwand, es war das Wissen
- Die Korridore wurden von schummrigem orangefarbenem Licht erhellt. Es fühlte
sich an wie ein sich zusammenziehender Zwang in Harrys Brust, der Wunsch, es
einfach zu tun und es hinter sich zu bringen. Er könnte sterben, aber wenn er
nicht starb, konnte er sich wieder rein fühlen. Prinzipien haben, die mehr waren
als Ausreden für Untätigkeit. Es war sein Leben. Er konnte es ausgeben, wenn er
wollte. Er konnte es jederzeit tun, wenn er wollte... ...wenn er ein guter
Mensch war.

Der Junge stand auf dem Dach, seine eigenen Augen waren mit zwei Feuerpunkten
verbunden. Die Sterne hatten vielleicht Zeit, ihre Konstellation zu verändern,
während er da stand und sich mit der Entscheidung quälte... die sich nicht...
ändern würde. Die Augen des Jungen flackerten einmal zu den Sternen über ihm,
und dann sah er den Phönix an.

\glqq Noch nicht\grqq{}, sagte der Junge mit einer kaum hörbaren Stimme. \glqq
Noch nicht. Ich habe noch zu viel zu tun. Bitte komm später wieder, wenn ich
andere gefunden habe, die den Wahren Patronus wirken können - in sechs Monaten
vielleicht -\grqq{}

Ohne ein Wort, ohne ein Geräusch umgab eine Feuerkugel die Gestalt des Vogels,
knisternd und lodernd mit weißen und karminroten Adern, als wolle sie verzehren,
was darin lag; und als das Feuer sich in grauen Rauch auflöste, blieb kein
Phönix mehr übrig. Es herrschte Stille auf der Spitze des Ravenclaw-Turms. Der
Junge ließ allmählich die Hände von den Ohren sinken und hielt nur inne, um sich
über die nassen Wangen zu wischen. Langsam drehte sich der Junge um - dann
schrie er auf, sprang zurück und wäre fast vom Ravenclaw-Turm gefallen; obwohl
der Fehltritt kaum ins Gewicht gefallen wäre, wenn dieser andere Zauberer dort
gestanden hätte.

\glqq Und erfüllt es sich...\grqq{}, sagte Albus Dumbledore, fast flüsternd.
\glqq So ist es entschieden.\grqq{}

Fawkes stand auf seiner Schulter und starrte mit einem nicht zu entziffernden
vogelartigen Blick auf die Stelle, an der der andere Phönix gestanden hatte.

\glqq Was machst du hier?\grqq{}

\glqq Ah?\grqq{}, sagte der alte Mann, der auf der gegenüberliegenden Ecke der
Dachplattform stand. \glqq Ich habe die Anwesenheit eines Wesens gespürt, das
Hogwarts nicht kennt, und bin natürlich gekommen, um nachzusehen.\grqq{} Langsam
fuhr die zitternde Hand des alten Zauberers hoch, um die Halbmondbrille
abzunehmen, mit der anderen Hand wischte er sich mit dem Ärmel seiner Robe über
Augen und Stirn. \glqq Ich habe es gewagt - ich habe es nicht gewagt zu sprechen
- ich wusste, ich wusste, dass diese Wahl vor allen anderen Entscheidungen deine
eigene sein muss -\grqq{}

Eine seltsame Befürchtung begann Harry zu erfüllen, quoll in ihm auf wie ein
flaues Gefühl im Magen.

\glqq Dass alles davon abhängt\grqq{}, sagte Albus Dumbledore, immer noch in
diesem Fast-Flüsterton, \glqq so viel wusste ich. Aber welche Wahl in die
Dunkelheit führte, das konnte ich nicht ahnen. Wenigstens war die Wahl deine
eigene.\grqq{}

\glqq Ich habe nicht -\grqq{} sagte Harry, und dann stockte seine Stimme. Eine
schreckliche Hypothese, die an Glaubwürdigkeit gewann...

\glqq Der Phönix kommt\grqq{}, sagte der alte Zauberer. \glqq Zu denen, die
kämpfen wollen, zu denen, die handeln wollen, selbst wenn es sie das Leben
kostet, kommt der Phönix. Phönixe sind nicht weise, Harry, sie wissen nicht, wie
sie uns beurteilen sollen, außer dass sie die Wahl miterleben. Ich dachte, ich
würde in den Tod gehen, als der Phönix mich zum Kampf gegen Grindelwald mitnahm.
Ich wusste nicht, dass Fawkes mir beistehen, mich heilen und an meiner Seite
bleiben würde -\grqq{} Die Stimme des alten Zauberers zitterte einen Moment
lang. \glqq Es wird nicht darüber gesprochen - du solltest begreifen, Harry,
warum nie darüber gesprochen wird - wenn der eine es wüsste, könnte der Phönix
nicht urteilen. Aber zu dir, Harry, darf ich es jetzt sagen, denn der Phönix
kommt nur einmal.\grqq{}

Der alte Zauberer schritt über die Spitze des Ravenclaw-Turms zu dem Jungen, der
wie angewurzelt stand, in dämmerndem Entsetzen, in dämmerndem und vollkommenem
Entsetzen.

\glqq In meinem Duell mit Grindelwald konnte ich nicht gewinnen, nur stundenlang
gegen ihn kämpfen, bis er vor Erschöpfung zusammenbrach; und ich wäre danach
daran gestorben, wenn nicht Fawkes gewesen wäre -

Harry wusste nicht einmal, dass er sprach, bis ihm das Flüstern entwichen war -
\glqq d\emph{ann hätte ich} -\grqq{}

\glqq Hättest du?\grqq{}, sagte der alte Zauberer, seine Stimme klang viel älter
als sein normaler Ton. \glqq Dreimal ist nun ein Phönix gekommen, um einen
meiner Schüler zu holen. Eine hat ihn fortgeschickt, und der Kummer darüber hat
sie gebrochen, glaube ich. Und der letzte war der Cousin deiner jungen Freundin
Lavender Brown, und er -\grqq{} Die Stimme des alten Zauberers brach. \glqq Er
kehrte nicht zurück, der arme John, und er rettete keinen von denen, die er
retten wollte. Unter den wenigen Gelehrten der Phönix-Lehre heißt es, dass nicht
einmal jeder Vierte von seiner Tortur zurückkehrt. Und selbst wenn du überlebt
hättest - für das Leben, das du führen musst, Harry James Potter-Evans-Verres -
die Entscheidungen, die du treffen musst und den Weg, den du gehen musst - immer
die Schreie des Phönix zu hören - wer kann sagen, dass es dich nicht in den
Wahnsinn getrieben hätte?\grqq{} Der alte Zauberer hob wieder seinen Ärmel und
zog ihn erneut über sein Gesicht. \glqq Ich hatte mehr Freude an Fawkes'
Gesellschaft, in den Tagen, bevor ich gegen Voldemort kämpfte.\grqq{}

Der Junge schien nicht zuzuhören, alle seine Augen waren auf den rot-goldenen
Vogel auf der Schulter des alten Zauberers gerichtet.

\glqq Fawkes?\grqq{}, fragte der Junge mit zitternder Stimme. \glqq Warum
schaust du mich nicht an, Fawkes?\grqq{}

Fawkes neigte den Kopf, um den Jungen neugierig anzuschauen, dann drehte er sich
um und starrte wieder seinen Meister an.

\glqq Siehst du?\grqq{}, sagte der alte Zauberer. \glqq Er weist dich nicht
zurück. Fawkes ist vielleicht nicht ganz so an dir interessiert, und er
weiß\grqq{}, der Zauberer lächelte schief, \glqq dass du seinem Herrn gegenüber
nicht gerade loyal bist.Aber einer, zu dem der Phönix überhaupt kommt - kann
nicht einer sein, den ein Phönix nicht mögen würde.\grqq{} Die Stimme des
Zauberers sank wieder zu einem Flüstern. \glqq Es wurde nie ein Vogel auf Godric
Gryffindors Schulter gesehen. Obwohl es nicht einmal in seinen Geheimnissen
steht, denke ich, dass er seinen Phönix weggeschickt haben muss, bevor er Rot
und Gold als seine Farben wählte. Vielleicht hat ihn die Schuld zu größeren
Taten getrieben, als er es sonst je gewagt hätte. Oder es hat ihn vielleicht
Demut gelehrt und Respekt vor menschlicher Schwäche und Versagen ...\grqq{} Der
Zauberer senkte den Kopf. \glqq Ich weiß wirklich nicht, ob deine Entscheidung
weise war. Ich weiß wirklich nicht, ob es das Richtige oder das Falsche war.
Wenn ich es wüsste, Harry, hätte ich gesprochen. Aber ich...\grqq{} Dann brach
Dumbledores Stimme. \glqq Ich bin nichts weiter als ein törichter Junge, der zu
einem törichten alten Mann geworden ist, und ich habe keine Weisheit.\grqq{}

Harry konnte nicht atmen, die Übelkeit schien seinen ganzen Körper auszufüllen
und zu überfluten, der Magen war fest verschlossen. Er war sich plötzlich und
furchtbar sicher, dass er versagt hatte, in irgendeinem endgültigen Sinn versagt
hatte, versagt in dieser Nacht -

Der Junge wirbelte herum und rannte hinaus auf den Rand des Ravenclaw-Daches.
\glqq Komm zurück!\grqq{} Seine Stimme brach und steigerte sich zu einem Schrei.
\glqq \textbf{Komm zurück!}\grqq{}

Endgültiges Nachspiel: Sie erwachte mit einem Keuchen des Entsetzens, mit einem
stimmlosen Schrei auf den Lippen, und es kamen keine Worte heraus, sie konnte
nicht verstehen, was sie gesehen hatte, sie konnte nicht verstehen, \emph{was
sie gesehen hatte} -

\glqq Wie spät ist es?\grqq{}, flüsterte sie.

Ihr goldener juwelenbesetzter Wecker flüsterte zurück: \glqq Gegen elf Uhr
nachts. Geh wieder schlafen.\grqq{}

Ihre Laken waren schweißgetränkt, ihr Nachthemd nass, sie nahm ihren Zauberstab
neben dem Kopfkissen und säuberte sich, bevor sie versuchte, wieder
einzuschlafen, was ihr schließlich auch gelang.

Sybill Trelawney schlief wieder ein.

... Im Verbotenen Wald hörte ein Zentaur, der von einem namenlosen Grauen
geweckt worden war, auf, den Nachthimmel abzusuchen, da er dort nur Fragen und
keine Antworten gefunden hatte; und mit einem Zusammenklappen seiner vielen
Beine schlief Firenze wieder ein.

... In den fernen Ländern des magischen Asiens schlief eine uralte Hexe namens
Fan Tong die müden Tage weg, sagte ihrem ängstlichen Ur-Ur-Enkel, dass es ihr
gut ginge, es sei nur ein Alptraum gewesen, und ging wieder schlafen.

... In einem Land, in dem Muggelgeborene keinerlei Briefe erhielten, wurde ein
Mädchen, das zu jung war, um einen eigenen Namen zu haben, in den Armen ihrer
verärgerten, aber liebevollen Mutter geschaukelt, bis sie aufhörte zu weinen und
wieder einschlief.

<p style=\grqq{}.ext-align:center\grqq{}.\emph{Keiner von ihnen schlief gut.
}</p>

