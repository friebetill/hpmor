\chapter{Selbstverwirklichung, Teil 4}

Aus dem Augenwinkel sah Hermine Granger es, eine Reflexion auf dem polierten
Metall einer Statue an der Kreuzung zweier Korridore, ein Aufblitzen von Gold,
ein Aufblitzen von Rot, so etwas wie ein Bild von Feuer; nur für einen Moment
sah sie es, und dann war es weg.

Sie hielt inne, verwirrt, und fast wäre sie weggegangen, aber da war etwas
Vertrautes an diesem kurzen Aufleuchten gewesen - Hermine ging nach vorne,
dorthin, wo die Statue gestanden hatte, schaute auf den Korridor, von dem sie
dachte, dass das feurige Spiegelbild gekommen sein könnte.

Schwach, wie von einem fernen Ort, hörte sie den Schrei, den Ruf. Hermine begann
zu rennen. Sie rannte eine Weile; jedes Mal, wenn sie an eine Kreuzung kam,
hielt sie inne, holte so viel Atem, wie sie konnte, und dann sah sie einen
Feuerschein, der aus der einen oder anderen Richtung reflektiert wurde, oder sie
hörte diesen fernen Ruf. Wäre sie nicht in der Armee ausgebildet worden, wäre
sie vor Erschöpfung umgefallen, so zu rennen. Sie sah den Phönix nicht. Und dann
kam sie an eine Abzweigung, und da war nichts, kein Zeichen, sie wartete lange
Sekunden, und sie hörte keinen Schrei und sah kein Feuer, und sie begann sich
gerade mit einem kranken, traurigen Gefühl zu fragen, ob sie sich das Ganze
eingebildet hatte, als sie eine Person schreien hörte.

Als ihre schnell rasenden Füße um die Ecke bogen, nahm sie die ganze Szene mit
einem Blick auf: drei riesige Jungen in grüngeschmückten Gewändern, die sich
bereits zu ihr umdrehten, und ein kleinerer, gelb gekleideter Junge, der an
einem Fuß in der Luft baumelte, der von einer unsichtbaren Hand hochgehalten
wurde. Die Sonnenschein-Generalin dachte nicht einmal darüber nach, Leute, die
aufhörten zu denken, legten keine besonders guten Hinterhalte. Sie hatte ihren
Zauberstab in der Hand, ihre Finger machten eine Drehung und ihre Lippen sagten
\glqq Somnium!\grqq{} und der größte Rüpel fiel um, der Hufflepuff-Junge fiel
mit einem dumpfen Schlag aus der Luft und die beiden anderen versuchten, mit
ihren Zauberstäben auf sie zu zielen und sie sagte wieder \glqq Somnium!\grqq{}
und ein weiterer großer Junge kippte um - derjenige, der mit seinem Zauberstab
schneller gezielt hatte, das war der, auf den sie gefeuert hatte. Leider war es
selbst für sie schwer, zwei Schlafverhexungen hintereinander zu zaubern, und sie
konnte keine dritte abschießen, bevor - der letzte schrie \glqq Protego!\grqq{}
und wurde von einem schimmernden blauen Leuchten umgeben. Vor vierundzwanzig
Stunden wäre Hermine bei so etwas in Panik geraten, ein echter Schutzzauber
hätte den Rüpel sogar in durch seinen Schild auf sie zaubern lassen. Jetzt -

\glqq Stupor!\grqq{}, schrie der Junge. Der karmesinrote Blitz schoss mit einem
furchtbaren Glanz auf sie zu und leuchtete weit heller als jeder Zauber, der
Harrys Zauberstab entsprungen war. Hermine schwankte leicht nach links, und der
Spruch verfehlte sie, denn der Rüpel hatte nicht annähernd so gut gezielt wie
Harry; und ihr kam der Gedanke, dass sich Rüpel und Professor Quirrells Armeen
vielleicht nicht vertragen.

\glqq Stupor!\grqq{}, schrie der Rüpel wieder. \glqq Expelliarmus!
Stupor!\grqq{}

Wie auch immer, jetzt hatte sie gerade eine ganze Stunde damit verbracht, an all
die anderen Zaubersprüche zu denken, die sie auf Harry und Neville hätte
anwenden können -

\glqq Jellyfy!\grqq{} schrie der Tyrann, ein weitstrahlender Fluch, dem man
nicht ausweichen konnte, und ihre Knie fühlten sich plötzlich fast zu schwach
an, um sie zu tragen. Und dann, mit einem wütenden Brüllen, das eine noch
hellere Purpurflamme erzeugte: \glqq Stupor!\grqq{}

Dem wich sie aus, indem sie sich absichtlich fallen ließ, und inzwischen hatte
sie sich genug für den nächsten Zauber erholt, der -

\glqq Glisseo\grqq{}, sagte Hermine und richtete ihren Zauber auf den Boden.

\glqq Uff\grqq{}, sagte der Rüpel, als ihm die Füße unterm Hintern weggezogen
wurden und er tatsächlich seinen Zauberstab fallen ließ. Der Protego erlosch.
\glqq Somnium\grqq{}, sagte Hermine.

Sie atmete immer noch keuchend, als sie zu dem Hufflepuff-Jungen hinüberkroch,
der sich aufsetzte, stöhnte und sich den Schädel rieb, wo er kopfüber auf den
Boden gefallen war; es war gut, dass er kein Muggel war, stellte Hermine fest,
sonst hätte er sich vielleicht das Genick gebrochen. Daran hatte sie eigentlich
nicht gedacht.

\glqq Äh\grqq{}, sagte der Junge, sein Haar hatte eine Farbe, die man 'brünett'
genannt hätte, wenn er ein Mädchen gewesen wäre, seine Augen waren ein
unauffälliges Braun, das irgendwie genau richtig für Hufflepuff zu sein schien,
es waren keine Tränen in seinem Gesicht, aber er sah irgendwie blass aus. Sie
schätzte ihn etwa auf das vierte oder dritte Schuljahr. Dann weiteten sich die
braunen Augen, als er sie anblickte.

\glqq General Sonnenschein?!\grqq{}

\glqq Ja\grqq{}, sagte sie. \glqq Das bin (\emph{keuch}) ich.\grqq{}

\emph{Wenn der Hufflepuff-Junge irgendetwas darüber sagte, dass sie Harry Potters Liebesinteresse war, beschloss sie, dass er sterben würde.}

\glqq Wow\grqq{}, sagte der Hufflepuff-Junge. \glqq Das war - du hast gerade -
ich meine, ich habe dich vor Weihnachten auf den Bildschirmen gesehen, aber -
wow! Ich kann nicht glauben, dass du das gerade getan hast!\grqq{}

Es gab eine Pause.

\emph{Ich kann nicht glauben, dass ich das gerade getan habe,} dachte Hermine
Granger, die sich auf einmal ein wenig schwach fühlte, das muss das ganze Laufen
gewesen sein.

\glqq Entschuldigung\grqq{}, sagte sie, \glqq könntest du mir die Beine
entzaubern?\grqq{}

Der Junge nickte, drückte sich auf die Beine und griff in seinen Umhang nach
seinem Zauberstab; aber Hermine musste seine Geste korrigieren, bevor der
Gegenzauber richtig funktionierte.

\glqq Ich bin Michael Hopkins\grqq{}, sagte der Junge, nachdem Hermine wieder
auf ihre eigenen Füße gerollt war. Er streckte seine Hand aus. \glqq Oder
einfach nur Mike in Hufflepuff, es gibt in diesem Jahr keine anderen Mikes in
ganz Hufflepuff, kannst du dir das vorstellen?\grqq{}

Sie schüttelten sich die Hand, und Mike sagte: \glqq Wie auch immer,
danke.\grqq{}

Hermine war nicht auf den Rausch der Euphorie vorbereitet, der sie in diesem
Moment überkam, jemanden auf diese Weise zu retten, fühlte sich buchstäblich
besser an als alles, was sie in ihrem ganzen Leben je empfunden hatte. Sie
drehte sich um und sah die Slytherins an. Sie waren sehr groß und sahen, so
schätzte sie, etwa fünfzehn Jahre alt aus, und ihr wurde plötzlich klar, wie
groß der Unterschied zwischen Hogwarts-Schülern, die sich für alle
außerschulischen Aktivitäten von Professor Quirrell angemeldet hatten, und
Schülern, die jahrelang von den schlimmsten Professoren aller Zeiten
unterrichtet worden waren, war. Die Fähigkeit, Dinge zu treffen, auf die man
zielt, zum Beispiel; oder die Fähigkeit, mitten in einem Kampf gut genug zu
denken, um zu erkennen, dass man seine gefallenen Verbündeten innervieren
sollte. Und andere Dinge, die Professor Quirrell gesagt hatte, wie zum Beispiel,
dass in der realen Welt fast jeder Kampf durch einen Überraschungsangriff
entschieden werden würde, machten plötzlich viel mehr Sinn für sie.

Sie versuchte immer noch, Luft zu holen, und schaute wieder zu Mike. \glqq
Kannst du dir vorstellen\grqq{}, sagte Hermine Granger, \glqq dass ich vor fünf
Minuten noch Schwierigkeiten hatte, herauszufinden, wie man ein Held
wird?\grqq{}.emph{ }

\emph{Hatte sie wirklich gedacht, sie bräuchte die Erlaubnis von jemandem, oder
dass Helden herumsaßen und darauf warteten, dass jemand anderes ihnen Quests
gab? Es war eigentlich ganz einfach, man ging einfach dorthin, wo das Böse war,
das war alles, was man brauchte, um ein Held zu sein.} \emph{Sie hätte sich
daran erinnern sollen, sie hätte keinen Phönix brauchen sollen, der ihr sagte,
dass hier in Hogwarts manchmal schlimme Dinge passierten. }

Dann blickte Hermine nervös zurück zu den drei älteren Jungen, die bewusstlos
dalagen, als ihr klar wurde, dass sie sie gesehen hatten, dass sie wissen
könnten, wer sie war, dass sie sich an sie heranschleichen und sie überrumpeln
könnten und - und sie könnten ihr wirklich wehtun - Hermine blieb stehen. Sie
erinnerte sich daran, dass Harry Potter sich am ersten Schultag in die Mitte von
fünf Slytherins gestellt hatte, als er noch nicht einmal wusste, wie man seinen
Zauberstab benutzt. Sie erinnerte sich daran, wie der Schulleiter sagte, dass
man erwachsen wird, indem man in erwachsene Situationen gebracht wird, und dass
die meisten Menschen ihr Leben in einem einschränkenden Kreis der Angst leben.
Und sie erinnerte sich an Professor McGonagalls Stimme, die sagte: \glqq Du bist
erst zwölf\grqq{}.

Hermine holte tief Luft, einmal, zweimal und dreimal. Sie fragte Mike, ob er in
Madam Pomfreys Büro gehen müsse, was nicht der Fall war; und ließ sich von ihm
die Namen der Slytherin-Jungen sagen, nur für den Fall. Und dann schlenderte
Hermine Granger von dem Haufen bewusstloser weg, wobei sie darauf achtete, ein
Lächeln auf ihr Gesicht zu zaubern, während sie ging. Sie wusste, dass sie
wahrscheinlich früher oder später verletzt werden würde. Aber wenn man zu viel
Angst davor hatte, verletzt zu werden, um das Richtige zu tun, dann konnte man
kein Held sein, so einfach war das; und wenn man ihr in diesem Moment den
Sprechenden Hut auf den Kopf gesetzt hätte, hätte er keine Sekunde gewartet,
bevor er \glqq \textbf{GRYFFINDOR}!\grqq{} gerufen hätte.

Sie dachte immer noch darüber nach, als sie zum Abendessen herunterkam; die
Euphorie, jemanden gerettet zu haben, war immer noch nicht abgeklungen, und sie
begann sich zu sorgen, dass es etwas in ihrem Gehirn kaputt gemacht hatte. Als
sie sich dem Ravenclaw-Tisch näherte, brach eine plötzliche Flüsterepidemie aus,
und Hermine fragte sich, ob der Hufflepuff-Junge schon etwas gesagt hatte, bevor
ihr klar wurde, dass das Flüstern wahrscheinlich nicht davon handelte. Sie
setzte sich Harry Potter gegenüber, der extrem nervös aussah, wahrscheinlich
weil sie immer noch lächelte.

\glqq Äh -\grqq{}, sagte Harry, als sie sich frisch getoastetes Brot, Butter,
Zimt, keinerlei Obst oder Gemüse und drei Portionen Schokoladenbrownies
servierte. \glqq Äh -\grqq{}

Sie ließ ihn so weitermachen, bis sie sich ein Glas Grapefruitsaft eingegossen
hatte, und dann sagte sie: \glqq Ich habe eine Frage an Sie, Mr. Potter. Was
glauben Sie, wie die Menschen daran scheitern, sie selbst zu werden?\grqq{}

\glqq Was?\grqq{}, sagte Harry.

Sie schaute ihn an. \glqq Tun Sie so, als gäbe es all diese Dinge nicht\grqq{},
sagte sie, \glqq und sagen Sie einfach das, was Sie gestern gesagt
hätten.\grqq{}

\glqq Ähm ...\grqq{} Harry sagte und sah sehr verwirrt und besorgt aus. \glqq
Ich denke, wir sind schon wir selbst ... es ist nicht so, dass ich eine
unvollkommene Kopie von jemand anderem bin. Aber ich schätze, wenn ich versuche,
dem Sinn der Frage nachzugehen, dann würde ich sagen, dass die Menschen nicht zu
sich selbst werden, weil wir all dieses verrückte Zeug aus der Umwelt aufnehmen
und dann wieder auswürgen. Ich meine, wie viele Leute, die Quidditch spielen,
würden so ein Spiel spielen, wenn sie das Spiel selbst erfunden hätten? Oder in
Muggel-Britannien, wie viele Leute, die sich für Labour oder Konservative oder
Liberaldemokraten halten, würden genau dieses Bündel an politischen
Überzeugungen unterstützen, wenn sie sich alles selbst ausdenken müssten?\grqq{}

Hermine dachte darüber nach. Sie hatte sich gefragt, ob Harry etwas von
Slytherin oder vielleicht sogar von Gryffindor sagen würde, aber das schien
nicht in die Liste des Schulleiters zu passen; und es kam Hermine in den Sinn,
dass es viel mehr Standpunkte zu diesem Thema geben könnte als nur vier.

\glqq Okay\grqq{}, sagte Hermine, \glqq andere Frage. Was macht jemanden zu
einem Helden?\grqq{}

\glqq Einen Helden?\grqq{}, sagte Harry.

\glqq Ja\grqq{}, sagte Hermine.

\glqq Ah ...\grqq{} sagte Harry. Seine Gabel und sein Messer sägten nervös an
einem Stück Steak und schnitten es in immer kleinere Stücke. \glqq Ich denke,
viele Leute können Dinge tun, wenn die Welt sie dazu kanalisiert... als würden
die Leute erwarten, dass du es tust, oder es nutzt nur Fähigkeiten, die du
bereits kennst, oder es gibt eine Autorität, die zusieht, um deine Fehler
aufzufangen und sicherzustellen, dass du deinen Teil tust. Aber solche Probleme
werden wahrscheinlich schon gelöst, weißt du, und dann braucht es keine Helden.
Ich denke also, dass die Menschen, die wir als \glqq Helden\grqq{} bezeichnen,
selten sind, weil sie sich alles selbst ausdenken müssen, und die meisten
Menschen fühlen sich dabei nicht wohl. Warum fragst du?\grqq{} Harry stach mit
der Gabel in drei Stücke des gründlich zerkleinerten Steaks und hob sie zu
seinem Mund.

\glqq Oh, ich habe gerade drei ältere Slytherins betäubt und einen Hufflepuff
gerettet habe\grqq{}, sagte Hermine. \glqq Ich werde ein Held sein.\grqq{}

Als Harry damit fertig war, sich an seinem Essen zu verschlucken (einige der
anderen Ravenclaws in Hörweite husteten immer noch), sagte er: \glqq
Was?!\grqq{}

Hermine erzählte die Geschichte, sie aß weiter vor sich hin, noch während sie
sprach. (Obwohl sie den Teil über den Phönix ausließ, denn das schien eine
private Sache zwischen ihnen beiden zu sein. Hermine hatte sich überrascht
gefühlt, als sie im Nachhinein darüber nachdachte, dass ein Phönix für jemanden
erscheinen würde, der ein Held sein wollte; es schien ein bisschen egoistisch,
wenn sie so darüber nachdachte; aber vielleicht war es den Phönixen egal,
solange sie sahen, dass man bereit war, Menschen zu helfen.) Als sie mit dem
Reden fertig war, starrte Harry sie über den Tisch hinweg an und sagte kein
einziges Wort. \glqq Es tut mir leid, wie ich mich vorhin verhalten habe\grqq{},
sagte Hermine. Sie nippte von ihrem Glas Grapefruitsaft. \glqq Ich hätte daran
denken sollen, dass es okay ist, wenn ich dich im Zauberkunstunterricht
verprügle, wenn du in Verteidigung besser bist.\grqq{}

\glqq Bitte versteh das nicht falsch\grqq{}, sagte Harry. Er sah jetzt zu
erwachsen aus, und grimmig. \glqq Aber bist du dir sicher, dass \emph{du} das
bist und nicht, um es ganz offen zu sagen, \emph{ich}?\grqq{}

\glqq Ich bin mir ganz sicher\grqq{}, sagte Hermine. \glqq Denn mein Name
buchstabiert sich praktisch wie 'Heldin', bis auf das zusätzliche 'm', das ist
mir bis heute nie aufgefallen.\grqq{}

\glqq Ein Held zu sein ist nicht nur Spaß und Spiel\grqq{}, sagte Harry. \glqq
Kein echtes Heldentum, wie es Erwachsene machen müssen, es ist nicht so, es wird
nicht so einfach sein."

\glqq Ich weiß\grqq{}, sagte Hermine.

\glqq Es ist schwer und es ist schmerzhaft und man muss Entscheidungen treffen,
auf die es keine gute Antwort gibt -\grqq{}

\glqq Ja, Harry, ich habe diese Bücher auch gelesen.\grqq{}

\glqq Nein\grqq{}, sagte Harry, \glqq du verstehst nicht, auch wenn die Bücher
dich warnen, du kannst es nicht verstehen, bis -\grqq{}

\glqq Das hält dich nicht auf\grqq{}, sagte Hermine. \glqq Es hält dich nicht
einmal ein bisschen auf. Ich wette, du hast nie auch nur in Erwägung gezogen,
deswegen kein Held zu sein. Warum denkst du also, dass es mich aufhalten
wird?\grqq{}

Es gab eine Pause.

Ein plötzliches breites Lächeln erhellte Harrys Gesicht, ein Lächeln, das so
hell und jungenhaft war, wie das Stirnrunzeln grimmig und erwachsen gewesen war,
und alles war wieder in Ordnung zwischen ihnen.

\glqq Das wird irgendwie furchtbar verblüffend schief gehen\grqq{}, sagte Harry
und lächelte immer noch gewaltig. \glqq Das weißt du doch, oder?\grqq{}

\glqq Oh, ich weiß\grqq{}, sagte Hermine. Sie aß einen weiteren Bissen Toast.
\glqq Da fällt mir ein, Dumbledore hat sich geweigert, mein geheimnisvoller
alter Zauberer zu sein, kann ich irgendwo hinschreiben, um einen anderen zu
bekommen?\grqq{}

\textbf{Nachsatz}: \glqq ...und Professor Flitwick sagt, dass ihre
Entschlossenheit unerschütterlich scheint\grqq{}, sagte Minerva knapp und
starrte den silberbärtigen alten Zauberer an, der dafür verantwortlich war.

Albus Dumbledore saß nur schweigend da und hörte ihr mit einem fernen, traurigen
Blick in den Augen zu.

\glqq Miss Granger hat nicht einmal mit der Wimper gezuckt, als Professor
Flitwick ihr gedroht hat, sie nach Gryffindor versetzen zu lassen, und hat nur
gesagt, dass sie, wenn sie geht, alle Bücher mitnehmen würde. Hermine Granger
hat sich entschlossen, eine Heldin zu werden, und sie akzeptiert kein Nein als
Antwort. Ich bezweifle, dass du sie noch mehr dazu hättest bringen können, wenn
du sie ermutigt hättest -"

\emph{Es dauerte ganze fünf Sekunden, bis Minervas Gehirn diese Erkenntnis verarbeitet hatte. }

\glqq \textbf{ALBUS}!\grqq{}, kreischte sie.

\glqq Meine Liebe\grqq{}, sagte der alte Zauberer, \glqq nachdem du mit deinem
dreißigsten Helden oder so zu tun hattest, wirst du feststellen, dass sie
ziemlich vorhersehbar auf bestimmte Dinge reagieren; zum Beispiel, wenn man
ihnen sagt, dass sie zu jung sind, oder dass sie nicht dazu bestimmt sind,
Helden zu sein, oder dass es unangenehm ist, ein Held zu sein; und wenn du
wirklich sicher sein willst, solltest du ihnen alle drei Dinge sagen.
Obwohl\grqq{}, mit einem kurzen Seufzer, \glqq es nicht gut ist, zu
offensichtlich zu sein, sonst könnte dich deine stellvertretende Schulleiterin
erwischen.\grqq{}

\glqq Albus\grqq{}, sagte Minerva, ihre Stimme noch fester, \glqq wenn sie
verletzt wird, schwöre ich, dass ich diesmal -\grqq{}

\glqq Sie wäre zu gegebener Zeit zu denselben Schluss gekommen\grqq{}, sagte
Albus, der ferne traurige Blick immer noch in seinen Augen. \glqq Wenn jemand
dazu bestimmt ist, ein Held zu werden, dann wird er nicht auf unsere Warnungen
hören, Minerva, egal wie sehr wir uns bemühen. Und in Anbetracht dessen ist es
besser für Harry, wenn Miss Granger nicht zu weit hinter ihm
zurückbleibt.\grqq{}

Albus brachte wie aus dem Nichts eine Dose hervor, die sich aufklappte und
kleine gelbe Klumpen zum Vorschein brachte; sie hatte nie herausfinden können,
wo er sie aufbewahrte, und sie war auch nicht in der Lage gewesen, die damit
verbundene Magie zu spüren.

\glqq Zitronenbonbon?\grqq{}

\glqq Sie ist ein zwölfjähriges Mädchen, Albus!\grqq{}

\textbf{Nachsatz 2}: Zwischen den Fenstern, kaum sichtbar in der abendlichen
Düsternis, schwammen Fische im schwarzen Wasser; beleuchtet vom hellen Schein
des Slytherin-Gemeinschaftsraums als sie näher kamen, dann verblassten sie in
der Dunkelheit, als sie davonschwammen.

Daphne Greengrass saß in einem bequemen schwarzen Ledersofa, den Kopf in die
Hände gestützt, und glühte goldgelb, während um sie herum helle Funken aus
weißem Licht ein- und ausblitzten.

Sie war bereit gewesen, das die Leute sich darüber lustig machten, dass sie
Neville Longbottom mochte. Sie hatte erwartet, eine Menge abfälliger Bemerkungen
über Hufflepuffs zu hören. Sie hatte sich auf dem Rückweg zu den
Slytherin-Kerkern eine ganze Reihe von bissigen Kontern dafür ausgedacht. Sie
hatte sich darauf gefreut, gehänselt zu werden, weil sie Neville mochte. Mit so
etwas gehänselt zu werden, bedeutete, dass man zu einem richtigen Mädchen
herangewachsen war.

Wie sich herausstellte, hatte niemand herausgefunden, das es bedeutete dass Sie
Neville mochte wenn Sie ihn zu einem Duell herausforderte. Sie hatte gedacht, es
wäre offensichtlich, aber nein, daran hatte anscheinend noch niemand gedacht.

\emph{Es waren immer die Flüche, die man nicht sah, die einen trafen}.\emph{ Sie
hätte sich einfach Daphne von Sonnenschein nennen sollen, wie Neville von Chaos.
Oder Sunny Daphne, wie Sunny Ron. Oder irgendwas anderes als Greengrass von
Sonnenschein. Greengrass of Sunshine.}

Von da an wurde es zu Greengrass von Sonnenschein und Blauem Himmel. Dann hatte
jemand schneebedeckte Berge und herumtollende Waldkreaturen hinzugefügt. Jetzt
nannte man sie die funkelnde Einhornprinzessin aus dem edlen und uralten Haus
Glitzerschein ('\emph{Sparklypoo', Anm. des Übersetzers}). Und irgendein
verfluchtes Mädchen aus der sechsten Klasse hatte sie mit einem funkel-Fluch
verhext, und Sie hatte nicht einmal gewusst, dass es so etwas wie einen
funkelnden Fluch gab, und Finite Incantatem hatte nicht funktioniert, und sie
hatte ältere Mädchen gefragt, von denen sie dachte, sie seien ihre Freundinnen
(da hatte sie sich offenbar geirrt), und dann hatte sie dem Mädchen mit schwerem
politischen Chaos gedroht, das ihr Vater anrichten würde, und trotzdem saß
Daphne Greengrass immer noch im Slytherin-Gemeinschaftsraum, den Kopf in den
Händen, strahlte, funkelte und fragte sich, wie sie als einzige vernünftige
Person in Hogwarts gelandet war.

Es war nach dem Abendessen und sie waren immer noch dabei, und wenn sie bis
morgen früh nicht aufhörten, würde sie nach Durmstrang wechseln und die nächste
Dunkle Lady werden.

\glqq Hey, Leute!\grqq{}, riefen die Carrow-Zwillinge dramatisch und wedelten
mit einer Ausgabe des Tagespropheten. \glqq Habt ihr die Nachrichten gehört? Das
Zaubergamot hat gerade entschieden, dass '\emph{Zeig, was du drauf hast'} eine
rechtmäßige Herausforderung ist, die so lange ausgefochten wird, bis der
Herausforderer sich hinlegt und ein Nickerchen macht!"

\glqq Wie kannst du es wagen, die Ehre der funkelnden Einhornprinzessin zu
beleidigen!\grqq{}, rief Tracey. \glqq Zeig mal, was du drauf hast!\grqq{}

Dann legte sich Tracey flach auf ihr Sofa und begann laut zu schnarchen.

Daphnes funkelnder Kopf sank weiter in ihre glühenden Hände. \glqq Nachdem meine
Familie die Macht übernommen hat, werde ich euch alle mit
Anti-Apparitionszaubern belegen und ins Meer treiben lassen\grqq{}, sagte sie zu
niemandem speziell. \glqq Ihr seid alle damit einverstanden, oder?\grqq{}

\emph{Poch Poch Poch}
Daphne schaute überrascht auf; das war ein Codesignal von Sonnenschein -

\glqq Ich höre jemanden klopfen!\grqq{}, brüllte Mr. Goyle. \glqq Klopfen an der
Tür!\grqq{}

\glqq Zeig mal, was du drauf hast, Tür!\grqq{}, rief ein älterer Junge in der
Nähe der Tür und riss die Tür auf.

Es gab einen Moment der völligen Überraschung.

\glqq Ich bin gekommen, um mit Miss Greengrass zu sprechen\grqq{}, sagte die
Sonnenschein-Generalin, und es klang, als ob sie versuchte, selbstbewusst zu
klingen. \glqq Könnte bitte jemand -\grqq{} Dem Gesichtsausdruck von Hermine
nach zu urteilen, hatte sie gerade bemerkt, dass Daphne funkelte.

Und das war der Moment, in dem Millicent Bulstrode aus den unteren Schlafsälen
herbeieilte und rief: \glqq Hey, Leute, ratet mal was passiert ist, jetzt ist
Granger losgezogen und hat Derrick und das, was von seiner Crew übrig ist
verprügelt, und sein Vater hat ihm eine Eule geschickt und gesagt, dass er, wenn
er nicht -\grqq{} Millicent erblickte Hermine, die in der Tür stand.

Es herrschte eine sehr laute Stille.

\glqq Äh\grqq{}, sagte Daphne. \emph{Was}? sagte ihr Gehirn. \glqq Äh, was
machst du hier, General?"

\glqq Nun\grqq{}, sagte Hermine Granger mit einem seltsamen Lächeln auf dem
Gesicht, \glqq ich habe beschlossen, dass es nicht fair ist, wenn mysteriöse
alte Zauberer einigen Leuten eine Chance geben, Helden zu sein, und anderen
nicht, und außerdem habe ich Geschichtsbücher gelesen und es gibt nicht
annähernd genug weibliche Helden darin. Also dachte ich, ich komme einfach mal
vorbei und frage, ob du ein Held sein willst und warum du so glitzerst.\grqq{}

Wieder herrschte Schweigen.

\glqq Das\grqq{}, sagte Daphne, \glqq war wahrscheinlich nicht der beste
Zeitpunkt, mir diese Frage zu stellen -\grqq{}

\glqq Ich komme mit!\grqq{}, rief Tracey Davis und sprang vom Sofa auf.

Und so wurde die Gesellschaft zur Förderung der heroischen Gleichberechtigung
von Hexen geboren.

