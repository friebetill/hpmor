\wrapchapter{The Truth, Part 5,}{Answers and Riddles}

\lettrine{T}{he} Defense
Professor had set up a cauldron, floating it into place with a wave of his
wand, another wave starting a fire beneath it. A brief circling of the Defense
Professor's finger had set in motion a long-handled spoon, and it had continued
stirring the cauldron without being held. Now the Defense Professor was
measuring out a heap of flowers from a large jar, what Harry supposed to be
bellflowers; the indigo petals seemed luminous in the white light of the walls,
and curved inward in a way that gave the impression of a desire for privacy.
The first of these flowers had been added to the potion at once, but then the
cauldron had just gone on stirring itself for a while.

The Defense Professor had assumed a position from which he could see Harry just
by turning his head slightly, and Harry knew that he was within the Defense
Professor's peripheral vision.

In the corner a Fiendfyre phoenix waited, some of the nearby stone beginning to
gloss over as it melted to greater smoothness. The burning wings shed crimson
light that gave everything in the room a tint of blood, and reflected in
scarlet sparks from the glassware.

"Time is wasting," said Professor Quirrell. "Ask your questions, if you have
them."

\emph{Why, Professor Quirrell, why, why must you be this way, why make yourself
the monster, why Lord Voldemort, I know you might not want the same things I
do, but I can't imagine what you want that makes} this \emph{the best way to
get it{\el}}

That was what Harry's brain wanted to know.

What Harry \emph{needed} to know was{\el} some way out of what was going to
happen next. But the Defense Professor had said that he wouldn't talk about his
future plans. It was strange enough that the Defense Professor was willing to
talk about \emph{anything,} that had to contradict one of his Rules{\el}

"I'm thinking," Harry said aloud.

Professor Quirrell smiled slightly. He was using a pestle to grind the potion's
first magical ingredient, a glowing red hexagon. "I \emph{quite} understand,"
said the Defense Professor. "But do not think over-long, child."

\emph{Goals: Prevent Lord Voldemort from harming people, find a way to kill or
neutralize him, but first get the Stone and resurrect Hermione{\el}}

\emph{{\el}convince Professor Quirrell to STOP THIS{\el}}

Harry swallowed, pushing down the emotion, trying not to let the water reach
his eyes. Tears probably wouldn't make a good impression on Lord Voldemort.
Professor Quirrell was already frowning, though from the direction of his gaze
he was examining a leaf colored in vivid shades of white, green, and purple.

There wasn't any obvious way to reach any of the goals, not yet. All Harry
could do was ask the questions that seemed most likely to provide useful
information, even if Harry didn't yet have a plan.

\emph{So we just ask about whatever seems most interesting?} said Harry's
Ravenclaw side. \emph{I'm up for that.}

\emph{Shut up,} Harry told the voice; and then, on further reflection, decided
that he was no longer pretending it was there.

Four topics came to Harry's mind as being priorities from the standpoint of
curiosity about important things. Four questions, then, four major subjects, to
try to fit in while this potion was still being brewed.

Four questions{\el}

"I ask my first question," Harry said. "What really happened on the night of
October 31st, 1981?" \emph{Why was that night different from all other
nights{\el}} "I would like the entire story, please."

The question of how and why Lord Voldemort had survived his apparent death
seemed likely to matter for future planning.

"I expected you would ask that," Professor Quirrell said, dropping a bellflower
and a white glittering stone into the potion. "To begin, everything I told you
about the Horcrux spell is true; as you should realize, since I spoke in
Parseltongue."

Harry nodded.

"Within seconds after you learned the details of the spell, you perceived the
central flaw, and began pondering how the spell might be improved. Do you think
the young Tom Riddle was any different?"

Harry shook his head.

"Well, he was," said Professor Quirrell. "Whenever I was tempted to despair of
you, I reminded myself how I was an idiot at twice your age. When I was fifteen
I made myself a Horcrux as a certain book had shown me, using the death of
Abigail Myrtle beneath the eyes of Slytherin's basilisk. I planned to make a
new Horcrux every year after I left Hogwarts, and call that my fallback plan if
my other hopes of immortality did not come to fruition. In retrospect, the
young Tom Riddle was grasping straws. The thought of making a \emph{better}
Horcrux, of not being content with the spell I had already learned{\el} this
thought did not come to me until I had grasped the stupidity of ordinary
people, and realized which follies of theirs I had imitated. But in time I
learned the habit that you inherited from me, to ask in every instance how it
might be done better. To be content with the spell I had learned from a book,
when it bore only a faint resemblance to what I truly wanted? Absurd! And so I
set forth to create a better spell."

"You have true immortality, now?" Harry was aware that, even with everything
else going on, this was a question more important than war and strategy.

"Indeed," said Professor Quirrell. He paused in his Potions work and turned to
face Harry fully; there was a look of exultation in the man's eyes that Harry
had never seen there before. "In all the Darkest Arts I could find, in all the
interdicted secrets to which Slytherin's Monster gave me keys, in all the lore
remembered among wizardkind, I found only hints and smatterings of what I
needed. So I rewove it and remade it, and devised a new ritual based on new
principles. I kept that ritual burning in my mind for years, perfecting it in
imagination, pondering its meaning and making fine adjustments, waiting for the
intention to stabilize. At last I dared to invoke my ritual, an invented
sacrificial ritual, based on a principle untested by all known magic. And I
lived, and yet live." The Defense Professor spoke with quiet triumph, as though
the act itself was so great that no words could ever do it justice. "I still
use the word `Horcrux', but only from sentiment. It is a new thing entirely,
the greatest of all my creations."

"As one of my questions you said you'd answer, I ask how to cast that spell,"
Harry said.

"Denied." The Defense Professor turned back to his potion, dropping in a
grey-flecked white feather and a bellflower. "I had thought perhaps to teach
you when you were older, for no Tom Riddle would be content otherwise; but I
have changed my mind."

Memory is a hard thing to recall, sometimes, and Harry had been trying to
remember if Professor Quirrell had dropped any hints about this subject before.
Something about Professor Quirrell's phrasing sparked a memory: \emph{Perhaps
you will be told when you are older{\el}}

"There are still physical anchors for your immortality," Harry said aloud. "It
resembles the old Horcrux spell by that much, which is another reason you still
call them Horcruxes." It was dangerous to say aloud, but Harry needed to
\emph{know.} "If I'm wrong, you can always deny it in Parseltongue."

Professor Quirrell was smiling evilly. "\parsel{Your guesss iss right, boy, for
all the good it doess you.}"

Unfortunately, that wasn't a difficult vulnerability to cover if the Enemy was
smart. Harry wouldn't ordinarily have made the suggestion, just in case the
Enemy \emph{hadn't} thought of it for themselves, but in this case he'd already
made it. "One horcux dropped into an active volcano, weighted so it would sink
into the Earth's mantle," Harry said heavily. "The same place I thought of
dropping the Dementor if I couldn't destroy it. And then you asked me where
else I would hide something if I didn't want anyone to find it ever again. One
Horcrux buried kilometers down, in an anonymous cubic meter of the Earth's
crust. One Horcrux you dropped into the Mariana Trench. One Horcrux floating
high in the stratosphere, transparent. Even you don't know where they are,
because you Obliviated the exact details from your memory. And the last Horcrux
is the Pioneer 11 plaque that you snuck into NASA and modified. It's where you
get your image of the stars, when you cast the spell of starlight. Fire, earth,
water, air, void." \emph{Something of a riddle,} the Defense Professor had
called it, and therefore Harry had remembered it. Something of a Riddle.

"Indeed," said the Defense Professor. "It did give me something of a shock when
you remembered it that quickly, but I suppose it makes no difference; all five
are beyond my reach, or yours."

That might not be true, especially if there was some way to trace the magical
connection somehow and determine the location{\el} though presumably
Voldemort would have done his best to obscure it{\el} but what magic had
done, magic might be able to defeat. Pioneer 11 might be far away by wizard
standards, but NASA knew exactly where it was, and it was probably a lot more
reachable if you could use magic to tell the Tsiolkovsky rocket equation to
bugger off{\el}

A sudden note of worry plucked at Harry's mind. There was no rule saying the
Defense Professor needed to have told the truth about \emph{which} interstellar
probe he'd Horcruxed, and if Harry recalled correctly, communication and
tracking of the Pioneer 10 probe had been lost shortly after the Jupiter fly-by.

Why wouldn't Professor Quirrell have just Horcruxed them both?

The obvious next thought came to Harry. It was something that ought not to be
suggested, if the Enemy had not thought of it. But it seemed extremely probable
that the Enemy had thought of it.

"\parsel{Tell me, teacher,}" Harry hissed, "\parsel{would desstroying thosse five
anchors sslay you?}"

"\parsel{Why do you assk?}" hissed the Defense Professor, with a lilt to the hiss
that Parseltongue translated as snakish amusement. "\parsel{Do you ssusspect that
ansswer is no?}"

Harry couldn't think of how to answer, though he strongly suspected that it
didn't matter in any case.

"\parsel{Your ssusspicion iss right, boy. Desstroying thosse five would not
render me mortal.}"

Harry's throat felt a bit dry again. If the spell had no disastrous cost
associated with it{\el} "\parsel{How many anchorss did you make?}"

"\parsel{Would not ordinarily ssay, but iss clear you have already guesssed.}"
The Defense Professor's smile widened. "\parsel{Ansswer iss that I do not know.
Sstopped counting ssomewhere around one hundred and sseven. Ssimply made a
habit of it each time I murdered ssomeone in private.}"

Over \emph{one hundred} murders, in private, before Lord Voldemort had stopped
counting. And even worse news---"Your immortality spell still requires a human
death? \emph{Why?}"

"\parsel{Great creation maintainss life and magic within devicess created by
ssacrificing life and magic of otherss.}" Again that hissing snake laughter.
"\parsel{Liked falsse desscription of previousss Horcrux sspell sso much, sso
dissappointed when realissed truth of it, thoughtss of improved verssion came
out in that sshape.}"

Harry wasn't sure why the Defense Professor was giving him all this vital
information, \emph{but there had to be a reason,} and that was making him
nervous. "So you really are a disembodied spirit possessing Quirinus Quirrell."

"\parsel{Yess. I sshall return sswiftly, if thiss body iss killed. Will be
greatly annoyed, and vengeful.} I am telling you this, boy, so that you do not
try anything stupid."

"I understand," Harry said. He did his best to organize his thoughts, remember
what he'd meant to ask next, while the Defense Professor turned his eyes back
to the potion. The man's left hand was dribbling crushed seashell into the
cauldron, while his right hand dropped in another bellflower. "So what did
happen on October 31st? You{\el} tried to turn the baby Harry Potter into a
Horcrux, either the new kind or the old kind. You did it deliberately, because
you told Lily Potter," Harry took a breath. Now that he knew \emph{why} the
chills were there, he could endure them. "Very well, I accept the bargain.
Yourself to die, and the child to live. Now drop your wand so that I can murder
you." In retrospect, it was clear that Harry had remembered that event mainly
from Lord Voldemort's perspective, and only at the very end had he seen it
through the baby Harry Potter's eyes. "What did you do? \emph{Why} did you do
it?"

"Trelawney's prophecy," Professor Quirrell said. His hand tapped a bellflower
with a strip of copper before dropping it in. "I spent long days pondering it,
after Snape brought the prophecy to me. Prophecies are never trivial things.
And how shall I put this in a way that does not make you think stupid
things{\el} well, I shall say it, and if you are stupid I shall be annoyed.
I was fascinated by the prophecy's assertion that someone would be my equal,
because it might mean that person could hold up the other end of an intelligent
conversation. After fifty years of being surrounded by gibbering stupidity, I
no longer cared whether my reaction might be considered a literary cliche. I
was not about to pass up on that opportunity without thinking about it first.
And then, you see, I had a \emph{clever idea}." Professor Quirrell sighed. "It
occurred to me how I might fulfill the Prophecy my own way, to my own benefit.
I would mark the baby as my equal by casting the old Horcrux spell in such
fashion as to imprint my own spirit onto the baby's blank slate; it would be a
purer copy of myself, since there would be no old self to mix with the new. In
some years, when I had become bored with ruling Britain and moved on to other
things, I would arrange with the other Tom Riddle that he should appear to
vanquish me, and he would rule over the Britain he had saved. We would play the
game against each other forever, keeping our lives interesting amid a world of
fools. I knew a dramatist would predict that the two of us would end by
destroying each other; but I pondered long upon it, and decided that both of us
would simply decline to play out the drama. That was my decision and I was
confident that it would remain so; both Tom Riddles, I thought, would be too
intelligent to truly go down that road. The prophecy seemed to hint that if I
destroyed all but a remnant of Harry Potter, then our spirits would not be so
different, and we could exist in the same world."

"Something went wrong," Harry said. "Something that blew off the top of the
Potters' home in Godric's Hollow, gave me the scar on my forehead, and left
your burnt body behind."

Professor Quirrell nodded. His hands had slowed in their Potions work. "The
resonance in our magic," Professor Quirrell said quietly. "When I had shaped
the baby's spirit to be like my own{\el}"

Harry remembered the moment in Azkaban when Professor Quirrell's Killing Curse
had collided with his Patronus. The burning, tearing agony in his forehead,
like his head had been about to split in half.

"I cannot count how many times I have thought of that night, rehearsing my
mistake, thinking of wiser things I should have done," said Professor Quirrell.
"I later decided that I should have thrown my wand from my hand and changed
into my Animagus form. But that night{\el} that night, I instinctively tried
to control the chaotic fluctuations in my magic, even as I felt myself burning
up from inside. That was the wrong decision, and I failed. So my body was
destroyed, even as I overwrote the infant Harry Potter's mind; \emph{either} of
us destroying all but a remnant of the other. And then{\el}" Professor
Quirrell's expression was controlled. "And then, when I regained consciousness
inside my Horcruxes, it turned out that my great creation did not work as I had
hoped. I should have been able to float free of my Horcruxes and possess any
victim that consented to me, or that was too weak to refuse me. \emph{That} was
the part of my great creation that failed my intent. As with the original
Horcrux spell, I would only be able to enter a victim who contacted the
physical Horcrux{\el} and I had hidden my unnumbered Horcruxes in places
where nobody would ever find them. Your instinct is correct, boy, \emph{this
would not be a good time to laugh.}"

Harry stayed very quiet.

The Potions-making had come to a temporary pause, a space where no ingredients
were added while the cauldron simmered for a time. "I spent most of my time
looking at the stars," Professor Quirrell said, his voice quieter now. The
Defense Professor had turned from the potion, staring at the white-illuminated
walls of the room. "My remaining hope was the Horcruxes I had hidden in the
hopeless idiocy of my youth. Imbuing them into ancient lockets, instead of
anonymous pebbles; guarding them beneath wells of poison in the center of a
lake of Inferi, instead of Portkeying them into the sea. If someone found one
of those, and penetrated their ridiculous protections{\el} but that seemed
like a distant hope. I was not sure I would ever be embodied again. Yet at
least I was immortal. The worst of all fates had been averted, my great
creation had done that much. I had little left to hope for, and little left to
fear. I decided that I would not go insane, since there seemed to be no
advantage in it. Instead, I gazed out at the stars and thought, as the Sun
slowly diminished behind me. I reflected on the errors of my past life; they
were many, in that hindsight. In my imagination I constructed powerful new
rituals I might attempt, if I were free to use my magic once more, and yet
confident of my immortality. I contemplated ancient riddles at greater length
than before, for all that I had once thought myself patient. I knew that if I
won free, I would be more powerful by far than in my previous life; but I
mostly did not expect that to happen." Professor Quirrell turned back to the
potion. "Nine years and four months after that night, a wandering adventurer
named Quirinus Quirrell won past the protections guarding one of my earliest
Horcruxes. The rest you know. And now, boy, you may say what we both know you
are thinking."

"Um," Harry said. "It doesn't seem like a very smart thing to say---"

"Indeed, Mr.~Potter. It is not a clever thing to say to me. Not even a little.
Not in the slightest. But I \emph{know you're thinking it,} and you will
\emph{go on thinking it} and I will \emph{go on knowing that} until you say it.
So speak."

"So. Um. I realize that this is something that is more obvious in hindsight
than in foresight, and I'm certainly not suggesting that you try to correct the
error now, but if you are a Dark Lord and you happen to hear about a child who
has been prophesied to defeat you, there is a certain spell which is
unblockable, unstoppable, and works every single time on anything with a
brain---"

"\emph{Yes thank you Mr.~Potter that thought occurred to me several times over
the next nine years.}" Professor Quirrell picked up another bellflower and
began crumbling it in his bare fist. "I made that principle the centerpiece of
my Battle Magic curriculum after I learned its centrality the hard way. It was
\emph{not} the first Rule on the younger Tom Riddle's list. It is only by harsh
experience that we learn which principles take priority over which other
principles; as mere words they all sound equally persuasive. In retrospect it
would have been better if I had sent Bellatrix to the Potters' home in my
place; but I had a Rule telling me that for such matters I must go myself and
not try sending a trusted lieutenant. \emph{Yes,} I considered the Killing
Curse; but I wondered if casting the Killing Curse at an infant would somehow
cause the curse to bounce off and hit me, thus fulfilling the prophecy. How was
I to know?"

"So use an axe, it's hard to get a prophecy-fulfilling spell backfire out of an
axe," Harry said and then shut up.

"I decided the safest path was to try to fulfill the prophecy on my own terms,"
Professor Quirrell said. "Needless to say, the next time I hear a prophecy I do
not like, I will tear it apart at \emph{every possible point of intervention},
rather than trying to play along." Professor Quirrell was crushing a rose as
though to squeeze the juice out of it, still using his bare fist. "And now
everyone thinks the Boy-Who-Lived is somehow immune to the Killing Curse, even
though Killing Curses do not ruin houses or leave burnt bodies behind them,
\emph{because it has not occurred to them that Lord Voldemort would ever use
any other spell.}"

Harry again stayed quiet. It had occurred to Harry that there was another
obvious way that Lord Voldemort could have avoided his mistake. Something that
might perhaps be easier to see given a Muggle upbringing, instead of the
wizarding way of looking at things.

Harry had not yet decided whether to tell Professor Quirrell about his thought;
there were both pros and cons to pointing out that particular error.

After a time Professor Quirrell picked up the next Potions ingredient, a strand
of what looked like unicorn hair. "I tell you this as a caution," said
Professor Quirrell. "Do not expect me to be delayed another nine years, if you
somehow destroy this body of mine. I set Horcruxes in better places at once,
and now even that is unnecessary. Thanks to you, I learned where to find the
Resurrection Stone. The Resurrection Stone does not bring back the dead, of
course; but it holds a more ancient magic than my own for projecting the
seeming of a spirit. And since I am one who has defeated death, Cadmus's Hallow
acknowledged me its master, and answered all my will. I have now incorporated
it into my great creation." Professor Quirrell smiled slightly. "I had many
years earlier considered making that device a Horcrux, but decided against it
at the time, since I realized that the ring had magic of unknown nature{\el}
ah, such ironies does life play upon us. But I digress. \emph{You}, boy, you
brought that about, you freed my spirit to fly where it pleases and seduce the
most opportune victim, by being too casual with your secrets. It is a
catastrophe for any who oppose me, and you wrought it with one finger drawing
wetness on a tea-saucer. This world will be a safer place for all, if you learn
the rectitude that wizardborns absorb in childhood. \parsel{And all thiss that I
have jusst said iss the truth.}"

Harry closed his eyes, and his own hand massaged his forehead; if he had seen
it from the outside, it would have looked the mirror of Professor Quirrell in
deep thought.

The problem of defeating Professor Quirrell was looking increasingly difficult,
even by the standards of the sort of impossible problems that Harry had solved
already. If communicating that difficulty was what Professor Quirrell was
trying to do, he was \emph{succeeding.} Harry was starting to seriously
consider the possibility that it might be better to offer to rule Britain as
Voldemort's \emph{non-homicidal} delegate, if Professor Quirrell himself would
just agree to \emph{stop killing people all the time.} Even \emph{mostly}.

But that wasn't likely to happen.

Harry stared at his hands, from where he had sat down upon the floor, feeling
sadness shading over into despair. The Lord Voldemort who'd given Harry his
dark side had spent \emph{that long} thinking things over and reflecting on his
own thought processes{\el} and had emerged as the calm, clear-headed, and
still homicidal Professor Quirrell.

Professor Quirrell added a pinch of golden hair to the \emph{potion of
effulgence,} and that reminded Harry that time was continuing to move; the
locks of bright hair were rarer than the bellflowers.

"I ask my second question," Harry said. "Tell me about the Philosopher's Stone.
Does it do anything besides making Transfigurations permanent? Is it possible
to make more Stones, and why is that problem hard?"

Professor Quirrell was bent over the potion, and Harry could not see his face.
"Very well, I shall tell you the Stone's story as I have inferred it. The one
and only power of the Stone is the imposition of permanency, to render a
temporary form into a true and lasting substance---a power absolutely beyond
ordinary spells. Conjurations such as the castle Hogwarts are maintained by a
constant well of magic. Even Metamorphmagi cannot manifest golden fingernails
and then trim them for sale. It is theorized that the Metamorphmagus curse
merely rearranges the substance of their flesh, like a Muggle smith manipulates
iron with hammer and tongs; and their body contains no gold. If Merlin himself
could create gold from thin air, history does not record it. So the Stone, we
can guess even before research, must be a very old thing indeed. In contrast,
Nicholas Flamel has been known to the world for a mere six centuries. Tell me
the obvious next question to ask, boy, if you wanted to trace the Stone's
history."

"Um," Harry said. He rubbed his forehead, concentrating. If the Stone was old,
but the world had only known Nicholas Flamel for six centuries{\el} "Was
there some other very long-lived wizard who disappeared at around the same time
Nicholas Flamel showed up?"

"Close," said Professor Quirrell. "You recall that six centuries ago there was
a Dark Lady called undying, the sorceress Baba Yaga? She was said to be able to
heal any wound in herself, to change shape into any form she pleased{\el}
she held the Stone of Permanency, obviously. And then one year Baba Yaga agreed
to teach Battle Magic at Hogwarts, under an old and respected truce." Professor
Quirrell looked{\el} \emph{angry,} a look such as Harry had rarely seen on
him. "But she was not trusted, and so there was invoked a curse. Some curses
are easier to cast when they bind yourself and others alike; Slytherin's
Parselmouth curse is an example of such. In this case, Baba Yaga's signature,
and signatures from every student and teacher of Hogwarts, were placed within
an ancient device known as the Goblet of Fire. Baba Yaga swore not to shed a
drop of students' blood, nor take from the students anything that was theirs.
In return, the students swore not to shed a drop of Baba Yaga's blood, nor take
from her anything that was hers. So they all signed, with the Goblet of Fire to
witness it and punish the transgressor."

Professor Quirrell picked up a new ingredient, a loose thread of gold wrapped
around a pinch of foul-looking substance. "Entering her sixth year at Hogwarts,
then, was a witch named Perenelle. And although Perenelle was new-come into the
beauty of her youth, her heart was already blacker than Baba Yaga's own---"

"\emph{You're} calling her evil?" Harry said, then realized he had just
committed the fallacy of \emph{ad hominem tu quoque.}

"Hush, boy, I am telling the story. Where was I? Ah, yes, Perenelle, the
beautiful and covetous. Perenelle seduced the Dark Lady over the months, with
gentle touches and flirtations and the shy pretense of innocence. The Dark
Lady's heart was captured, and they became lovers. And then one night Perenelle
whispered how she had heard of Baba Yaga's shape-changing power and how this
thought had inflamed her desires; thus Perenelle swayed Baba Yaga to come to
her with the Stone in hand, to assume many guises in a single night, for their
pleasures. Among other forms Perenelle bid Baba Yaga take the form of a man;
and they lay together in the fashion of a man and a woman. But Perenelle had
been a virgin until that night. And since they were all rather old-fashioned in
those days, the Goblet of Fire accounted that as the shedding of Perenelle's
blood, and the taking of what was hers; thus Baba Yaga was tricked into being
forsworn, and the Goblet rendered her defenseless. Then Perenelle killed the
unsuspecting Baba Yaga as she slept in Perenelle's bed, killed the Dark Lady
who had loved her and come peacefully to Hogwarts under truce; and that was the
end of the pact by which Dark Wizards and Witches taught Battle Magic at
Hogwarts. For the next few centuries the Goblet of Fire was used to oversee
pointless inter-school tournaments, and then it resided in a disused chamber at
Beauxbatons, until I finally stole it." Professor Quirrell dropped a pale
beige-pink twig into the cauldron, and its color changed to white just as it
touched the surface. "But I digress. Perenelle took the Stone from Baba Yaga,
and assumed the guise and name of Nicholas Flamel. She also kept her identity
as Perenelle, calling herself Flamel's wife. The two have appeared together in
public, but that might be done by any number of obvious methods."

"And the Stone's manufacture?" said Harry, his brain working to process all
this. "I saw an alchemical recipe for it, in a book---"

"Another lie. Perenelle was making it appear as though `Nicholas Flamel' had
earned the right to live forever by completing a great magic that any could
attempt. And she was giving others a false path to pursue, instead of seeking
the one true Stone as Perenelle had sought Baba Yaga's." Professor Quirrell
looked rather sour. "It should come as no surprise that I spent years trying to
master that false recipe. Next you will ask why I did not kidnap, torture, and
kill Perenelle after I learned the truth."

This had not in fact been a question that had come into Harry's mind.

Professor Quirrell continued to speak. "The answer is that Perenelle had
foreseen and forestalled the ambitions of Dark Wizards like myself. `Nicholas
Flamel' publicly took Unbreakable Vows not to be coerced by any means into
relinquishing his Stone---to guard immortality from the covetous, he claimed,
as if that were a public service. I was afraid the Stone would be lost forever,
if Perenelle died without saying where it was hidden, and her Vow prevented
attempts at torture. Further, I had hopes of gaining Perenelle's knowledge, if
I could find the right strategy to extract it from her. Though Perenelle began
with little lore of her own, she has held hostage the lives of wizards greater
than herself, holding out dribs and drabs of healing in exchange for secrets,
and small reversals of age in exchange for power. Perenelle does not condescend
to bestow any real youth upon others---but if you hear of a wizard who lived,
greybearded, to the age of two hundred and fifty, you may be sure that her hand
was in play. By my own generation, the centuries had given Perenelle enough of
an advantage that she could raise up Albus Dumbledore as a counterweight to the
Dark Lord Grindelwald. When I appeared as Lord Voldemort, Perenelle raised up
Dumbledore yet further, parceling out another drop of her hoarded lore whenever
Lord Voldemort seemed to gain an advantage. I felt like I ought to be able to
figure out something clever to do with that situation, but I never did. I did
not attack her directly, for I was not sure of my great creation; it was not
impossible that I would someday need to go begging to her for a dollop of
reversed age." Professor Quirrell dropped two bellflowers at once into the
potion, and they seemed to merge as they touched the bubbling liquid. "But now
I am sure of my creation, and so I have decided that the time has come to take
the Stone by force."

Harry hesitated. "I would like to hear you answer in Parseltongue, was all of
that true?"

"\parsel{None of it iss known to me to be falsse,}" said Professor Quirrell.
"Telling a tale implies filling in certain gaps; I was not present to observe
when Perenelle seduced Baba Yaga. \parsel{The bassicss sshould be mosstly
correct, I think.}"

Harry had noticed a trace of confusion. "Then I don't understand why the
Stone is here in Hogwarts. Wouldn't the best defense just be hiding it under an
anonymous rock in Greenland?"

"Perhaps she respected my abilities as a particularly good finder," said the
Defense Professor. He appeared focused on his cauldron as he dipped a
bellflower into a jar of liquid labeled with the Potions symbol for rainwater.

\emph{We are very much alike, the Defense Professor and I, in some ways if not
others. If I imagine what I'd do, given his problem{\el}}

"Did you bluff everyone into \emph{believing} you had some way of finding the
Stone?" Harry said aloud. "So that Perenelle would put it inside Hogwarts,
where Dumbledore could guard it?"

The Defense Professor sighed, not looking up from the cauldron. "I suppose that
stratagem would be futile to conceal from you. Yes, after I possessed Quirrell
and returned, I implemented a strategy I had conceived while gazing at the
stars. First I made sure to be accepted as Defense Professor at Hogwarts, for
it would not do to have suspicions raised while I was still seeking employment.
When that was done, I arranged for one of Perenelle's curse-breaking
expeditions to discover a falsified but credible inscription describing how the
Crown of the Serpent could be used to seek out the Stone wherever it was
hidden. Immediately after, before Perenelle could buy up the Crown, it was
stolen; furthermore I left clear indications that the thief had possessed the
power to speak to snakes. So Perenelle thought that I could infallibly find the
Stone's location, and that it needed a guardian powerful enough to defeat me.
That is how the Stone came to be held in Hogwarts, in Dumbledore's domain. Just
as I intended, naturally, since I had already gained access to Hogwarts for the
year. I think that is all of this that concerns you, if I speak not of future
plans."

Harry frowned. Professor Quirrell should not have told him that. Unless the
strategy had somehow become irrelevant to any future deception of
Perenelle{\el}? Or unless, by answering so quickly, the Defense Professor
had hoped to have people conclude that it was a double-bluff, and that the
Crown of the Serpent really could find the Stone{\el}

Harry decided not to question this answer in Parseltongue.

Another lock of bright hair, seeming white but not with age, was gently
dribbled into the cauldron, again reminding Harry that they were on a time
limit. Harry considered, but he couldn't see any further path to pursue this
line of questioning; there was no known way to manufacture more Philosopher's
Stones and no obvious way to invent such, which was probably the
\emph{objectively} worst news Harry had heard all day.

Harry took a deep breath. "I ask my third question," Harry said. "What's the
truth behind this entire school year? All the plots you ran, all the plots you
know about."

"Hm," said Professor Quirrell, dropping another bellflower into the potion,
accompanied by a plant-shape like a tiny cross. "Let me see{\el} the most
shocking twist is that the Defense Professor turns out to be secretly
Voldemort."

"Well, obviously," Harry said, with a good deal of self-directed bitterness.

"Then where do you wish me to start?"

"Why did you kill Hermione?" The question just slipped out.

Professor Quirrell's pale eyes glanced up from the potion, watched him
intently. "One would think that should be evident---but I suppose I cannot
blame you for distrusting what seems evident. To understand the object of an
obscure plot, observe its consequences and ask who might have intended them. I
killed Miss Granger to improve your position relative to that of Lucius Malfoy,
since my plans did not call for him to have so much leverage over you. I admit
I am impressed by how far you managed to parlay that opening."

Harry unclenched his teeth, which took an effort. "That's after your failed
attempt to \emph{frame} Hermione for the attempted murder of Draco and
\emph{send her to Azkaban} because of \emph{why?} Because you didn't like the
influence she was having on me?"

"Don't be ridiculous," Professor Quirrell said. "If I had only wished to remove
Miss Granger, I would not have brought the Malfoys into it. I observed your
game with Draco Malfoy and found it amusing, but I knew it could not continue
for very long before Lucius learned and intervened; and then your folly would
have brought you great trouble, for Lucius would not take it lightly. Had you
just been able to \emph{lose} during the Wizengamot trial, \emph{lose} as I had
taught you, then in only two more weeks, ironclad evidence would have shown
that Lucius Malfoy, after discovering his son's seeming perfidy, had Imperiused
Professor Sprout into using the Blood-Cooling Charm on Mr.~Malfoy and casting
the False Memory Charm on Miss Granger. Lucius would have been swept off the
political gameboard, sent to exile if not Azkaban; Draco Malfoy would have
inherited the wealth of House Malfoy, and your influence over him would have
been unchallenged. Instead I had to abort that plot in mid-course. You managed
to completely disrupt the real plan in the course of sacrificing double your
entire fortune, by giving Lucius Malfoy the perfect opportunity to prove his
true concern for his son. You have an incredible anti-talent for meddling, I
must say."

"And you also thought," Harry said, even with his dark side's patterns he had
to work to keep his voice level and cool, "that two weeks in Azkaban would
improve Miss Granger's disposition, and get her to stop being a bad influence
on me. So you somehow arranged for there to be newspaper stories calling for
her to be sent to Azkaban, rather than some other penalty."

Professor Quirrell's lips drew up in a thin smile. "Good catch, boy. Yes, I
thought she might serve as your Bellatrix. That particular outcome would also
have provided you with a constant reminder of how much respect was due the law,
and helped you develop appropriate attitudes toward the Ministry."

"Your plot was stupidly complicated and had no chance of working." Harry knew
he ought to be more tactful, that he was engaging in more of what Professor
Quirrell called \emph{folly,} but in that instant he could not bring himself to
care.

"It was less complicated than Dumbledore's plot to have the three armies tie in
the Christmas Battle, and not much more complicated than my own plot to make
you think Dumbledore had blackmailed Mr.~Zabini. The insight you are missing,
Mr.~Potter, is that these were not plots that \emph{needed} to succeed."
Professor Quirrell continued to casually stir the potion, smiling. "There are
plots that \emph{must} succeed, where you keep the core idea as simple as
possible and take every precaution. There are also plots where it is acceptable
to fail, and with those you can indulge yourself, or test the limits of your
ability to handle complications. It was not as if something going wrong with
any of those plots would have killed me." Professor Quirrell was no longer
smiling. "Our journey into Azkaban was of the first type, and I was less amused
by your antics there."

"What \emph{exactly} did you do to Hermione?" Some part of Harry wondered at
the evenness of his voice.

"Obliviations and False Memory Charms. I could not trust anything else to go
undetected by the Hogwarts wards and the scrutiny I knew her mind would
undergo." A flicker of frustration crossed Professor Quirrell's face. "Part of
what you rightly call complication is because the first version of my plot did
not go as planned, and I had to modify it. I came to Miss Granger in the
hallways wearing the appearance of Professor Sprout, to offer her a conspiracy.
My first attempt at suasion failed. I Obliviated her and tried again with a new
presentation. The second bait failed. The third bait failed. The \emph{tenth}
bait failed. I was so frustrated that I began going through my entire library
of guises, including those more appropriate to Mr.~Zabini. \emph{Still} nothing
worked. The child \emph{would not} violate her childish code."

"\emph{You} do not get to call her childish, Professor." Harry's voice sounded
strange in his own ears. "Her code \emph{worked.} It prevented you from
tricking her. The whole point of having deontological ethical injunctions is
that arguments for violating them are often much less trustworthy than they
look. You don't get to criticize her rules when they worked exactly as
intended." After they resurrected Hermione, Harry would tell her that Lord
Voldemort himself hadn't been able to tempt her into doing wrong, and that was
why he'd killed her.

"Fair enough, I suppose," said Professor Quirrell. "There is a saying that even
a stopped clock is right twice a day, and I do not think Miss Granger was
actually being reasonable. Still, Rule Ten: one must not rant about the
opposition's unworthiness after they have foiled you. Regardless. After two
full hours of failed attempts, I realized that I was being over-stubborn, and
that I did not need Miss Granger to carry out the exact part I had planned for
her. I gave up on my original intent, and instead imbued Miss Granger with
False Memories of watching Mr.~Malfoy plotting against her under circumstances
that implied she should not tell you or the authorities. In the end it was
Mr.~Malfoy who gave me the opening I needed, entirely by luck." Professor
Quirrell dropped a bellflower and a scrap of parchment into the cauldron.

"Why did the wards show the Defense Professor as having killed Hermione?"

"I wore the mountain troll as a false tooth while Dumbledore was identifying me
to the Hogwarts wards as the Defense Professor." A slight smile. "Other living
weapons cannot be Transfigured; they will not survive the disenchantment for
the requisite six hours to avoid being traced by Time-Turner. The fact that a
mountain troll was used as a weapon of assassination was a clear sign that the
assassin had needed a proxy weapon that could be Transfigured safely. Combined
with the evidence of the wards, and Dumbledore's own knowledge of how he had
identified me to Hogwarts, you could have deduced who was responsible---in
theory. However, experience has taught me that such puzzles are far harder to
solve when you do not already know the solution, and I considered it a small
risk. Ah, that reminds me, I have a question of my own." The Defense Professor
was now giving Harry an intent look. "What gave me away at the last, in the
corridor outside these chambers?"

Harry put aside other emotions to weigh up the cost and benefit of answering
honestly, came to the conclusion that the Defense Professor was giving away far
more information than he was getting (\emph{why?}) and that it was best not to
give the appearance of reticence. "The main thing," Harry said, "was that it
was too improbable that everyone had arrived in Dumbledore's corridor at the
same time. I tried running with the hypothesis that everyone who arrived had to
be coordinated, including you."

"But I had said that I was following Snape," the Defense Professor said. "Was
that not plausible?"

"It was, but{\el}" Harry said. "Um. The laws governing what constitutes a
good explanation don't talk about plausible excuses you hear afterward. They
talk about the probabilities we assign in advance. That's why science makes
people do advance predictions, instead of trusting explanations people come up
with afterward. And I wouldn't have predicted in advance for you to follow
Snape and show up like that. Even if I'd known in advance that you could put a
trace on Snape's wand, I wouldn't have \emph{expected} you to do it and follow
him just then. Since your explanation didn't make me feel like I would have
predicted the outcome in advance, it remained an improbability. I started to
wonder if Sprout's mastermind might have arranged for you to show up, too. And
then I realized the note to myself hadn't really come from future-me, and that
gave it away completely."

"Ah," said the Defense Professor, and sighed. "Well, I think it is all working
out for the best. You did understand only too late; and there would have been
inconveniences as well as benefits to you remaining unaware."

"What on \emph{Earth} were you trying to do? The reason I was trying so hard to
figure it out was that the whole thing was just so weird."

"That should have pointed at Dumbledore, not myself," said Professor Quirrell,
and frowned. "The fact is that Miss Greengrass was not supposed to arrive in
that corridor for several hours{\el} though I suppose, since I did have
Mr.~Malfoy give her the clue I assigned her, it is not too surprising they
banded together. Had Mr.~Nott arrived seemingly alone, events would have played
out less farcically. But I consider myself a specialist in battlefield control
magics, and I was able to ensure that the fight went as I wished. I suppose it
did end up looking a bit ridiculous." The Defense Professor dropped a peach
slice and a bellflower into the cauldron. "But let us defer our discussion of
the Mirror until we reach it. Did you have any more questions concerning Miss
Granger's regrettable and hopefully temporary demise?"

"Yes," Harry said in an even voice. "What did you do to the Weasley twins?
Dumbledore thought---I mean, the school saw the Headmaster go to the Weasley
twins after Hermione was arrested. Dumbledore thought you, as Voldemort, had
wondered why Dumbledore had done so, and that you'd checked on the Weasley
twins, found and took their map, and Obliviated them afterward?"

"Dumbledore was quite correct," Professor Quirrell said, shaking his head as
though in wonderment. "He was also an utter fool to leave the Hogwarts Map in
the possession of those two idiots. I had an unpleasant shock after I recovered
the Map; it showed my name and yours correctly! The Weasley idiots had thought
it a mere malfunction, especially after you received your Cloak and your
Time-Turner. If Dumbledore had kept the Map himself---if the Weasleys had ever
spoken of it to Dumbledore---but they did not, thankfully."

\emph{Showed my name and yours correctly---}

"I would like to see that," Harry said.

Without taking his eyes from the cauldron, Professor Quirrell drew a folded
parchment from within his robes, hissed at it "\parsel{Sshow our
ssurroundingss}", and tossed the folded parchment toward Harry. It cut
unerringly through the air, an increase of doom breathing on Harry's senses as
it moved toward him, and then it fluttered gently to Harry's feet.

Harry picked up the parchment and unfolded it.

At first the parchment seemed blank. Then, as though an unseen pen were moving
across it, the outline of walls and doors appeared, all drawn in handwritten
lines. The writing outlined a series of chambers, most of them shown as empty;
the last chamber in the series had a confused scribble in its center, as though
the Map were trying to indicate its own bewilderment; and the second-to-last
chamber showed two names within, written in positions within the chamber
corresponding to where Harry was sitting and Professor Quirrell was standing.

\emph{Tom~M. Riddle.}

\emph{Tom~M. Riddle.}

Harry gazed at the parchment, an unpleasant chill coming over him. It was one
thing to hear Lord Voldemort claim that your name was Tom Riddle; it was
another thing to find that Hogwarts's magic agreed. "\parsel{Did you tamper with
thiss map to achieve thiss ressult, or did it appear before you by ssurprisse?}"

"\parsel{Wass ssurprisse,}" replied Professor Quirrell, with an overtone of
hissing laughter. "\parsel{No trickss.}"

Harry folded the Map and threw it back in Professor Quirrell's direction; some
force caught it in midair before it reached the floor, and drew the Map back
into Professor Quirrell's robes.

The Defense Professor spoke. "I should also like to volunteer that Snape was
guiding Miss Granger and her underlings toward bullies, and sometimes
intervening to protect them."

"I knew that."

"Interesting," said Professor Quirrell. "Did Dumbledore also learn of this?
Answer in Parseltongue."

"\parsel{Not sso far ass I know,}" hissed Harry.

"Fascinating," said Professor Quirrell. "You may be interested to know this as
well: \parsel{Potionss-maker had to work in ssecret because hiss plot oppossed
sschoolmasster's plot.}"

Harry thought about this, while Professor Quirrell blew on the potion as though
to cool it, though the fire still burned under the cauldron; then added a pinch
of dirt and a drop of water and a bellflower. "Please explain," Harry said.

"Has it never occurred to you to wonder why Dumbledore chose Severus Snape as
the Head of House Slytherin? To say that it was a cover for his work as
Dumbledore's spy explains nothing. Snape could have been a Potions Master only,
and not the Head of Slytherin at all. Snape could have been made Keeper of
Grounds and Keys, if he needed to stay within Hogwarts! Why the \emph{Head of
House Slytherin?} Surely it occurred to you that this could not have good
effects upon the Slytherins, according to Dumbledore's moral pretenses?"

The thought hadn't occurred to Harry in \emph{exactly} those terms, no{\el}
"I wondered something like it. I didn't put the dilemma in that precise form."

"And now that you have, is the solution obvious?"

"No," Harry said.

"Disappointing. You have not learned enough cynicism, you have not grasped the
\emph{flexibility} of what moralists call morality. To fathom a plot, look at
the consequences and ask if they might be intended. Dumbledore was deliberately
sabotaging Slytherin House---don't give me that look, boy, \parsel{I am sspeaking
truth.} During the last Wizarding War, Slytherins filled out my ranks of
underlings, and other Slytherins in the Wizengamot supported me. Look at it
from Dumbledore's perspective, and remember that he has no native understanding
of Slytherin's ways. Think of Dumbledore becoming increasingly sad over this
Hogwarts House that seems the source of so much ill-doing. And then behold,
Dumbledore puts in as Head of Slytherin the person of Snape. Snape! Severus
Snape! A man who would teach his House neither cunning nor ambition, a man who
would impose lax discipline and make its children weak! A man who would offend
students of other Houses, who would ruin Slytherin's name among them! A man
whose surname was unknown in magical Britain and certainly not noble, who went
about half in rags! Do you think Dumbledore ignorant of the consequence? When
Dumbledore was the one who brought it about, and had motive to bring it about?
I expect Dumbledore told himself that more lives would be saved during the next
Wizarding War if Voldemort's future Death Eaters were weakened." Professor
Quirrell dropped into the cauldron a chip of ice, slowly melting as it touched
the surface froth. "Continue the process long enough, and no child would want
to go to Slytherin. The House would be retired, and if the Hat kept calling the
name, it would become a mark of ignominy among children who would afterward be
distributed among the other three Houses. From that day on, Hogwarts would have
three upstanding Houses of courage and scholarship and industry, with no House
of Bad Children added to the mix; just as if the three Founders of Hogwarts had
been wise enough in the beginning to refuse Salazar Slytherin their company.
That, I expect, was Dumbledore's intended end-game; a short-term sacrifice for
the greater good." Professor Quirrell smiled sardonically. "And Lucius let it
all happen without protest or even, I expect, \emph{noticing} that anything was
going awry. I fear that in my absence my former servants have been quite
outmatched in this battle of wits."

Harry was having a bit of trouble taking this in, but decided, after some
thought, that now was not the time to try to work it out. Whether Lord
Voldemort believed it was not decisive; Harry would have to evaluate this
accusation on his own.

Professor Quirrell's mention of his \emph{servants} had reminded Harry of
something else that he was{\el} obligated, Harry supposed, to ask. The bad
news was predictable. On any other day it would have been horrible. Today it
would just wash out in the flood. "Bellatrix Black," Harry said. "What was the
truth about her?"

"She was broken inside before I ever met her," Professor Quirrell said. He
picked up what looked like a white-grey rubber band and held it over the
cauldron; as the rubber was held within the steam, it turned black. "Using
Legilimency on her was a mistake. But that glimpse showed me how easy it would
be to make her fall in love with me, so I did. Ever after she was the most
faithful of all my servants, the only one I could almost trust. I had no
intention of giving her what she wanted from me; so I commended her to the
Lestrange brothers for their use, and the three of them were happy in their own
special way."

"I doubt it," Harry's mouth said, mostly on autopilot. "If that were true,
Bellatrix wouldn't have remembered who the Lestrange brothers were, when we
found her in Azkaban."

Professor Quirrell shrugged. "You may be right."

"What the hell were we actually doing there?"

"Finding out where Bellatrix had put my wand. I had told the Death Eaters of my
immortality, in the hope---now proven futile---that they would stay together
for at least a few \emph{days} if I appeared to die. Bellatrix's instructions
were to recover my wand from wherever my body had been slain; and take that
wand to a certain graveyard where my spirit would appear before her."

Harry swallowed. The image came to him of Bellatrix Black waiting, waiting,
waiting at the graveyard, in increasing desperation{\el} it was no wonder
she hadn't been thinking strategically when she attacked the Longbottom
household. "What did you do with Bellatrix once she was out?"

"\parsel{Ssent her to a peaceful place to recover sstrength}," Professor Quirrell
said. A cold smile. "I had a use remaining for her, or rather a certain portion
of her, and on my future plans I shall not answer questions."

Harry breathed deeply, trying to maintain control. "Were there any other secret
plots in this school year?"

"Oh, a fair number, but not many more that concern you, not that I can think of
offhand. The true reason I demanded to try to teach the Patronus Charm to
first-years was to bring a Dementor before your own person, and then I arranged
for your wand to fall where the Dementor could continue to drain you through
it. \parsel{Wass no malice in it, only hopess that you would recover ssome of
your true memoriess.} That was also why I arranged for certain witches to pull
you down from the air during your rooftop episode, so I could appear to save
your life; just in case any suspicion fell on me during the Dementor incident I
had scheduled for shortly after. \parsel{Alsso no malice there.} I arranged some
of the attacks on Miss Granger's group, so that the attacks could be defeated;
I do rather dislike bullies. \parsel{Think that iss all ssecret plotss concerning
you from thiss sschool-year, unless I have forgotten ssomething.}"

\emph{Life lesson learned,} said his Hufflepart. \emph{Try to resist the
temptation to randomly meddle in other people's lives. Like, you know, Padma
Patil's life. If you don't want to end up like this, that is.}

A pinch of red-brown dust was gently sifted into the potions cauldron, and
Harry asked his fourth and final question, the one that had seemed to have the
lowest priority, but still mattered.

"What was your objective during the Wizarding War?" Harry said. "I mean,
what---" His voice wobbled. "What was the \emph{point} of the \emph{entire
thing?}" His brain repeating endlessly, \emph{Why, why, why Lord
Voldemort{\el}}

Professor Quirrell lifted an eyebrow. "They told you about David Monroe, did
they not?"

"Yes you were both David Monroe and Lord Voldemort during the Wizarding War, I
understood that part. You killed David Monroe, disguised yourself as him, and
wiped out David Monroe's family so they wouldn't notice any differences---"

"Indeed."

"You planned to control whichever side won the Wizarding War, regardless of
which side won. But why did one side have to be \emph{Voldemort?} I, I mean,
wouldn't it have been easier to gain public support with someone less{\el}
with someone less Voldemort?"

Professor Quirrell's mallet made an unusually loud \emph{thud} as it crushed
white butterfly wings, mixing them with another bellflower. "I \emph{planned,}"
Professor Quirrell said harshly, "for Lord Voldemort to \emph{lose} to David
Monroe. The flaw in that strategy was the absolute wretchedness of---"
Professor Quirrell stopped. "No, I am telling the tale out of order. Listen,
boy, when I had devised my great creation and come into the fullness of my
magic, I thought the time had come for me to take political power into my
hands. It would be inconvenient, certainly, and take up my time in ways that
were not enjoyable. But I knew the Muggles would eventually destroy the world
or make war on wizardkind or both, and something had to be done if I were not to
wander a dead or dull world through my eternity. Having attained immortality I
needed a new ambition to occupy my decades, and to prevent the Muggles from
ruining everything seemed a goal of acceptable scope and difficulty. It is a
source of continual amusement to me that I, of all people, am the only one
really taking action towards that end. Though I suppose it would make sense for
the mortal insects not to care about their world's end; why should they, when
they are just going to die regardless, and can save themselves the
inconvenience of trying to do anything difficult along the way? But I digress.
I saw how Dumbledore had risen to power from his defeat of Grindelwald, so I
thought I would do the same. I had long ago taken my vengeance on David
Monroe---he was an annoyance from my year in Slytherin---so I bethought to also
steal his identity, and wipe out his family to make myself heir of his House.
And I conceived also a great foe for David Monroe to fight, the most terrifying
Dark Lord imaginable, clever beyond reckoning; more dangerous by far than
Grindelwald, for his intelligence would be perfected in all the ways that
Grindelwald had been flawed and self-destructive. A Dark Lord who would do his
cunning utmost to disrupt the alliances who would fight him, a Dark Lord who
would command the deepest loyalty from his followers through his oratorical
skills. The most dreadful Dark Lord who had ever threatened Britain or the
world, that was who David Monroe would defeat."

Professor Quirrell's mallet struck a bellflower and then a different pale
flower with two more thuds. "But then, while I had sometimes played the part of
Dark Wizard in my wanderings, I had never adopted the identity of a
full-fledged Dark Lord with underlings and a political agenda. I had no
practice at the task, and I was mindful of the story of Dark Evangel and the
disaster of her first public appearance. According to what she said afterward,
she had meant to call herself the Walking Catastrophe and the Apostle of
Darkness, but in the excitement of the moment she introduced herself as the
Apostrophe of Darkness instead. After that she had to ruin two entire villages
before anyone took her seriously."

"So you decided to try a small-scale experiment first," Harry said. A sickness
rose up in him, because in that moment Harry \emph{understood,} he saw himself
reflected; the next step was just what Harry himself would have done, if he'd
had no trace of ethics whatsoever, if he'd been that empty inside. "You created
a disposable identity, to learn how the ropes worked, and get your mistakes out
of the way."

"Indeed. Before becoming a truly terrible Dark Lord for David Monroe to fight,
I first created for practice the persona of a Dark Lord with glowing red eyes,
pointlessly cruel to his underlings, pursuing a political agenda of naked
personal ambition combined with blood purism as argued by drunks in Knocturn
Alley. My first underlings were hired in a tavern, given cloaks and skull
masks, and told to introduce themselves as Death Eaters."

The sick sense of understanding deepened, in the pit of Harry's stomach. "And
you called yourself Voldemort."

"Just so, General Chaos." Professor Quirrell was grinning, from where he stood
by the cauldron. "I wanted it to be an anagram of my name, but that would only
have worked if I'd conveniently been given the middle name of `Marvolo', and
then it would have been a stretch. Our actual middle name is Morfin, if you're
curious. But I digress. I thought Voldemort's career would last only a few
months, a year at the longest, before the Aurors brought down his underlings
and the disposable Dark Lord vanished. As you perceive, I had vastly
overestimated my competition. And I could not \emph{quite} bring myself to
torture my underlings when they brought me bad news, no matter what Dark Lords
did in plays. I could not \emph{quite} manage to argue the tenets of blood
purism as incoherently as if I were a drunk in Knockturn Alley. I was not
trying to be clever when I sent my underlings on their missions, but neither
did I give them entirely pointless orders---" Professor Quirrell gave a rueful
grin that, in another context, might have been called charming. "One month
after that, Bellatrix Black prostrated herself before me, and after three
months Lucius Malfoy was negotiating with me over glasses of expensive
Firewhiskey. I sighed, gave up all hope for wizardkind, and began as David
Monroe to oppose this fearsome Lord Voldemort."

"And then what happened---"

A snarl contorted Professor Quirrell's face. "The absolute inadequacy of every
single institution in the civilization of magical Britain is what happened! You
cannot comprehend it, boy! I cannot comprehend it! It has to be seen and even
then it cannot be believed! You will have observed, perhaps, that of your
fellow students who speak of their family's occupations, three in four seem to
mention jobs in some part or another of the Ministry. You will wonder how a
country can manage to employ three of its four citizens in bureaucracy. The
answer is that if they did not all prevent each other from doing their jobs,
none of them would have any work left to do! The Aurors were competent as
individual fighters, they did fight Dark Wizards and only the best survived to
train new recruits, but their leadership was in absolute disarray. The Ministry
was so busy routing papers that the country had \emph{no} effective opposition
to Voldemort's attacks except myself, Dumbledore, and a handful of untrained
irregulars. A shiftless, incompetent, cowardly layabout, Mundungus Fletcher,
was considered a key asset in the Order of the Phoenix---because, being
otherwise unemployed, he did not need to juggle another job! I tried weakening
Voldemort's attacks, to see if it was \emph{possible} for him to lose; at once
the Ministry committed fewer Aurors to oppose me! I had read Mao's Little Red
Book, I had trained my Death Eaters in guerilla tactics---for nothing! For
nothing! I was attacking all of magical Britain and in every engagement my
forces \emph{outnumbered} their opposition! In desperation, I ordered my Death
Eaters to systematically assassinate every single incompetent managing the
Department of Magical Law Enforcement. One paper-pusher after another
volunteered to accept higher positions despite the fate of their predecessors,
gleefully rubbing their hands at the prospect of promotion. Every one of them
thought they would cut a deal with Lord Voldemort on the side. It took
\emph{seven months} to murder our way through them all, and not a single Death
Eater asked why we were bothering. And then, even with Bartemius Crouch risen
to Director and Amelia Bones as Head Auror, it was still too little. I could
have done better fighting \emph{alone.} Dumbledore's aid was not worth his
moral restraints, and Crouch's aid was not worth his respect for the law."
Professor Quirrell turned up the fire beneath the potion.

"And eventually," Harry said through the heart-sickness, "you realized you were
just having more fun as Voldemort."

"It is the least annoying role I have ever played. If Lord Voldemort says that
something is to be done, people \emph{obey him} and \emph{do not argue.} I did
not have to suppress my impulse to Cruciate people being idiots; for once it
was all part of the role. If someone was making the game less pleasant for me,
I just said \emph{Avadakedavra} regardless of whether that was strategically
wise, and they never bothered me again." Professor Quirrell casually chopped a
small worm into bits. "But my true epiphany came on a certain day when David
Monroe was trying to get an entry permit for an Asian instructor in combat
tactics, and a Ministry clerk denied it, smiling smugly. I asked the Ministry
clerk if he understood that this measure was meant to \emph{save his life} and
the Ministry clerk only smiled more. Then in fury I threw aside masks and
caution, I used my Legilimency, I dipped my fingers into the cesspit of his
stupidity and \emph{tore} out the truth from his mind. I did not understand and
I \emph{wanted to understand.} With my command of Legilimency I forced his tiny
clerk-brain to live out alternatives, seeing what his clerk-brain would think
of Lucius Malfoy, or Lord Voldemort, or Dumbledore standing in my place."
Professor Quirrell's hands had slowed, as he delicately peeled bits and small
strips from a chunk of candle-wax. "What I finally realized that day is
complicated, boy, which is why I did not understand it earlier in life. To you
I shall try to describe it anyway. Today I know that Dumbledore does not stand
at the top of the world, for all that he is the Supreme Mugwump of the
International Confederation. People speak ill of Dumbledore openly, they
criticize him proudly and to his face, in a way they would not dare stand up to
Lucius Malfoy. \emph{You} have acted disrespectfully toward Dumbledore, boy, do
you know why you did so?"

"I'm{\el} not sure," Harry said. Having Tom Riddle's leftover neural
patterns was certainly an obvious hypothesis.

"Wolves, dogs, even chickens, fight for dominance among themselves. What I
finally understood, from that clerk's mind, was that to him Lucius Malfoy had
dominance, Lord Voldemort had dominance, and David Monroe and Albus Dumbledore
did not. By taking the side of good, by professing to abide in the light, we
had made ourselves \emph{nonthreatening.} In Britain, Lucius Malfoy has
dominance, for he can call in your loans, or send Ministry bureaucrats against
your shop, or crucify you in the \emph{Daily Prophet,} if you go openly against
his will. And the most powerful wizard in the world has no dominance, because
everyone knows that he is," Professor Quirrell's lips curled, "\emph{a hero out
of stories,} relentlessly self-effacing and too humble for vengeance. Tell me,
child, have you ever seen a drama where the hero, before he consents to save
his country, demands so much gold as a barrister might receive for a court
case?"

"Actually there have been a \emph{lot} of heroes like that in Muggle fiction,
I'll name Han Solo just to start---"

"Well, in magical drama it is not so. It is all humble heroes like Dumbledore.
It is the fantasy of the powerful \emph{slave} who will never truly rise above
you, never demand your respect, never even ask you for pay. Do you understand
now?"

"I{\el} think so," Harry said. Frodo and Samwise from \emph{Lord of the
Rings} did seem to match the archetype of a completely non-threatening hero.
"You're saying that's how people think of Dumbledore? I don't believe the
Hogwarts students see him as a hobbit."

"In Hogwarts, Dumbledore does punish certain transgressions against his will,
so he is feared to some degree---though the students still make free to mock
him in more than whispers. Outside this castle, Dumbledore is sneered at; they
began to call him mad, and he aped the part like a fool. Step into the role of
a savior out of plays, and people see you as a slave to whose services they are
entitled and whom it is their enjoyment to criticize; for it is the privilege
of masters to sit back and call forth helpful corrections while the slaves
labor. Only in the tales of the ancient Greeks, from when men were less
sophisticated in their delusions, may you see the hero who is also high.
Hector, Aeneas, those were heroes who retained their right of vengeance upon
those who insulted them, who could demand gold and jewels in payment for their
services without sparking indignation. And if Lord Voldemort conquered Britain,
he might then condescend to show himself noble in victory; and nobody would
take his goodwill for granted, nor chirp corrections at him if his work was not
to their liking. When he won, he would have \emph{true} respect. I understood
that day in the Ministry that by envying Dumbledore, I had shown myself as
deluded as Dumbledore himself. I understood that I had been trying for the
wrong place all along. You should know this to be true, boy, for you have made
freer to speak ill of Dumbledore than you ever dared speak ill of me. Even in
your own thoughts, I wager, for instinct runs deep. You knew that it might be
to your cost to mock the strong and vengeful Professor Quirrell, but that there
was no cost in disrespecting the weak and harmless Dumbledore."

"Thank you," Harry said through the pain, "for that valuable lesson, Professor
Quirrell, I see that you are right about what my mind was doing." Though Tom
Riddle's memories had probably also had something to do with the way he had
sometimes lashed out at Dumbledore for no good reason, Harry hadn't been like
that around Professor McGonagall{\el} who admittedly had the power to deduct
House Points and didn't have Dumbledore's air of tolerance{\el} no, it was
still true, Harry would have been more respectful even in his own thoughts if
Dumbledore had not seemed \emph{safe} to disrespect.

So that had been David Monroe, and that had been Lord Voldemort{\el}

It still hadn't answered the most puzzling question, and Harry wasn't sure that
asking it would be wise. If, somehow, Lord Voldemort had managed \emph{not to
think of it,} and then Professor Quirrell had still managed not to think of it
during nine years of contemplation, then it wasn't wise to say{\el} or maybe
it was; the agonies of the Wizarding War had not been good for Britain.

Harry decided, and spoke. "One thing that did confuse me was why the Wizarding
War lasted so long," Harry ventured. "I mean, maybe I'm underestimating the
difficulties that were facing Lord Voldemort---"

"You want to know why I did not Imperius some of the stronger wizards who could
Imperius others, slay the very strongest wizards who could have resisted my
Imperius, and take over the Ministry in, oh, perhaps three days."

Harry nodded silently.

Professor Quirrell looked contemplative; his hand was sifting grass clippings
into the cauldron, bit by bit. That ingredient, if Harry remembered correctly,
was something like four-fifths towards the end of the recipe.

"I wondered that myself," the Defense Professor said finally, "when I heard
Trelawney's prophecy from Snape, and I contemplated the past as well as the
future. If you had asked my past self why he did not use the Imperius, he would
have spoken of the need to be \emph{seen} to rule, to openly command the
Ministry bureaucracy, before it was time to turn his eyes outward to other
countries. He would have remarked on how a quick and silent victory might bring
challenges later. He would have remarked on the obstacle presented by
Dumbledore and his incredible defensive prowess. And he would have had similar
excuses for every other quick path he considered. Somehow it was never the
right time to bring my plans to their final phase, there was always one more
thing to do first. Then I heard the prophecy and I \emph{knew} that it was
time, for Time itself was taking notice of me. That the span for hesitation was
done. And I looked back, and realized somehow this had been going on for years.
I think{\el}" The occasional bit of grass was still dropping down from his
hand, but Professor Quirrell did not seem to pay it any mind. "I thought, when
I was contemplating my past beneath the starlight, that I had become too
accustomed to playing against Dumbledore. Dumbledore was intelligent, he tried
diligently to be cunning, he did not wait for me to strike but presented me
with surprises. He made bizarre moves that played out in fascinating and
unpredictable ways. In retrospect, there were many obvious plans for destroying
Dumbledore; but I think some part of me did not want to go back to playing
solitaire instead of chess. It was when I had the prospect of creating another
Tom Riddle to plot against, someone even more worthy than Dumbledore, that I
was first willing to contemplate the end of my war. Yes, in retrospect that
sounds stupid, but sometimes our emotions are more foolish than we can bring
our reason to admit. I would never have espoused such a policy deliberately. It
would have violated Rules Nine, Sixteen, Twenty, and Twenty-two and that is too
much even if you are enjoying yourself. But to repeatedly decide that there was
one more thing left to be done, one more advantage left to be gained, one more
piece that I simply \emph{had} to move into place, before abandoning an
enjoyable time in my life and moving on to the more tedious rulership of
Britain{\el} well, even I am not immune to a mistake like that, if I do not
realize that I am making it."

And that was when Harry knew what was going to happen at the end of this, after
the Philosopher's Stone had been retrieved.

At the end of this, Professor Quirrell was going to kill him.

Professor Quirrell didn't want to kill him. It was possible that Harry was the
only person in the world against whom Professor Quirrell \emph{wouldn't} be
able to use a Killing Curse. But Professor Quirrell thought he had to do it,
for whatever reason.

That was why Professor Quirrell had decided that it was necessary to brew the
\emph{potion of effulgence} the long way. That was why Professor Quirrell had
been so easily negotiated into answering these questions, into finally talking
about his life with someone who might understand. Just like Lord Voldemort had
delayed the end of the Wizarding War to play longer against Dumbledore.

Harry couldn't exactly recall what Professor Quirrell had said earlier about
not killing Harry. It hadn't been anything straightforward along the lines of
`I am absolutely not planning to kill you in any way, shape, or form unless you
positively insist on doing something stupid'. Harry had been reluctant himself
to push the promise too far and insist on unambiguous terms because Harry had
already known that he would need to neutralize Lord Voldemort and had expected
more precise language to reveal that fact, if they tried to exchange truly
binding promises. So there certainly would have been loopholes, whatever had
been said.

There was no particular shock to the realization, just an increased sense of
urgency; some part of Harry had already known this, and had simply been waiting
for an excuse to make it known to deliberation. There had been too many things
said here that Professor Quirrell would not reveal to anyone with an expected
lifespan measured in more than hours. The overwhelming isolation and loneliness
of the life Professor Quirrell had described might explain why he was willing
to violate his Rules and talk with Harry, \emph{given} that Harry was going to
die soon and that the world did not actually work like a play where the villain
disclosing his plans would always fail to kill the hero afterward. But Harry's
death certainly had to be in those future plans somewhere.

Harry swallowed, controlling his breathing. Professor Quirrell had just added a
tuft of horsehair to the \emph{potion of effulgence,} and that was very late in
the potion, if Harry remembered correctly. There weren't many bellflowers left
in the heap to be added, either.

It was probably time to stop worrying so much about risk and play this
conversation less conservatively, all things considered.

"If I point out one of Lord Voldemort's mistakes," Harry said, "does he punish
me for it?"

Professor Quirrell lifted his eyebrows. "Not if the mistake is a real one. I do
not suggest that you moralize at me. But I would not curse the bearer of bad
news, nor the subordinate who makes an honest attempt to point out a problem.
Even as Lord Voldemort I could never bring myself to that stupidity. Of course,
there were some fools who mistook my policy for weakness, who tried to thrust
themselves forward by pushing me down in their public counsel, thinking me
obliged to tolerate it as criticism." Professor Quirrell smiled reminiscently.
"The Death Eaters were better off without them, and I do not advise you make
the same mistake."

Harry nodded, a slight shiver going through him. "Um, when you told me about
what happened in Godric's Hollow, on Halloween night, in 1981 I mean,
um{\el} I thought I saw another flaw in your reasoning. A way you could have
avoided disaster. But, um, I think you have a blind spot, a class of strategies
you don't consider, so you didn't see it even afterward---"

"I hope you are not about to say anything stupid along the lines of `don't try
to kill people'," Professor Quirrell said. "I shall be unhappy if that is the
case."

"\parsel{Not valuess difference. True misstake, given your goalss. Will you hurt
me, if I act the part of the teacher toward you, and teach lessson? Or if
misstake is ssimple and obviousss, and makess you feel sstupid?}"

"\parsel{No,}" hissed Professor Quirrell. "\parsel{Not if lessson iss true.}"

Harry swallowed. "Um. Why didn't you test the Horcrux system before you
actually had to use it?"

"Test it?" said Professor Quirrell. He looked up from the brewing potion, and
indignation came into his voice. "What do you mean, \emph{test it?}"

"Why didn't you test if the Horcrux system was working correctly, before you
needed it on Halloween?"

Professor Quirrell looked disgusted. "You ridiculous---I didn't want to
\emph{die,} Mr.~Potter, and that was the only way to test my great creation!
What good would it have done to risk my life sooner rather than later? How
would I have been better off?"

Harry swallowed a lump in his throat. "\parsel{There wass way for you to tesst
your Horcrux ssysstem without dying.} The general lesson is important. Do you
see it now?"

"No," Professor Quirrell said after a while. The Defense Professor gently
crumbled one of the last bellflowers together with a strand of long blonde hair
and then dropped it into the potion, which was bubbling brighter, now. Only two
more bellflowers remained on the Potions table. "And I do hope your lesson is a
sensible one, for your sake."

"Suppose, Professor, that I learned how to cast the improved Horcrux spell and
I was willing to use it. What would I do with it?"

Professor Quirrell answered at once. "You would find some person whom you found
morally abhorrent and whose death you could convince yourself would save other
lives, and murder them to create a Horcrux."

"And then what?"

"Make more Horcruxes," said the Defense Professor. He picked up a jar of what
looked like dragon scales.

"Before that," Harry said.

After a time the Defense Professor shook his head. "I still do not see it, and
you will cease this game and tell me."

"I would make Horcruxes for my friends. If you'd ever really cared about one
single other person in the entire world, if there'd been just one person who
gave your immortality \emph{meaning,} someone that you wanted to live forever
\emph{with} you---" Harry's throat choked. "Then, then the idea of making a
Horcrux for someone else wouldn't have been such a counterintuitive thought."
Harry was blinking hard. "You have a blind spot around strategies that involve
doing nice things for other people, to the point where it stops you from
achieving your selfish values. You think{\el} it's not your style, I
suppose. That{\el} particular part of your self-image{\el} is what cost
you those nine years."

The dropper of mint oil that the Defense Professor was holding added liquid to
the cauldron, drip by drip.

"I see{\el}" the Defense Professor said slowly. "I see. I should have taught
Rabastan the advanced Horcrux ritual, and forced him to test the invention.
Yes, that is supremely obvious in retrospect. For that matter, I could have
ordered Rabastan to try marking himself onto some disposable infant, to see
what happened, before I took myself to Godric's Hollow to create you."
Professor Quirrell shook his head bemusedly. "Well. I am glad I am realizing
this now and not ten years earlier; I had enough to chide myself for at that
time."

"You don't see nice ways to do \emph{the things you want to do,}" Harry said.
His ears heard a note of desperation in his own voice. "Even when a nice
strategy would be \emph{more effective} you don't see it because you have a
self-image of \emph{not being nice.}"

"That is a fair observation," said Professor Quirrell. "Indeed, now that you
have pointed it out, I have just now thought of some nice things I can do this
very day, to further my agenda."

Harry just looked at him.

Professor Quirrell was smiling. "Your lesson is a good one, Mr.~Potter. From
now on, until I learn the trick of it, I shall keep diligent watch for cunning
strategies that involve doing kindnesses for other people. Go and practice acts
of goodwill, perhaps, until my mind goes there easily."

Cold chills ran down Harry's spine.

Professor Quirrell had said this without the slightest visible hesitation.

Lord Voldemort was absolutely certain that he could never be redeemed. He
wasn't the tiniest bit afraid of it happening to him.

The second-to-last bellflower was dropped into the potion, gently.

"Any other valuable lessons you would like to teach to Lord Voldemort, boy?"
said Professor Quirrell. He was looking up from the potion, and grinning as
though he knew exactly what Harry was thinking.

"Yes," Harry said, his voice almost breaking. "If your goal is to obtain
happiness, then doing nice things for other people feels better than doing them
for yourself---"

"Do you \emph{really} think I never thought of that, boy?" The smile had
vanished. "Do you think I am stupid? After graduating Hogwarts I wandered the
world for years, before I returned to Britain as Lord Voldemort. I have put on
more faces than I bothered counting. Do you think I never tried to play the
hero, just to see how it would feel? Have you come across the name of Alexander
Chernyshov? Under that guise, I sought out a forlorn hellhole ruled over by a
Dark Wizard, and I freed the wretched inhabitants from their bondage. They wept
tears of gratitude for me. It did not feel like anything in particular. I even
stayed about and killed the next five Dark Wizards to try taking command of the
place. I spent my own Galleons---well, not my own Galleons, but the same
principle applies---to prettify their little country and introduce a semblance
of order. They groveled all the more, and named one in three of their infants
Alexander. I still felt nothing, so I nodded to myself, wrote it off as a fair
try, and went upon my way."

"And were you happy as Lord Voldemort, then?" Harry's voice had risen, grown
wild.

Professor Quirrell hesitated, then shrugged. "It appears you already know the
answer to that."

"Then \emph{why?} Why be Voldemort if it \emph{doesn't even make you happy?}"
Harry's voice broke. "I'm \emph{you}, I'm based on you, so \emph{I know} that
Professor Quirrell isn't just a mask! I \emph{know} he's somebody you really
could have been! Why not just stay that way? Take your curse off the Defense
Position and just \emph{stay here}, use the Philosopher's Stone to take David
Monroe's shape and let the real Quirinus Quirrell go free, if you say you'll
stop killing people I'll swear not to tell anyone who you really are,
just \emph{be Professor Quirrell,} for always! Your students \emph{would}
appreciate you, my father's students appreciate \emph{him}---"

Professor Quirrell was chuckling over the cauldron as he stirred it. "There are
perhaps fifteen thousand wizards living in magical Britain, child. There used
to be more. There's a reason they're afraid to speak my name. You'd forgive me
that because you liked my Battle Magic lessons?"

\emph{Seconded,} said Harry's inner Hufflepuff. \emph{Seriously, what the hell?}

Harry kept his head raised, though it was trembling. "It's not my place to
forgive anything you've done. But it's better than another war."

"Ha," said the Defense Professor. "If you ever find a Time-Turner that goes
back forty years and can alter history, be sure to tell Dumbledore that before
he rejects Tom Riddle's application for the Defense position. But alas, I fear
that Professor Riddle would not have found lasting happiness in Hogwarts."

"\emph{Why not?}"

"Because I still would've been surrounded by idiots, and I wouldn't have been
able to kill them," Professor Quirrell said mildly. "Killing idiots is my great
joy in life, and I'll thank you not to speak ill of it until you've tried it
for yourself."

"There's \emph{something} that would make you happier than that," Harry said,
his voice breaking again. "There has to be."

"Why?" said Professor Quirrell. "Is this some scientific law I have not yet
encountered? Tell me of it."

Harry opened his mouth, but couldn't find any words, there had to be something
\emph{had to be something} if he could just find the right thing to say---

"And \emph{you}," said Professor Quirrell, "have no right to speak of happiness
either. Happiness is not what you hold precious above all. You decided that in
the beginning, all the way back in the beginning of this year, when the Sorting
Hat offered you Hufflepuff. Which I know about, because I received a similar
offer and warning all those years ago, and I refused it just as you did. Beyond
this there is little more to say, between Tom Riddles." The Defense Professor
turned back to the cauldron.

Before Harry could think of any way to reply, Professor Quirrell dropped in the
last bellflower, and a burst of glowing bubbles boiled up from the cauldron.

"I believe we are done here," Professor Quirrell said. "If you have further
questions, they must wait."

Harry shakily rose to his feet; even as Professor Quirrell took up the cauldron
and poured out a ridiculously huge volume of effulgent liquid, more than seemed
like it could fit in a dozen cauldrons, onto the purple fire that guarded the
doorway.

The purple fire winked out.

"Now for the Mirror," said Professor Quirrell, and he drew forth the Cloak of
Invisibility from his robes, and floated it to drop before Harry's shoes.
