% We want the total number of real pages (sheets) to be a multiple of 4, and we
% want the colophon to land on the last page (which will necessarily be an even
% page). So we might need to insert up to 3 blank pages before the colophon.
%
% Note that the \include command used to include this file makes a new page, so
% if the current page here is a multiple of 4, we don't need to insert any blank
% page.
%
% First we calculate the current page number modulo 4:
\makeatletter
\@tempcnta=\c@sheetsequence% The sheetsequence counter defined by the memoir
class
\newcount\blobb \blobb=\c@sheetsequence \divide\@tempcnta by 4
\multiply\@tempcnta by 4 \multiply\@tempcnta by -1 \advance\@tempcnta by
\c@sheetsequence
%
% Now @tempcnta holds @sheetsequence % 4. We will add pages and increment
% @tempcnta until it reaches 4 (unless it is 0, in which case there is no page
% to add).
\ifnum\@tempcnta>0
\loop
\thispagestyle{empty}% Disable headers/footers
\null% There must be "something" on the current page for \newpage to have an
effect
\newpage% Make new page
\advance\@tempcnta by 1 \ifnum\@tempcnta<4
\repeat
\fi
\makeatother
%
% Now the current page is a muliple of 4. We put the colophon here.
{\null\vfil%
\thispagestyle{empty}\noindent\rightskip=0pt plus 1fil\leftskip=0pt plus
1fil\parfillskip=0pt%
\vbox{ \parshape 1 0em 21em
\noindent
\leftskip=0pt\rightskip=0pt\parfillskip=0pt plus 1fil{}\emph{Dieses Buch wurde
von Fans der Geschichte formatiert. Der Text wurde in 10½~Punkt
URW~Garamond~No.~8 geschrieben. Die Parseltongue wurde in Alegreya Sans Light
Italic von Huerta Tipografica geschrieben. Die Kapitelüberschriften wurden in
Lumos von Sarah McFalls geschrieben, inspiriert von der Schriftart, die in den
US-Ausgaben der Harry-Potter-Bücher verwendet wird. Das Cover wurde von
\CoverAuthor{} erstellt. Der Schriftsatz erfolgte mit \emph{\LaTeX}; der Quellcode
kann unter \mbox{\url{\PdfSourceUrl}} gefunden werden. Dieses Buch wurde am
\today{} erstellt.}}\vskip\footskip%
\vfil\vfil \newpage }
