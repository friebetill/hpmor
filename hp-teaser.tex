\thispagestyle{empty}
{
    \small
    \parindent=0pt
    \begin{center}
        \emph{Irgendetwas, irgendwo, irgendwann, muss anders gewesen sein\el}
    \end{center}

    \MakeUppercase{Petunia Evans} heiratete Michael Verres, einen Professor
    für Biochemie in Oxford.

    \bigskip
    \MakeUppercase{Harry James Potter-Evans-Verres} wuchs in einem Haus auf,
    das bis zum Rand mit Büchern gefüllt war. Er hat einmal einen Mathelehrer
    gebissen, der nicht wusste, was ein Logarithmus ist. Er hat \emph{Gödel, Escher,
    Bach} und \emph{Judgment Under Uncertainty: Heuristics and Biases} und Band
    eins der \emph{Feynman Lectures on Physics} gelesen. Und entgegen den
    Befürchtungen aller, die ihn kennen gelernt haben, will er nicht der
    nächste Dunkle Lord werden. Er wurde zu etwas Besserem erzogen. Er will
    die Gesetze der Magie entdecken und ein Gott werden.

    \bigskip
    \MakeUppercase{Hermine Granger} ist in allen Fächern besser als er, außer
    beim Besenreiten.

    \bigskip
    \MakeUppercase{Draco Malfoy} ist genau das, was man von einem elfjährigen
    Jungen erwarten würde, wenn Darth Vader sein vernarrter Vater wäre.

    \bigskip
    \MakeUppercase{Professor Quirrell} lebt seinen Lebenstraum, Verteidigung
    gegen die dunklen Künste zu unterrichten, oder, wie er es vorzieht, seine
     Klasse zu nennen, Kampfmagie. Seine Schüler fragen sich alle, was diesmal
      mit dem Verteidigungsprofessor schiefgehen wird.

    \bigskip
    \MakeUppercase{Dumbledore} ist entweder verrückt, oder er spielt ein viel
    tiefer gehendes Spiel, bei dem es darum geht, ein Huhn in Brand zu setzen.

    \bigskip
    \MakeUppercase{Deputy Headmistress Minerva Mcgonagall} muss irgendwo
    hingehen und eine Weile schreien.

    \begin{center}
        Präsentiert:

        \bigskip
        \MakeUppercase{Harry Potter und die Methoden der Rationalität}

        \bigskip
        Du wirst nicht erraten, wohin die Reise führt.
    \end{center}
}
