\documentclass[11pt,extrafontsizes,twoside,openany,final]{memoir} % 11pt is scaled down to 10.5 in \setmainfont call

% THIS DOCUMENT REQUIRES LuaLaTeX TO COMPILE!
% XeLaTeX should work (if the \ShowLoosenessHelp hand-tuning macro is not used)
% but there might be formatting artefacts. In particular, it seems that the
% letter and word spacing settings are processed differently.

\usepackage{lettrine}		% Used for the fancy caps at the start of each chapter
\usepackage{amsmath}		% Provides the align environment, used in chapter 13 for the notes
\usepackage[protrusion=basictext]{microtype}
                            % Makes right margin neater by letting certain punctuation protrude slightly
\usepackage{fontspec}		% For the many fonts
\usepackage{censor}         % For redacted name in chapter 89
\usepackage{xstring}        % Needed for \IfInteger...

\usepackage[bookmarks=true,unicode=true,pdfborder={0 0 0},
	pdftitle={HPMOR},
	pdfauthor={LessWrong}, breaklinks={true},
	pdfkeywords={rationality},pdfencoding=auto
]{hyperref}                 % Hyperlinks and bookmarks in PDF

\usepackage{ellipsis}        % Must be loaded after fontspec and hyperref
\renewcommand{\ellipsisgap}{.16667em}
\newcommand{\ellipsisgapitalic}{.115em} % Gap must be reduced for italic (to be used manually)

\usepackage[style=german]{csquotes}
\MakeOuterQuote{"}	    % Automatic open and closing quotes

\usepackage[german]{babel}  % German language support
\usepackage{fmtcount}       % Provides \NUMBERstringnum
\usepackage{graphicx}       % for \reflectbox in header
\usepackage[normalem]{ulem} % For \censor in chapter 64

%%\usepackage{chickenize}
%\colorstretch

% Show the last row of paragraphs in red when \looseness=-1 would work
\newcommand{\ShowLoosenessHelp}{
\usepackage{luatexbase}
\directlua{dofile("widow-assist.lua")}
}

% Show grid lines in red, for grid typesetting checks
\newcommand{\ShowGrid}{
\usepackage{atbegshi,picture,xcolor}
\AtBeginShipout{%
  \AtBeginShipoutUpperLeft{%
    \color{red}%
    \put(\dimexpr 1in+\oddsidemargin,
         -\dimexpr 1in+\topmargin+\headheight+\headsep+\topskip)%
      {%
       \vtop to\dimexpr\vsize+\baselineskip{
         \hrule
         \leaders\vbox to\baselineskip{\hrule width\hsize\vfill}\vfill
       }%
      }%
  }%
}
}

%\ShowLoosenessHelp
%\ShowGrid

\renewcommand*{\thefootnote}{\fnsymbol{footnote}}% We have only one footnote in the introduction

\providecommand{\LineSpread}{1.11}% Won't override a different value set in hpmor-X.tex
\setlength{\parskip}{0pt}% Default is 0 with rubber
\setSingleSpace{\LineSpread}
\SingleSpace

\setlength{\parindent}{0em}
\setlength{\parskip}{15pt plus 1pt minus 2pt}


% \overfullrule=5pt % Mark overfull hbox with black rectangle

% Make framed boxes (for debugging) take no additional space
% \fboxrule=1pt
% \fboxsep=-1pt

%
% Set-up page sizes and margins
%
%\setstocksize{205mm}{135mm}
%\settrimmedsize{\stockheight}{\stockwidth}{*}
%\settrims{0mm}{0mm}
\setstocksize{211mm}{141mm}% Size of paper fed to the printer
\settrimmedsize{205mm}{135mm}{*}% Size of paper after trimming the edges
\settrims{3mm}{3mm}% Amount trimmed from the top and fore-edge
\settypeblocksize{164mm}{107mm}{*} % Size of main text (will be adjusted below)
%\setbinding{5mm}% Space reserved on spine side for binding: not used, margins set explicitely
%\setlrmargins{*}{*}{1.1}% Ratio between spine and edge margins: unused, margins set explicitely
% Spine and fore-edge margins: spine margin should look a bit smaller (as there
% are two next to each other) but we must account for the binding.
\setlrmarginsandblock{21mm}{17mm}{*}
\setulmargins{*}{*}{1.2}% Ratio between up and down margins
\setheadfoot{\baselineskip}{2\baselineskip}% Header height and skip from text bottom to footer bottom
\setheaderspaces{*}{*}{*}% Equal spacing above and below header
\settypeoutlayoutunit{mm}% Print dimensions in millimeters in compilation output
\checkandfixthelayout% Apply the layout

%
% Font setup: URW Garamond No. 8
%
\setmainfont[
, ExternalLocation=fonts/garamond/
, Ligatures={Common,TeX}
, UprightFont=*-Regular
, ItalicFont=*-Italic
, BoldFont=*-Bold
, BoldItalicFont=*-Bold-Italic
, Scale=0.95
]{GaramondNo8}
%
% Chapters, headings, etc.: Lumos (extra spacing looks much better)
\newfontface\hp[ExternalLocation=fonts/lumos/, LetterSpace=6.0, WordSpace=1.85]{Lumos}
% Slightly scaled up font, so that we can get a size between the 11pt \footnotesize and \scriptsize
\newfontface\hpScaledUp[ExternalLocation=fonts/lumos/, LetterSpace=6.0, WordSpace=1.85, Scale=1.05]{Lumos}
% Uppercase Lumos, used in the title pages
\newcommand{\lumos}[1]{{\hp\MakeUppercase{#1}}}
% Chapter numbers: Lumos (extreme spacing for effect)
\newfontface\hpchap[ExternalLocation=fonts/lumos/, LetterSpace=40.0, WordSpace=5.0]{Lumos}
%
% Font used to draw the magic star
\newfontface\starfont[ExternalLocation=fonts/]{Miscelanea.ttf}
%
% Draws a "magic star", used as a decoration
\newcommand*{\Star}{{\starfont{}*}}
%
% Draws three "magic stars", used as a decoration everywhere
\newcommand*{\Stars}{{\large\Star\kern-.6ex\lower1.3ex\hbox{\large\Star}%
    \kern-.1ex\raise.2ex\hbox{\tiny\Star}\spacefactor1000}}
%
% Parseltongue
\newfontface\ptsansi[
, ExternalLocation=fonts/alegreya-sans/
, Ligatures={Common,TeX}
, UprightFont=*-Light
, ItalicFont=*-LightItalic
, BoldFont=*-Regular
, BoldItalicFont=*-Italic
, Scale=0.95
]{AlegreyaSans}

%
% Section break with three spaced stars
%
\newcommand{\sbreak}{%
    \par
    \SomeVSpace
    \hfill\raisebox{-.5ex}{\Star\qquad\Star\qquad\Star}\hfill\null% So the \hfill is not removed
    \SomeVSpace
    \par\noindent
}

%
% Custom chapter style
%
\makechapterstyle{evans}{%
	\renewcommand*{\chapnamefont}{\hpchap\normalsize}
	\renewcommand*{\chapnumfont}{\chapnamefont\normalsize}
	\renewcommand*{\chaptitlefont}{\hp\large}

	\setlength{\beforechapskip}{0pt}
	\setlength{\midchapskip}{1.1cm plus .5\baselineskip minus .5\baselineskip}
	\setlength{\afterchapskip}{1.1cm plus .5\baselineskip minus .5\baselineskip}% Some stretchable glue is necessary to avoid vbox fill messages

	\renewcommand*{\printchapternum}{% " C H A P T E R   O N E "
		\begin{center} \chapnumfont \hyperref[contents]{KAPITEL \NUMBERstringnum{\thechapter}}\end{center}}

	\renewcommand*{\printchaptername}{}% Not needed, "CHAPTER" added above

	\renewcommand*{\printchaptertitle}[1]{% "A DAY OF VERY LOW PROBABILITY"
		\begin{center}\chaptitlefont \MakeUppercase{##1}\end{center}}

	\renewcommand*{\chaptermark}[1]{% "CHAPTER ONE" and "A DAY OF..."
		\markboth{\MakeUppercase{\chaptername}~\thechapter}%
			{\MakeUppercase{##1}}}

	\renewcommand*{\tocmark}{\markboth{}{\MakeUppercase{Contents}}}

	\renewcommand{\tocheadstart}{\chapterheadstart}
	\renewcommand{\aftertoctitle}{\thispagestyle{empty}\afterchaptertitle}
}
\chapterstyle{evans}% Actually implements the above chapter style code

% Macro to convert a number to uppercase Roman
\makeatletter
\newcommand*{\RomNum}[1]{\expandafter\@slowromancap\romannumeral #1@}
\makeatother

% Command to choose where to break the line in long chapter titles
\newcommand*{\wrapchapter}[2]{
	\chapter[#1 #2]           % TOC and page header
	{#1\protect\linebreak #2}}% Title

% Macro to format an abbreviation such as "TSPE" for "the Stanford Prison Experiment"
\newcommand{\abbrev}[1]{{\scshape \MakeLowercase{#1}}}

% Formatting of the part numbers.
% Numbers are formatted as roman numerals (0 is left unchanged).
% Non-numbers are also left unchanged, and without a "Part~" prefix. This is used for chapter 77.
% This implementation is expandable and so causes no issue with hyperref.
\newcommand{\formatPart}[1]{\ifnum0<0#1\relax Part~\RomNum{#1}\else \if 0#1\relax Part~0\else #1\fi\fi}
\newcommand{\formatPartShort}[1]{\ifnum0<0#1\relax \RomNum{#1}\else #1\fi}% Same case for 0 and non-numeric

% \partchapter{The Stanford Prison Experiment}{1}
% TOC: The Stanford Prison Experiment, Part I
% Page header: The Stanford Prison Experiment I
% Title: The Stanford Prison Experiment, Part I
\newcommand{\partchapter}[2]{%
	\chapter[#1, \formatPart{#2}]%
		[#1 \formatPartShort{#2}]%
		{#1, \formatPart{#2}}}

% \namedpartchapter{Taboo Tradeoffs}{2}{The Horns Effect}
% TOC: Taboo Tradeoffs, Part II: The Horns Effect
% Page header: Taboo Tradeoffs II: The Horns Effect
% Title: Taboo Tradeoffs, Part II: \\? The Horns Effect
% The optional argument affects whether the line is broken after "Part ...:"
\newcommand{\namedpartchapter}[5][1]{%
	\chapter[#2, \formatPart{#3}: #4]%
		[\mbox{#2 \formatPartShort{#3}:} \mbox{#4}]%
		{#2, \formatPart{#3}:\protect\linebreak[#1] #4}}

% \abbrevnamedpartchapter{The Stanford Prison Experiment}{TSPE}{6}{Constrained Optimization}
% TOC: \abbrev{TSPE}, Part VI: Constrained Optimization
% Page header: TSPE VI: Constrained Optimization
% Title: The Stanford Prison Experiment, Part VI: \\? Constrained Optimization
% The optional argument affects whether the line is broken after "Part ...:"
\newcommand{\abbrevnamedpartchapter}[6][1]{%
	\chapter[\texorpdfstring{\abbrev{#3}, \formatPart{#4}: #5}{#3, \formatPart{#4}: #5}]%
		[\mbox{#3 \formatPartShort{#4}:} \mbox{#5}]%
		{#2, \formatPart{#4}:\protect\linebreak[#1] #5}}

% Single space after a period (also avoids having to take care of spacing after
% abbreviations)
\frenchspacing

% If a paragraph cannot be typeset without overfull boxes, try adding some
% stretchable glue in each line
\setlength{\emergencystretch}{.04\textwidth}

% Lettrine font pick and configuration
%
\renewcommand{\LettrineFontHook}{\hp} % Uses Lumos for the large letter
\renewcommand{\LettrineTextFont}{} % Normal font for the rest of the first word
% For when the first word is in italics, use this command instead
\newcommand{\lettrinemph}[3][]{\lettrine[#1]{#2}{\emph{#3}}}
\setcounter{DefaultLines}{1}
% Large cap instead of drop cap because of many short first lines
\renewcommand{\DefaultLoversize}{0}
\renewcommand{\DefaultLraise}{0}

%
% Page numbering, footer, header
%
\newcommand{\pageInFooter}{{\small\makebox[2em][c]{\thepage}}}
% Page number centered in footer
% Same on all pages in body text
\makeevenfoot{plain}{}{\pageInFooter}{}
\makeoddfoot{plain}{}{\pageInFooter}{}
\makeevenfoot{headings}{}{\pageInFooter}{}
\makeoddfoot{headings}{}{\pageInFooter}{}
%
% Left side header: chapter number
% Right side header: chapter name
\makeevenhead{headings}{}{\hpScaledUp\hyperref[contents]{\scriptsize\leftmark}}{}
\makeoddhead{headings}{}{\hpScaledUp\scriptsize\rightmark}{}
% Style definitions and hacks
%
% In the text, the ellipsis is encoded with \el, which we define here to \dots
% preceded by the same amount of space as used between the dots (see the
% ellipsis package configuration above). Note that the ellpsis package is
% necessary for \dots to generate three spaced dots when compiling with
% lualatex or xelatex (otherwise one gets a single glyph of 3 packed dots). We
% use \dots instead of \ldots because of a buggy interaction with the amsmath
% package.
\newcommand*{\el}{\leavevmode\kern \ellipsisgap\dots} % Ellipsis definition
%
% Logical formatting macros
\newcommand*\parsel[1]{{\ptsansi\textit{#1}}} % Parseltongue
\newcommand*{\prophecy}[1]{\textsc{#1}} % Prophecies
\newcommand*{\headline}[1]{\begin{center}\textsc{#1}\end{center}} % Newspaper headlines
\newcommand*{\inlineheadline}[1]{\textsc{#1}} % Newspaper headlines inline with text
%
% Macros for time of day. Using '/' suffix to avoid eating a following space.
\def\AM/{{\,\scshape am}} % AM in small caps
\def\PM/{{\,\scshape pm}} % PM in small caps

% Define vertical spaces between .5 and 1.5 lines to keep text on a grid as much as possible.
% (these spaces are typically used before and after a list, note etc. Using
% twice .5 lines or twice 1.5 lines means we get back to grid typesetting after
% the second space.)
\newlength{\gridvskip}
\setlength{\gridvskip}{1\baselineskip}
\newlength{\gridsmallvskip}
\setlength{\gridsmallvskip}{.5\baselineskip}
\newcommand*{\SomeVSpace}{\vskip \gridvskip\relax}
\newcommand*{\SmallVSpace}{\vskip \gridsmallvskip\relax}

% Subsections style and spacing (Aftermath:, Act:, etc.)
\setsubsecheadstyle{\scshape} % Subsection titles in small caps
\setbeforesubsecskip{0\baselineskip}% Note: a negative value here means no indent
\setaftersubsecskip{1\baselineskip}

% Space around lists
%
% Remove extra space around lists and between items. It looks better and helps
% typesetting on the grid. This command must come first, or it would overwrite
% some of the other changes below.
\tightlists*
% Set space before and after lists and center environments
\setlength{\topsep}{\gridvskip}
% Set additional space added to lists that start a new paragraph. Also used by
% adjustwidth
\setlength{\partopsep}{0pt}

% The tabular environment results in some vertical space missing after the tabular,
% allow TeX to fill it with some stretchable glue
\let\Oldendtabular\endtabular
\def\endtabular{\Oldendtabular\vskip0pt plus .1\baselineskip\relax}

% Makes a written note in italics (first used, Hogwarts acceptance letter)
\newenvironment{writtenNote}{%
    \SomeVSpace
    \begin{adjustwidth}{\parindent}{\parindent}%
        \begin{em}\par\noindent
}{%
        \end{em}%
    \end{adjustwidth}%
    \SomeVSpace
}
% Makes a letter address and closing, for use with above writtenNote environment
\newcommand{\letterAddress}[1]{\pagebreak[1]\noindent{}#1\nopagebreak[4]\par}
\newcommand{\letterClosing}[2][\vskip 1\baselineskip]{\nopagebreak[4]#1\par\nopagebreak[5]\noindent#2}

% Format a "thought" paragraph: vertical space around, consistent and controllable indent
\newenvironment{thought}[1][\parindent]{%
    \SmallVSpace% vertical space, also terminates previous paragraph
    \hangindent=#1% apply indent given in optional argument
    \hangafter=0% apply indent from the first line
    \parindent=0pt% disable extra indent for first line
}{%
    \SmallVSpace% vertical space, also terminates paragraph
}

% Commands for formatting the titles of the fourth omake, in chapter 64
\newcommand{\omakeFourTitle}[2]{%
\vskip 0pt plus 1\baselineskip%
\noindent\hfill\scalebox{#2}{#1}\hfill\mbox{}%
\vskip 1\baselineskip plus 0pt minus .5\baselineskip
}
%
\newcommand{\omakeFourFontTextLotR}{%
\fontspec[ExternalLocation=fonts/ringbearer/]{RingBearer}
\settowidth{\versewidth}{\mbox{the}}
Lord\scalebox{.40}{\parbox[b]{\versewidth}{
	\centering of\\\nointerlineskip\vskip 4pt the}}Rationalit\raisebox{-.32ex}{Y}
}
%
\newcommand{\omakeFourFontTextWardrobe}{%
\fontspec[ExternalLocation=fonts/narnia/]{NarniaBLL}
456}% 4, 5 and 6 encode "The Witch", "and" and "the Wardrobe"
%
\newcommand{\omakeFourFontTextThunder}{%
\fontspec[ExternalLocation=fonts/thundercats/]{Thundercats}
ThunderSmarts}
%
\newcommand{\omakeFourFontTextTwilight}{%
\fontspec[ExternalLocation=fonts/]{Twilight}
Utilitarian Twilight}
%
\newcommand*{\omakeFourLogo}[3][]{%
\omakeFourTitle{%
    \includegraphics[width=#2\linewidth,#1]{omake4-titles/#3}%
}{1}}

% Hyphenation corrections
%
\hyphenation{Her-mi-ne Gran-ger bru-shes Gryf-fin-dor Gryf-fin-dors Le-strange
some-where which-ev-er Hog-warts re-pli-cat-ed ran-dom sta-tis-ti-cal
Wi-zen-gam-ot an-aly-se an-aly-sis re-mem-ber Mc-Gon-a-gall fun-da-men-tal
Dum-ble-dore thou-sand Huf-fle-puff Huf-fle-puffs Slyther-in Slyther-ins Har-ry
Flit-wick win-gar-di-um le-viosa Dra-co Tre-lawney Sev-erus Mal-foy}

\newcommand{\UnsetValue}{UNSET}
\newcommand{\CoverAuthor}{Lily Yao Lu}% Set cover author here, to appear in colophon
\newcommand{\PdfSourceUrl}{https://github.com/friebetill/hpmor}% Set URL for the book's source code, to appear in colophon

% Check that the cover author and PDF URL are set and if not, print a message to the log
\newcommand{\CheckIsSet}[1]{{
    \edef\first{\UnsetValue}
    \edef\second{#1}
    \ifx\first\second
        \typeout{^^JYOU NEED TO SET \string#1\space IN hp-header.tex^^J}
    \fi}}
\CheckIsSet{\CoverAuthor}
\CheckIsSet{\PdfSourceUrl}
