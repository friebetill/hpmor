% This file includes all the generic formatting for HPatMoR. This mostly entails configuring
% the memoir package, though “configuring” on occasion means “completely messing it up”.

\RequirePackage{ucntn} % Provides \NUMTONAME, for all-caps output
%~ \RequirePackage{hp-hacks}
\RequirePackage{calc}

%
% Set-up page sizes
%
\setstocksize{9in}{6in}
\settrimmedsize{\stockheight}{\stockwidth}{*}
\settypeblocksize{\topskip + 36\baselineskip}{4.6in}{*}
\setlrmargins{*}{*}{0.8}
\setulmargins{*}{*}{0.8}
\setheadfoot{\topskip + \baselineskip}{2\baselineskip}
\checkandfixthelayout[fixed]

\fixdvipslayout % fix for xelatex

%
% Fonts used generally (specific fonts used only once or twice are not here).
%

\setmainfont[
, Numbers=OldStyle
, Ligatures={Common,TeX}
, Extension=.ttf
, UprightFont=*-Regular
, ItalicFont=*-Italic
, BoldFont=*-Medium
, BoldItalicFont=*-Medium-Italic
]{GaramondNo8}

\newfontface\hp[ExternalLocation, LetterSpace=18.0, WordSpace=1.5]{Lumos}
\newcommand{\lumos}[1]{{\hp\MakeUppercase{#1}}}
\newfontface\hpchap[ExternalLocation, LetterSpace=96.0, WordSpace=2.5]{Lumos}

\newfontface\abysmal[ExternalLocation, Ligatures={Common,TeX}]{AlegreyaSans-LightItalic}
\newcommand\parsel[1]{{\abysmal #1}}

%
% Page numbering/footer
%
\def\pageInFooter{{\small\Star\ \makebox[2em][c]{\thepage}\Star}}
\makeevenfoot{plain}{}{\pageInFooter}{}
\makeoddfoot{plain}{}{\pageInFooter}{}
\makeevenfoot{headings}{}{\pageInFooter}{}
\makeoddfoot{headings}{}{\pageInFooter}{}

%
% Custom chapter style
%
\makeatletter
\makechapterstyle{evans}{%
	\renewcommand*{\chapnamefont}{\hpchap\normalsize}
	\renewcommand*{\chapnumfont}{\chapnamefont\normalsize}
	\renewcommand*{\chaptitlefont}{\hp\Large}

	\setlength{\beforechapskip}{2\baselineskip plus 1\baselineskip minus 1\baselineskip }
	\setlength{\midchapskip}{0pt}
	\setlength{\afterchapskip}{1\baselineskip}

	\renewcommand*{\printchapternum}{%
		\begin{center} \chapnumfont \hyperref[contents]{CHAPTER \NUMTONAME{\thechapter}\end{center}
%\IfInteger{\thechapter}{\NUMTONAME{\thechapter}}{\thechapter}
}}
	% \renewcommand*{\printchapternonum}{%
		% \chapnumfont \hyperref[contents]{Something no num}}

	\renewcommand*{\printchaptername}{%
%		\centering \chapnamefont \hyperref[contents]{\MakeUppercase{\@chapapp}}
	}

	\renewcommand*{\printchaptertitle}[1]{%
		\vskip 1cm
		\begin{center}\chaptitlefont \MakeUppercase{##1}\end{center}\par
		\vskip 1cm
	}

	\renewcommand*{\chaptermark}[1]{
		\markboth{
                  \MakeUppercase{##1}
                }{
		  \MakeUppercase{\chaptername}~
%                  \IfInteger{\thechapter}{
%                    \NUMTONAME{\thechapter}
%                    spell: \thechapter
%                  }{
                    \NUMTONAME{\thechapter}
%                  }
                }
        }

	\renewcommand*{\tocmark}{\markboth{}{\MakeUppercase{Contents}}}

	\renewcommand{\tocheadstart}{\chapterheadstart}
	\renewcommand{\aftertoctitle}{\thispagestyle{empty}\afterchaptertitle}

  % \renewcommand*{\afterchapternum}{\par\nobreak\vskip 25pt}
  % \def\chapterheadstart{\vspace*{\beforechapskip}}%
  % \def\printchaptername{\chapnamefont \@chapapp}%
  % \def\chapternamenum{\space}%
  % \def\printchapternum{\chapnumfont \thechapter}%
  % \def\afterchapternum{\par\nobreak\vskip \midchapskip}%
  % \def\printchapternonum{}%
  % \def\printchaptertitle##1{\chaptitlefont ##1}%
  % \def\afterchaptertitle{\par\nobreak\vskip \afterchapskip}%
}
\makeatother
\chapterstyle{evans}

\def\chapterheadstart{\vspace*{-1\baselineskip}\vspace*{-1\topskip}\vspace*{\beforechapskip}}

%
% Subsection
%
\setsubsecheadstyle{\scshape}
%\bottomsectionskip=0pt plus 1fill
%~ \raggedbottomsection
%\subsecindent %            heading indent
\beforesubsecskip=1.5\baselineskip %        skip before the heading
\aftersubsecskip=.5\baselineskip plus .5\baselineskip %         skip after the heading
\setsubsechook{\nopagebreak\vskip 0pt plus 3\baselineskip}

%
% Lettrine font pick
%
\renewcommand{\LettrineFontHook}{\hp}
\renewcommand{\LettrineTextFont}{}

\newcommand\lettrinemph[3][]{\lettrine[#1]{#2}{\emph{#3}}}
% \setcounter{DefaultLines}{1}
\renewcommand{\DefaultLoversize}{0.4}
\renewcommand{\DefaultLraise}{0}

%
% Epigraph configuration
%
\setlength{\epigraphwidth}{\textwidth}

\epigraphtextposition{flushleftright}
\epigraphfontsize{\footnotesize}
\setlength{\epigraphrule}{0pt}
\setlength{\beforeepigraphskip}{0pt}
\setlength{\afterepigraphskip}{\baselineskip}

\makeatletter
\renewcommand{\epigraph}[2]{%
	\vspace{\beforeepigraphskip}%
	{%
		\epigraphsize%
		\begin{\epigraphflush}%
			\begin{minipage}{\epigraphwidth}%
				\centering\emph{#1}%
			\end{minipage}%
		\end{\epigraphflush}%
	}%
	\mbox{}\sbreak%
}
\makeatother

%
%
%
\usepackage{graphicx} % for \reflectbox
\makeevenhead{headings}{\Stars}{
	\hp\hyperref[contents]{\rightmark}}{\reflectbox{\Stars}}
\makeatletter
\makeoddhead{headings}{\Stars}{\parbox{97mm}{\centering\hp\leftmark}}{\reflectbox{\Stars}}

\copypagestyle{cleared}{empty}
\makepsmarks{headings}{%
\createmark{chapter}{right}{shownumber}{\@chapapp\ }{. \ }}
\makeatother

%%%%%%%%%%%%%
\setlength{\emergencystretch}{.06\textwidth}

\clubpenalty=50
\widowpenalty=100
\brokenpenalty=10000

% Allow linebreaks after em-dash and hyphens, except when they’re followed by punctuation

\newXeTeXintercharclass \punctuationClass

\XeTeXcharclass `\’ \punctuationClass
\XeTeXcharclass `\‘ \punctuationClass
\XeTeXcharclass `\“ \punctuationClass
\XeTeXcharclass `\” \punctuationClass
\XeTeXcharclass `\. \punctuationClass
\XeTeXcharclass `\, \punctuationClass
\XeTeXcharclass `\: \punctuationClass
\XeTeXcharclass `\? \punctuationClass
\XeTeXcharclass `\! \punctuationClass
\XeTeXcharclass `\: \punctuationClass

\newXeTeXintercharclass \digitClass
\XeTeXcharclass `\0 \digitClass
\XeTeXcharclass `\1 \digitClass
\XeTeXcharclass `\2 \digitClass
\XeTeXcharclass `\3 \digitClass
\XeTeXcharclass `\4 \digitClass
\XeTeXcharclass `\5 \digitClass
\XeTeXcharclass `\6 \digitClass
\XeTeXcharclass `\7 \digitClass
\XeTeXcharclass `\8 \digitClass
\XeTeXcharclass `\9 \digitClass

\newXeTeXintercharclass \dashClass
\XeTeXcharclass `\— \dashClass % em
\XeTeXcharclass `\– \dashClass % en

\newXeTeXintercharclass \hyphenClass
\XeTeXcharclass `\- \hyphenClass % hyphen

\XeTeXinterchartokenstate = 1

\def\morhyphenpenalty{75}
\exhyphenpenalty=10000

\XeTeXinterchartoks \hyphenClass 0 = {\hskip 0pt\penalty \morhyphenpenalty}

\XeTeXinterchartoks \dashClass 0 = {\,}
\XeTeXinterchartoks 0 \dashClass = {\,}
\XeTeXinterchartoks \dashClass 255 = {\,}
\XeTeXinterchartoks 255 \dashClass = {\,}

%
% Adjust space around lists
%
\setlength{\topsep}{.5\baselineskip plus 1\baselineskip minus .5\baselineskip}
\setlength{\partopsep}{.5\baselineskip plus 1\baselineskip minus .5\baselineskip}

\usepackage[normalem]{ulem}

\usepackage{xfrac}

\usepackage{censor}

\usepackage[useregional]{datetime2}

%\usepackage[xspace]{ellipsis}
%\makeatletter
%\renewcommand{\ellipsis@before}{\kern \ellipsisgap}
%\makeatother
